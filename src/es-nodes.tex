%
% %CopyrightBegin%
%
% Copyright Ericsson AB 2017. All Rights Reserved.
%
% Licensed under the Apache License, Version 2.0 (the "License");
% you may not use this file except in compliance with the License.
% You may obtain a copy of the License at
%
%     http://www.apache.org/licenses/LICENSE-2.0
%
% Unless required by applicable law or agreed to in writing, software
% distributed under the License is distributed on an "AS IS" BASIS,
% WITHOUT WARRANTIES OR CONDITIONS OF ANY KIND, either express or implied.
% See the License for the specific language governing permissions and
% limitations under the License.
%
% %CopyrightEnd%
%

\chapter{Nodes}

\label{chapter:nodes}

\section{Single-node and multi-node systems}

\label{section:single-multi-node}

An \Erlang\ node is the operating system for \Erlang\ processes.  
Every process runs on some specific node and once a process has been created
it remains on that node.

\index{node!communicating|(}
\index{node!isolated|(}
By default an \Erlang\ node is \emph{isolated} and constitutes a single-node system.
In order to become a \emph{communicating} node that can be part of a cluster
of \Erlang\ nodes, it
must have a process registered under the name
\T{net_kernel}\index{net_kernel process name@\T{net_kernel} process name} that has
opened a port through which it communicates
with other nodes using the protocol described in \S\ref{chapter:external-format}.
\index{node!communicating|)}
\index{node!isolated|)}

\index{node!name|(}
Every \Erlang\ node has a name, which is an atom.  The printname of
the atom always contains exactly one \T{@} character\index{"@
character@\T{"@} character} (\T{\char`\\\ifStd u0040\fi\ifOld
100\fi}).  An isolated node always has the name
\T{nonode@nohost}\index{nonode"@nohost node name@\T{nonode"@nohost} node name}.
In a communicating node, the part of the printname after the \T{@}\
must be the network name of the host computer on which the node
resides (typically an Internet domain name).  We call an atom with
such a printname a \emph{valid node name}.  The name of each
communicating node must be unique.  As the node name contains the
network name of the host computer (which is assumed to be unique), it
is sufficient to ensure that all nodes running on the same host
computer have unique names.  This is ensured by the use of EPMD, as
described in \S\ref{section:registering-nodes}.  Because node names
are unique and that nodes are always referred to through their names
in the language, we will identify nodes with their names.
\index{node!name|)}

\index{node!communicating|(}
A node with a \T{net_kernel}\index{net_kernel process name@\T{net_kernel} process name}
process that has opened a port \TZ{R} as described above
becomes a communicating node with name \TZ{A} by some process making a BIF call
\ifOld \T{erlang:set_node(\TZ{A},\TZ{R})}\index{set_node/2 BIF@\T{set_node/2} BIF}
(\S\ref{section:setnode2}).\fi
\ifStd \T{node:set_node(\TZ{A},\TZ{R})}\index{set_node/2 BIF@\T{set_node/2} BIF}
(\S\ref{section:node:setnode2}).\fi
CHECK THE ARGUMENTS!!!
\ifOld It is also possible to make a node communicating from the beginning
\cite[pp.~5--9]{otp-dev-ref}.\fi
\ifStd An implementation should provide a
way to make a node communicating from the beginning.\fi
\index{node!communicating|)}

\index{node!isolated|(}
On an isolated node the effects and result is undefined when
any other node name than
\T{nonode@nohost}\index{nonode"@nohost node name@\T{nonode"@nohost} node name}
is used in a BIF that have an explicit node argument
or when sending messages.  (This is not stated again in the descriptions
of BIFs and send expressions.)
In the remainder of this chapter, \Erlang\ nodes are assumed to be
communicating unless
it is stated explicitly that a node is or may be isolated.  Moreover, as
every communicating node has a unique name and there is no way to refer to a node except
through its name, we will identify nodes with their names.  Thus, when
we write ``node \TZ{A}'', where \TZ{A} is an atom, we mean the node with
name \TZ{A}, provided that such a node exists.
\index{node!isolated|)}

\index{node!magic cookie|(}
Each node has a term associated with it, called the \emph{magic cookie} of the node.
In order for a processes on some node $\TZ{N}_1$ to communicate successfully with a
process on another node $\TZ{N}_2$, the \emph{magic cookie} of node
$\TZ{N}_2$ must be provided
in the communication from node $\TZ{N}_1$.
A group of nodes can be protected from communication
originating on other nodes by sharing a magic cookie and not disclosing it
to other nodes.  Magic cookies are discussed in more detail in
\S\ref{section:magic-cookie}.
\index{node!magic cookie|)}

\index{friend (of a node)!see{node, friendship}}
\index{node!friendship|(}
At any given time, each \Erlang\ node has a number of other nodes as
\emph{friends}.  The set of such friends may change dynamically over time.
When a process running on some node $\TZ{N}_1$ attempts to communicate with
a process
on another node $\TZ{N}_2$, node $\TZ{N}_1$
attempts to set up a friendship with node $\TZ{N}_2$ through negotiation
between the \T{net_kernel}\index{net_kernel process name@\T{net_kernel} process name}
processes of the two nodes.  Friendship is a
symmetric property so in the process, $\TZ{N}_1$
also becomes a friend of node $\TZ{N}_2$.  The friendship is terminated
when the nodes can no longer communicate or when a process residing on one
of the nodes calls the BIF
\ifOld \T{erlang:disconnect_node/1}\index{erlang:disconnect_node/1
BIF@\T{erlang:disconnect_node/1} BIF} (\S\ref{section:disconnectnode1}) \fi
\ifStd \T{node:disconnect/1}\index{disconnect/1 BIF@\T{disconnect/1} BIF}
(\S\ref{section:node:disconnect1}) \fi
with the name of the other node as argument.
Friendship is described in more detail in \S\ref{section:friendship}.
\index{node!friendship|)}

\section{Registering a node}

\label{section:registering-nodes}
\index{EPMD|(}
\index{net_kernel process name@\T{net_kernel} process name|(}

MAKE COHERENT WITH ABOVE AND FINISH!!!

Any computer that can be the host of \Erlang\ nodes must provide the service
\emph{EPMD}\index{EPMD} (\Erlang\ Port Mapper Daemon).
The EPMD service on the host computer
stores the names of all \Erlang\ nodes running on the computer, ensuring that
the names are unique.  It will provide a handle to the port of the \T{net_kernel}
process of any node residing on the host computer, given the name of the node
(cf.\ \S\ref{section:friendship}).

MUST OBTAIN INFO ABOUT SET_NODE/2-3 BEFORE FIXING NEXT PARAGRAPH!!!

An \Erlang\ node becomes a communicating node by spawning a process registered
with the name \T{net_kernel}.  It should
open a port through which other nodes can communicate with it.  It then
contacts the
EPMD service on its host computer, requesting to register the node with a certain
name and the opened port.  The protocol for this registration is described
in \S\ref{section:epmd-register}.
\index{net_kernel process name@\T{net_kernel} process name|)}
\index{EPMD|)}

\section{Initializing and terminating friendship}

\label{section:friendship}
\index{node!friendship|(}

\index{net_kernel process name@\T{net_kernel} process name|(}
When a signal is to be delivered from a process residing on some node $\TZ{N}_1$
to a process residing on a different node $\TZ{N}_2$, $\TZ{N}_2$ must be a friend
of $\TZ{N}_1$.  NOT CORRECT!!!:
Concretely, this means that the \T{net_kernel} process of
node $\TZ{N}_1$ has opened a port to the \T{net_kernel} process of
node $\TZ{N}_2$.  Associated with this port is an
\emph{atom table}\index{node!atom table}, as
described in \S\ref{section:atom-tables}.
\index{net_kernel process name@\T{net_kernel} process name|)}

Making node $\TZ{N}_2$ a friend of node $\TZ{N}_1$ is thus initiated when
a process on node $\TZ{N}_1$ needs to communicate with a process on node $\TZ{N}_2$
and node $\TZ{N}_2$ is not yet a friend of node $\TZ{N}_1$.  The friendship
remains until node $\TZ{N}_1$ detects that it cannot communicate with node
$\TZ{N}_2$ or either node calls the BIF
\ifOld \T{erlang:disconnect_node/1}\index{erlang:disconnect_node/1
BIF@\T{erlang:disconnect_node/1} BIF} (\S\ref{section:disconnectnode1}) \fi
\ifStd \T{node:disconnect/1}\index{disconnect/1 BIF@\T{disconnect/1} BIF}
(\S\ref{section:node:disconnect1}) \fi
with the name of the other
node as argument.

When node $\TZ{N}_1$ makes node $\TZ{N}_2$ a friend, node $\TZ{N}_1$ should also
become a friend of node $\TZ{N}_2$.  Under normal circumstances, friendship is thus
a reflexive relation.  When the means of communication is broken
between two nodes that are friends, they may not detect this simultaneously
and friendship may then temporarily be unilateral.  Should communications be restored
in a situation where node $\TZ{N}_1$ is still a friend of node $\TZ{N}_2$ but
node $\TZ{N}_2$ is no longer a friend of node $\TZ{N}_1$, then
$\TZ{N}_1$ should cease to be a friend of node $\TZ{N}_2$ before (bilateral)
friendship is resumed.  (This is to ensure that both processes on $\TZ{N}_1$ monitoring
friendship with $\TZ{N}_2$ and processes on $\TZ{N}_2$ monitoring
friendship with $\TZ{N}_1$ will be notified that friendship has been broken,
cf.\ \S\ref{section:monitor-node}.)

NEXT PARAGRAPH IS HIGHLY UNCERTAIN!!!

\index{net_kernel process name@\T{net_kernel} process name|(}
When a node $\TZ{N}_1$ wishes to initiate friendship with a node $\TZ{N}_2$ the
\T{net_kernel} process of node $\TZ{N}_1$
first contacts the EPMD\index{EPMD} on the host computer of node $\TZ{N}_2$ and requests a
handle of the port of the \T{net_kernel} process of node $\TZ{N}_2$, as described
in \S\ref{section:epmd-find}.  If this succeeds, then the \T{net_kernel}
process of node $\TZ{N}_1$
contacts the \T{net_kernel} process of node $\TZ{N}_2$ and completes the setup of
the friendship as described in \S\ref{chapter:external-format}.  That communication
also establishes node $\TZ{N}_2$ as a friend of node $\TZ{N}_1$.
\index{net_kernel process name@\T{net_kernel} process name|)}

The current set of friends of a node \TZ{N} is referred to in this document as
\T{friends[\Z{N}]}\index{friends node property@\T{friends} node property}.
\index{node!friendship|)}

\section{Remote communication and magic cookies}

\label{section:magic-cookie}
\index{node!magic cookie|(}

All forms of communication between two
processes residing on different nodes go through the
\T{net_kernel}\index{net_kernel process name@\T{net_kernel} process name}
processes on those nodes.  Note that all communication between processes
is in the form of signals\index{signal!communication} (\S\ref{section:signals}).

Each node \TZ{N} has associated with it an atom
\T{magic_cookie[\Z{N}]}\index{magic_cookie node property@\T{magic_cookie} node property}
that is called the \emph{magic cookie} of the node.  Any process residing on node \TZ{N}
can obtain the magic cookie of node \TZ{N} through a BIF call
\ifOld \T{erlang:get_cookie()}\index{get_cookie/0 BIF@\T{get_cookie/0} BIF}
(\S\ref{section:getcookie0}).\fi
\ifStd \T{node:get_cookie()}\index{get_cookie/0 BIF@\T{get_cookie/0} BIF}
(\S\ref{section:node:getcookie0}).\fi
The magic cookie of node \TZ{N} can be set
by any process residing on node \TZ{N}
to some atom \TZ{A} through a BIF call
\ifOld \T{erlang:set_cookie(\Z{N},\Z{A})}\index{set_cookie/2 BIF@\T{set_cookie/2} BIF}
(\S\ref{section:setcookie2}).\fi
\ifStd \T{node:set_cookie(\Z{N},\Z{A})}\index{set_cookie/2 BIF@\T{set_cookie/2} BIF}
(\S\ref{section:node:setcookie2}).\fi

Each node also has a table
\T{magic_cookies[\Z{N}]}\index{magic_cookies node property@\T{magic_cookies} node property}
in which the keys are node names and the values are
the presumed magic cookies of those nodes.  The presumed magic
cookie on node $\TZ{N}_1$ of a node $\TZ{N}_2$ cannot be retrieved but can
be set to some atom \TZ{A} by any process residing on node $\TZ{N}_1$ through a BIF
call
\ifOld \T{erlang:set_cookie($\Z{N}_2$,\Z{A})}\index{set_cookie/2 BIF@\T{set_cookie/2} BIF}
(\S\ref{section:setcookie2}).\fi
\ifStd \T{node:set_cookie($\Z{N}_2$,\Z{A})}\index{set_cookie/2 BIF@\T{set_cookie/2} BIF}
(\S\ref{section:node:setcookie2}).\fi

\index{exit!signal!sending|(}
When a process $\TZ{P}_1$ residing on a node $\TZ{N}_1$ attempts
to send a signal \TZ{S} to a process $\TZ{P}_2$ on a different node $\TZ{N}_2$,
then the following happens:

\begin{itemize}
\item \index{net_kernel process name@\T{net_kernel} process name|(}
The signal \TZ{S}, source $\TZ{P}_1$ and destination $\TZ{P}_2$ are
passed to the \T{net_kernel} process on node $\TZ{N}_1$.
\item If $\TZ{N}_2$ is not in
\T{friends[\Z{N}]}\index{friends node property@\T{friends} node property}, then the \T{net_kernel}
attempts to make it a friend of $\TZ{N}_1$.  Should this fail, the
signal $s$ is simply discarded.
\item The \T{net_kernel} process on $\TZ{N}_1$ looks up
the atom $\TZ{N}_2$ in
\T{magic_cookies[$\Z{N}_1$]}\index{magic_cookies node property@\T{magic_cookies} node property} and 
if there is no value for $\TZ{N}_2$, then the atom
\T{nocookie}\index{nocookie magic cookie@\T{nocookie} magic cookie}
is inserted for it.  Let \TZ{C} be the result of the lookup
(possibly after insertion of \T{nocookie}).
\item The \T{net_kernel} process on $\TZ{N}_1$ passes
$\TZ{P}_1$, \TZ{C}, \TZ{S} and $\TZ{P}_2$ to the \T{net_kernel}
process on $\TZ{N}_2$
using the protocol described in \S\ref{chapter:external-format}.
\item The \T{net_kernel} process on $\TZ{N}_2$ compares
\TZ{C} with \T{magic_cookie[$\Z{N}_2$]}\index{magic_cookie node property@\T{magic_cookie} node property}.
\begin{itemize}
\item If \TZ{C} and \T{magic_cookie[$\Z{N}_2$]} are (exactly) equal
(\S\ref{section:equality}), then the signal \TZ{S} is passed on to
process $\TZ{P}_2$ as described in \S\ref{section:signal-arrival}.
\item Otherwise, if the signal \TZ{S} is a message with \TZ{T} as body, a message
\T{\{$\Z{P}_1$,badcookie,$\Z{P}_2$,\Z{T}\}} is placed at the end of the
message queue of the \T{net_kernel} process of node $\TZ{N}_2$.
\item Otherwise,
\ifStd
the implementation should carry out one of the following actions:
\begin{itemize}
\item pass the signal \TZ{S}
on to process $\TZ{P}_2$ as described in \S\ref{section:signal-arrival},
\item do nothing (i.e., silently drop the signal),
\item place an informative message at the end of the
message queue of the \T{net_kernel} process on node $\TZ{N}_2$.
\end{itemize}
\fi
\ifOld
the signal is passed on to
process $\TZ{P}_2$ as described in \S\ref{section:signal-arrival}.
\fi
\end{itemize}
\end{itemize}
\index{exit!signal!sending|)}
\index{net_kernel process name@\T{net_kernel} process name|)}

\index{security|(}
The magic cookie mechanism is built into the communication mechanism
of \Erlang\ in order to ensure that processes on a node cannot
disrupt processes running on an arbitrary node.
The security model of \Erlang\ is such that if a process $\TZ{P}_1$ resides
on the same node as on a process $\TZ{P}_2$, or $\TZ{P}_1$ resides on a node
that has the correct magic cookie for the node on which $\TZ{P}_2$ resides,
then process $\TZ{P}_1$ has complete access to process $\TZ{P}_2$.  Process
$\TZ{P}_1$ can send arbitrary signals (including exit signals
with \T{kill} as reason, cf.\ \S\ref{section:signal-arrival})
to process $\TZ{P}_2$.  It is thus up to the programmer to ensure that
processes running on the same node or on nodes that have each others' correct
magic cookies can trust each other. (Typically but not necessarily one will
make sure that whenever
a node $\TZ{N}_1$ has the correct magic cookie of node $\TZ{N}_2$, node
$\TZ{N}_2$ has the correct magic cookie of node $\TZ{N}_1$.  In particular,
a cluster of nodes may share a magic cookie.)
\index{security|)}
\index{node!magic cookie|)}

\section{Process registry}

\label{section:process-registry}
\index{process!registry|(}

For each node there is a table
\T{registry[\Z{N}]}\index{registry node property@\T{registry} node property} in
which the keys are atoms naming processes residing on that node and the values
are the PIDs of the processes.
The process registry makes it possible to send messages to registered processes
on a node without knowing their PIDs.

Some processes, such as the \T{net_kernel}
of a communicating node, are spawned and registered automatically by the
\Erlang\ system.
It is also possible for any process on a node \TZ{N} to register a name \TZ{A}
for a live process \TZ{P}
residing on \TZ{N} through a BIF call
\T{register(\Z{A},\Z{P})}\index{register/2 BIF@\T{register/2} BIF}
(\S\ref{section:register2}).
There can be at most one name registered for a process.
A process name is
removed from the registry automatically when the process completes.
It is also possible for any process on a node \TZ{N} to remove the name \TZ{A}
from the registry
of \TZ{N} through a BIF call
\T{unregister(\Z{A})}\index{unregister/1 BIF@\T{unregister/1} BIF} (\S\ref{section:unregister1}).
A process \TZ{P} can look up a name
\TZ{A} in the registry on \T{node[\Z{P}]} and obtain the PID through a BIF call
\T{whereis(\Z{A})}\index{whereis/1 BIF@\T{whereis/1} BIF}
(\S\ref{section:whereis1})
or obtain a list of all currently registered names on \T{node[\Z{P}]} through
a BIF call
\T{registered()}\index{registered/0 BIF@\T{registered/0} BIF}
(\S\ref{section:registered0}).

The BIF \T{register/2} must ensure that
a process cannot be registered on node \TZ{N} unless it resides on
that node and is alive.  Also it must be ensured that when a process
completes, any name for it in the registry of the node on which it resides
is removed.
\index{process!registry|)}

\section{Monitoring of nodes}

\label{section:monitor-node}
\index{node!monitoring|(}
\ifOld \index{monitor_node/2 BIF@\T{monitor_node/2} BIF|(} \fi
\ifStd \index{node:monitor/2 BIF@\T{node:monitor/2} BIF|(} \fi

THERE ARE THINGS MISSING ABOUT THIS, ESPECIALLY WHEN DISCUSSING SIGNALS AND
THE DYNAMIC STATE.

Any process on a node $\TZ{N}_1$ can be notified when some node $\TZ{N}_2$ ceases to
be a friend of $\TZ{N}_1$.

After a process \TZ{P} on a node $\TZ{N}_1$ has made a BIF call
\ifOld \T{monitor_node($\TZ{N}_2$,true)} (\S\ref{section:monitornode2}),\fi
\ifStd \T{node:monitor($\TZ{N}_2$,true)} (\S\ref{section:node:monitor2}),\fi
where $\TZ{N}_2$ is an atom, a message \T{\{nodedown,$\TZ{N}_2$\}} will be sent to \TZ{P}
if $\TZ{N}_2$ ceases to be a friend of $\TZ{N}_1$.  If $\TZ{N}_2$ is not a friend of $\TZ{N}_1$
when the call is made, the message \T{\{nodedown,$\TZ{N}_2$\}} is sent to \TZ{P} immediately.
A BIF call
\ifOld \T{monitor_node($\TZ{N}_2$,false)} \fi
\ifStd \T{node:monitor($\TZ{N}_2$,false)} \fi
cancels the effect of a call
\ifOld \T{monitor_node($\TZ{N}_2$,true)}. \fi
\ifStd \T{node:monitor($\TZ{N}_2$,true)}. \fi
The effect of a BIF call
\ifOld \T{monitor_node($\TZ{N}_2$,false)} \fi
\ifStd \T{node:monitor($\TZ{N}_2$,false)} \fi
when there is no call
\ifOld \T{monitor_node($\TZ{N}_2$,true)} \fi
\ifStd \T{node:monitor($\TZ{N}_2$,true)} \fi
to cancel is undefined.  There can be
more than one call
\ifOld \T{monitor_node($\TZ{N}_2$,true)} \fi
\ifStd \T{node:monitor($\TZ{N}_2$,true)} \fi
in effect and \TZ{P} will receive one message
for each call
\ifOld \T{monitor_node($\TZ{N}_2$,true)} \fi
\ifStd \T{node:monitor($\TZ{N}_2$,true)} \fi
that has not been canceled when $\TZ{N}_2$ ceases
to be a friend of  $\TZ{N}_1$.
\ifOld \index{monitor_node/2 BIF@\T{monitor_node/2} BIF|)} \fi
\ifStd \index{node:monitor/2 BIF@\T{node:monitor/2} BIF|)} \fi
\index{node!monitoring|)}

\section{The state of a node}

\index{node!state|(}

When a node is started, some properties of the node are determined and will be
in effect until the node is terminated.  During that time also a dynamic state
is maintained, consisting of properties of the node that change as time passes.
This state is affected by and affects the behaviour of the node.

We refer to these as the \emph{static} and \emph{dynamic properties} of a node.  We
refer collectively to the values of the latter at a certain time as the \emph{state}
of the node at that time.

\subsection{Static properties}

\label{section:node-state-static}

\begin{Lentry}
\item[\T{creation[\Z{N}]}]
\index{creation@\T{creation}!node property|(}
When a node is started, the value of \T{creation[\Z{N}]} --- a nonnegative integer ---
is obtained from EPMD\index{EPMD}.  Its purpose is to
to distinguish the current instance of the node from previous instances
that had the same name.  The value is used when creating new refs, PIDs
and ports.
\index{creation@\T{creation}!node property|)}

\item[\T{communicating[\Z{N}]}]
\index{communicating node property@\T{communicating} node property|(}
\index{node!communicating|(}
\index{node!isolated|(}
This is a Boolean atom which is \T{false} if the node is isolated
and \T{true} if it is communicating.  When a node becomes communicating, the
property is changed from \T{false} to \T{true} but after that it cannot be changed.
(The property is
thus not strictly static as it can change once but in practice we can
treat it as being static.)
It can be accessed on the node itself through the BIF
\ifOld \T{erlang:is_alive/0}\index{is_alive/0 BIF@\T{is_alive/0} BIF}
(\S\ref{section:isalive0}).\fi
\ifStd \T{node:is_alive/0}\index{is_alive/0 BIF@\T{is_alive/0} BIF}
(\S\ref{section:node:isalive0}).\fi
\index{node!communicating|)}
\index{node!isolated|)}
\index{communicating node property@\T{communicating} node property|)}

\item[\T{name[\Z{N}]}]
\index{name node property@\T{name} node property|(}
\index{node!name|(}
This is the name of the node, which is an atom.  For an isolated node it
is always \T{nonode@nohost}\index{nonode"@nohost node name@\T{nonode"@nohost} node name}.
When a node becomes communicating, the
name is changed but after that it cannot be changed.  (The property is
thus not strictly static as it can change once but in practice we can
treat it as being static.)
It can be accessed on the node itself through the BIF
\T{node/0}\index{node/0 BIF@\T{node/0} BIF} (\S\ref{section:node0}).
\index{node!name|)}
\index{name node property@\T{name} node property|)}

\item[\T{preloaded[\Z{N}]}]
\index{preloaded node property@\T{preloaded} node property|(}
\index{module!preloaded on a node|(}
This is a set of the names of the modules that were loaded as part of
starting the node.  Although there should normally never be
any reason to delete these modules, their presence in the (static)
set does not imply that
they are still loaded.
The set can be accessed on the node itself through the BIF
\ifOld \T{erlang:preloaded/0}\index{preloaded/0 BIF@\T{preloaded/0} BIF}
(\S\ref{section:preloaded0}).\fi
\ifStd \T{node:preloaded/0}\index{preloaded/0 BIF@\T{preloaded/0} BIF}
(\S\ref{section:node:preloaded0}).\fi
\index{module!preloaded on a node|)}
\index{preloaded node property@\T{preloaded} node property|)}
\end{Lentry}

\subsection{Dynamic properties}

\label{section:node-state-dynamic}

\begin{Lentry}
\item[\T{atom_tables[\Z{N}]}]
\index{atom_tables node property@\T{atom_tables} node property|(}
\index{node!atom table|(}
This is a table that for each friend of \TZ{N} contains a row with the name
and the atom table (\S\ref{section:atom-tables}) for that friend as key and value,
respectively.
When \TZ{N} gets a new friend, a row should be added to \T{atom_tables[\Z{N}]} with
the name of the friend as key and an empty atom table as value.
When a node ceases to be a friend of \TZ{N}, its row in
\T{atom_tables[\Z{N}]} must be removed.
\index{node!atom table|)}
\index{atom_tables node property@\T{atom_tables} node property|)}

\item[\T{distribution_port[\Z{N}]}]
\index{distribution_port node property@\T{distribution_port} node property|(}
On a communicating node, this is the port that was given as second argument
to \T{node:alive/2} ???.
The EPMD\index{EPMD} server for the computer on which the node resides
will communicate with the \T{net_kernel}\index{net_kernel process name@\T{net_kernel} process name}
process of \TZ{N} through that port.
\index{distribution_port node property@\T{distribution_port} node property|)}

\item[\T{entry_points[\Z{N}]}]
\index{entry_points node property@\T{entry_points} node property|(}
This is a table mapping triples consisting of a module name (an atom), a function
symbol (an atom) and an arity (a nonnegative integer) to
\ifStd implementation-defined values that denote \fi
entry points to executable code.  The table cannot be accessed
directly but is used when evaluating a function application where the function is specified
through a module name, a function symbol and an arity (\S\ref{section:appl-named-function}).
The table is updated by BIFs for module
dynamics (\S\ref{chapter:module-dynamics},
\ifOld\S\ref{section:module-bifs}\fi
\ifStd\S\ref{section:module-module}\fi).
\index{entry_points node property@\T{entry_points} node property|)}

\item[\T{friends[\Z{N}]}]
\index{friends node property@\T{friends} node property|(}
\index{node!friendship|(}
This is a set of atoms that are names of nodes that are friends of node \TZ{N}
(\S\ref{section:friendship}).
It can be accessed on node \TZ{N} by calling the BIF
\T{nodes/0}\index{nodes/0 BIF@\T{nodes/0} BIF}
(\S\ref{section:nodes0}).  The set is
added to when a friendship has been established with another node, typically
because a process on one of the nodes has attempted to communicate with a process
on the other node.  The set is subtracted from when node \TZ{N} loses contact
with a friend node (e.g., because communication between host computers has been
lost, the other node has been terminated, or the host computer on which it
resides has been restarted), or when a process on either node calls the
BIF
\ifOld \T{erlang:disconnect_node/1}\index{erlang:disconnect_node/1
BIF@\T{erlang:disconnect_node/1} BIF} (\S\ref{section:disconnectnode1}) \fi
\ifStd \T{node:disconnect/1}\index{disconnect/1 BIF@\T{disconnect/1} BIF}
(\S\ref{section:node:disconnect1}) \fi
with the name of the other node as argument.
\index{node!friendship|)}
\index{friends node property@\T{friends} node property|)}

\item[\T{garbage_collection[\Z{N}]}]
\index{garbage_collection node property@\T{garbage_collection} node property|(}
\index{current_gc node operation@\T{current_gc} node operation|(}
This is the state of the garbage collection gauge, which is updated
automatically when processes collect garbage to reflect the number of
garbage collection operations that have been carried out by processes
on the node and the number of memory words reclaimed by such
operations.  It can be accessed through the operation
\T{current_gc[\Z{N}]}, which returns a 3-tuple
\T{\{\Z{NumberOfGCs},\Z{WordsReclaimed},0\}} of integers
where \TZ{NumberOfGCs} is the total number of garbage
collection operations that have been carried out by
processes on the node and \TZ{WordsReclaimed} is the total
number of memory words reclaimed by such operations.  (The
third integer is always zero and is there to confuse the
enemy.)
\index{current_gc node operation@\T{current_gc} node operation|)}
\index{garbage_collection node property@\T{garbage_collection} node property|)}

\item[\T{magic_cookie[\Z{N}]}]
\index{magic_cookie node property@\T{magic_cookie} node property|(}
\index{node!magic cookie|(}
This term must be provided by any process on another node that wishes
to communicate with a process on node \TZ{N} (cf.\
\S\ref{section:magic-cookie}).  The value of \T{magic_cookie[\Z{N}]}
can be obtained by a process on node \TZ{N} through the BIF
\ifOld \T{erlang:get_cookie/0}\index{get_cookie/0 BIF@\T{get_cookie/0} BIF}
(\S\ref{section:getcookie0}).\fi
\ifStd \T{node:get_cookie/0}\index{get_cookie/0 BIF@\T{get_cookie/0} BIF}
(\S\ref{section:node:getcookie0}).\fi
\T{magic_cookie[\Z{N}]} can be set by a process on node \TZ{N} by calling
the BIF
\ifOld \T{erlang:set_cookie/2}\index{set_cookie/2 BIF@\T{set_cookie/2} BIF}
(\S\ref{section:setcookie2}) \fi
\ifStd \T{node:set_cookie/2}\index{set_cookie/2 BIF@\T{set_cookie/2} BIF}
(\S\ref{section:node:setcookie2}) \fi
with \T{name[\Z{N}]} and the new magic cookie as arguments.
\index{node!magic cookie|)}
\index{magic_cookie node property@\T{magic_cookie} node property|)}

\item[\T{magic_cookies[\Z{N}]}]
\index{magic_cookies node property@\T{magic_cookies} node property|(}
\index{node!magic cookie|(}
This is a mapping from atoms to terms.  Its role in process communication
across nodes is described in \S\ref{section:magic-cookie}.
\T{magic_cookies[\Z{N}]} cannot be accessed directly but
can be modified by a process on node \TZ{N} by calling
the BIF
\ifOld \T{erlang:set_cookie/2}\index{set_cookie/2 BIF@\T{set_cookie/2} BIF}
(\S\ref{section:setcookie2}) \fi
\ifStd \T{node:set_cookie/2}\index{set_cookie/2 BIF@\T{set_cookie/2} BIF}
(\S\ref{section:node:setcookie2}) \fi
with the name of another node and
the magic cookie to be used as arguments.
\index{node!magic cookie|)}
\index{magic_cookies node property@\T{magic_cookies} node property|)}

\item[\T{module_table[\Z{N}]}]
\index{module_table node property@\T{module_table} node property|(}
\index{node!module table|(}
This is a table where the keys are module names, i.e., atoms.  The table contains
a row with key \TZ{Mod} if a module named \TZ{Mod} has ever been loaded on the node.
The value \TZ{R} of each row contains two fields:
\T{current_version[\Z{R}]} and \T{old_version[\Z{R}]}.
\begin{itemize}
\item \T{current_version[\Z{R}]} is either a binary representing the compiled code
for the current version of the module, or \T{none}.
\item \T{old_version[\Z{R}]} is either a binary representing the compiled code
for the current version of the module, or \T{none}.
\end{itemize}
The table is initially empty.  For a new row both fields are \T{none}.
It cannot be accessed or modified directly but is accessed or updated
by most BIFs for module
dynamics (\S\ref{chapter:module-dynamics},
\ifOld\S\ref{section:module-bifs}\fi
\ifStd\S\ref{section:module-module}\fi).
\index{node!module table|)}
\index{module_table node property@\T{module_table} node property|)}

\item[\T{monitored_nodes[\Z{N}]}]
\index{monitored_nodes node property@\T{monitored_nodes} node property|(}
\index{node!monitored|(}
This is a table where each row has a node name as key and a table as
value.  Each such table has the PID of a process residing on \TZ{N} as
key and a nonnegative integer as value.  If \T{monitored_nodes[\Z{N}]}
contains a row with key $\TZ{N}'$ and value $t$ and $t$ contains a row
with key \TZ{P} and value \TZ{I}, then each time node $\TZ{N}'$ ceases
to be a friend of node \TZ{N}, $\Er[\TZ{I}]$ messages
\T{\{node_down,$\TZ{N}'$\}} will be sent to process \TZ{P}.
The value of \T{monitored_nodes[\Z{N}]} cannot be accessed directly.  It can be
modified using the BIF
\T{monitor_node/2}\index{monitor_node/2@\T{monitor_node/2}} (\S\ref{section:monitornode2}).
It is initially empty.
\index{node!monitored|)}
\index{monitored_nodes node property@\T{monitored_nodes} node property|)}

\item[\T{ports[\Z{N}]}]
\index{ports node property@\T{ports} node property|(}
This is a set of the ports that are open on node \TZ{N}.
It can be accessed by processes on \TZ{N} through the BIF
\ifOld \T{ports/0}\index{ports/0 BIF@\T{ports/0} BIF}
(\S\ref{section:ports0}).\fi
\ifStd \T{node:ports/0}\index{ports/0 BIF@\T{ports/0} BIF}
(\S\ref{section:node:ports0}).\fi
The set is updated implicitly when a
new port is opened on \TZ{N} (\S\ref{section:opening-ports})
and when a port on \TZ{N} is closed (\S\ref{section:closing-ports}).
\index{ports node property@\T{ports} node property|)}

\item[\T{processes[\Z{N}]}]
\index{processes node property@\T{processes} node property|(}
This is a set of the PIDs of the processes that run on node \TZ{N}.
It consists of the PIDs of all processes that have been spawned on
\TZ{N} and that have not yet completed.
It can be accessed by processes on \TZ{N} through the BIF
\ifOld \T{processes/0}\index{processes/0 BIF@\T{processes/0} BIF}
(\S\ref{section:processes0}).\fi
\ifStd \T{node:processes/0}\index{processes/0 BIF@\T{processes/0} BIF}
(\S\ref{section:node:processes0}).\fi
The set is updated implicitly when a
new process is spawned on \TZ{N} (\S\ref{section:spawning-processes})
and when a process on \TZ{N} completes (\S\ref{section:process-completion}).
\index{processes node property@\T{processes} node property|)}

\item[\T{reductions[\Z{N}]}]
\index{reductions@\T{reductions}!node property|(}
This is the state of the reduction gauge.  It
\ifStd should be \fi \ifOld is \fi
updated automatically to reflect the
number of function calls that have been made on node \TZ{N}.
\ifStd
Implementations may
differ in how the number of function calls is measured.  However, the count must
be strictly increasing and it should increase at least
each time a call of remote function application begins.
\fi

\index{current_reductions node operation@\T{current_reductions} node operation|(}
\T{reductions[\Z{N}]} cannot be accessed directly but the atomic operation
\T{current_reductions[\Z{N}]} uses and updates the value of \T{reductions[\Z{N}]}
to return a 2-tuple \T{\{\Z{Total},\Z{SinceLast}\}} of integers where both
measure the number of function calls on node \TZ{N}.
The first is the number of reductions since node \TZ{N} was started while the second is
since the previous invocation of \T{current_reductions[\Z{N}]}.

The first time \T{current_reductions[\Z{N}]} is invoked,
\TZ{Total} and \TZ{SinceLast} will be the same.
This operation should only be invoked as part of an explicit
application \T{statistics(reductions)} (\S\ref{section:statistics1}) in a user program.
\index{current_reductions node operation@\T{current_reductions} node operation|)}
\index{reductions@\T{reductions}!node property|)}

\item[\T{ref_state[\Z{N}]}]
\index{ref_state node property@\T{ref_state} node property|(}
\index{next_ref node operation@\T{next_ref} node operation|(}
This is the state of the ref generator.  It cannot be accessed directly
but the atomic operation \T{next_ref[\Z{N}]} uses the value of \T{ref_state[\Z{N}]}
to provide the ID for a new ref and at the same time increments the value of
\T{ref_state[\Z{N}]}
so that the next invocation of \T{next_ref[\Z{N}]} will produce a unique ref.
\index{ref_state node property@\T{ref_state} node property|)}
\index{next_ref node operation@\T{next_ref} node operation|)}

\item[\T{registry[\Z{N}]}]
\index{registry node property@\T{registry} node property|(}
\index{process!registry|(}
This is a mapping from atoms to PIDs.
\ifStd The implementation must ensure \fi
\ifOld It is ensured \fi
that at any time,
\begin{itemize}
\item the mapping is invertible (i.e., that at most one name is registered
for each PID),
\item that all PIDs refer to processes residing on node \TZ{N}, and
\item that all PIDs refer to live processes.
\end{itemize}
This implies that when a process completes, any name registered for it
must be removed from the registry.

\T{registry[\Z{N}]} can be accessed
by processes running on node \TZ{N} through the BIFs \T{register/2}, \T{unregister/1},
\T{registered/0} and \T{whereis/1}
(\ifOld\S\ref{section:process-registry-bifs}\fi
\ifStd\S\ref{section:process-module}\fi).
It can be accessed indirectly by processes running
on any node when sending messages (\S\ref{section:send-expr}).
\index{process!registry|)}
\index{registry node property@\T{registry} node property|)}

\item[\T{runtime[\Z{N}]}]
\index{runtime node property@\T{runtime} node property|(}
\index{current_runtime node operation@\T{current_runtime} node operation|(}
This is the state of the run time gauge.  It is updated automatically to reflect the
amount of time spent running processes on node \TZ{N}.
It cannot be accessed directly
but the atomic operation \T{current_runtime[\Z{N}]} uses and updates
the value of \T{runtime[\Z{N}]}
to return a 2-tuple \T{\{\Z{Total},\Z{SinceLast}\}} of integers where both
measure the total time in milliseconds spent on running processes on node \TZ{N}.
The first element is the time since node \TZ{N} was started while the second is
since the previous invocation of \T{current_runtime[\Z{N}]}.

The first time \T{current_runtime[\Z{N}]} is invoked,
\TZ{Total} and \TZ{SinceLast} will be the same.
This operation should only be invoked as part of an explicit
BIF application \T{statistics(runtime)}\index{statistics/1 BIF@\T{statistics/1} BIF}
(\S\ref{section:statistics1}) in a user program.
\index{current_runtime node operation@\T{current_runtime} node operation|)}
\index{runtime node property@\T{runtime} node property|)}

\item[\T{wall_clock[\Z{N}]}]
\index{wall_clock node property@\T{wall_clock} node property|(}
\index{current_wall_clock node operation@\T{current_wall_clock} node operation|(}
This is the state of the real time gauge.  It is updated automatically
to reflect the passing of real time in the world where node \TZ{N}
resides.  It cannot be accessed directly but the value of
\T{wall_clock[\Z{N}]} is used and updated by the atomic operation
\T{current_wall_clock[\Z{N}]} to return a 2-tuple
\T{\{\Z{Total},\Z{SinceLast}\}} of integers where both measure the
time in milliseconds that has passed in the world.  The first element
is the time since node \TZ{N} was started while the second is since
the previous invocation of \T{current_wall_clock[\Z{N}]}.

The first time \T{current_wall_clock[\Z{N}]} is invoked, \TZ{Total}
and \TZ{SinceLast} will be the same.  This operation should only be
invoked as part of an explicit BIF application
\T{statistics(wall_clock)}\index{statistics/1 BIF@\T{statistics/1}
BIF} (\S\ref{section:statistics1}) in a user program.
\index{wall_clock node property@\T{wall_clock} node property|)}
\index{current_wall_clock node operation@\T{current_wall_clock} node operation|)}
\end{Lentry}
\index{node!state|)}
