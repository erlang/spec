%
% %CopyrightBegin%
%
% Copyright Ericsson AB 2017. All Rights Reserved.
%
% Licensed under the Apache License, Version 2.0 (the "License");
% you may not use this file except in compliance with the License.
% You may obtain a copy of the License at
%
%     http://www.apache.org/licenses/LICENSE-2.0
%
% Unless required by applicable law or agreed to in writing, software
% distributed under the License is distributed on an "AS IS" BASIS,
% WITHOUT WARRANTIES OR CONDITIONS OF ANY KIND, either express or implied.
% See the License for the specific language governing permissions and
% limitations under the License.
%
% %CopyrightEnd%
%

\chapter{Lexical structure}

\label{chapter:lexical}

This chapter describes the lexical structure of \Erlang\ programs and
how a sequence of \ifStd Unicode \else ASCII \fi characters is
translated to a sequence of tokens and full stops.

\ifStd
\section{Unicode}

\label{section:unicode}
\index{Unicode|(}

\StdErlang\ programs are written using the 16-bit \emph{Unicode} character set,
version 2.0, which contains the 7-bit ASCII character set as its first
128 characters and the 8-bit Latin-1 character set as its first 256
characters.

All \Erlang\ keywords, separators and operators use only the ASCII
subset.  It is therefore possible to write entire \Erlang\ programs
using only that subset, provided that identifiers, (string and
character) literals and comments use only ASCII characters.
\index{Unicode|)}

\index{escape!Unicode|(}
There is also an escape mechanism which allows non-ASCII characters to be
written using only ASCII characters.  This allows editing of \Erlang\
programs also using editors that support only ASCII or Latin-1.
Software tools for \Erlang\ (e.g., interactive debuggers)
should also use this escape mechanism when displaying
\Erlang\ programs on devices that do not support Unicode.
\index{escape!Unicode|)}
\fi

\section{Lexical translation}

\label{section:lexical-translation}
\index{translation!lexical|(}

A sequence of \ifStd Unicode \else ASCII \fi characters is translated
into a sequence of \Erlang\ tokens by applying the following four
steps, in order:

\begin{enumerate}
\ifStd
\item Each Unicode escape in the sequence of characters
is replaced by the corresponding Unicode character (\S\ref{section:unicode-escapes}).
\fi
\item \label{item:unicode-line-sep} The sequence of
\ifStd Unicode characters from step~1 \else characters \fi
is translated to a sequence
of input characters and line terminators (\S\ref{section:line-terminators}).
If the sequence does not end with a line terminator, one is added at the end.
\item The sequence of input characters and line terminators is translated
to a sequence of \Erlang\ input elements (\S\ref{section:input-elements}).
\item The sequence of \Erlang\ input elements is translated to a sequence of tokens
and full stops by discarding white space (\S\ref{section:white-space}) and comments
(\S\ref{section:comments}) and detecting full
stops (\S\ref{section:full-stop}). The resulting
sequence of tokens and full stops is the result of the lexical processing.
\end{enumerate}
\index{translation!lexical|)}

\ifStd
\section{Unicode escapes}

\label{section:unicode-escapes}
\index{escape!Unicode|(}

(The processing of Unicode escapes is intentionally made identical with that in Java\index{Java}
\cite[pp.~12--13]{javaspec}.)

\begin{rules}
\ifStd
\grrule{UnicodeInputCharacter}
       {\NT{UnicodeEscape} \OR
        \NT{RawInputCharacter}}

\grrule{UnicodeEscape}
       {\TXT{\char`\\} \NT{UnicodeMarker} \NT{HexDigit} \NT{HexDigit} \NT{HexDigit} \NT{HexDigit}}

\grrule{UnicodeMarker}
       {\TXT{u} \OR
        \NT{UnicodeMarker} \TXT{u}}

\grrule{RawInputCharacter}
       {any Unicode character}

\grrule{HexDigit}
       {\NT{Digit}[16]}
\fi
\end{rules}
\NT{Digit}[16] is as in \S\ref{section:integer-literals}.

In addition to what is specified by the grammar above, a backslash\index{\ character@\T{\char`\\} character}
(\TXT{\char`\\}) character begins a Unicode escape only if
\begin{itemize}
\item there is an even number of backslash characters separating it
from preceding non-backslash characters or the beginning of the input
stream, and
\item it is followed by \TXT{u}.
\end{itemize}
Unless both these conditions are met, the backslash should be treated
as a \NT{RawInputCharacter}.

If an eligible backslash is followed by one or more \TXT{u} and the
last \TXT{u} is not followed by four hexadecimal digits, then a
compile-time error should occur.

An eligible sequence consisting of backslash, one or more \TXT{u}, and
four hexadecimal digits with values $d_3$, $d_2$, $d_1$ and $d_0$ is
replaced in the input stream by a Unicode character with character
code $4096d_3+256d_2+16d_1+d_0$.  The resulting character is never
part of further Unicode escape processing.

Note that there exists an invertible transformation from a sequence of
\NT{UnicodeInputCharacter} to a sequence of \NT{UnicodeInputCharacter}
that only contains ASCII (or Latin-1) characters:
\begin{itemize}
\item Replace any \NT{UnicodeEscape} with a \NT{UnicodeEscape} having
one more \TXT{u}.
\item Replace any non-ASCII (or non-Latin-1) \NT{RawInputCharacter}
with a corresponding \NT{UnicodeEscape}.
\item Leave any ASCII (or Latin-1) \NT{RawInputCharacter} as is.
\end{itemize}
The resulting stream of characters consists only of ASCII (or Latin-1)
characters and can thus be processed by tools that handle only any
ASCII (or Latin-1) characters, e.g., many standard UNIX utilities.
The inverse transformation is obvious.

An \Erlang\ system should use Unicode escapes to display Unicode
characters that cannot be displayed in a current font.
\index{escape!Unicode|)}
\fi

\section{Character classes}
\index{section:character-classes}

\begin{rules}
\grrule{ErlangUppercase}
       {the capital \ifStd ISO 8859-1 \else ASCII \fi letters
\T{A}--\T{Z} (\T{\char`\\\ifStd u0041\else 101\fi} to \T{\char`\\\ifStd u005a\else 132\fi})\ifStd,
\T{�}--\T{�} (\T{\char`\\u00c0} to \T{\char`\\u00d6}) and % 192 - 214
\T{\char'037}--\T{[THORN]} % Should be `�'
(\T{\char`\\u00d8} to \T{\char`\\u00de})\fi}

\grrule{ErlangLowercase}
       {the small
\ifStd ISO 8859-1 \fi\ifOld ASCII \fi letters
\T{a}--\T{z} (\T{\char`\\\ifStd u0061\fi\ifOld 141\fi} to \T{\char`\\\ifStd u007a\fi\ifOld 172\fi})\ifStd,
\T{\char'031}--\T{�} (\T{\char`\\u00df} to \T{\char`\\u00f6}) and % 246
\T{\char'034}--\T{�} (\T{\char`\\u00f8} to \T{\char`\\u00ff}) \fi} % 255

\grrule{ErlangLetter}
       {\NT{ErlangLowercase} \OR \NT{ErlangUppercase}}

\grrule{ErlangDigit}
       {the
ASCII decimal digits \T{0}--\T{9} (\T{\char`\\\ifStd u0030\fi\ifOld 060\fi} to
\T{\char`\\\ifStd u0039\fi\ifOld 071\fi})}
\end{rules}

\section{Line terminators}

\label{section:line-terminators}
\index{line terminator|(}

Line terminators need to be recognized uniformly across platforms
so \Erlang\ compilers and tools can report line numbers and determine
the end of comments coherently.
\begin{rules}
\ifStd
\grrule{LineTerminator}
       {the ASCII LF character (``linefeed''\index{Linefeed character} or ``newline''\index{Newline character})
\OR
        the ASCII CR character (``return''\index{Return character}) \OR
	the ASCII CR character followed by the ASCII LF character}
\else
\grrule{LineTerminator}
       {the LF character (``linefeed''\index{Linefeed character} or ``newline''\index{Newline character})}
\fi

\ifStd
\grrule{InputCharacter}
       {\NT{UnicodeInputCharacter} but not ASCII CR or LF}
\else
\grrule{InputCharacter}
       {\NT{AsciiInputCharacter} but not LF}
\fi
\end{rules}
\ifStd
A CR character followed by a LF character is thus considered one
line terminator, not two.
\fi
\index{line terminator|)}

\section{Input elements}

\label{section:input-elements}
\index{input element|(}

The sequence of input characters and line terminators is translated to
a sequence of input elements, i.e., white space
(\S\ref{section:white-space}), comments (\S\ref{section:comments}) and
tokens.

\index{grammar!token|(}
There are six kinds of tokens:
separators (\S\ref{section:separators}),
keywords (\S\ref{section:keywords}),
operators (\S\ref{section:operators}),
integer literals (\S\ref{section:integer-literals}),
float literals (\S\ref{section:float-literals}),
character literals (\S\ref{section:char-literals}),
string literals (\S\ref{section:string-literals}),
atom literals (\S\ref{section:atom-literals}),
variables (\S\ref{section:variables}),
universal patterns (\S\ref{section:universal-pattern}).
\index{grammar!token|)}

\iffalse
Note that the input characters and line terminators may optionally be
followed by a control-Z character (as provided by some operating
systems).
\fi

Note \iffalse also\fi that white space and comments always will
separate tokens.

\begin{rules}
\grrule{Input}
       {\OPT{InputElements}\iffalse \OPT{Sub}\fi}

\grrule{InputElements}
       {\NT{InputElement} \OR
        \NT{InputElements} \NT{InputElement}}

\grrule{InputElement}
       {\NT{WhiteSpace} \OR
        \NT{Comment} \OR
        \NT{Token}}

\grrule{Token}
       {\NT{Separator} \OR
        \NT{Keyword} \OR
        \NT{Operator} \OR
        \NT{IntegerLiteral} \OR
        \NT{FloatLiteral} \OR
        \NT{CharLiteral} \OR
        \NT{StringLiteral} \OR
        \NT{AtomLiteral}
        \NT{Variable} \OR
        \NT{UniversalPattern}}
\iffalse
\grrule{Sub}
       {the ASCII SUB character, also known as ``control-Z''}
\fi
\end{rules}
\index{input element|)}

\section{White space}

\label{section:white-space}
\index{white space|(}

White space is defined as the characters with ASCII codes less than or equal to that
of ASCII space, i.e., the control characters and space.  The line terminators
are composed of control characters but have been identified in a preceding step
and are therefore included explicitly below.

\begin{rules}
\grrule{WhiteSpace}
       {\NT{LineTerminator} \OR
        \NT{ControlCharacter} \OR
        the ASCII SP character, also known as ``space''\index{space character}}

\grrule{ControlCharacter}
       {any ASCII control character\index{control character} (\T{\char`\\\ifStd u0000\else 000\fi} to \T{\char`\\\ifStd u001f\else 037\fi})}
\end{rules}
\index{white space|)}

\section{Comments}

\label{section:comments}
\index{comment|(}

A comment begins with an ASCII \T{\%} character\index character} and extends up to and including
the next line terminator.

\begin{rules}
\grrule{Comment}
       {\TXT{\%} \OPT{InputCharacters} \NT{LineTerminator}}

\grrule{InputCharacters}
       {\NT{InputCharacter} \OR
        \NT{InputCharacters} \NT{InputCharacter}}
\end{rules}
Examples of comments:
\begin{verbatim}
        %A space after the percent is not obligatory but...
% The comment can begin a line.
     %% There can be additional %s in the comment.
\end{verbatim}
\index{comment|)}

\section{Separators}

\label{section:separators}
\index{separator|(}

The following 15 tokens are \emph{separators} in \Erlang\ifStd,
formed from ASCII characters\fi:
\begin{rules}
{\obeyspaces%
\grruleoneof{Separator}{\TXT{(    )    \char`\{    \char`\}    [    ]    .    :}\\
                        \TXT{\char`\|~~~~\char`\|\char`\|~~~;~~~~,~~~~?~~~~->~~~\char`\#}}}
\end{rules}
\index{separator|)}

\section{Keywords}

\label{section:keywords}
\index{keyword|(}

The following \ifStd 15 \else 13 \fi tokens are the \emph{keywords} of
\Erlang\ifStd, formed from ASCII characters\fi:
\begin{rules}
\ifStd
{\obeyspaces%
\grruleoneof{Keyword}{\TXT{after     catch     if        some_true}\\
                      \TXT{all_true  cond      let       try}\\
                      \TXT{begin     end       of        when}\\
                      \TXT{case      fun       receive   }}}
\else
{\obeyspaces%
\grruleoneof{Keyword}{\TXT{after     cond      let       when}\\
                      \TXT{begin     end       of}\\
                      \TXT{case      fun       query}\\
                      \TXT{catch     if        receive}}}
\fi
\end{rules}
Thus they cannot be used as atom literals (\S\ref{section:atom-literals}).

\ifStd
The keyword \T{let} is not defined but is
reserved for a possible future
extension of the language.
The keyword \T{query} is also not defined in this document but is reserved for
the Mnesia subsystem of OTP \cite{otp-mnesia}.
\else
The keywords \T{all_true}, \T{cond}, \T{let} and \T{some_true} are
not defined in \OldErlang\ but are reserved for possible future
extensions of the language.
\fi
\index{keyword|)}

\section{Operators}

\label{section:operators}
\index{operator|(}

The following \ifStd 31 \else 29 \fi tokens are the \emph{operators}
of \Erlang\ifStd, formed from ASCII characters\fi:
\begin{rules}
{\obeyspaces%
\grruleoneof{Operator}{\TXT{+    -    *    /    div  rem  \ifStd//   mod\fi}\\
                       \TXT{or   xor  bor  bxor bsl  bsr  and  band}\\
                       \TXT{==   /=   =:=  =/=  <    =<   >    >=}\\
                       \TXT{not  bnot ++   --   =    !~~~~<-}}}
\end{rules}
Thus they cannot be used as atom literals (\S\ref{section:atom-literals}).
\index{operator|)}

\section{Escape sequences}

\label{section:escapes}
\index{escape!character|(}

These are the escape sequences for character literals, string literals
and quoted atom literals.  All escape sequences begin with a
backslash (\T{\char`\\}).
\iffalse
\index{ \b@\TXT{\char`\\b}|(}
\index{ \d@\TXT{\char`\\d}|(}
\index{ \e@\TXT{\char`\\e}|(}
\index{ \f@\TXT{\char`\\f}|(}
\index{ \n@\TXT{\char`\\n}|(}
\index{ \r@\TXT{\char`\\r}|(}
\index{ \s@\TXT{\char`\\s}|(}
\index{ \t@\TXT{\char`\\t}|(}
\index{ \v@\TXT{\char`\\v}|(}
\index{ \\@\TXT{\char`\\\char`\\}|(}
\index{ \'@\TXT{\char`\\'}|(}
\index{ \"@\TXT{\char`\\""}|(}
\index{ \1@\TXT{\char`\\} $\langle\I{digits}\rangle$|(}
\index{ \^@\TXT{\char`\\\char`\^}|(}
\fi
\begin{rules}
\grrule{EscapeSequence}
       {\TXT{\char`\\} \TXT{b} & \TXT{\% \char`\\\ifStd u0008\else 008\fi: \R{backspace BS}} \OR
        \TXT{\char`\\} \TXT{d} & \TXT{\% \char`\\\ifStd u007f\else 177\fi: \R{delete DEL}} \OR
        \TXT{\char`\\} \TXT{e} & \TXT{\% \char`\\\ifStd u001b\else 033\fi: \R{escape ESC}} \OR
        \TXT{\char`\\} \TXT{f} & \TXT{\% \char`\\\ifStd u000c\else 014\fi: \R{form feed FF}} \OR
        \TXT{\char`\\} \TXT{n} & \TXT{\% \char`\\\ifStd u000a\else 012\fi: \R{linefeed LF}} \OR
        \TXT{\char`\\} \TXT{r} & \TXT{\% \char`\\\ifStd u000d\else 015\fi: \R{carriage return CR}} \OR
        \TXT{\char`\\} \TXT{s} & \TXT{\% \char`\\\ifStd u0020\else 040\fi: \R{space SPC}} \OR
        \TXT{\char`\\} \TXT{t} & \TXT{\% \char`\\\ifStd u0009\else 011\fi: \R{horizontal tab HT}} \OR
        \TXT{\char`\\} \TXT{v} & \TXT{\% \char`\\\ifStd u000b\else 013\fi: \R{vertical tab VT}} \OR
        \TXT{\char`\\} \TXT{\char`\\} & \TXT{\% \char`\\\ifStd u005c\else 008\fi: \R{backslash} \char`\\} \OR
        \NT{ControlEscape} & \TXT{\% \char`\\\ifStd u0000 \else 000 \fi to \char`\\\ifStd u001f\else 037\fi: \R{64 less than the char}} \OR
        \TXT{\char`\\} \TXT{'} & \TXT{\% \char`\\\ifStd u0027\else 047\fi: \R{single quote} '} \OR
        \TXT{\char`\\} \TXT{"} & \TXT{\% \char`\\\ifStd u0022\else 042\fi: \R{double quote} "} \OR
        \NT{OctalEscape} & \TXT{\% \char`\\\ifStd u0000 \else 000 \fi to \char`\\\ifStd u00ff\else 777\fi: \R{from octal value}}}

\grrule{ControlEscape}
       {\TXT{\char`\\} \TXT{\char`\^} \NT{ControlName}}

\grrule{ControlName}
       {any \ifStd Unicode \fi character between \T{\char`\\\ifStd u0040\else 100\fi} and \T{\char`\\\ifStd u005f\else 137\fi}}

\grrule{OctalEscape}
       {\ifStd 
	\TXT{\char`\\} \NT{ZeroToThree} \NT{OctalDigit} \NT{OctalDigit}
	\else
	\TXT{\char`\\} \NT{OctalDigit} \OR
        \TXT{\char`\\} \NT{OctalDigit} \NT{OctalDigit} \OR
        \TXT{\char`\\} \NT{OctalDigit} \NT{OctalDigit} \NT{OctalDigit}
	\fi}

\grruleoneof{OctalDigit}{\TXT{0 1 2 3 4 5 6 7}}

\ifStd
\grruleoneof{ZeroToThree}{\TXT{0 1 2 3}}
\fi
\end{rules}

\iffalse
\index{ \b@\TXT{\char`\\b}|)}
\index{ \d@\TXT{\char`\\d}|)}
\index{ \e@\TXT{\char`\\e}|)}
\index{ \f@\TXT{\char`\\f}|)}
\index{ \n@\TXT{\char`\\n}|)}
\index{ \r@\TXT{\char`\\r}|)}
\index{ \s@\TXT{\char`\\s}|)}
\index{ \t@\TXT{\char`\\t}|)}
\index{ \v@\TXT{\char`\\v}|)}
\index{ \\@\TXT{\char`\\\char`\\}|)}
\index{ \'@\TXT{\char`\\'}|)}
\index{ \"@\TXT{\char`\\""}|)}
\index{ \1@\TXT{\char`\\} $\langle\I{digits}\rangle$|)}
\index{ \^@\TXT{\char`\\\char`\^}|)}
\fi
In the case of a control escape, it denotes the character which has a
\ifStd Unicode value \else code \fi
that is 64 less than that of the \NT{ControlName}
following \T{\char`\\} and \T{\char`\^}.

In the case of an octal escape, it denotes
\ifStd
the character which has the Unicode value
\else
the integer
\fi
denoted by the octal numeral.
\ifStd
Note that octal escapes can express only characters with Unicode
values \T{\char`\\u0000} to \T{\char`\\u00ff}.
\else
Note that only the octal escapes \T{\char`\\000} to \T{\char`\\177}
denote characters.  The octal escapes \T{\char`\\200} to
\T{\char`\\377} denote noncharacter bytes while
\T{\char`\\400} to \T{\char`\\777} denote larger integers.
\fi
\index{escape!character|)}

\section{Integer literals}

\label{section:integer-literals}
\index{integer!literal|(}

An integer literal consists of an optional sign,
an optional radix\index{radix (of integer literal)} specifier and a sequence of digits in the
specified radix (which is decimal if omitted).
\ifStd
Between
any two digits there may be an underscore\index{_ character@\T{_} character}, which is
ignored. (This is to simplify correct entry of long
numerals by making grouping possible.)
\fi
\begin{rules}
\grrule{IntegerLiteral}
       {\NT{DecimalLiteral} \OR
        \NT{ExplicitRadixLiteral}}

\grrule{DecimalLiteral}
       {$\NT{Digits}[10]$}

\grrule{ExplicitRadixLiteral}
       {\TXT{2} \TXT{\#} $\NT{Digits}[2]$ \OR
        \TXT{3} \TXT{\#} $\NT{Digits}[3]$ \OR
        \TXT{4} \TXT{\#} $\NT{Digits}[4]$ \OR
        \TXT{5} \TXT{\#} $\NT{Digits}[5]$ \OR
        \TXT{6} \TXT{\#} $\NT{Digits}[6]$ \OR
        \TXT{7} \TXT{\#} $\NT{Digits}[7]$ \OR
        \TXT{8} \TXT{\#} $\NT{Digits}[8]$ \OR
        \TXT{9} \TXT{\#} $\NT{Digits}[9]$ \OR
        \TXT{1} \TXT{0} \TXT{\#} $\NT{Digits}[10]$ \OR
        \TXT{1} \TXT{1} \TXT{\#} $\NT{Digits}[11]$ \OR
        \TXT{1} \TXT{2} \TXT{\#} $\NT{Digits}[12]$ \OR
        \TXT{1} \TXT{3} \TXT{\#} $\NT{Digits}[13]$ \OR
        \TXT{1} \TXT{4} \TXT{\#} $\NT{Digits}[14]$ \OR
        \TXT{1} \TXT{5} \TXT{\#} $\NT{Digits}[15]$ \OR
        \TXT{1} \TXT{6} \TXT{\#} $\NT{Digits}[16]$}

\grruleindex{Digits[k]}{$\NT{Digits}[k]$}
       {$\NT{Digit}[k]$ \OR
        $\NT{Digits}[k]$ \ifStd \OPT{DigitSeparator} \fi $\NT{Digit}[k]$}

\ifStd
\grrule{DigitSeparator}
       {\TXT{_}}
\fi

\grruleoneofthefirst{Digit[k]}{$\NT{Digit}[k]$}{$k$}
       {\TXT{0 1 2 3 4 5 6 7 8 9 Aa Bb Cc Dd Ee Ff}}
\end{rules}
An integer literal without an explicit radix should thus be composed of decimal digits
and will be interpreted arithmetically as a decimal numeral as described in
\S\ref{section:numeral-to-integer}.

If an explicit radix is given, it consists of a decimal numeral between \TXT{2} and
\TXT{16} followed by a \TXT{\#} character.  Suppose that that the decimal interpretation
of this numeral is $r$.  The characters following it should then
be drawn from the first $r$ ``extended digits'' \TXT{0}, \ldots,
\TXT{9}, \TXT{A}, \ldots, \TXT{F}, where the letters may alternatively be in lower case.
Its interpretation as a numeral in radix $r$ is given in \S\ref{section:radix-numeral-to-integer}.

A compile-time error occurs if an integer literal is too
large or too small to be representable as an \Erlang\ integer.

Examples of integer literals:
\ifOld
\begin{verbatim}
0   1499   -54   2#0010010   -8#377
10#1499   16#fa66   16#FA66
\end{verbatim}
\fi
\ifStd
\begin{verbatim}
0   1499   -54   2#0010010   -8#377
10#1499   16#fa66   16#FA66   1_234_567
\end{verbatim}
\fi
(Their values are 0, 1499, -54, 18, -255, 1499, 64\,102, 64\,102 and 1\,234\,567, respectively.)
\index{integer!literal|)}

\section{Float literals}

\label{section:float-literals}
\index{float!literal|(}

A float literal has five parts: an optional sign, a whole number part,
a decimal point\index{decimal point (of float literal)}, a fractional
part and an optional exponent part.  The exponent part, if present, is
indicated by \TXT{E}\index{E@\TXT{E} character} or
\TXT{e}\index{e@\TXT{e} character} and is followed by the
exponent\index{exponent (of float literal)} as a decimal numeral with
an optional sign.
\ifStd
As for integer numerals, between any two digits in the whole number
part, the fractional part and the exponent there may be an underscore\index{_ character@\T{_} character},
which is ignored.
\fi
\begin{rules}
\grrule{FloatLiteral}
       {\NT{DecimalLiteral} \TXT{.}\ \NT{DecimalLiteral} \OPT{ExponentPart}}

\grrule{ExponentPart}
       {\NT{ExponentIndicator} \OPT{Sign} \NT{DecimalLiteral}}

\grruleoneof{Sign}{\TXT{+ -}}

\grruleoneof{ExponentIndicator}{\TXT{E e}}
\end{rules}
(The rule for \NT{DecimalLiteral} appears in
\S\ref{section:integer-literals}.)  A compile-time error occurs if the
magnitude of a float literal is too large or too small to be
representable as an \Erlang\ float.

The interpretation of a float literal as a float is given in
\S\ref{section:numeral-to-float}.

Examples of float literals:
\begin{verbatim}
1.0   -5.1e6   5.1e+6   0.346E-20
\end{verbatim}
\index{float!literal|)}

\section{Character literals}

\label{section:char-literals}
\index{character!literal|(}

A character literal is indicated by \TXT{\$} followed either by a single
character \ifStd (which may not be whitespace) \fi or an escape sequence.
\begin{rules}
\grrule{CharLiteral}
       {\TXT{\$} \NT{CharLiteralChar}}

\grrule{CharLiteralChar}
       {\NT{InputCharacter} \ifStd but not \TXT{\char`\\} or \NT{WhiteSpace} \fi \OR
	\NT{EscapeSequence}}
\end{rules}
The escape sequences are described in \S\ref{section:escapes}.

Note that \TXT{\$} \ifStd may \else should \fi not be followed by
white space\index{white space!not after \TXT{\$}}
(\S\ref{section:white-space})\footnote{The reasons that \TXT{\$}
\ifStd may \else should \fi
not be followed by whitespace are (i) it is impossible to determine
safely from the printed representation of a program which character
\TXT{\$} followed by whitespace and/or a line break actually denotes,
and (ii) whitespace (especially at the end of a line) often gets
transformed by text processing or text transmission tools.} which
\ifStd must \else should \fi instead be expressed through an escape
sequence.

\ifStd
\ifDiff\footnote{Compatibility note: Whitespace may appear
after \TXT{\$} in \OldErlang.  Its appearance should generate warnings
in \NewErlang\ and cause a compile-time error in some future
version.}\fi
\fi

Examples of character literals:
\begin{verbatim}
$x   $R   $$   $\n   $\s   $\\   $\^T   $\125
\end{verbatim}
The values of these literals are the characters `x', `R', dollar, newline, space, backslash,
control-T and `U' (which has ASCII code eightyfive, which is denoted by the octal numeral
\T{125}).
\index{character!literal|)}

\section{String literals}

\label{section:string-literals}
\index{string!literal|(}

A string literal is enclosed in \T{"} characters\index{"" character@\T{""} character}.
\begin{rules}
\grrule{StringLiteral}
       {\TXT{"} \OPT{StringCharacters} \TXT{"}}

\grrule{StringCharacters}
       {\NT{StringCharacter} \OR
        \NT{StringCharacters} \NT{StringCharacter}}

\grrule{StringCharacter}
       {\NT{InputCharacter} but not \NT{ControlCharacter} or \TXT{\char`\\} or \TXT{"} \OR
        \NT{EscapeSequence}}
\end{rules}
The escape sequences are described in \S\ref{section:escapes}.

\ifStd
Note that line terminators\index{line terminator!not ``naked''} and
control characters\index{control character!not ``naked''} cannot
appear ``naked'' in a string literal (and thus they also cannot appear
as Unicode escapes such as
\T{\char`\\u000c}) but must be expressed through escape
sequences.  A compiler need not generate an error in this case but
should at least produce a warning.\ifDiff\footnote{Compatibility note:
Line terminators and control characters may appear in strings and
quoted atoms in \OldErlang.  Their appearance should generate warnings
in \NewErlang\ and cause a compile-time error in some future
version.}\fi
\fi

Examples of string literals:
\begin{verbatim}
"Fred"   ""   "\n"   "\e42n"   "Ludwig Van Beethoven"
\end{verbatim}
The second literal denotes an empty string.
\index{string!literal|)}

\section{Atom literals}

\label{section:atom-literals}
\index{atom!literal|(}

An \emph{atom literal}, which we will also refer to simply as an
\emph{atom}, is on one of two forms:
\begin{itemize}
\item An \emph{unquoted} atom\index{atom!unquoted} is a nonempty sequence of \Erlang\ letters, \Erlang\
digits and the ASCII character `\TXT{@}', where the first character must be a lowercase \Erlang\ letter.
Such an atom cannot consist of the same sequence of characters
as a keyword (\S\ref{section:keywords}) or an operator (\S\ref{section:operators}).
\item A \emph{quoted} atom\index{atom!quoted} is a sequence
of input characters and escape sequences enclosed in single quotes (\T{'})\index{' character@\T{'} character}.
As for string literals,  line terminators\index{line terminator!not ``naked''} and
control characters\index{control character!not ``naked''} cannot appear
``naked'' in a quoted atom.
\end{itemize}

\begin{rules}
\grrule{AtomLiteral}
       {\NT{AtomLiteralChars} but not a \NT{Keyword} or \NT{Operator} \OR
        \TXT{'} \OPT{QuotedCharacters} \TXT{'}}

\grrule{AtomLiteralChars}
       {\NT{ErlangLowercase} \OPT{NameChars}}

\grrule{NameChars}
       {\NT{NameChar} \OR
	\NT{NameChars} \NT{NameChar}}

\grrule{NameChar}
       {\NT{ErlangLetter} \OR
        \NT{ErlangDigit} \OR \TXT{@} \OR \TXT{_}}

\grrule{QuotedCharacters}
       {\NT{QuotedCharacter} \OR
        \NT{QuotedCharacters} \NT{QuotedCharacter}}

\grrule{QuotedCharacter}
       {\NT{InputCharacter} but not \NT{ControlCharacter} or \TXT{\char`\\} or \TXT{'} \OR
        \NT{EscapeSequence}}
\end{rules}
The escape sequences are described in \S\ref{section:escapes}.

\iffalse
An \emph{\Erlang\ letter} is a character for which the function
\T{char:is_alpha/1} returns \T{true}.  An \emph{\Erlang\ letter-or-digit}
is a character for which the function \T{char:is_alpha_num/1} returns
\T{true}.
\fi

Examples of atoms:
\ifOld
\begin{verbatim}
friday   tv@lf@rs@lj@re   one_2_three   '!%r@(\'.\ $\\[[#'
\end{verbatim}
\fi
\ifStd
\begin{verbatim}
friday   tv�lf�rs�lj@re   one_2_three   '!�r@(\'.\ �\\[[#'
\end{verbatim} % 182
Note that the second example contains non-ASCII characters (but which belong
to ISO 8859-1).
\fi
The fourth example shows several escape sequences and nonletters.

We say that the \emph{printname}\index{atom!printname} of an atom is
\begin{itemize}
\item the atom literal as such, in the case of an unquoted atom;
\item the sequence of \ifStd Unicode \fi \ifOld ASCII \fi characters resulting from decoding
of escape sequences and removal of the surrounding quotes,
in the case of a quoted atom.
\end{itemize}

Two atoms are the same if and only if they have the same printname.
For example, the atoms \T{foo} and \T{'foo'} are the same and so are the atoms \T{'foo
bar'}\ifStd, \T{'foo\char`\\u0020bar'} \fi and \T{'foo\char`\\040bar'}.
\ifStd
(Note that the first and second atom literals are identical already after
Unicode escape processing, cf.~\S\ref{section:lexical-translation}.)
\fi
\index{atom!literal|)}

\section{Variables}

\label{section:variables}
\index{variable|(}

A \emph{variable} is a nonempty sequence of \Erlang\ letters, \Erlang\
digits and the character `\TXT{@}', where the first character must be an uppercase \Erlang\ letter.
Since no keyword, operator or literal begins with an uppercase \Erlang\ letter,
there can be no ambiguity between them.

\begin{rules}
\grrule{Variable}
       {\TXT{_} \NT{NameChars} \OR
	\NT{ErlangUppercase} \OPT{NameChars}}
\end{rules}
Note that a \emph{single} underscore is not a \NT{Variable} but a
\NT{UniversalPattern} (cf.~\S\ref{section:universal-pattern}).

Examples of variables:
\begin{verbatim}
MostSignificantDigit   X   Best_guess   _Rest   LOUD
\end{verbatim}

It is recommended that compilers warn about a variable that has only a
binding occurrence (and thus no applied occurrences) unless the
variable begins with an underscore.
\index{variable|)}

\section{Universal pattern}

\label{section:universal-pattern}
\index{universal pattern|(}

A \emph{universal pattern} consists of a single underscore\index{_ character@\T{_} character}.

\begin{rules}
\grrule{UniversalPattern}
       {\TXT{_}}
\end{rules}
It is thus syntactically similar to a variable but may only appear in patterns
(cf.~\S\ref{section:pattern-matching}) where occurrences of
the universal pattern are like binding occurrences of distinct variables that have
no other occurrences.  Matching against the universal pattern thus always succeeds.
\index{universal pattern|)}

\section{Separated token sequences}

\label{section:full-stop}
\label{section:tokenseq}

In the final step of lexical processing, white space and comments are discarded.
At the same time, each occurrence of a single period\index{ .@\TXT{.} character} token
(i.e., `\TXT{.}', which is a separator) that is followed by white space or a comment
is replaced by a \NT{FullStop}\index{full stop}.  The final result is a sequence of tokens and
full stops where each full stop should be thought of as terminating the preceding
sequence of tokens.

\begin{rules}
\grrule{TerminatedTokens}
       {\OPT{TokenSequences}}

\grrule{TokenSequences}
       {\NT{TokenSequence} \OR
        \NT{TokenSequences} \NT{TokenSequence}}

\grrule{TokenSequence}
       {\NT{Tokens} \NT{FullStop}}

\grrule{Tokens}
       {\NT{Token} \OR
        \NT{Tokens} \NT{Token}}

\grrule{FullStop}
       {\TXT{.} \NT{WhiteSpace} \OR
        \TXT{.} \NT{Comment}}
\end{rules}
