%
% %CopyrightBegin%
%
% Copyright Ericsson AB 2017. All Rights Reserved.
%
% Licensed under the Apache License, Version 2.0 (the "License");
% you may not use this file except in compliance with the License.
% You may obtain a copy of the License at
%
%     http://www.apache.org/licenses/LICENSE-2.0
%
% Unless required by applicable law or agreed to in writing, software
% distributed under the License is distributed on an "AS IS" BASIS,
% WITHOUT WARRANTIES OR CONDITIONS OF ANY KIND, either express or implied.
% See the License for the specific language governing permissions and
% limitations under the License.
%
% %CopyrightEnd%
%

\chapter{Implementation constants}

This appendix summarizes the constants that characterize a \StdErlang\ implementation
and requirements on the values for these constants.  An implementation should
document the values, most of which are also available to programs.

\section*{Atoms}

The constant
\I{maxatomlength} (\S\ref{section:atoms}) must be at least $2^8-1$ ($255$)
and at most $2^{16}-1$ ($65\,535$).

The constant $\mathit{atom\_table\_size}$ gives
the size of the atom tables used when transforming terms to and from
the external representation (\S\ref{section:atom-tables}).
It must be at least $2^8$ ($256$).

\section*{Integers}

The integers in an implementation of \StdErlang\ are characterized by
five constants $\mathit{bounded}$, $\mathit{maxint}$, $\mathit{minint}$,
$\mathit{minfixnum}$ and $\mathit{maxfixnum}$
(\S\ref{section:integer-type}):
\begin{itemize}
\item There is no requirement on the constant $\mathit{bounded}$.
\item The constant $\mathit{maxint}$ is only relevant if $\it{bounded}=\B{true}$,
in which case $\mathit{maxint}$ must be at least $2^{59}-1$ (576\,460\,752\,303\,423\,487).
\item The only requirement on the constant $\mathit{maxfixnum}$ is the obvious
condition that $0 < \I{maxfixnum} \leq \I{maxint}$.
\item Either
$\I{minint} = -(\I{maxint}+1)$ or $\I{minint} = -\I{maxint}$ must hold.
\end{itemize}

\section*{Floats}

The floating-point numbers in an implementation of \StdErlang\ are characterized by
five constants $r$, $p$, $\mathit{emin}$, $\mathit{emax}$ and $\mathit{denorm}$
(\S\ref{section:float-type}):
\begin{itemize}
\item The radix $r$ should be even.
\item The precision $p$ should be such that $r^{p-1}\geq 10^6$.
\item For the constants $\mathit{emin}$ and $\mathit{emax}$ it should hold that
$\mathit{emin}-1 \leq k*(p-1)$ and  $\mathit{emax} > k*(p-1)$,
with $k\geq 2$ and $k$ as large an integer as practical, and that
$-2 \leq (emin-1) + emax \leq 2$.
\item There is no requirement on the constant $\mathit{denorm}$.
\end{itemize}

\section*{Refs}

The refs in an implementation are characterized by two constants:
\I{refs\_bounded} and \I{maxrefs} (\S\ref{section:refs}).
If \I{refs\_bounded} is \B{true},
then the value of \I{maxrefs} must be at least XXX.

\section*{PIDs}

The PIDs in an implementation are characterized by two constants:
\I{pids\_bounded} and \I{maxpids} (\S\ref{section:pids}).
If \I{pids\_bounded} is \B{true},
then the value of \I{maxpids} must be at least XXX.

\section*{Ports}

The ports in an implementation are characterized by two constants:
\I{ports\_bounded} and \I{maxports} (\S\ref{section:ports}).
If \I{ports\_bounded} is \B{true},
then the value of \I{maxports} must be at least XXX.

\section*{Tuples}

The constant \I{maxtuplesize} (\S\ref{section:tuples})
must be at least $2^{16}-1$ ($65\,535$) and at most
$2^{32}-1$ ($4\,294\,967\,296$).

\section*{Scheduling}

The constant \I{normal\_advantage} (\S\ref{section:scheduling})
should be between $4$ and $32$.
