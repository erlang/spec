%
% %CopyrightBegin%
%
% Copyright Ericsson AB 2017. All Rights Reserved.
%
% Licensed under the Apache License, Version 2.0 (the "License");
% you may not use this file except in compliance with the License.
% You may obtain a copy of the License at
%
%     http://www.apache.org/licenses/LICENSE-2.0
%
% Unless required by applicable law or agreed to in writing, software
% distributed under the License is distributed on an "AS IS" BASIS,
% WITHOUT WARRANTIES OR CONDITIONS OF ANY KIND, either express or implied.
% See the License for the specific language governing permissions and
% limitations under the License.
%
% %CopyrightEnd%
%

% Choose one of the following
\newcommand{\STYLE}{0} % For \OldErlang
%\newcommand{\STYLE}{1} % For \StdErlang

%Choose draft until we're almost there.
\newif\ifdraft
\drafttrue
%\draftfalse

% Choose \Difftrue to generate a chapter with differences.
\newif\ifDiff
%\Difftrue
\Difffalse

%Choose according to LaTeX implementation.  LaTeX2e strongly recommended.
\newif\ifsecondedition
\secondeditiontrue
%\secondeditionfalse

\ifsecondedition
\documentclass[11pt,a4paper]{book}
\else
\documentstyle[11pt,a4,twoside,grammar,alltt,swedish\ifdraft,draftstamp,drafthead\fi]{report}
\fi

%\includeonly{}

\edef\tdots{{}\dots{}} % Must do this before amstex gobbles it up.

\ifsecondedition
\usepackage[latin1]{inputenc} % Understand Latin-1 characters.
\usepackage{varioref} % Clever cross-references.
\usepackage{alltt} % Verbatim-like.
\usepackage{verbatim} % Better verbatim environment and \verb.
\usepackage{pifont} % Dingbats.
\usepackage{latexsym} % Extra math symbols.
\usepackage{amsmath} % Alignment, etc. [amstex on Bingo]
%\usepackage{amstext}
%usepackage[swedish,english]{babel} % Multiple languages.
% Not yet available!
%\usepackage{pict2e} % Better pictures.
\usepackage{grammar} % Grammar rules.
%\ifdraft
%\usepackage{draftstamp} % Print ``Draft'' shaded on pages.
%\usepackage{drafthead} % Print ``Draft'' etc as page header.
%\fi
%\usepackage{calc}
\usepackage{ifthen}
\usepackage{makeidx}
\fi

\ifdraft
\newcommand{\draftvsn}{0.7}
\newcommand{\Draft}[1]{\textsc{Draft}~#1}
%\renewcommand{\draftheadnote}{\Draft{\draftvsn}}
\fi

\newif\ifOld \if1\STYLE\Oldfalse\else\Oldtrue\fi
\newif\ifStd \if0\STYLE\Stdfalse\else\Stdtrue\fi
\newif\ifNew \if0\STYLE\Newfalse\else\Newtrue\fi

\ifsecondedition\else
\newcommand\textsc[1]{{\sc #1}}
\fi

% Define names for various types of Erlang.
% First generic name and name with version
\newcommand{\Erlang}{\textsc{Erlang}}
\newcommand{\ErlVsn}[1]{\Erlang~#1}

% Current old Erlang version
\newcommand{\OldVsn}{4.7.3}
\newcommand{\OldErlang}{\ErlVsn{\OldVsn}}

% Standard Erlang
\newcommand{\Std}{\textsc{Standard}}
\newcommand{\StdErlang}{\Std~\Erlang}

% Current version of Standard Erlang
\newcommand{\NewVsn}{5.0}
\newcommand{\NewErlang}{\ErlVsn{\NewVsn}}

\ifStd
\newcommand{\CrtClause}{CrtClause}
\newcommand{\CrtClauses}{CrtClauses}
\else
\newcommand{\CrtClause}{CrClause}
\newcommand{\CrtClauses}{CrClauses}
\fi

\ifStd
\title{Specification of the \\
       \StdErlang \\
       programming language \ifdraft \\[\bigskipamount]
       \Draft{\draftvsn}\fi}
\else
\title{\ErlVsn{4.7.3} \\ Reference Manual \ifdraft \\[\bigskipamount]
       \Draft{\draftvsn}\fi}
\fi

\author{Jonas Barklund \\ (barklund@hotmail.com) \and
	Robert Virding \\ (rv@cslab.ericsson.se)}

\newif\iftypedecl \typedeclfalse
\newif\ifstruct \structfalse

% Sometimes we want something one way in the text grammar but
% in another way in the appendix grammar.

\newif\ifappendix \appendixfalse

\newcounter{example}[chapter]
\renewcommand{\theexample}{\thechapter.\arabic{example}}
\newenvironment{example}{\refstepcounter{example}%
  \subsubsection*{Example \theexample}}{}

\ifsecondedition
\iffalse
\newcommand{\I}[1]{\relax\ifmmode\mathit{#1}\else\textit{#1}\fi}
\newcommand{\B}[1]{\relax\ifmmode\mathbf{#1}\else\textbf{#1}\fi}
\newcommand{\R}[1]{\relax\ifmmode\mathrm{#1}\else\textrm{#1}\fi}
\newcommand{\T}[1]{\relax\ifmmode\mathtt{#1}\else\texttt{#1}\fi}
\newcommand{\U}[1]{\relax\ifmmode\mathup{#1}\else\textup{#1}\fi}
\newcommand{\Z}[1]{\relax\ifmmode\mathsl{#1}\else\textsl{#1}\fi}
\newcommand\Tm[1]{\mathtt{#1}}
\else
\let\I=\textit
\let\B=\textbf
\let\R=\textrm
\let\T=\texttt
\let\U=\textup
\let\Z=\textsl
\fi
\newcommand\Zm[1]{\Z{\scriptsize#1}}
\newcommand\TZm[1]{\TZ{\scriptsize#1}}
\else
\newcommand{\I}[1]{{\it#1}}
\newcommand{\B}[1]{{\bf#1}}
\newcommand{\R}[1]{{\rm#1}}
\newcommand{\T}[1]{{\tt#1}}
\newcommand{\U}[1]{{\ddt#1}}
\newcommand{\Z}[1]{{\sl#1}}
\newcommand\Zm[1]{\mbox{\scriptsize\sl{#1}}}
\newcommand\TZm[1]{\mbox{\scriptsize\tt{\sl{#1}}}}
\newcommand{\emph}[1]{{\em#1\/}}
\fi

\newcommand{\TZ}[1]{\T{\Z{#1}}}

\iffalse
\newcommand{\decnum}{\B{dec}}
\newcommand{\inttype}{\B{int}}
\newcommand{\expr}[1]{\B{expr}[#1]}
\newcommand{\SemDec}[1]{[\![#1]\!]_{\decnum}}
\newcommand{\SemInt}[1]{[\![#1]\!]_{\inttype}}
\fi

\newcommand{\OTP}{OTP}

\newcommand{\etc}{etc.}

\newenvironment{textdisplay}{\begin{trivlist}\item}{\end{trivlist}}

%\newenvironment{suggestion}{\begin{dinglist}{224}\item}{\end{dinglist}}

%\newcommand{\THINK}{\par\begin{suggestion}THINK ABOUT IT BEFORE READING FURTHER!\end{suggestion}}

%\newif\iffemale \femaletrue
%\newcommand{\his}{\relax\iffemale her\global\femalefalse\else his\global\femaletrue\fi}

\newcommand{\domain}[1]{\R{dom}(#1)}

\ifsecondedition
\catcode`\_=13
{\catcode`\#=8 \gdef_{\relax\ifmmode#\else\texttt{\char`\_}\fi}}

\catcode`\^=13
{\catcode`\#=7 \gdef^{\relax\ifmmode#\else\texttt{\char`\^}\fi}}
\else
\catcode`\_=13
{\catcode`\#=8 \gdef_{\relax\ifmmode#\else{\tt\char`\_}\fi}}

\catcode`\^=13
{\catcode`\#=7 \gdef^{\relax\ifmmode#\else{\tt\char`\^}\fi}}
\fi

\def\{{\relax\ifmmode\lbrace\else\texttt{\char`\{}\fi}
\def\}{\relax\ifmmode\rbrace\else\texttt{\char`\}}\fi}

\def\tilde{\symbol{`\~}}

% From the LaTeX companion.
% Mentry is similar to the description environment but does not let
% labels run into the text.

\newcommand{\entrylabel}[1]{\mbox{\textbf{#1:}}\hfil}
\newenvironment{entry}
  {\begin{list}{}%
         {\renewcommand{\makelabel}{\entrylabel}%
          \setlength{\labelwidth}{35pt}%
%          \setlength{\leftmargin}{\labelwidth+\labelsep}%
          \setlength{\leftmargin}{\labelwidth}%
          \addtolength{\leftmargin}{\labelsep}%
         }%
  }%
  {\end{list}}

\newlength{\Mylen}
\newcommand{\Lentrylabel}[1]{%
  \settowidth{\Mylen}{\textbf{#1}}%
  \ifthenelse{\lengthtest{\Mylen > \labelwidth}}%
      {\parbox[b]{\labelwidth}%        term > labelwidth
         {\makebox[0pt][l]{\textbf{#1}}\\\mbox{}}}%
      {\textbf{#1}}%                  term < labelwidth
  \hfil\relax}
\newenvironment{Lentry}
   {\renewcommand{\entrylabel}{\Lentrylabel}%
    \begin{entry}}
   {\end{entry}}

% Customize itemize!

%\renewcommand{\labelitemi}{\ding{70}}
%\renewcommand{\labelitemii}{\ding{84}}
\renewcommand{\labelitemii}{$\ast$}

\iffalse
% Now handled by the inputenc package.
\catcode`\�=\active\def�{\AA}          % shift-option A
\catcode`\�=\active\def�{\"{A}}        % option u, then A
\catcode`\�=\active\def�{\"{O}}        % option u, then O
\catcode`\�=\active\def�{\aa}          % option a
\catcode`\�=\active\def�{\"{a}}        % option u, then a
\catcode`\�=\active\def�{\"{o}}        % option u, then o
\catcode`\�=\active\def�{\'{e}}
\fi

\newcommand{\INTS}{\mathcal{Z}}
\newcommand{\REALS}{\mathcal{R}}
\newcommand{\BOOLEANS}{\mathcal{B}}

\newcommand{\arccosh}{\mathop{\rm arccosh}\nolimits}
\newcommand{\arcsinh}{\mathop{\rm arcsinh}\nolimits}
\newcommand{\arctanh}{\mathop{\rm arctanh}\nolimits}

% An exit.

\newcommand{\badarith}{badarg}
\newcommand{\badindex}{badindex}

% Writing the results of evaluation

\newcommand{\yields}[2]{$\T{#1}\RETURNS\T{#2}$}

%!!!temporary
%\gdef\rightsquigarrow{\diamond}

\newcommand{\RETURNS}{\relax\ifmmode\Rightarrow\else$\Rightarrow$\fi}
\newcommand{\EXITSWITH}{\relax\ifmmode\leadsto\else$\leadsto$\fi}

\newcommand{\Er}{\Re^{-1}}

% For the representation of Erlang programs.
\newcommand{\Rep}{\R{\I{Rep}}}
\newcommand{\LINE}{\R{\B{L}}}

% These are used in the descriptions of expressions
% I want them to produce a bold italicized display heading.
% Use paragraph for this!

\makeatletter
\renewcommand{\paragraph}{\@startsection
   {paragraph}                       % the name
   {2}                               % the level
   {0mm}                             % the indent
   {-\baselineskip}                  % the beforeskip
   {0.5\baselineskip}                % the afterskip
   {\normalfont\normalsize\itshape\bfseries}} % the style
\makeatother

\newcommand{\SYNTAX}{\paragraph*{Syntax}}
\newcommand{\EVALUATION}{\paragraph*{Evaluation}}
\newcommand{\ENVIRONMENTS}{\paragraph*{Output environment}}
\newcommand{\NOTE}{\paragraph*{Note}}
\newcommand{\TYPE}{\paragraph*{Type}}
\newcommand{\EXITS}{\paragraph*{Exits}}
\newcommand{\EXAMPLES}{\paragraph*{Examples}}

%% Move part title and the dotted lines in contents to leave
%% enough space for full part number

\makeatletter
\renewcommand*\l@section{\@dottedtocline{1}{1.5em}{3em}}
\renewcommand*\l@subsection{\@dottedtocline{2}{4.5em}{4em}}
\renewcommand*\l@subsubsection{\@dottedtocline{3}{8.5em}{5em}}
\renewcommand*\l@paragraph{\@dottedtocline{4}{13.5em}{6em}}
\renewcommand*\l@subparagraph{\@dottedtocline{5}{19.5em}{7em}}
\makeatother

% Conversion from integer to float (LIA-1)

%\newcommand{\IntToFloat}[1]{\mathit{cvt}_{I\rightarrow F}(#1)}

% Make the bibliography appear in the table of contents.

\makeatletter
\renewenvironment{thebibliography}[1]
     {\chapter*{\bibname
        \@mkboth{\MakeUppercase\bibname}{\MakeUppercase\bibname}}%
      \addcontentsline{toc}{chapter}{\numberline{}\bibname}% Added by Jonas.
      \list{\@biblabel{\@arabic\c@enumiv}}%
           {\settowidth\labelwidth{\@biblabel{#1}}%
            \leftmargin\labelwidth
            \advance\leftmargin\labelsep
            \@openbib@code
            \usecounter{enumiv}%
            \let\p@enumiv\@empty
            \renewcommand\theenumiv{\@arabic\c@enumiv}}%
      \sloppy
      \clubpenalty4000
      \@clubpenalty \clubpenalty
      \widowpenalty4000%
      \sfcode`\.\@m}
     {\def\@noitemerr
       {\@latex@warning{Empty `thebibliography' environment}}%
      \endlist}
\makeatother

\makeindex

\begin{document}

\frontmatter

\begin{titlepage}

\maketitle

\end{titlepage}

%\thispagestyle{empty}\mbox{}\clearpage

%\pagenumbering{roman}

\tableofcontents

\ifdraft
%
% %CopyrightBegin%
%
% Copyright Ericsson AB 2017. All Rights Reserved.
%
% Licensed under the Apache License, Version 2.0 (the "License");
% you may not use this file except in compliance with the License.
% You may obtain a copy of the License at
%
%     http://www.apache.org/licenses/LICENSE-2.0
%
% Unless required by applicable law or agreed to in writing, software
% distributed under the License is distributed on an "AS IS" BASIS,
% WITHOUT WARRANTIES OR CONDITIONS OF ANY KIND, either express or implied.
% See the License for the specific language governing permissions and
% limitations under the License.
%
% %CopyrightEnd%
%

\chapter*{``Release notes'' for \Draft{0.7}}

This draft has the following important differences from \Draft{0.6}:

\begin{itemize}
\item Cleaned up meaning of \Erlang.
\item Removed \StdErlang\ lexical rules from lexical summary. (Used
Std test in ALL \Std\ rules)
\item Switch expression of case fixed.
\item LineTerminator is only LF in \OldErlang.  Warn about using
whitespace after \TXT{\$}.
\item Fix recognizer BIFs in \OldErlang\ to be only guard
recognizers.
\item Re-did macro definition and application syntax.  Hopefully
correct.
\item All numeric literals (Integer, Decimal, ExplicitRadix and Float)
are now unsigned in the lexical structure.
\item Made the logical operators have the same priorities as the
additive and multiplicative operators, as they should.
\item Atomic literals now defined in the expression chapter as this is
where values are explained.
\end{itemize}

\chapter*{``Release notes'' for \Draft{0.6}}

This draft has the following important differences from \Draft{0.5}:

\begin{itemize}
\item There is now an index (laboriously compiled).
\item There is a neat summary of all \Erlang\ expressions in \S\ref{chapter:summary-exprs}.
\item Function application is now described.
\item Records are now described.
\item Ports are now fully (?) described.
\item Dynamic code loading is now fully (?) described.
\item The external term format is now described.
\item The portable (?) hash function is now described.
\item Parse trees are now described.
\item Added \T{try \Z{E} end} to be almost equivalent with
\T{catch \Z{E}}.
\item The description of atoms has changed (\S\ref{section:atoms}).
\item All BIFs of \OldErlang\ are described (\S\ref{chapter:bifs}), except two:
\T{set_node/2} and \T{set_node/3}.
\item Exact equality and arithmetic equality (formerly called coerced equality)
are now well-defined, I think (\S\ref{section:equality}).
\item The section about Unicode escapes \ifStd (\S\ref{section:unicode-escapes}) \fi
is now written.
\item Lowered minimum \I{maxint} to $2^{59}-1$ (it was $2^{63}-1$)
so that on a 64-bit machine
an implementation without bignums can store all numbers as scalars with four
bits for a tag etc.
%\item The exit cause \T{badarith} has disappeared and been replaced by \T{badarg}.
\item Operators \T{//} and \T{mod} have been added (computing the same as
\T{div} and \T{rem} except that they round towards negative infinity rather
than zero).
\item Extent of function calls and last call optimization has been defined
(\S\ref{section:extent-function-calls}).
\item Dropped type and rule declarations completely from the specification.
\item Scheduling and waiting in \T{receive} expressions is described more precisely.
\end{itemize}

I am considering:
\begin{itemize}
\item Moving \S\ref{section:coding-pattern-matching} to an appendix.
\end{itemize}

Note:
\begin{itemize}
\item In the version for \StdErlang, I have assumed that the proposal
about new BIFs will be accepted, hence there are quite a lot of
references to such BIFs.  However, they are not described yet, hence
there are also quite a lot of unresolved cross-references in the
\StdErlang\ version.  If the BIF proposal is not accepted, then
I will change to use the current BIFs also in that version, of course.
\end{itemize}

The following is what remains:
\begin{itemize}
\item Checking release notes for R4.
\item The BIFs \T{set_node/2} (\S\ref{section:setnode2}) and
\T{set_node/3} (\S\ref{section:setnode3}).
\item Adding appendix about EPMD and \Erlang\ to \Erlang\ node communication.
\item If remaining proposals are accepted, incorporating them (the new BIFs is a hog,
the rest is fairly easy).
\item Checking that monitoring of nodes is described properly.
\end{itemize}

PLEASE, DRAFT READERS: THERE ARE PLACES IN THE TEXT WHERE I REQUEST
INFORMATION, ESPECIALLY ABOUT VARIOUS PARAMETERS OF ERLANG 4.7.  IF
YOU KNOW THESE THINGS, EMAIL ME!

\chapter*{``Release notes'' for \Draft{0.5}}

This draft has the following important differences from \Draft{0.4}:

\begin{itemize}
\item It has been added to the introduction how the specification
relates to the \Erlang\ language and the implementations \OldErlang\
and \NewErlang.
\item Some notions have been added to the glossary: expression,
operator, macro and term.
\item Macro \T{?VERSION} added.  Comments?
\item The description of equality tests is not quite accurate, will rewrite.
\item The evaluation order has been updated to be strictly left-to-right.
\item The section about guards (\S\ref{section:guards}) has been rewritten on
considerably simpler form.
\item The directionality when relating input and output environments has been made more clear.
\item It should now be even clearer that clauses are tried left to right and that
in \T{receive} expressions, all clauses are tried for a term in the message queue before
the next term is tried.
\item The terminology for abrupt completion has been made more uniform.
\item New chapter (\S\ref{chapter:arithmetics}) about arithmetics (LIA-1 based).
\item New (but not yet complete) chapter (\S\ref{chapter:more-about-ports}) about ports.
\item The format for BIF descriptions has evolved and many BIFs described.
\item The section about scheduling (\S\ref{section:scheduling}) has been improved.
\item New (short) sections about lifetime of terms (\S\ref{section:life-time}) and about
memory management (\S\ref{section:memory-management}).
\end{itemize}

Some things that are planned to be improved (and thus do not need to
be commented upon in the \Draft{0.5} if you agree):
\begin{itemize}
\item I will try to reduce the number of forward references further.
\end{itemize}

Known unclear issues:
\begin{itemize}
\item How should shadowing of BIFs work for those that are guard BIFs?
\end{itemize}

I have not gone through all comments from Torbj\"{o}rn Keisu.  They
should be incorporated in release 0.5.1.

\chapter*{``Release notes'' for \Draft{0.4}}

This draft has the following important differences from \Draft{0.3.1}:

\begin{itemize}
\item `\TXT{.}'\ and `\TXT{?}' are now listed as separators.
\item Full stops are detected properly (before white space and comments have been thrown
away).
\item The structure of a module is now described as consisting of
\begin{itemize}
\item First a module declaration \T{-module(\Z{ModuleName})}.
\item Then any \emph{header forms}, i.e., \T{export}, \T{import} and \T{compile}
attributes, together with \T{type} and \T{rule} declarations.
\item Then the \emph{program forms}, i.e., function declarations mixed with
\T{file} and wild attributes, record declarations and macro definitions.
\end{itemize}
\item The compilation of \Erlang\ programs is now described in detail in
\S\ref{chapter:compilation}.
\item A \emph{list} is now either \T{[]} or a cons in which the right part is a list.
\item The description of coercion has been improved (\S\ref{section:coercion}).
\end{itemize}

I have taken (most of) the written comments on 0.3 and/or 0.3.1 by
H�kan and Per Mildner into account.

Some things that are planned to be improved (and thus do not need to
be commented upon in the \Draft{0.4} if you agree):
\begin{itemize}
\item Express better the directionality when relating input and output environments
of expressions.
\item Express arithmetics in terms of LIA-1.
\item Change the writing so that becomes clear that clauses (in \T{case} etc) are
tried left to right.
\item Make the writing for \T{receive} expressions \emph{even} clearer concerning
the fact that for each message, each clause is tried before the next message is
considered.
\item Use a more consistent terminology for abrupt completion and its \\ causes.
\item The BIF Chapter (\S\ref{chapter:bifs}) is both incomplete and requiring changes.
\item Assuming that the revised proposal about evaluation order is accepted,
the evaluation order will be changed to strict left-to-right everywere.
\end{itemize}

\chapter*{``Release notes'' for \Draft{0.3.1}}

This draft has the following important differences from \Draft{0.3}:

\begin{itemize}
\item Various typos have been fixed (in particular the ones in \S\ref{section:match-expr}
and \S\ref{section:send-expr}
about match and send expressions and in \S\ref{section:moduleinfo1} about \T{module_info/1}
for \T{imports}).
\item The operator \T{and} is no longer also listed as a keyword
(\S\ref{section:keywords}).
\item The proposed new escape sequence \verb|\s| (for space) is described
(\S\ref{section:escapes}).
\item The blurb about process communcation has been removed from the
introduction of binaries (\S\ref{section:binaries}).
\item Records are now in the term order (between tuples and lists)
described in \S\ref{section:term-order}.
\item The \T{unregistered_name} error handler has been removed
(\S\ref{section:send-expr}), on request from Klacke and Tony.
\item The description of the list difference operator \T{--} has been
fixed so now it may be correct (\S\ref{section:list-subtraction}).
\item The description of \T{receive} expressions has been somewhat adjusted
to accord better with reality and with the intentions (\S\ref{section:receive-expr}).
\item The descriptions of how the clauses in \T{if}, \T{case}, \T{receive}, \T{try}
and \T{fun}
expressions are checked have been improved to increase clarity
(\S\ref{section:if-expr}, \S\ref{section:case-expr}, \S\ref{section:receive-expr},
\ifStd \S\ref{section:try-expr}, \fi \S\ref{section:fun-exprs}).
\item The description of \T{receive} expressions has been significantly
improved and may be correct (\S\ref{section:fun-exprs}).
\item The missing Section~\ref{section:guards} about guards is no longer missing.
\item \S\ref{chapter:processes} has been almost completely rewritten and
uses a new concept of \emph{signals} that comprises messages and exit signals,
as well as link/unlink requests, etc.  (This is actually how it is implemented
in the `5.0' system and the presentation becomes more precise.)
\item The priority \T{high} has been added (\S\ref{chapter:processes}).
\end{itemize}

I have taken the written comments on 0.3 by Klacke and Tony into account.
The comments by Per Mildner remain (I forgot to bring them to Stockholm).

\include{es-todo}
\fi

%\cleardoublepage

%\pagenumbering{arabic}

\mainmatter

%
% %CopyrightBegin%
%
% Copyright Ericsson AB 2017. All Rights Reserved.
%
% Licensed under the Apache License, Version 2.0 (the "License");
% you may not use this file except in compliance with the License.
% You may obtain a copy of the License at
%
%     http://www.apache.org/licenses/LICENSE-2.0
%
% Unless required by applicable law or agreed to in writing, software
% distributed under the License is distributed on an "AS IS" BASIS,
% WITHOUT WARRANTIES OR CONDITIONS OF ANY KIND, either express or implied.
% See the License for the specific language governing permissions and
% limitations under the License.
%
% %CopyrightEnd%
%

\chapter{Introduction}

\ifStd
\index{Erlang@\Erlang!\Std|(}
This document is a specification of the language \StdErlang, which will
be available from Ericsson Software Technology AB as \NewErlang. There is a
companion document \cite{olderlang}, which is a specification of the
closely related \Erlang\ implementation
called \OldErlang, developed at Ericsson Software Technology AB.
\index{Erlang@\Erlang!\Std|)}
\else
\index{Erlang@\Erlang!\OldVsn|(}
This document is a specification of the \Erlang\ implementation called
\ErlVsn{4.7.3}, developed at Ericsson Telecom AB.
\index{Erlang@\Erlang!\OldVsn|)}
\fi

\Erlang\ was originally designed by Joe Armstrong\index{Armstrong, Joe},
Robert Virding\index{Virding, Robert}, Claes
Wikstr\"{o}m\index{Wikstr\"{o}m, Claes} and Mike
Wil\-liams\index{Williams, Mike} at the Computer Science Laboratory of
Ericsson Telecommunications Systems Laboratories.

This specification is primarily designed to be useful for \Erlang\
programmers and implementors of \Erlang\ by providing clear although
mostly informally expressed semantics for all language constructs.
It should be of use also for those developing analysis tools for
\Erlang\ although for these purposes the semantics may have to be
reformulated as a formal system.

The specification should thus be able to function well as a reference
manual for the language.  As such it should be a good companion to the
book \emph{Concurrent Programming in ERLANG, Second Edition}
\cite{erlbook}, which is more of a tutorial, text book and evangel
than a language reference.  That edition of the book, however, uses an
earlier version of the language (the implementation \ErlVsn{4.3}).

It is also intended that the specification should be useful as a basis for
a future international standardization of the language.


%
% %CopyrightBegin%
%
% Copyright Ericsson AB 2017. All Rights Reserved.
%
% Licensed under the Apache License, Version 2.0 (the "License");
% you may not use this file except in compliance with the License.
% You may obtain a copy of the License at
%
%     http://www.apache.org/licenses/LICENSE-2.0
%
% Unless required by applicable law or agreed to in writing, software
% distributed under the License is distributed on an "AS IS" BASIS,
% WITHOUT WARRANTIES OR CONDITIONS OF ANY KIND, either express or implied.
% See the License for the specific language governing permissions and
% limitations under the License.
%
% %CopyrightEnd%
%

\chapter{Notation and glossary}

\emph{In this chapter we define the notation used in the remainder of
the specificaition, including that used for grammars.  Notation that
is mostly local to a single chapter is described in that chapter.
There is also a glossary that explains various terms that are used in
the specification (some of them specific to \Erlang, some of them
not).}

\ifStd
\section{Should and must}

\index{should|(}
\index{must|(}
When this specification states that an implementation \emph{should} do
in a certain way it expresses a preference that an implementation
ought to follow unless there is a good reason for not doing so.  (For
example, an implementation optimizing heavily for performance may
choose not to satisfy some ``should'' directives.)

When the specification simply states what how something is to be done,
whether the word ``must'' is used or not, an implementation must
appear to do as stated.  By this we means that an implementation may
deviate from what is literally stated, if it is not possible to
observe the deviation.  (Evaluation order is an example of something
that can sometimes be violated without any possibility of
observation.)
\index{must|)}
\index{should|)}
\fi

\section{Symbols}

We use typewriter style for \Erlang\ tokens, e.g., `\T{foo(X) -> bar(X)}'.
The symbol `\T{X}' is thus an \Erlang\ variable.

We use slanted typewriter style for metavariables\index{variable!meta-}, e.g.,
for the `\TZ{T}' in `\T{throw(\Z{T})}'.  A metavariable always consists of
letters but may have an index as
in `$\TZ{E}_2$' or `$\TZ{v}_1$'. A metavariable stands for some unspecified sequence
of \Erlang\ tokens.  The letter (and case of the letter) chosen for the main symbol of the
metavariable is an indication of the token sequences over which the metavariable usually ranges
(but the actual range is always given explicitly in the text).
\begin{itemize}
\item An `\TZ{E}' indicates an arbitrary \Erlang\ expression.
\item A `\TZ{T}', `\TZ{t}', `\TZ{v}' or `\TZ{w}' indicates an \Erlang\ term.  (The two latter
symbols are typically used for terms obtained as the value of some expression, rather
than a term occurring as part of the program text.)
%\item An `\TZ{U}' indicates an \Erlang\ inner guard expression (cf.~\S\ref{section:inner-guardexpr}).
\item A `\TZ{V}' indicates an \Erlang\ variable.
\item An `\TZ{A}' indicates an \Erlang\ atom.
\item A `\TZ{B}' indicates an \Erlang\ Boolean atom or an \Erlang\ binary.
\item An `\TZ{O}' indicates an \Erlang\ operator.
\item An `\TZ{I}' indicates an \Erlang\ integer numeral.
\item An `\TZ{F}' indicates an \Erlang\ floating point numeral.
\item A `\TZ{P}' indicates an arbitrary \Erlang\ pattern in some contexts
and an \Erlang\ PID in others.
\item An `\TZ{R}' indicates an \Erlang\ port or an \Erlang\ ref.
\item An `\TZ{M}' indicates an \Erlang\ module (i.e., a binary that is the representation
of a module).
\end{itemize}

We use italic letters in non-\Erlang\ expressions to stand for numbers or \Erlang\
terms.

\section{Sets}

\index{set notation|(}
In set expressions, $x\cup y$\index{  union@$\cup$} and $x\cap y$\index{  intersection@$\cap$}
mean the union and intersection of the sets $x$ and $y$, respectively.
$\overline{x}$\index{  complement@$\overline{\cdot}$}
means the complement
of a set $x$. $x\supseteq y$\index{  supset@$\supseteq$} means that $x$ contains
every element of $y$.
\index{set notation|)}

\section{Mappings}

\label{section:mappings}
\index{mapping|(}

A \emph{mapping} is a set of pairs such that each pair has a distinct
left half.  We write such a pair of $v$ and $t$ as $v\mapsto t$.  A
mapping $\epsilon$ can be \emph{applied}
to a value $v$, which is
written $\epsilon(v)$.  If there is a pair $v\mapsto t$ in $\epsilon$,
then the application denotes $t$.  Otherwise the denotation is
undefined.

We define the \emph{domain}
of a mapping $\epsilon$ to be the set of
values occurring in the left half of a pair in $\epsilon$ and
denote it by $\domain{\epsilon}$.

The \emph{restriction} of a mapping $\epsilon$ to a set of
values $d$ means the mapping which is the largest subset of $\epsilon$
such that its domain is contained in $d$ and is written $\epsilon|d$.

We say that that a mapping $\epsilon'$ \emph{extends} a mapping
$\epsilon$ if $\domain{\epsilon'}\supseteq\domain{\epsilon}$ and for every
value $v$ in $\domain{\epsilon}$, $\epsilon'(v)=\epsilon(v)$.  That is,
$\epsilon$ is a subset of $\epsilon'$.  Note that the extension can be
trivial --- $\epsilon$ extends $\epsilon$.

If $\epsilon'$ extends $\epsilon$ then the \emph{difference} between
$\epsilon'$ and $\epsilon$ is the mapping consisting of the pairs
that occur in $\epsilon'$ but not in $\epsilon$ and is denoted by
$\epsilon'\setminus\epsilon$.  In other words, $\epsilon'\setminus\epsilon =
\epsilon'|\overline{\domain{\epsilon}}$.

With the usual set notation we may denote a finite mapping
$\epsilon$ with domain $\{\TZ{v}_1,\ldots,\TZ{v}_n\}$ by
$\{\TZ{v}_1\mapsto\epsilon(\TZ{v}_1),\ldots,\TZ{v}_n\mapsto\epsilon(\TZ{v}_n)\}$.
(All mappings that will be discussed in this specification are finite.)

\index{  extension@$\oplus$|(}
Let $\epsilon$ and $\delta$ be mappings.  The result of \emph{extending}
$\epsilon$ with $\delta$ is a mapping that contains all pairs of $\delta$
and those pairs of $\epsilon$ having left halves not in the domain of $\delta$
and we denote it by $\epsilon\oplus\delta$.
In other words, $\epsilon\oplus\delta = \epsilon|\overline{\domain{\delta}}\cup\delta$.
\index{  extension@$\oplus$|)}

We say that an mapping is \emph{defined} for some value if the value is
in the domain of that mapping.

An \emph{empty} mapping has an empty set as its domain.
\index{mapping|)}

Most of this terminology is introduced for describing
environments\index{environment!is a mapping} in the evaluation of
expressions (\S\ref{chapter:expressions-evaluation}).

\section{Tables}

\label{section:tables}
\index{table|(}
A \emph{table} is an object with a state representing a mapping
between \Erlang\ terms.  We use the table metaphor because it is
suggestive to use a terminology of adding and removing rows.

A table consists of a finite number of rows where each row contains a
\emph{key}\index{key!of a table row} and a
\emph{value}\index{value!of a table row}, both of which are typically
\Erlang\ terms.
The keys of all rows in a table are distinct.  An empty table has no
rows.  If $t$ is a table having a row with key \TZ{k}, then we may
write $t(\TZ{k})$ for the value of that row.

Suppose that a table at a particular time contains $n$ rows having the
keys $\TZ{k}_1$, \ldots, $\TZ{k}_n$ and values $\TZ{v}_1$, \ldots, $\TZ{v}_n$.
\index{list!association|(}
An association list (\S\ref{section:assocationlists}) representing the
contents of the table contains $n$ 2-tuples
\T{\{$\Z{k}_i$,$\Z{v}_i$\}}, $1\leq i\leq n$, in some order.
\index{list!association|)}
A list representing the keys of the table contains $n$ terms $\Z{k}_i$,
$1\leq i\leq n$, in some order.
\index{table|)}

\iffalse
% replaced by tables
\section{Dictionaries}

\label{section:dictionaries}
\index{dictionary|(}

\iftrue
A dictionary is a set of pairs of terms, each pair with two
parts known as the \emph{key}\index{key!in a dictionary} and the
\emph{value}\index{value!in a dictionary},
such that the there are no two pairs with the same key.
% and there is no
%pair with the value \T{undefined}.
If $d$ is a dictionary, we denote the set of all keys in the dictionary
with $\mathit{keys}(d)$\index{keys@$\mathit{keys}$}.

A list \TZ{lst} \emph{represents}\index{dictionary!list representing a} a dictionary $d$ if
the following
three conditions are satisfied:
\begin{itemize}
\item If $d$ contains a key-value pair $\TZ{k}/\TZ{t}$, then \TZ{lst}
contains an element \T{\{\Z{k},\Z{t}\}}.
\item If \TZ{lst} contains an element \T{\{\Z{k},\Z{t}\}}, then
$d$ contains a key-value pair $\TZ{k}/\TZ{t}$.
\item There are no duplicates in \TZ{lst}
\end{itemize}
Note that the order of elements in \TZ{lst} is not significant.
\else
A dictionary is a function from some domain of \emph{keys} to some
domain of \emph{values}.  We assume the following functions operating
on dictionaries:
\begin{itemize}
\item $\mathit{has\_key}(d,k)$ is \B{true} if the dictionary $d$ contains
a value for the key $k$.
\item $\mathit{value\_of}(d,k)$ is the value associated with the key $k$
in the dictionary $d$ if such a value exists, otherwise it is undefined.
\item $\mathit{without}(d,k)$ is a dictionary which is exactly like $d$
except that it contains no value for the key $k$.
\item $\mathit{update}(d,k,v)$ is a dictionary which is exactly like $d$
except that it contains the value $v$ for the key $k$.
\end{itemize}
We also assume that the constant $\mathit{empty\_dictionary}$ denotes an
empty dictionary.
\fi
\index{dictionary|)}
\fi

\section{Types}

\label{section:type-notation}
\index{type!notation|(}

When describing BIFs (\S\ref{chapter:bifs}) we include information
about their types.  This information imposes constraints on which
should be the types of the evaluated arguments and what the type of
the result can be.  The \emph{signature} of a BIF is written in
accordance with that used for the experimental type checker for
\Erlang\
\cite{erltc,subtyping}.
The type expressions listed in Table~\ref{table:types}
are used in this description:
\begin{table}
\begin{center}
\begin{tabular}{@{}ll@{}}
\hline
Type expression & Type \\
\hline
\NT{AtomicLiteral} & the denoted term \\
\T{atom()} & atom \\
\T{bool()} & Boolean \\
\T{char()} & character \\
\T{int()} & integer \\
\T{float()} & float \\
\T{num()} & integer or float \\
\T{ref()} & ref \\
\T{bin()} & binary \\
\T{pid()} & PID \\
\T{port()} & port \\
\T{function()} & function \\
\T{tuple()} & tuple \\
\T{\{$\Z{T}_1$,\tdots,$\Z{T}_k$\}} & $k$-tuple with elements of types
$\Z{T}_1$, \ldots, $\Z{T}_k$ \\
\T{[\Z{T}]} & list with elements of type \TZ{T} \\
\T{string()} & string \\
\T{term()} & any \Erlang\ term \\
\hline
\end{tabular}
\caption{Types in \Erlang.}
\label{table:types}
\end{center}
\end{table}
The type of a function \T{\Z{F}/\Z{n}} is expressed on the form
\begin{alltt}
\Z{F}(\(\Z{T}_{1,1}\),\tdots,\(\Z{T}_{1,\TZm{n}}\)) -> \Z{T}_1 ;
\Z{F}(\(\Z{T}_{2,1}\),\tdots,\(\Z{T}_{2,\TZm{n}}\)) -> \Z{T}_2 ;
\ldots
\Z{F}(\(\Z{T}_{k,1}\),\tdots,\(\Z{T}_{k,\TZm{n}}\)) -> \Z{T}_k
\end{alltt}
and says that
\begin{itemize}
\item If \T{\Z{F}/\Z{n}} is applied to
\TZ{n} arguments that are of the types $\TZ{T}_{1,1}$, \ldots,
$\TZ{T}_{1,\TZm{n}}$, then the result is of type $\TZ{T}_1$.
\item Otherwise, if \ldots
\item Otherwise, if \T{\Z{F}/\Z{n}} is applied to
\TZ{n} arguments that are of the types $\TZ{T}_{k,1}$, \ldots,
$\TZ{T}_{k,\TZm{n}}$, then the result is of type $\TZ{T}_k$.
\end{itemize}
This should cover all valid combinations of argument types for \T{\Z{F}/\Z{n}}.
\index{type!notation|)}

\section{Grammar}

\subsection{Grammar notation}

\index{grammar!notation|(}

The purpose of a grammar is to define sets of well-formed sequences of
terminals, called \emph{syntactic categories}\index{grammar!syntactic
category}.  Some examples of syntactic categories in this
specification are \NT{AtomicLiteral} and \NT{Expr}, which consist of
the atomic literals and the expressions of \Erlang, respectively.

The \emph{terminals}\index{grammar!terminal} can be, e.g., individual
characters of some alphabet, or elements of some other set of tokens.
The set of tokens\index{token} may be infinite but it must be possible
to partition them into a finite number of categories.  The grammar
will only be concerned with the category to which a token belongs.

We use a standard grammar
formalism as explained, e.g., by Aho\index{Aho, Alfred V.}
\& Ullman\index{Ullman, Jeffrey D.} \cite{aho+ullman:foundations}
(although we use a somewhat different syntax than in their
exposition).  A grammar consists of a set of
\emph{productions}\index{grammar!production|(}.  Each production consists of
a \emph{head}, which is a syntactic category, and a
\emph{body}, which is a finite (and possibly
empty) sequence of terminals and syntactic categories.

A production is written as in
\begin{rules}
\grrule{RecordField}
       {\NT{AtomLiteral} \TXT{=} \NT{Expr}}
\end{rules}
where \NT{RecordField} is the head and \NT{AtomLiteral} \TXT{=}
\NT{Expr} is the body consisting of a syntactic category
\NT{AtomLiteral}, a terminal \TXT{=} and a syntactic category
\NT{Expr}.

We use two shorthands for productions.  First, we may write a production
having one head but $k$ bodies, where $k>1$, with each body on a separate line.
This is shorthand for $k$ productions, all with the same head.  For example,
we write the single production
\begin{rules}
\grrule{CompareExpr}
       {\NT{ListConcExpr} \NT{RelationalOp} \NT{ListConcExpr} \OR
        \NT{ListConcExpr} \NT{EqualityOp} \NT{ListConcExpr} \OR
        \NT{ListConcExpr}}
\end{rules}
instead of the three productions
\begin{rules}
\grrule{CompareExpr}
       {\NT{ListConcExpr} \NT{RelationalOp} \NT{ListConcExpr}}

\grrule{CompareExpr}
       {\NT{ListConcExpr} \NT{EqualityOp} \NT{ListConcExpr}}

\grrule{CompareExpr}
       {\NT{ListConcExpr}}
\end{rules}

\index{opt@\I{opt} (subscript)|(}
Second, we may attach a subscript \I{opt} (for `optional')
to a body element (as a matter of fact
we only do so for elements that are syntactic categories).  A production having
such an annotation in the body is shorthand for two productions, one where the body
element is present (but without the subscript) and one where it is not.
For example, the production
\begin{rules}
\grrule{TupleSkeleton}
       {\TXT{\char`\{} \OPT{Exprs} \TXT{\char`\}}}
\end{rules}
is short for the two productions
\begin{rules}
\grrule{TupleSkeleton}
       {\TXT{\char`\{} \NT{Exprs} \TXT{\char`\}}}

\grrule{TupleSkeleton}
       {\TXT{\char`\{} \TXT{\char`\}}}
\end{rules}
If more than one body element of a production has an \I{opt}
subscript, then this expansion can be repeated and will produce $2^k$ productions,
where $k$ is the number of subscripted elements.
For example, the production
\begin{rules}
\grrule{ModuleDeclaration}
       {\NT{ModuleAttribute} \OPT{HeaderForms} \OPT{ProgramForms}}
\end{rules}
is short for the two productions
\begin{rules}
\grrule{ModuleDeclaration}
       {\NT{ModuleAttribute} \NT{HeaderForms} \OPT{ProgramForms}}
\grrule{ModuleDeclaration}
       {\NT{ModuleAttribute} \OPT{ProgramForms}}
\end{rules}
which are short for the four productions
\begin{rules}
\grrule{ModuleDeclaration}
       {\NT{ModuleAttribute} \NT{HeaderForms} \NT{ProgramForms}}

\grrule{ModuleDeclaration}
       {\NT{ModuleAttribute} \NT{HeaderForms}}

\grrule{ModuleDeclaration}
       {\NT{ModuleAttribute} \NT{ProgramForms}}

\grrule{ModuleDeclaration}
       {\NT{ModuleAttribute}}
\end{rules}
\index{opt@\I{opt} (subscript)|)}
In \S\ref{section:integer-literals} an additional ad-hoc shorthand is used
where a single production
\begin{rules}
\grruleoneofthefirst{Digit[k]}{$\NT{Digit}[k]$}{$k$}
       {\TXT{0 1 2 3 4 5 6 7 8 9 Aa Bb Cc Dd Ee Ff}}
\end{rules}
is used to summarize sixteen productions on the form
\begin{rules}
\grrule{Digit2}
       {\TXT{0} \OR
        \TXT{1}}
\vdotsrule
\grrule{Digit16}
       {\TXT{0} \OR
        \TXT{1} \OR
        $\vdots$ \OR
        \TXT{E} \OR
        \TXT{e} \OR
        \TXT{F} \OR
        \TXT{f}}
\end{rules}
\index{grammar!production|)}
\index{grammar!notation|)}

\subsection{The lexical grammar}

\label{section:lexical-grammar}
\index{grammar!lexical}

The productions in in \S\ref{chapter:lexical} can be viewed as
constituting a grammar on their own.  That grammar has individual
\ifStd Unicode characters (\S\ref{section:unicode}) \else ASCII characters \fi
as terminals.  It defines two syntactic categories \NT{Token}
(\S\ref{section:input-elements}) and \NT{FullStop}
(\S\ref{section:full-stop}) from which the terminals of the main
grammar (\S\ref{section:main-grammar}) are drawn.

The subgrammar for \NT{Token} actually consists of regular expressions
and can therefore be coded in a system such as the \T{lex} utility
program, which translates the grammar to a finite automaton that can
be implemented very efficiently.

The lexical grammar is summarized in \S\ref{section:lex-gram-summary}.

\subsection{The main grammar}

\label{section:main-grammar}
\index{grammar!main}

In the main grammar presented in \S\ref{chapter:expressions-evaluation} and
\S\ref{chapter:programs-modules},
the terminals are the syntactic categories of the lexical grammar
and anything in typewriter style (all of which belong to
\NT{Token} as described in \S\ref{chapter:lexical}).

That grammar is almost, but not quite, a LALR(1)
grammar\index{grammar!not LALR(1)}.  The problem is that match
expressions (\S\ref{section:match-expr}) and generators
(\S\ref{section:list-comprehensions}) begin with a pattern, which is
typically indistinguishable from an expression using a lookahead of
only one token.  One way to make the grammar a LALR(1) grammar is to
change the productions
\begin{rules}
\grrule{MatchExpr}
       {\NT{Pattern} \TXT{=} \NT{SendExpr} \OR
        $\vdots$}
\grrule{Generator}
       {\NT{Pattern} \TXT{<-} \NT{Expr}}
\end{rules}
to
\begin{rules}
\grrule{MatchExpr}
       {\NT{ApplicationExpr} \TXT{=} \NT{SendExpr} \OR
        $\vdots$}
\grrule{Generator}
       {\NT{ApplicationExpr} \TXT{<-} \NT{Expr}}
\end{rules}
% ??? Should really be primaryexpr but it could be a record.
% [980514] What the heck did I mean with the above comment?
and add \NT{UniversalPattern} as a \NT{PrimaryExpr}.

Then it must be verified after parsing that the left-hand operand of \TXT{=} and
\TXT{<-} are indeed patterns and that no \NT{UniversalPattern} appears in an
expression.

(Note that this problem would not have appeared if the syntax of
match expressions and generators
had included a keyword before the pattern.)

The main grammar is summarized in \S\ref{section:main-gram-summary}.

\section{Glossary}

The purpose of this section is to explain some terms that are used
throughout the specification and either do not have a natural point of
definition or that need to be used before their point of definition.
In the latter case, this glossary is intended to give a summarical
explanation that will be sufficient until the full description is
reached.  Most concepts explained here are thus described in more
detail elsewhere.  When a word in an explanation is in bold face, it
is explained separately.

\begin{Lentry}
\item[Abrupt completion]
\index{completion!abrupt|(}
The evaluation of an expression completes abruptly either because a
problem has been encountered that makes it impossible or meaningless
to continue evaluation (cf.\ \B{exit}), or because it has been
requested to \B{throw} a value.  Abrupt completion always has an
associated reason\index{reason (for abrupt completion)} which is an
\Erlang\ term.  Abrupt completion can be trapped if it occurs in the
body of a
\ifStd\T{try} expression (\S\ref{section:try-expr}) \else
\T{catch} expression (\S\ref{section:catch}) \fi
and the reason can then be accessed.  Abrupt completion for
expressions is described in more detail in \S\ref{section:completion}.

A process may also complete abruptly, e.g., because the evaluation of
its original function application completes abruptly or it receives an
untrapped exit signal.  Abrupt completion for processes is described
in more detail in \S\ref{section:process-completion}.
\index{completion!abrupt|)}

\item[Application]
\index{function!application|(}
\index{application!in OTP sense|(}
We use this word with two overloaded meanings.  We can mean a function
application,
i.e., the application \T{\Z{F}($\Z{E}_1$,\tdots,$\Z{E}_k$)}
of an \Erlang\ function \TZ{F} to a sequence of \B{arguments} $\TZ{E}_1$, \ldots,
$\TZ{E}_k$, or we can
mean an application in the sense of \B{OTP}, i.e., a collection of
related modules.
\index{function!application|)}
\index{application!in OTP sense|)}

\item[Argument]
\index{argument!of a function application|(}
A function \B{application} expression has two main parts: a function to be
applied and a sequence of \emph{arguments}, or actual parameters, of the function
application.  The arguments are evaluated before the function call begins, so
the called function can only observe the values of the arguments.
\index{argument!of a function application|)}

\item[Arity]
\index{function!arity|(}
The \emph{arity} of a function is the number of arguments to which
it expects should be applied.  Each function has a specific arity (so the
clauses defining it must expect the same number of arguments) but there may be
functions having the same \B{function symbol} but different arities.
\index{function!arity|)}

\item[ASCII]
\index{ASCII|(}
ASCII is the popular acronym for the 7-bit code for representation of
characters properly called ANSI X3.4 \cite{ascii}.
\index{ASCII|)}

\item[BIF]
\index{BIF|(}
A BIF is a built-in function of \Erlang.
\ifStd
This implies that the if the function is provided in an \Erlang\
implementation conforming to this specification, it must have the
described semantics concerning result, effects, exits and order of
growth of time and space.
\fi

Being built-in does not imply any particular form of implementation: a
BIF could be implemented, for example, through a virtual machine
instruction, a procedure in another language, or an \Erlang\ function.
A BIF must not be redefined during the lifetime of a node so the
\Erlang\ compiler and loader are permitted to use all information in
the description of the BIF to make execution efficient.

Some BIFs have unqualified names (e.g., \T{length/1}) while others
must be referred to using a qualified name (e.g., \T{lists:map/2})
unless they have been explicitly imported
(\S\ref{section:import-attribute}).

The operators of \Erlang\ (\S\ref{section:operators}) are not BIFs,
but with the exception of match expressions\index{match expression}
(with the \T{=} operator), binary operator expressions behave exactly
like applications of binary functions.

\Erlang\ constructs such as \T{if} expressions, \T{catch} expressions, etc.,
are not even functions and thus not BIFs.

The BIFs of \Erlang\ are described in detail in \S\ref{chapter:bifs}.
\index{BIF|)}

\item[Big-endian]
\label{page:big-endian}
\index{big-endian|(}
When a sequence of \B{bytes} $\TZ{b}_1$, \ldots, $\TZ{b}_k$ is interpreted as a
\emph{big-endian} numeral, the significance decreases monotonically.
If the numeral has base 256 and is interpreted as unsigned, its value is
\[256(256(\ldots 256\TZ{b}_1+\TZ{b}_2\ldots)+\TZ{b}_{k-1})+\TZ{b}_k = \sum_{i=1}^k 256^{k-i}\TZ{b}_i,\]
and we denote this by
$\I{BigEndianValue}(\langle\TZ{b}_1, \ldots, \TZ{b}_k\rangle)$\index{BigEndianValue@\I{BigEndianValue}}.

For example, the four bytes
\T{22 188 72 209} when interpreted as a big-endian numeral represent
381,438,161 because $256^3\cdot22+256^2\cdot188+256^1\cdot72+256^0\cdot209 =
16777216\cdot22+65536\cdot188+256\cdot72+1\cdot209 = 369098752+12320768+18432+209 =
381438161$.
%  (Cf.\ \B{little-endian}.)

If the numeral is interpreted as signed, then its value is
\[(\I{BigEndianValue}(\langle\TZ{b}_1, \ldots, \TZ{b}_k)+2^{8k-1}\rangle) \bmod 2^{8k} - 2^{8k-1}\]
and we denote this by
$\I{BigEndianSignedValue}(\langle\TZ{b}_1, \ldots, \TZ{b}_k\rangle)$\index{BigEndianSignedValue@\I{BigEndianSignedValue}}.
\index{big-endian|)}

\item[Bignum]
\index{integer!bignum|(}
A bignum is an \Erlang\ integer that is not a fixnum.
\ifStd Typically, an implementation would represent bignums
\else Bignums are represented \fi
in such a way that integers with very large magnitudes can be
represented.  However, arithmetic operations on bignums, even
comparatively small ones, should be expected to be much more expensive
than the corresponding operations on \B{fixnums}.
\S\ref{section:binaries}.
\index{integer!bignum|)}

\item[Binary]
\index{binary|(}
When used as a noun, a binary is an \Erlang\ term that represents a
finite sequence of \B{bytes}.  Binaries are described in more detail
in \S\ref{section:binaries}.
\index{binary|)}

\item[Body]
\index{body|(}
A body (syntactic category \NT{Body}) is a nonempty sequence of
expressions.  Evaluating a body means to evaluate the constituent
expressions in order.  The value of the body is the value of the last
expression.
\index{body|)}

\item[Byte]
\index{byte|(}
A byte is an integer in the range $[0,255]$.
\index{byte|)}

\item[Call]
\index{function!call|(}
To \emph{call} a \B{function} (e.g., a \B{BIF}) means to evaluate an
application of the function.  However, we often mean only the part of
the evaluation that follows argument evaluation.
\index{function!call|)}

\item[Clause]
\index{clause|(}
In expressions where there is a choice between several alternatives,
each alternative is specified through a clause.  Every clause has a
\B{body} that will be evaluated if the clause is chosen. What the
other parts are depends on the surrounding expression.  In function
declarations and in \T{fun}, \T{case}, \T{receive} and \T{try}
expressions, each clause has a \B{pattern} and a corresponding
\B{guard}.  In \T{if} expressions, the clauses have only a \B{guard}.
\ifStd
In \T{cond} expressions, each clause has an arbitrary Boolean
expression.
\fi
\index{clause|)}

\item[Compile time]
\index{compile-time!definition of|(}
When we write that something is carried out at compile time or that a
compile time error should be raised, we mean that it should happen as
part of the process of transforming an \Erlang\ module definition into
a loadable binary.

However, one must also consider compilers that are applied to source
code that has already been loaded, e.g., in order to be used in an
interpreter.  In that case, enough processing must have taken place as
the source code was loaded so that all errors that should be detected
at compile-time according to this specification were detected at load
time.

Compilation of \Erlang\ modules is described in more detail in
\S\ref{chapter:compilation}.
\index{compile-time!definition of|)}

\item[Context]
\index{context!input|(}
\index{context!output|(}
A context is a set of variables.  Each expression that occurs as part
of an \Erlang\ program (i.e., a module declaration) has an \emph{input
context}, which is the set of variables that will have bindings at
\B{run time}, i.e., when the expression is evaluated.  (In other
words: the input context is the domain of the \B{environment} in which
the expression will be evaluated.)  It also has an \emph{output
context}, which is the input context extended with any variables for
which the expression will provide bindings.  Input and output contexts
are described in more detail in \S\ref{section:REB}.
\index{context!input|)}
\index{context!output|)}

\item[Effect]
\index{effect|(}
Evaluation of an \Erlang\ expression by an \Erlang\ process
may produce an \emph{effect}, regardless of
whether it completes
normally or abruptly.  An effect is one of:
\begin{itemize}
\item changing the state of a \B{process}\index{process!changing state of};
\item changing the state of a \B{port}\index{port!changing state of};
\item changing the state of a \B{node}\index{node!changing state of}.
\end{itemize}
Typical effects are sending a \B{message} (which adds to the message queue of some process)
or receiving a message (which removes from the message queue of the own process).

Effects that are observable externally are those that change the state
of a port, and in some cases those that change the state of a node.  A
typical example of such an effect is sending a message to a port in
order to produce output.
\index{effect|)}

\item[Environment]
\index{environment|(}
An expression is evaluated in an environment, which is a mapping from
variables to terms.  Environments are described in more detail in
\S\ref{section:REB}.
\index{environment|)}

\item[\Erlang]
\index{Erlang@\Erlang|(}
In this specification, the unqualified name \Erlang\ refers to
\ifStd\StdErlang\else\OldErlang\fi.
\ifStd\StdErlang\else\OldErlang\fi\ may be written explicitly when
referring to different behaviour of different versions.
\index{Erlang@\Erlang|)}

\item[Error]
\index{compile-time!error|(}
A \B{compile-time} error is a violation of a syntactic or semantic
rule that governs \Erlang\ programs and that can be verified in the
process of compiling an \Erlang\ module (\S\ref{chapter:compilation}).
If a compile-time error occurred while compiling a module, the result
of loading or using it is undefined.
\index{compile-time!error|)}

\index{run-time!error|(}
A \B{run-time} error is a violation of a required condition in the
semantic rules that govern \Erlang\ programs and that is verified
during evaluation.  If the description states that the evaluation
should \B{exit} with some reason, then this behaviour is required.
(The exit reason is then a term describing the error.)  If a stated
precondition is violated and the description does not state an exit
reason, then computation may proceed but the result and effects are
undefined.

If a process completes abruptly due to an error that was encountered
while evaluating its function application, the \B{exit signal}
provides information about the error.
\index{run-time!error|)}

\item[Exit]
\index{exit!reason|(}
When we say that the evaluation of an expression \emph{exits with
reason \TZ{T}}, this is short for saying that the evaluation of the
expression completes abruptly with an associated reason
\T{\char`\{'EXIT',\Z{T}\char`\}}.

An exit indicates that an exceptional situation such as a \B{run-time}
\B{error} has occurred.
\index{exit!reason|)}

\item[Exit signal]
\index{exit!signal|(}
An exit signal is an \Erlang\ term communicated from one process to
another, typically to inform the latter that the former process has
completed (or to simulate that this has happened). Exit signals are
described in more detail in
\S\ref{section:exit-signals}.
\index{exit!signal|)}

\item[Expression]
\index{expression|(}
An \Erlang\ \emph{expression} is a well-formed sequence of tokens
(syntactic category \NT{Expr}) that can be evaluated.  Either the
evaluation completes normally (cf.\ \B{normal completion}), in which
case the result is the value of the expression, or it completes
abruptly, in which case there is an associated reason for the
\B{abrupt completion}.
\index{expression|)}

\item[Fixnum]
\index{integer!fixnum|(}
A fixnum is an \Erlang\ integer in a range that can be represented
efficiently on the platform hosting the implementation.  Arithmetic
operations on fixnums should be expected to be fast.  The largest and
smallest fixnums are given by the implementation parameters
\I{maxfixnum} and \I{minfixnum}.
\S\ref{section:binaries}
\index{integer!fixnum|)}

\item[Float]
\index{float|(}
\emph{Float} is a synonym for floating-point number.
Floats are described in more detail in \S\ref{section:float-type}.
\index{float|)}

\item[Free variable]
\index{variable!free|(}
A \emph{free variable} in a \B{context} (cf.~\S\ref{section:REB})
is a variable that does not belong to that context.
\index{variable!free|)}

\item[Function]
\index{function|(}
In \Erlang\ a \emph{function} may compute a function in the
mathematical sense but does not necessarily do so.  This is because it
may have \B{effects}, such as sending or receiving messages or
modifying the state of the process
(\S\ref{section:process-state-dynamic}), the node
(\S\ref{section:node-state-dynamic}) or some port
(\S\ref{section:port-state-dynamic}), and it may also depend on these
states and what is received.

All \Erlang\ functions (including the \B{BIFs}) are
\emph{strict}\index{function!strict},
i.e., in a function \B{application} all \B{arguments} are fully
evaluated before any part of the function body is evaluated.
\index{function|)}

\item[Function symbol]
\index{function!symbol|(}
\index{function!name|(}
A named function is declared through a \NT{FunctionDeclaration},
in which case its name consists of a function symbol
that is an atom and an \B{arity} which is a nonnegative integer.
\index{function!name|)}
\index{function!symbol|)}

\item[Garbage collection]
\index{memory management|(}
Garbage collection is the activity of automatically reclaiming memory
that can no longer be referenced. In \Erlang\ this is done fully
automatically.  It is commonly believed that such automatic memory
management reduces the number of severe programming errors and thus
shortens development time and product quality, possibly at a modest
expense of execution time.  (One of the differences between the newer
language Java \cite{javaspec} over the older language C \cite{iso-c}
is that Java provides garbage collection.)
\index{memory management|)}

\item[Guard]
\index{guard|(}
A \emph{guard} (syntactic category \NT{Guard}) is a sequence of guard
tests (syntactic category \NT{GuardTest}), which are Boolean-valued
expressions.  They are used in \B{clauses} (such as those of function
declarations and \T{receive} expressions).  When a clause is
considered for selection, all guard tests of the guard must evaluate
to \T{true}.  (In contexts where the guard is optional, an omitted
guard is equivalent to the trivially true guard \T{true}.)

The guard tests and their constituent guard expressions (syntactic
category \NT{GuardExpr}) have been chosen so that they are independent
of the state (i.e., their result depends exclusively on their
arguments), have no side effect and (with a few exceptions) take
$O(1)$ time (with respect to the size of their arguments).
\index{guard|)}

\item[Implementation]
\ifStd
\index{Erlang@\Erlang!implementation of \Std|(}
An \emph{implementation} of \StdErlang\ is a system that satisfies
this description.  It may provide additional functionality, provided
that such functionality is documented.  When we write that something
is implementation-defined, no portable \StdErlang\ program can depend
on the choice made in a particular implementation.
\index{Erlang@\Erlang!implementation of \Std|)}
\else
Although we referred to \OldErlang\ as an \Erlang\ implementation,
there are actually implementations on various platforms that may
differ slightly, for example, due to different operating systems.
When we write that something is implementation-defined, no portable
\OldErlang\ program can depend on the choice made for a particular
platform.
\fi

\item[Latin-1]
\index{Latin-1|(}
\index{ISO/IEC 8859-1|(}
Latin-1 is the popular name for the 8-bit code for representation of
characters properly called ISO/IEC 8859-1 \cite{latin-1}.
\index{ISO/IEC 8859-1|)}
\index{Latin-1|)}

\item[List]
\index{list|(}
A list is a sequence of terms that either is empty, or consists of a
first element (the head) and a remaining list (the tail).  Lists are
described in more detail in
\S\ref{section:lists}.
\index{list|)}

\item[Literal]
\index{literal|(}
A literal is an expression for which the value is considered obvious
so no evaluation is necessary.  Numerals are obvious examples of
literals, other examples from \Erlang\ are string literals and atom
literals (\S\ref{chapter:types-terms})

All \Erlang\ literals are \B{terms} but there are \B{terms} for which
there are no literals, e.g., \B{PIDs} and \B{refs}.
\index{literal|)}

\item[Little-endian]
\label{page:little-endian}
\index{little-endian|(}
When a sequence of \B{bytes} $\TZ{b}_1$, \ldots, $\TZ{b}_k$ is interpreted as a
\emph{little-endian} numeral, the significance increases monotonically.
If the numeral has base 256 and is interpreted as unsigned, its value
is
\[\TZ{b}_1+256(\TZ{b}_2+256(\ldots\TZ{b}_{k-1}+256\TZ{b}_k\ldots)) = \sum_{i=1}^k 256^{i-1}\TZ{b}_i\]
and we denote this by
$\I{LittleEndianValue}(\langle\TZ{b}_1, \ldots, \TZ{b}_k\rangle)$\index{LittleEndianValue@\I{LittleEndianValue}}.

For example, the four bytes
\T{22 188 72 209} when interpreted as a little-endian numeral represent
3,511,204,886 because $256^0\cdot22+256^1\cdot188+256^2\cdot72+256^3\cdot209 =
1\cdot22+256\cdot188+65536\cdot72+16777216\cdot209 = 22+48128+4718592+3506438144 =
3511204886$.
%  (Cf.\ \B{big-endian}.)
\index{little-endian|)}

\item[Macro]
\index{macro|(}
A \emph{macro} is a token abstraction and each macro application is
replaced as part of the \B{preprocessing} by a sequence of
tokens. Macros are described in more detail in \S\ref{section:macros}.
\index{macro|)}

\item[Message]
\index{message|(}
A message is an \Erlang\ term communicated from one process to
another.  Each message is queued at the receiving process, which can
subsequently read a message by evaluating a \T{receive} expression
(\S\ref{section:receive-expr}).  Messages are described in more detail
in \S\ref{section:messages}.
\index{message|)}

\item[Module]
\index{module|(}
A module is a named unit of executable \Erlang\ code.  Its external
interface is a mapping from (exported) function names to functions.  A
module is loaded onto a \B{node}.  Modules are described in more
detail in \S\ref{chapter:programs-modules}.  Dynamic replacement of
modules is described in \S\ref{chapter:module-dynamics}.
\index{module|)}

\item[Node]
\index{node|(}
A node hosts \Erlang\ \B{processes} and \B{ports}.  There could be
more than one node on a computer and a node could run on a
multiprocessor computer.  All processes on a node share certain
resources. Nodes are described in more detail in
\S\ref{chapter:nodes}.
\index{node|)}

\item[Normal completion]
\index{completion!normal|(}
When the evaluation of an expression completes normally, a result has
been computed that is the value of the expression.
\index{completion!normal|)}

\item[Obsolete]
\index{obsolete|(}
When it is stated that something is \emph{obsolete} then we mean that
it is an old construction which should no longer be used.  Usually it
has only been kept for backwards compatibility.
\index{obsolete|)}

\item[Operator]
\index{operator!binary|(}
A \emph{binary operator} is a symbol that is written between two
expressions, called its \emph{operands}\index{operand (of operator)}.
For every \Erlang\ binary operator (\S\ref{section:operators}) except
\T{=}, such an operator expression is evaluated exactly like an
application of a binary function.
\index{operator!binary|)}

\index{operator!prefix|(}
A \emph{prefix operator} is a symbol that is written before an
expression, called its \emph{operand}\index{operand (of operator)}.
For every \Erlang\ prefix operator (\S\ref{section:operators}), such
an operator expression is evaluated exactly like an application of a
unary function.
\index{operator!prefix|)}

\item[OTP]
\index{OTP|(}
OTP stands for Open Telecom Platform \cite{otp-dev-ref} and is a
programming environment for telecom applications developed at Ericsson
Telecom AB.  The \Erlang\ language is a core component of OTP.
\index{OTP|)}

\item[Pattern]
\index{pattern|(}
A pattern expresses the structure of a term and its syntax (syntactic
category \NT{Pattern}) resembles that of a term.  It can be matched
(\S\ref{section:pattern-matching}) against a term in an input
\B{context}, which will either fail or succeed with an output context
that contains bindings for any \B{variables} in the pattern that were
not in the input context.  A \emph{universal pattern} (syntactic
category \NT{UniversalPattern}) is a wild card, i.e., it will match
any term.
\index{pattern|)}

\item[PID]
\index{PID|(}
A PID stands for a Process IDentifier and is an \Erlang\ term that
uniquely identifies a \B{process} that exists or has existed.  PIDs
are (ideally, cf.\ \S\ref{section:pids}) never `reused' so spawning a
new process always yields a new PID.  PIDs are described in more
detail in \S\ref{section:pids}.
\index{PID|)}

\item[Port]
\index{port|(}
A port is an \Erlang\ term that uniquely identifies an external
resource.  Communication with a port has been designed to be similar
to communication with a \B{process}. Ports are described in more
detail in \S\ref{chapter:more-about-ports}.
\index{port|)}

\item[Preprocessing]
\index{module!preprocessing of|(}
A \B{module} declaration is preprocessed as part of compilation (cf.\
\B{compile time}).  This involves macro expansion, elimination of the
record abstraction and conditional compilation.  Preprocessing is
described in more detail in \S\ref{section:preprocessing}.
\index{module!preprocessing of|)}

\item[Process]
\index{process|(}
A process is an entity that carries out the evaluation of an
`original' function application and can send and receive signals
(\S\ref{section:signals}), e.g., \B{messages} and
\B{exit signals}.  Processes are described in more detail in
\S\ref{chapter:processes}.
\index{process|)}

\item[Ref]
\index{ref|(}
A \emph{ref} is an \Erlang\ term.  The interesting property of refs is
that each call of the BIF \T{make_ref/0} (\S\ref{section:makeref0}) is
guaranteed to return a universally unique ref and that there is no
other way to obtain a ref (e.g., there are no ref literals).  Refs are
discussed in more detail in \S\ref{section:refs}.  (``Ref'' is
obviously short for ``reference'', but we write simply ``ref'' here as
``reference'' has a rather different meaning in some programming
languages.)
%(In \emph{Concurrent Programming in ERLANG, Second Edition}
%\cite{erlbook}, refs are called \emph{references}.)
\index{ref|)}

\item[Run time]
\index{run-time!definition of|(}
When we write that something is carried out at run time, we mean that
it should happen at the time an expression is being evaluated, etc.
For example, testing the type of \B{arguments} to \B{BIFs} is carried
out at run time in \Erlang\ (i.e., the types of the values of the
arguments are tested).
\index{run-time!definition of|)}

\ifStd
\item[Should]
\index{should!an implementation|(}
When it is stated that an implementation \emph{should} do something,
it means that it is strongly recommended that the implementation does
as specified.
\index{should!an implementation|)}
\fi

\item[Skeleton]
\index{skeleton|(}
A skeleton is an \Erlang\ expression that reveals the structure of a
data structure but for which the individual parts are given by
arbitrary expressions.

A skeleton is a \B{literal} if, and only if, all its subexpressions
are \B{literals}.
\index{skeleton|)}

\ifstruct
\item[Struct]
\index{struct|(}
An \Erlang\ struct is a mapping from a finite set of atoms $\TZ{A}_1$,
\ldots, $\TZ{A}_k$, called the \emph{domain} of the struct, to terms
$\TZ{t}_1$, \ldots, $\TZ{t}_k$.  A view of structs that is closer
to a natural form of implementation is that they consist of $k$
\emph{fields} named by the atoms, each of which contains an \Erlang\ term.
\index{struct|)}
\fi

\item[Syntactic sugar]
\index{syntactic sugar|(}
When we write that an expression \TZ{E} is \emph{syntactic sugar} for
some expression $\TZ{E}'$, it means that evaluation of \TZ{E} should
behave exactly like evaluation of $\TZ{E}'$.  The compiler could
implement \TZ{E} by replacing it with $\TZ{E}'$ or some expression
equivalent to $\TZ{E}'$.
\index{syntactic sugar|)}

\item[Term]
\index{term|(}
An \Erlang\ term is an \Erlang\ expression for which it is obvious
without any evaluation what value it denotes.  \Erlang\ terms are
described in more detail in \S\ref{chapter:types-terms}.

In mathematics, a term is usually a syntactic entity.  That is, it can
be written using some formal language.  This is not the case for all
\Erlang\ terms.

All \Erlang\ terms except refs, ports, pids, functions and binaries
have \B{literals}\index{literal!term denoted by} denoting them.  All
occurrences of a literal denote the same term.
\index{term|)}

\item[Throw]
\index{throw|(}
When we say that the evaluation of an expression \emph{throws \TZ{T}},
this is short for saying that the evaluation of the expression
completes abruptly with an associated reason
\T{\char`\{'THROW',\Z{T}\char`\}}.

This is always caused by evaluation of an application \T{throw(\Z{T})}
(\S\ref{section:throw1}) and indicates a programmer-controlled
nonlocal exit.
\index{throw|)}

\item[Token]
\index{grammar!token|(}
When we discuss the main grammar of \Erlang\
(\S\ref{section:main-grammar}), tokens are its terminals; the tokens
themselves are defined by the lexical grammar
(\S\ref{section:lexical-grammar}).
\index{grammar!token|)}

\item[Tuple]
\index{tuple|(}
An \Erlang\ $k$-tuple (where $k\geq0$) is a mapping from the integers
$1$, \ldots, $k$ to terms $\TZ{t}_1$, \ldots, $\TZ{t}_k$.  A view
of $k$-tuples that is closer to
\ifStd a natural form of \else the \fi
implementation is that they consist of $k$ enumerated \emph{fields},
each of which contains an \Erlang\ term.
\index{tuple|)}

\item[Type]
\index{type|(}
The terms of \Erlang\ are partitioned into a collection of
\emph{types} and many operations are only meaningful for terms of a
certain type.  \Erlang\ is a \emph{dynamically typed} language, which
means that the type of a term is always obvious but that the type of
any other kind of expressions, e.g., a variable, is not stated
explicitly and generally cannot be inferred at compile time.

The types and terms of \Erlang\ are described in more detail in
\S\ref{chapter:types-terms}.

In the descriptions of the \B{BIFs} of \Erlang, we
describe what are the expected types of their arguments and what
is the type of their result.  The notation used for these types is described in
\S\ref{section:type-notation}.
\index{type|)}

\item[Unicode]
\index{Unicode|(}
The \emph{Unicode} standard, version 2.0, \cite{unicode} is a 16-bit
code for representation of multilingual text.  It contains symbols for
most scripts used in the world today.  It includes the 7-bit
ASCII\index{ASCII} character set as its first 128 characters and the
8-bit Latin-1\index{Latin-1} character set as its first 256
characters.
\index{Unicode|)}

\item[Variable]
\index{variable|(}
A \emph{variable} (syntactic category \NT{Variable}, cf.\ \S\ref{section:variables})
stands for a term.
It can be part of an \B{expression} or of a \B{pattern}.
In the evaluation of an expression, the value must be found in the
\B{environment} (\S\ref{section:variables-eval}).
In pattern matching (\S\ref{section:pattern-matching}),
the value may already be in the
environment, or a value will be added to the environment when the variable
is first encountered.

In other words, a variable will be bound locally to a certain term.
Unlike in conventional (imperative) programming languages, there is no concept
of updating the value of a variable.
\index{variable|)}
\end{Lentry}


\grammarindextrue

%
% %CopyrightBegin%
%
% Copyright Ericsson AB 2017. All Rights Reserved.
%
% Licensed under the Apache License, Version 2.0 (the "License");
% you may not use this file except in compliance with the License.
% You may obtain a copy of the License at
%
%     http://www.apache.org/licenses/LICENSE-2.0
%
% Unless required by applicable law or agreed to in writing, software
% distributed under the License is distributed on an "AS IS" BASIS,
% WITHOUT WARRANTIES OR CONDITIONS OF ANY KIND, either express or implied.
% See the License for the specific language governing permissions and
% limitations under the License.
%
% %CopyrightEnd%
%

\chapter{Lexical structure}

\label{chapter:lexical}

This chapter describes the lexical structure of \Erlang\ programs and
how a sequence of \ifStd Unicode \else ASCII \fi characters is
translated to a sequence of tokens and full stops.

\ifStd
\section{Unicode}

\label{section:unicode}
\index{Unicode|(}

\StdErlang\ programs are written using the 16-bit \emph{Unicode} character set,
version 2.0, which contains the 7-bit ASCII character set as its first
128 characters and the 8-bit Latin-1 character set as its first 256
characters.

All \Erlang\ keywords, separators and operators use only the ASCII
subset.  It is therefore possible to write entire \Erlang\ programs
using only that subset, provided that identifiers, (string and
character) literals and comments use only ASCII characters.
\index{Unicode|)}

\index{escape!Unicode|(}
There is also an escape mechanism which allows non-ASCII characters to be
written using only ASCII characters.  This allows editing of \Erlang\
programs also using editors that support only ASCII or Latin-1.
Software tools for \Erlang\ (e.g., interactive debuggers)
should also use this escape mechanism when displaying
\Erlang\ programs on devices that do not support Unicode.
\index{escape!Unicode|)}
\fi

\section{Lexical translation}

\label{section:lexical-translation}
\index{translation!lexical|(}

A sequence of \ifStd Unicode \else ASCII \fi characters is translated
into a sequence of \Erlang\ tokens by applying the following four
steps, in order:

\begin{enumerate}
\ifStd
\item Each Unicode escape in the sequence of characters
is replaced by the corresponding Unicode character (\S\ref{section:unicode-escapes}).
\fi
\item \label{item:unicode-line-sep} The sequence of
\ifStd Unicode characters from step~1 \else characters \fi
is translated to a sequence
of input characters and line terminators (\S\ref{section:line-terminators}).
If the sequence does not end with a line terminator, one is added at the end.
\item The sequence of input characters and line terminators is translated
to a sequence of \Erlang\ input elements (\S\ref{section:input-elements}).
\item The sequence of \Erlang\ input elements is translated to a sequence of tokens
and full stops by discarding white space (\S\ref{section:white-space}) and comments
(\S\ref{section:comments}) and detecting full
stops (\S\ref{section:full-stop}). The resulting
sequence of tokens and full stops is the result of the lexical processing.
\end{enumerate}
\index{translation!lexical|)}

\ifStd
\section{Unicode escapes}

\label{section:unicode-escapes}
\index{escape!Unicode|(}

(The processing of Unicode escapes is intentionally made identical with that in Java\index{Java}
\cite[pp.~12--13]{javaspec}.)

\begin{rules}
\ifStd
\grrule{UnicodeInputCharacter}
       {\NT{UnicodeEscape} \OR
        \NT{RawInputCharacter}}

\grrule{UnicodeEscape}
       {\TXT{\char`\\} \NT{UnicodeMarker} \NT{HexDigit} \NT{HexDigit} \NT{HexDigit} \NT{HexDigit}}

\grrule{UnicodeMarker}
       {\TXT{u} \OR
        \NT{UnicodeMarker} \TXT{u}}

\grrule{RawInputCharacter}
       {any Unicode character}

\grrule{HexDigit}
       {\NT{Digit}[16]}
\fi
\end{rules}
\NT{Digit}[16] is as in \S\ref{section:integer-literals}.

In addition to what is specified by the grammar above, a backslash\index{\ character@\T{\char`\\} character}
(\TXT{\char`\\}) character begins a Unicode escape only if
\begin{itemize}
\item there is an even number of backslash characters separating it
from preceding non-backslash characters or the beginning of the input
stream, and
\item it is followed by \TXT{u}.
\end{itemize}
Unless both these conditions are met, the backslash should be treated
as a \NT{RawInputCharacter}.

If an eligible backslash is followed by one or more \TXT{u} and the
last \TXT{u} is not followed by four hexadecimal digits, then a
compile-time error should occur.

An eligible sequence consisting of backslash, one or more \TXT{u}, and
four hexadecimal digits with values $d_3$, $d_2$, $d_1$ and $d_0$ is
replaced in the input stream by a Unicode character with character
code $4096d_3+256d_2+16d_1+d_0$.  The resulting character is never
part of further Unicode escape processing.

Note that there exists an invertible transformation from a sequence of
\NT{UnicodeInputCharacter} to a sequence of \NT{UnicodeInputCharacter}
that only contains ASCII (or Latin-1) characters:
\begin{itemize}
\item Replace any \NT{UnicodeEscape} with a \NT{UnicodeEscape} having
one more \TXT{u}.
\item Replace any non-ASCII (or non-Latin-1) \NT{RawInputCharacter}
with a corresponding \NT{UnicodeEscape}.
\item Leave any ASCII (or Latin-1) \NT{RawInputCharacter} as is.
\end{itemize}
The resulting stream of characters consists only of ASCII (or Latin-1)
characters and can thus be processed by tools that handle only any
ASCII (or Latin-1) characters, e.g., many standard UNIX utilities.
The inverse transformation is obvious.

An \Erlang\ system should use Unicode escapes to display Unicode
characters that cannot be displayed in a current font.
\index{escape!Unicode|)}
\fi

\section{Character classes}
\index{section:character-classes}

\begin{rules}
\grrule{ErlangUppercase}
       {the capital \ifStd ISO 8859-1 \else ASCII \fi letters
\T{A}--\T{Z} (\T{\char`\\\ifStd u0041\else 101\fi} to \T{\char`\\\ifStd u005a\else 132\fi})\ifStd,
\T{�}--\T{�} (\T{\char`\\u00c0} to \T{\char`\\u00d6}) and % 192 - 214
\T{\char'037}--\T{[THORN]} % Should be `�'
(\T{\char`\\u00d8} to \T{\char`\\u00de})\fi}

\grrule{ErlangLowercase}
       {the small
\ifStd ISO 8859-1 \fi\ifOld ASCII \fi letters
\T{a}--\T{z} (\T{\char`\\\ifStd u0061\fi\ifOld 141\fi} to \T{\char`\\\ifStd u007a\fi\ifOld 172\fi})\ifStd,
\T{\char'031}--\T{�} (\T{\char`\\u00df} to \T{\char`\\u00f6}) and % 246
\T{\char'034}--\T{�} (\T{\char`\\u00f8} to \T{\char`\\u00ff}) \fi} % 255

\grrule{ErlangLetter}
       {\NT{ErlangLowercase} \OR \NT{ErlangUppercase}}

\grrule{ErlangDigit}
       {the
ASCII decimal digits \T{0}--\T{9} (\T{\char`\\\ifStd u0030\fi\ifOld 060\fi} to
\T{\char`\\\ifStd u0039\fi\ifOld 071\fi})}
\end{rules}

\section{Line terminators}

\label{section:line-terminators}
\index{line terminator|(}

Line terminators need to be recognized uniformly across platforms
so \Erlang\ compilers and tools can report line numbers and determine
the end of comments coherently.
\begin{rules}
\ifStd
\grrule{LineTerminator}
       {the ASCII LF character (``linefeed''\index{Linefeed character} or ``newline''\index{Newline character})
\OR
        the ASCII CR character (``return''\index{Return character}) \OR
	the ASCII CR character followed by the ASCII LF character}
\else
\grrule{LineTerminator}
       {the LF character (``linefeed''\index{Linefeed character} or ``newline''\index{Newline character})}
\fi

\ifStd
\grrule{InputCharacter}
       {\NT{UnicodeInputCharacter} but not ASCII CR or LF}
\else
\grrule{InputCharacter}
       {\NT{AsciiInputCharacter} but not LF}
\fi
\end{rules}
\ifStd
A CR character followed by a LF character is thus considered one
line terminator, not two.
\fi
\index{line terminator|)}

\section{Input elements}

\label{section:input-elements}
\index{input element|(}

The sequence of input characters and line terminators is translated to
a sequence of input elements, i.e., white space
(\S\ref{section:white-space}), comments (\S\ref{section:comments}) and
tokens.

\index{grammar!token|(}
There are six kinds of tokens:
separators (\S\ref{section:separators}),
keywords (\S\ref{section:keywords}),
operators (\S\ref{section:operators}),
integer literals (\S\ref{section:integer-literals}),
float literals (\S\ref{section:float-literals}),
character literals (\S\ref{section:char-literals}),
string literals (\S\ref{section:string-literals}),
atom literals (\S\ref{section:atom-literals}),
variables (\S\ref{section:variables}),
universal patterns (\S\ref{section:universal-pattern}).
\index{grammar!token|)}

\iffalse
Note that the input characters and line terminators may optionally be
followed by a control-Z character (as provided by some operating
systems).
\fi

Note \iffalse also\fi that white space and comments always will
separate tokens.

\begin{rules}
\grrule{Input}
       {\OPT{InputElements}\iffalse \OPT{Sub}\fi}

\grrule{InputElements}
       {\NT{InputElement} \OR
        \NT{InputElements} \NT{InputElement}}

\grrule{InputElement}
       {\NT{WhiteSpace} \OR
        \NT{Comment} \OR
        \NT{Token}}

\grrule{Token}
       {\NT{Separator} \OR
        \NT{Keyword} \OR
        \NT{Operator} \OR
        \NT{IntegerLiteral} \OR
        \NT{FloatLiteral} \OR
        \NT{CharLiteral} \OR
        \NT{StringLiteral} \OR
        \NT{AtomLiteral}
        \NT{Variable} \OR
        \NT{UniversalPattern}}
\iffalse
\grrule{Sub}
       {the ASCII SUB character, also known as ``control-Z''}
\fi
\end{rules}
\index{input element|)}

\section{White space}

\label{section:white-space}
\index{white space|(}

White space is defined as the characters with ASCII codes less than or equal to that
of ASCII space, i.e., the control characters and space.  The line terminators
are composed of control characters but have been identified in a preceding step
and are therefore included explicitly below.

\begin{rules}
\grrule{WhiteSpace}
       {\NT{LineTerminator} \OR
        \NT{ControlCharacter} \OR
        the ASCII SP character, also known as ``space''\index{space character}}

\grrule{ControlCharacter}
       {any ASCII control character\index{control character} (\T{\char`\\\ifStd u0000\else 000\fi} to \T{\char`\\\ifStd u001f\else 037\fi})}
\end{rules}
\index{white space|)}

\section{Comments}

\label{section:comments}
\index{comment|(}

A comment begins with an ASCII \T{\%} character\index character} and extends up to and including
the next line terminator.

\begin{rules}
\grrule{Comment}
       {\TXT{\%} \OPT{InputCharacters} \NT{LineTerminator}}

\grrule{InputCharacters}
       {\NT{InputCharacter} \OR
        \NT{InputCharacters} \NT{InputCharacter}}
\end{rules}
Examples of comments:
\begin{verbatim}
        %A space after the percent is not obligatory but...
% The comment can begin a line.
     %% There can be additional %s in the comment.
\end{verbatim}
\index{comment|)}

\section{Separators}

\label{section:separators}
\index{separator|(}

The following 15 tokens are \emph{separators} in \Erlang\ifStd,
formed from ASCII characters\fi:
\begin{rules}
{\obeyspaces%
\grruleoneof{Separator}{\TXT{(    )    \char`\{    \char`\}    [    ]    .    :}\\
                        \TXT{\char`\|~~~~\char`\|\char`\|~~~;~~~~,~~~~?~~~~->~~~\char`\#}}}
\end{rules}
\index{separator|)}

\section{Keywords}

\label{section:keywords}
\index{keyword|(}

The following \ifStd 15 \else 13 \fi tokens are the \emph{keywords} of
\Erlang\ifStd, formed from ASCII characters\fi:
\begin{rules}
\ifStd
{\obeyspaces%
\grruleoneof{Keyword}{\TXT{after     catch     if        some_true}\\
                      \TXT{all_true  cond      let       try}\\
                      \TXT{begin     end       of        when}\\
                      \TXT{case      fun       receive   }}}
\else
{\obeyspaces%
\grruleoneof{Keyword}{\TXT{after     cond      let       when}\\
                      \TXT{begin     end       of}\\
                      \TXT{case      fun       query}\\
                      \TXT{catch     if        receive}}}
\fi
\end{rules}
Thus they cannot be used as atom literals (\S\ref{section:atom-literals}).

\ifStd
The keyword \T{let} is not defined but is
reserved for a possible future
extension of the language.
The keyword \T{query} is also not defined in this document but is reserved for
the Mnesia subsystem of OTP \cite{otp-mnesia}.
\else
The keywords \T{all_true}, \T{cond}, \T{let} and \T{some_true} are
not defined in \OldErlang\ but are reserved for possible future
extensions of the language.
\fi
\index{keyword|)}

\section{Operators}

\label{section:operators}
\index{operator|(}

The following \ifStd 31 \else 29 \fi tokens are the \emph{operators}
of \Erlang\ifStd, formed from ASCII characters\fi:
\begin{rules}
{\obeyspaces%
\grruleoneof{Operator}{\TXT{+    -    *    /    div  rem  \ifStd//   mod\fi}\\
                       \TXT{or   xor  bor  bxor bsl  bsr  and  band}\\
                       \TXT{==   /=   =:=  =/=  <    =<   >    >=}\\
                       \TXT{not  bnot ++   --   =    !~~~~<-}}}
\end{rules}
Thus they cannot be used as atom literals (\S\ref{section:atom-literals}).
\index{operator|)}

\section{Escape sequences}

\label{section:escapes}
\index{escape!character|(}

These are the escape sequences for character literals, string literals
and quoted atom literals.  All escape sequences begin with a
backslash (\T{\char`\\}).
\iffalse
\index{ \b@\TXT{\char`\\b}|(}
\index{ \d@\TXT{\char`\\d}|(}
\index{ \e@\TXT{\char`\\e}|(}
\index{ \f@\TXT{\char`\\f}|(}
\index{ \n@\TXT{\char`\\n}|(}
\index{ \r@\TXT{\char`\\r}|(}
\index{ \s@\TXT{\char`\\s}|(}
\index{ \t@\TXT{\char`\\t}|(}
\index{ \v@\TXT{\char`\\v}|(}
\index{ \\@\TXT{\char`\\\char`\\}|(}
\index{ \'@\TXT{\char`\\'}|(}
\index{ \"@\TXT{\char`\\""}|(}
\index{ \1@\TXT{\char`\\} $\langle\I{digits}\rangle$|(}
\index{ \^@\TXT{\char`\\\char`\^}|(}
\fi
\begin{rules}
\grrule{EscapeSequence}
       {\TXT{\char`\\} \TXT{b} & \TXT{\% \char`\\\ifStd u0008\else 008\fi: \R{backspace BS}} \OR
        \TXT{\char`\\} \TXT{d} & \TXT{\% \char`\\\ifStd u007f\else 177\fi: \R{delete DEL}} \OR
        \TXT{\char`\\} \TXT{e} & \TXT{\% \char`\\\ifStd u001b\else 033\fi: \R{escape ESC}} \OR
        \TXT{\char`\\} \TXT{f} & \TXT{\% \char`\\\ifStd u000c\else 014\fi: \R{form feed FF}} \OR
        \TXT{\char`\\} \TXT{n} & \TXT{\% \char`\\\ifStd u000a\else 012\fi: \R{linefeed LF}} \OR
        \TXT{\char`\\} \TXT{r} & \TXT{\% \char`\\\ifStd u000d\else 015\fi: \R{carriage return CR}} \OR
        \TXT{\char`\\} \TXT{s} & \TXT{\% \char`\\\ifStd u0020\else 040\fi: \R{space SPC}} \OR
        \TXT{\char`\\} \TXT{t} & \TXT{\% \char`\\\ifStd u0009\else 011\fi: \R{horizontal tab HT}} \OR
        \TXT{\char`\\} \TXT{v} & \TXT{\% \char`\\\ifStd u000b\else 013\fi: \R{vertical tab VT}} \OR
        \TXT{\char`\\} \TXT{\char`\\} & \TXT{\% \char`\\\ifStd u005c\else 008\fi: \R{backslash} \char`\\} \OR
        \NT{ControlEscape} & \TXT{\% \char`\\\ifStd u0000 \else 000 \fi to \char`\\\ifStd u001f\else 037\fi: \R{64 less than the char}} \OR
        \TXT{\char`\\} \TXT{'} & \TXT{\% \char`\\\ifStd u0027\else 047\fi: \R{single quote} '} \OR
        \TXT{\char`\\} \TXT{"} & \TXT{\% \char`\\\ifStd u0022\else 042\fi: \R{double quote} "} \OR
        \NT{OctalEscape} & \TXT{\% \char`\\\ifStd u0000 \else 000 \fi to \char`\\\ifStd u00ff\else 777\fi: \R{from octal value}}}

\grrule{ControlEscape}
       {\TXT{\char`\\} \TXT{\char`\^} \NT{ControlName}}

\grrule{ControlName}
       {any \ifStd Unicode \fi character between \T{\char`\\\ifStd u0040\else 100\fi} and \T{\char`\\\ifStd u005f\else 137\fi}}

\grrule{OctalEscape}
       {\ifStd 
	\TXT{\char`\\} \NT{ZeroToThree} \NT{OctalDigit} \NT{OctalDigit}
	\else
	\TXT{\char`\\} \NT{OctalDigit} \OR
        \TXT{\char`\\} \NT{OctalDigit} \NT{OctalDigit} \OR
        \TXT{\char`\\} \NT{OctalDigit} \NT{OctalDigit} \NT{OctalDigit}
	\fi}

\grruleoneof{OctalDigit}{\TXT{0 1 2 3 4 5 6 7}}

\ifStd
\grruleoneof{ZeroToThree}{\TXT{0 1 2 3}}
\fi
\end{rules}

\iffalse
\index{ \b@\TXT{\char`\\b}|)}
\index{ \d@\TXT{\char`\\d}|)}
\index{ \e@\TXT{\char`\\e}|)}
\index{ \f@\TXT{\char`\\f}|)}
\index{ \n@\TXT{\char`\\n}|)}
\index{ \r@\TXT{\char`\\r}|)}
\index{ \s@\TXT{\char`\\s}|)}
\index{ \t@\TXT{\char`\\t}|)}
\index{ \v@\TXT{\char`\\v}|)}
\index{ \\@\TXT{\char`\\\char`\\}|)}
\index{ \'@\TXT{\char`\\'}|)}
\index{ \"@\TXT{\char`\\""}|)}
\index{ \1@\TXT{\char`\\} $\langle\I{digits}\rangle$|)}
\index{ \^@\TXT{\char`\\\char`\^}|)}
\fi
In the case of a control escape, it denotes the character which has a
\ifStd Unicode value \else code \fi
that is 64 less than that of the \NT{ControlName}
following \T{\char`\\} and \T{\char`\^}.

In the case of an octal escape, it denotes
\ifStd
the character which has the Unicode value
\else
the integer
\fi
denoted by the octal numeral.
\ifStd
Note that octal escapes can express only characters with Unicode
values \T{\char`\\u0000} to \T{\char`\\u00ff}.
\else
Note that only the octal escapes \T{\char`\\000} to \T{\char`\\177}
denote characters.  The octal escapes \T{\char`\\200} to
\T{\char`\\377} denote noncharacter bytes while
\T{\char`\\400} to \T{\char`\\777} denote larger integers.
\fi
\index{escape!character|)}

\section{Integer literals}

\label{section:integer-literals}
\index{integer!literal|(}

An integer literal consists of an optional sign,
an optional radix\index{radix (of integer literal)} specifier and a sequence of digits in the
specified radix (which is decimal if omitted).
\ifStd
Between
any two digits there may be an underscore\index{_ character@\T{_} character}, which is
ignored. (This is to simplify correct entry of long
numerals by making grouping possible.)
\fi
\begin{rules}
\grrule{IntegerLiteral}
       {\NT{DecimalLiteral} \OR
        \NT{ExplicitRadixLiteral}}

\grrule{DecimalLiteral}
       {$\NT{Digits}[10]$}

\grrule{ExplicitRadixLiteral}
       {\TXT{2} \TXT{\#} $\NT{Digits}[2]$ \OR
        \TXT{3} \TXT{\#} $\NT{Digits}[3]$ \OR
        \TXT{4} \TXT{\#} $\NT{Digits}[4]$ \OR
        \TXT{5} \TXT{\#} $\NT{Digits}[5]$ \OR
        \TXT{6} \TXT{\#} $\NT{Digits}[6]$ \OR
        \TXT{7} \TXT{\#} $\NT{Digits}[7]$ \OR
        \TXT{8} \TXT{\#} $\NT{Digits}[8]$ \OR
        \TXT{9} \TXT{\#} $\NT{Digits}[9]$ \OR
        \TXT{1} \TXT{0} \TXT{\#} $\NT{Digits}[10]$ \OR
        \TXT{1} \TXT{1} \TXT{\#} $\NT{Digits}[11]$ \OR
        \TXT{1} \TXT{2} \TXT{\#} $\NT{Digits}[12]$ \OR
        \TXT{1} \TXT{3} \TXT{\#} $\NT{Digits}[13]$ \OR
        \TXT{1} \TXT{4} \TXT{\#} $\NT{Digits}[14]$ \OR
        \TXT{1} \TXT{5} \TXT{\#} $\NT{Digits}[15]$ \OR
        \TXT{1} \TXT{6} \TXT{\#} $\NT{Digits}[16]$}

\grruleindex{Digits[k]}{$\NT{Digits}[k]$}
       {$\NT{Digit}[k]$ \OR
        $\NT{Digits}[k]$ \ifStd \OPT{DigitSeparator} \fi $\NT{Digit}[k]$}

\ifStd
\grrule{DigitSeparator}
       {\TXT{_}}
\fi

\grruleoneofthefirst{Digit[k]}{$\NT{Digit}[k]$}{$k$}
       {\TXT{0 1 2 3 4 5 6 7 8 9 Aa Bb Cc Dd Ee Ff}}
\end{rules}
An integer literal without an explicit radix should thus be composed of decimal digits
and will be interpreted arithmetically as a decimal numeral as described in
\S\ref{section:numeral-to-integer}.

If an explicit radix is given, it consists of a decimal numeral between \TXT{2} and
\TXT{16} followed by a \TXT{\#} character.  Suppose that that the decimal interpretation
of this numeral is $r$.  The characters following it should then
be drawn from the first $r$ ``extended digits'' \TXT{0}, \ldots,
\TXT{9}, \TXT{A}, \ldots, \TXT{F}, where the letters may alternatively be in lower case.
Its interpretation as a numeral in radix $r$ is given in \S\ref{section:radix-numeral-to-integer}.

A compile-time error occurs if an integer literal is too
large or too small to be representable as an \Erlang\ integer.

Examples of integer literals:
\ifOld
\begin{verbatim}
0   1499   -54   2#0010010   -8#377
10#1499   16#fa66   16#FA66
\end{verbatim}
\fi
\ifStd
\begin{verbatim}
0   1499   -54   2#0010010   -8#377
10#1499   16#fa66   16#FA66   1_234_567
\end{verbatim}
\fi
(Their values are 0, 1499, -54, 18, -255, 1499, 64\,102, 64\,102 and 1\,234\,567, respectively.)
\index{integer!literal|)}

\section{Float literals}

\label{section:float-literals}
\index{float!literal|(}

A float literal has five parts: an optional sign, a whole number part,
a decimal point\index{decimal point (of float literal)}, a fractional
part and an optional exponent part.  The exponent part, if present, is
indicated by \TXT{E}\index{E@\TXT{E} character} or
\TXT{e}\index{e@\TXT{e} character} and is followed by the
exponent\index{exponent (of float literal)} as a decimal numeral with
an optional sign.
\ifStd
As for integer numerals, between any two digits in the whole number
part, the fractional part and the exponent there may be an underscore\index{_ character@\T{_} character},
which is ignored.
\fi
\begin{rules}
\grrule{FloatLiteral}
       {\NT{DecimalLiteral} \TXT{.}\ \NT{DecimalLiteral} \OPT{ExponentPart}}

\grrule{ExponentPart}
       {\NT{ExponentIndicator} \OPT{Sign} \NT{DecimalLiteral}}

\grruleoneof{Sign}{\TXT{+ -}}

\grruleoneof{ExponentIndicator}{\TXT{E e}}
\end{rules}
(The rule for \NT{DecimalLiteral} appears in
\S\ref{section:integer-literals}.)  A compile-time error occurs if the
magnitude of a float literal is too large or too small to be
representable as an \Erlang\ float.

The interpretation of a float literal as a float is given in
\S\ref{section:numeral-to-float}.

Examples of float literals:
\begin{verbatim}
1.0   -5.1e6   5.1e+6   0.346E-20
\end{verbatim}
\index{float!literal|)}

\section{Character literals}

\label{section:char-literals}
\index{character!literal|(}

A character literal is indicated by \TXT{\$} followed either by a single
character \ifStd (which may not be whitespace) \fi or an escape sequence.
\begin{rules}
\grrule{CharLiteral}
       {\TXT{\$} \NT{CharLiteralChar}}

\grrule{CharLiteralChar}
       {\NT{InputCharacter} \ifStd but not \TXT{\char`\\} or \NT{WhiteSpace} \fi \OR
	\NT{EscapeSequence}}
\end{rules}
The escape sequences are described in \S\ref{section:escapes}.

Note that \TXT{\$} \ifStd may \else should \fi not be followed by
white space\index{white space!not after \TXT{\$}}
(\S\ref{section:white-space})\footnote{The reasons that \TXT{\$}
\ifStd may \else should \fi
not be followed by whitespace are (i) it is impossible to determine
safely from the printed representation of a program which character
\TXT{\$} followed by whitespace and/or a line break actually denotes,
and (ii) whitespace (especially at the end of a line) often gets
transformed by text processing or text transmission tools.} which
\ifStd must \else should \fi instead be expressed through an escape
sequence.

\ifStd
\ifDiff\footnote{Compatibility note: Whitespace may appear
after \TXT{\$} in \OldErlang.  Its appearance should generate warnings
in \NewErlang\ and cause a compile-time error in some future
version.}\fi
\fi

Examples of character literals:
\begin{verbatim}
$x   $R   $$   $\n   $\s   $\\   $\^T   $\125
\end{verbatim}
The values of these literals are the characters `x', `R', dollar, newline, space, backslash,
control-T and `U' (which has ASCII code eightyfive, which is denoted by the octal numeral
\T{125}).
\index{character!literal|)}

\section{String literals}

\label{section:string-literals}
\index{string!literal|(}

A string literal is enclosed in \T{"} characters\index{"" character@\T{""} character}.
\begin{rules}
\grrule{StringLiteral}
       {\TXT{"} \OPT{StringCharacters} \TXT{"}}

\grrule{StringCharacters}
       {\NT{StringCharacter} \OR
        \NT{StringCharacters} \NT{StringCharacter}}

\grrule{StringCharacter}
       {\NT{InputCharacter} but not \NT{ControlCharacter} or \TXT{\char`\\} or \TXT{"} \OR
        \NT{EscapeSequence}}
\end{rules}
The escape sequences are described in \S\ref{section:escapes}.

\ifStd
Note that line terminators\index{line terminator!not ``naked''} and
control characters\index{control character!not ``naked''} cannot
appear ``naked'' in a string literal (and thus they also cannot appear
as Unicode escapes such as
\T{\char`\\u000c}) but must be expressed through escape
sequences.  A compiler need not generate an error in this case but
should at least produce a warning.\ifDiff\footnote{Compatibility note:
Line terminators and control characters may appear in strings and
quoted atoms in \OldErlang.  Their appearance should generate warnings
in \NewErlang\ and cause a compile-time error in some future
version.}\fi
\fi

Examples of string literals:
\begin{verbatim}
"Fred"   ""   "\n"   "\e42n"   "Ludwig Van Beethoven"
\end{verbatim}
The second literal denotes an empty string.
\index{string!literal|)}

\section{Atom literals}

\label{section:atom-literals}
\index{atom!literal|(}

An \emph{atom literal}, which we will also refer to simply as an
\emph{atom}, is on one of two forms:
\begin{itemize}
\item An \emph{unquoted} atom\index{atom!unquoted} is a nonempty sequence of \Erlang\ letters, \Erlang\
digits and the ASCII character `\TXT{@}', where the first character must be a lowercase \Erlang\ letter.
Such an atom cannot consist of the same sequence of characters
as a keyword (\S\ref{section:keywords}) or an operator (\S\ref{section:operators}).
\item A \emph{quoted} atom\index{atom!quoted} is a sequence
of input characters and escape sequences enclosed in single quotes (\T{'})\index{' character@\T{'} character}.
As for string literals,  line terminators\index{line terminator!not ``naked''} and
control characters\index{control character!not ``naked''} cannot appear
``naked'' in a quoted atom.
\end{itemize}

\begin{rules}
\grrule{AtomLiteral}
       {\NT{AtomLiteralChars} but not a \NT{Keyword} or \NT{Operator} \OR
        \TXT{'} \OPT{QuotedCharacters} \TXT{'}}

\grrule{AtomLiteralChars}
       {\NT{ErlangLowercase} \OPT{NameChars}}

\grrule{NameChars}
       {\NT{NameChar} \OR
	\NT{NameChars} \NT{NameChar}}

\grrule{NameChar}
       {\NT{ErlangLetter} \OR
        \NT{ErlangDigit} \OR \TXT{@} \OR \TXT{_}}

\grrule{QuotedCharacters}
       {\NT{QuotedCharacter} \OR
        \NT{QuotedCharacters} \NT{QuotedCharacter}}

\grrule{QuotedCharacter}
       {\NT{InputCharacter} but not \NT{ControlCharacter} or \TXT{\char`\\} or \TXT{'} \OR
        \NT{EscapeSequence}}
\end{rules}
The escape sequences are described in \S\ref{section:escapes}.

\iffalse
An \emph{\Erlang\ letter} is a character for which the function
\T{char:is_alpha/1} returns \T{true}.  An \emph{\Erlang\ letter-or-digit}
is a character for which the function \T{char:is_alpha_num/1} returns
\T{true}.
\fi

Examples of atoms:
\ifOld
\begin{verbatim}
friday   tv@lf@rs@lj@re   one_2_three   '!%r@(\'.\ $\\[[#'
\end{verbatim}
\fi
\ifStd
\begin{verbatim}
friday   tv�lf�rs�lj@re   one_2_three   '!�r@(\'.\ �\\[[#'
\end{verbatim} % 182
Note that the second example contains non-ASCII characters (but which belong
to ISO 8859-1).
\fi
The fourth example shows several escape sequences and nonletters.

We say that the \emph{printname}\index{atom!printname} of an atom is
\begin{itemize}
\item the atom literal as such, in the case of an unquoted atom;
\item the sequence of \ifStd Unicode \fi \ifOld ASCII \fi characters resulting from decoding
of escape sequences and removal of the surrounding quotes,
in the case of a quoted atom.
\end{itemize}

Two atoms are the same if and only if they have the same printname.
For example, the atoms \T{foo} and \T{'foo'} are the same and so are the atoms \T{'foo
bar'}\ifStd, \T{'foo\char`\\u0020bar'} \fi and \T{'foo\char`\\040bar'}.
\ifStd
(Note that the first and second atom literals are identical already after
Unicode escape processing, cf.~\S\ref{section:lexical-translation}.)
\fi
\index{atom!literal|)}

\section{Variables}

\label{section:variables}
\index{variable|(}

A \emph{variable} is a nonempty sequence of \Erlang\ letters, \Erlang\
digits and the character `\TXT{@}', where the first character must be an uppercase \Erlang\ letter.
Since no keyword, operator or literal begins with an uppercase \Erlang\ letter,
there can be no ambiguity between them.

\begin{rules}
\grrule{Variable}
       {\TXT{_} \NT{NameChars} \OR
	\NT{ErlangUppercase} \OPT{NameChars}}
\end{rules}
Note that a \emph{single} underscore is not a \NT{Variable} but a
\NT{UniversalPattern} (cf.~\S\ref{section:universal-pattern}).

Examples of variables:
\begin{verbatim}
MostSignificantDigit   X   Best_guess   _Rest   LOUD
\end{verbatim}

It is recommended that compilers warn about a variable that has only a
binding occurrence (and thus no applied occurrences) unless the
variable begins with an underscore.
\index{variable|)}

\section{Universal pattern}

\label{section:universal-pattern}
\index{universal pattern|(}

A \emph{universal pattern} consists of a single underscore\index{_ character@\T{_} character}.

\begin{rules}
\grrule{UniversalPattern}
       {\TXT{_}}
\end{rules}
It is thus syntactically similar to a variable but may only appear in patterns
(cf.~\S\ref{section:pattern-matching}) where occurrences of
the universal pattern are like binding occurrences of distinct variables that have
no other occurrences.  Matching against the universal pattern thus always succeeds.
\index{universal pattern|)}

\section{Separated token sequences}

\label{section:full-stop}
\label{section:tokenseq}

In the final step of lexical processing, white space and comments are discarded.
At the same time, each occurrence of a single period\index{ .@\TXT{.} character} token
(i.e., `\TXT{.}', which is a separator) that is followed by white space or a comment
is replaced by a \NT{FullStop}\index{full stop}.  The final result is a sequence of tokens and
full stops where each full stop should be thought of as terminating the preceding
sequence of tokens.

\begin{rules}
\grrule{TerminatedTokens}
       {\OPT{TokenSequences}}

\grrule{TokenSequences}
       {\NT{TokenSequence} \OR
        \NT{TokenSequences} \NT{TokenSequence}}

\grrule{TokenSequence}
       {\NT{Tokens} \NT{FullStop}}

\grrule{Tokens}
       {\NT{Token} \OR
        \NT{Tokens} \NT{Token}}

\grrule{FullStop}
       {\TXT{.} \NT{WhiteSpace} \OR
        \TXT{.} \NT{Comment}}
\end{rules}


%
% %CopyrightBegin%
%
% Copyright Ericsson AB 2017. All Rights Reserved.
%
% Licensed under the Apache License, Version 2.0 (the "License");
% you may not use this file except in compliance with the License.
% You may obtain a copy of the License at
%
%     http://www.apache.org/licenses/LICENSE-2.0
%
% Unless required by applicable law or agreed to in writing, software
% distributed under the License is distributed on an "AS IS" BASIS,
% WITHOUT WARRANTIES OR CONDITIONS OF ANY KIND, either express or implied.
% See the License for the specific language governing permissions and
% limitations under the License.
%
% %CopyrightEnd%
%

\chapter{Types and terms}

\label{chapter:types-terms}

\section{Types in \Erlang}

\Erlang\ is a \emph{dynamically typed} language, which means that the
type of a variable or expression generally cannot be determined at
compile time.  The dynamic typing offers a high degree of flexibility
in that a variable can take on, for example, an integer in one
invocation but a list in another invocation.  This corresponds to
having union types in a statically typed language.  Some of the
advantages of polymorphic static typing can be achieved also for
well-structured \Erlang\ programs by adding type declarations and type
analysis \cite{erltc}.

Every \Erlang\ term belongs to exactly one of the types below.

The types in \Erlang\ can be divided into \emph{elementary types} and
\emph{compound types}.  A term of an elementary type never properly
contains an arbitrary \Erlang\ term and is said to be an
\emph{elementary term}.  A term of a compound type is said to be a
\emph{compound term} and has a number of
\emph{immediate subterms}\index{subterm!immediate}.

\index{type!elementary|(}
\index{term!elementary|(}
The elementary types in \Erlang\ are:
%\S\ref{section:new-records}
\begin{itemize}
\item Atoms (\S\ref{section:atoms}).
\ifStd
\item Characters (\S\ref{section:chars}).
\fi
\item Numbers (integers and floats) (\S\ref{section:numbers}).
\item Refs (\S\ref{section:refs}).
\item Binaries (\S\ref{section:binaries}).
\ifStd
\item Functions (\S\ref{section:functions}).
\fi
\item Process identifiers (\S\ref{section:pids}).
\item Ports (\S\ref{section:ports}).
\end{itemize}
\index{type!elementary|)}
\index{term!elementary|)}

\index{type!compound|(}
\index{term!compound|(}
The compound types in \Erlang\ are:
\begin{itemize}
\item Tuples (\S\ref{section:tuples}).
\item Lists and conses (\S\ref{section:lists}).
\ifstruct
\item Structs (\S\ref{section:structs}).
\fi
\end{itemize}
\index{type!compound|)}
\index{term!compound|)}

The BIFs for recognizing terms of a certain type are described in
\S\S\ref{section:recognizer-bifs}.

\section{Atoms}

\label{section:atoms}
\index{atom!Erlang type@\Erlang\ type|(}

The only distinguishing property of an atom is its
\emph{printname}\index{atom!printname}
(cf.~\S\ref{section:atom-literals}).  Two atoms are
equal\index{atom!equality} if and only if they have the same
printname.
\index{maxatomlength@\I{maxatomlength}|(}
The printname of an atom must have at most \I{maxatomlength} characters.
\ifOld
In \OldErlang, \I{maxatomlength} is 255.
\fi
\ifStd
For any implementation of \StdErlang,
\I{maxatomlength} must be at least $2^8-1 = 255$ and at most $2^{16}-1 = 65\,535$.
(The reason for the upper limit is a restriction in the external term format,
cf.~\S\ref{chapter:external-format}.)
This parameter is available through the BIF
\T{atom:max_length/0} (\S\ref{section:atom:maxlength0}).
\fi
\index{maxatomlength@\I{maxatomlength}|)}

The atoms \T{true}\index{true@\T{true}} and
\T{false}\index{false@\T{false}} are called
\emph{Boolean}\index{atom!Boolean}.  Thus, when we say that an
expression is Boolean, we mean that its value is a Boolean atom.  The
Boolean atoms are distinguished in that \Erlang\ provides four
operators acting only on them:
\begin{itemize}
\item The logical complement operator \T{not} (\S\ref{section:booleannot}).
\item The logical operators \T{and} (\S\ref{section:booleanand}), \T{or}
(\S\ref{section:booleanor}) and \T{xor} (\S\ref{section:booleanxor}).
\end{itemize}
Boolean atoms are also distinguished
\ifStd in \T{cond}, \T{all_true} and \T{some_true}
expressions (\S\ref{section:alltrue-exprs},
\S\ref{section:sometrue-exprs}, \S\ref{section:cond-expr}), and \fi
in the filters of list comprehensions
(\S\ref{section:list-comprehensions}).

\ifStd
Comparison for equality between two atoms should be $O(1)$, i.e, it
cannot be implemented by comparing the printnames.  (This can be
accomplished by ``interning''\index{atom!interning}, i.e., making sure
that each occurrence of an atom is represented internally by the same
value, for example through a hashtable that maps printnames to atoms
and is used each time an atom is to be obtained from its printname.
Because such a table grows each time a new atom is created, many
consider it bad programming style to write programs that may add new
atoms at runtime, e.g., through the BIFs \T{list:to_atom/1} and
\T{binary:to_term/1}.  Strings\index{string!use instead of atom}
[\S\ref{section:strings}] can often be used instead when the $O(1)$
comparison for equality is not needed.)
\else
Comparison for equality between two atoms is $O(1)$.  This is
accomplished by ``interning''\index{atom!interning}, i.e., making sure
that each occurrence of an atom is represented internally by the same
value through a hashtable that maps printnames to atoms and is used
each time an atom is to be obtained from its printname.  Because such
a table grows each time a new atom is created, many consider it bad
programming style to write programs that may add new atoms at runtime,
e.g., through the BIFs \T{list_to_atom/1} and \T{binary_to_term/1}.
Strings\index{string!use instead of atom} [\S\ref{section:strings}]
can often be used instead when the $O(1)$ comparison for equality is
not needed.
\fi

Atoms are recognized by the BIF
\ifStd \T{is_atom/1} \else \T{atom/1} \fi (\S\ref{section:recognizer-bifs}).

There are no operators acting specifically on atoms (but note the
Boolean operators above).

Atom literals are described in \S\ref{section:atom-literals}.

\Erlang\ BIFs relating to atoms are described in \S\ref{section:atom-bifs}.

%The memory areas holding the already interned atoms are typically
%garbage collected much less frequently than other areas.
\index{atom!Erlang type@\Erlang\ type|)}

\ifStd
\section{Characters}

\label{section:chars}
\index{character!Erlang type@\Erlang\ type|(}

The character type of \StdErlang\ corresponds to the Unicode character
set \cite{unicode}.  That is, the invertible mapping between the
characters and the character codes, which is the integer range
$[0,2^{16}-1]$, i.e., $[0,65535]$, is the Unicode character encoding.

Characters are recognized by the BIF \T{is_char/1}
(\S\ref{section:recognizer-bifs}).

There are no operators acting specifically on characters.

Character literals are described in \S\ref{section:char-literals}.

\Erlang\ BIFs relating to characters are described in \S\ref{section:char-module}.

\index{character!may be integer|(}
It is permitted (but discouraged) for an implementation to make
characters indistinguishable from the integers with the corresponding
character codes except that the recognizer BIF \T{is_char/1} must only
be true of characters and \T{is_integer/1} must only be true of
integers (\S\ref{section:recognizer-bifs}).\footnote{The purpose of
this relaxation is to allow implementations that can run legacy
\OldErlang\ applications.  There was no character type in \OldErlang\
--- integers were used to represent characters --- but the BIFs
\T{is_char/1} and
\T{is_integer/1} did not exist either so characters and integers will
be completely indistinguishable in such applications.} Some
consequences of this would be:
\begin{itemize}
\item The BIF \T{integer} recognizes characters
(\S\ref{section:recognizer-bifs}).
\item Any value (e.g., an operand or a BIF argument) which should be a
number may instead be a character and its arithmetic value will be the
character code.  (The resulting integer value may be further coerced
to a float.)
\item Any value (e.g., an operand or a BIF argument) which should be a
character may instead be an integer and its character will be the one
having the integer as its code.  If an integer is used that is not the
code of any character, the result is not defined.
\item An integer \TZ{I} and a character with code \TZ{I} must be
(exactly) equal (\S\ref{section:equality}).  The characters thus do
not precede the numbers in the term order
(\S\ref{section:term-order}), instead they are ordered together with
the numbers according to their character codes.
\end{itemize}
\index{character!may be integer|)}
\index{character!Erlang type@\Erlang\ type|)}
\fi

\section{Numbers}

\label{section:numbers}
\label{section:integers}
\label{section:floats}
\index{number!is integer or float|(}
\index{integer!Erlang type@\Erlang\ type|(}
\index{float!Erlang type@\Erlang\ type|(}

\Erlang\ has two numeric types: \emph{integers} and \emph{floats}.
The arithmetic operations permit arbitrary combinations of integer and
float op\-er\-ands, when meaningful.  We therefore describe both types
together.

\Erlang\ integers and floats are described in detail in
\S\ref{section:integer-type} and \S\ref{section:float-type},
respectively.

Integer literals are described in \S\ref{section:integer-literals}.
Float literals are described in \S\ref{section:float-literals}.

Numbers, integers and floats are recognized by the BIFs
\ifStd \T{is_number/1}, \T{is_integer/1} and \T{is_float/1}
\else \T{number/1}, \T{integer/1} and \T{float/1} \fi, respectively
(\S\ref{section:recognizer-bifs}).

\Erlang\ provides the following operators acting on numeric terms:
\begin{itemize}
\item The unary plus and minus operators \T{+}, \T{-}
(\S\ref{section:unaryplus}, \S\ref{section:unaryminus}).
\item The multiplicative operators
\T{*} (\S\ref{section:multiplication}),
\T{/} (\S\ref{section:floatdiv}),
\ifStd \T{//} (\S\ref{section:intdiv-f}), \fi
\T{div} (\S\ref{section:intdiv})\ifOld\ \fi
\ifStd, \T{mod} (\S\ref{section:intmod}) \fi
and \T{rem} (\S\ref{section:intrem})\ifOld.\fi

\item The addition operators \T{+} and \T{-} (\S\ref{section:additionops})
\item The signed shift operators \T{bsl} and \T{bsr} (\S\ref{section:shift}).
\item The unary bitwise complement operator \T{bnot}
(\S\ref{section:bitwisecomp}).
\item The integer bitwise operators \T{band} (\S\ref{section:bitwiseand}),
\T{bor} (\S\ref{section:bitwiseor}) and \T{bxor} (\S\ref{section:bitwiseor}).
\end{itemize}

\Erlang\ BIFs relating to numbers are described in
\ifOld\S\ref{section:number-bifs}\fi
\ifStd\S\ref{section:number-module}\fi.
\index{number!is integer or float|)}
\index{float!Erlang type@\Erlang\ type|)}

% JoB 0.7 Took it out again.
\iffalse
\subsection{Bytes}

\index{byte!subset of integer|(}
A byte is an integer having a value in the range $[0,255]$.

There are no operators acting specifically on bytes.

There are no BIFs relating particularly to bytes, although various
BIFs expect arguments to be bytes or list of bytes, or return such values.

The byte literals are simply a subset of the integer literals
\ifOld
(but note that octal character escapes can be useful for denoting bytes).
\fi
\index{byte!subset of integer|)}
\fi

\ifOld
\subsection{Characters}

\label{section:chars}
\index{character!subset of integer|(}
A character is an integer having a value in the range $[0,255]$ and is
thus a byte.

There are no operators acting specifically on characters.

The characters literals are a subset of the integer
literals\footnote{It is possible, but \emph{not} recommended, to use
an integer literal to denote a character.}, plus the character
literals described in \S\ref{section:char-literals}.
\index{character!subset of integer|)}
\fi
\index{integer!Erlang type@\Erlang\ type|)}

\section{Refs}

\label{section:refs}
\index{ref!Erlang type@\Erlang\ type|(}

Refs are terms for which the only meaningful operations are
obtaining a new ref and comparing two refs for equality.

When we describe operations (such as transformation to the external
term format, cf.~\S\ref{chapter:external-format}) we shall assume that
the internal representation of a ref \TZ{R} consists of three parts:
\begin{itemize}
\item \T{node[\Z{R}]}, which is the node on which \TZ{R} was created,
represented by an atom;
\item \T{creation[\Z{R}]}, a nonnegative integer that is the value of
\T{creation[\Z{N}]} for the node \TZ{N} on which \TZ{R} was created;
\item \T{ID[\Z{R}]}, a nonnegative integer which is a
``serial number'' for \TZ{R} on the node on which it was created.
\end{itemize}
Two refs are equal if all these parts are equal.
\ifOld
In \OldErlang\ \T{ID[\Z{R}]} is limited to XXX.  Thus the BIF \T{make_ref/0}
may eventually produce duplicate values.
\fi
\ifStd
If in an implementation there is an upper limit to \T{ID[\Z{R}]}, then
the number of possible refs created on a node is bounded and the BIF
\T{make_ref/0} may eventually produce duplicate values.  If this is
the case, then \I{refs\_bounded} is \B{true} for that implementation,
otherwise \I{refs\_bounded} is \B{false}.  If \I{refs\_bounded} is
\B{true}, then \I{numrefs} is the number of distinct values that
\T{ID[\Z{R}]} can have.  The values of \I{refs\_bounded} and
\I{maxrefs} (if applicable) are available through the BIFs
\T{ref:bounded/0} and \T{ref:max/0}.
\fi

There are no ref literals.

Refs are recognized by the BIF
\ifOld \T{reference/1} \fi
\ifStd \T{is_ref/1} \fi (\S\ref{section:recognizer-bifs}).

There are no operators acting specifically on refs.

\Erlang\ BIFs relating to refs are described in
\ifOld \S\ref{section:misc-bifs}.\fi
\ifStd \S\ref{section:ref-module}.\fi
\index{ref!Erlang type@\Erlang\ type|)}

\section{Binaries}

\label{section:binaries}
\index{binary!Erlang type@\Erlang\ type|(}

A \emph{binary} is a sequence of bytes\index{byte}, i.e., a sequence of
integers between 0 and 255.

\iffalse
It's \emph{raison d'\^{e}tre} is communication between processes,
where low-level communication takes place through sequences of bytes
and terms are to be communicated.  There are thus BIFs for translating
from arbitrary terms to binaries and vice versa.  An arbitrary term
$t$ can be sent from a process $p$ to a process $q$ as follows:
\begin{enumerate}
\item process $p$ obtains the binary $b$ representing $t$;
\item the binary $b$ is sent via a port connecting $p$ and $q$;
\item process $q$ retrieves the original term $t$ from $b$.
\end{enumerate}
The internal representation of terms may allow \emph{sharing} of
structure between identical subterms.  If so, sharing in $p$'s
representation of $t$ and sharing in $q$'s representation of $t$ are
totally independent of each other and so are the memory requirements
for representing $t$ in $p$ and in $q$.
\fi

There are no binary literals.

Binaries are recognized by the BIF
\ifStd\T{is_binary/1}\else\T{binary/1}\fi
(\S\ref{section:recognizer-bifs}).

There are no operators acting specifically on binaries.

\Erlang\ BIFs relating to binaries are described in
\S\ref{section:binary-bifs}.

\ifOld
\NOTE

The \OldErlang\ implementation has disjoint address spaces for its
processes and thus copy terms sent as messages
(\S\ref{section:messages}).  However, in typical applications binaries
may be very large and copying them would therefore be expensive.
Therefore the \OldErlang\ implementation has a single memory area for
all binaries residing on a node and uses indirect addressing.  That
is, a binary would be represented in the memory of a process by a
pointer into the common binary area together with information about
length.  When sending a binary in a message, only the local
information is copied, not the elements of the binary.  This also
implies that a binary can be split (cf.~the BIF \T{split_binary/2},
\S\ref{section:splitbinary2}) in constant time as no copying of the
elements is necessary.  Of course this arrangement complicates memory
management\index{memory management}, as the binary area must be
maintained separately.
\fi
\index{binary!Erlang type@\Erlang\ type|)}

\ifStd
\section{Functions}

\label{section:functions}
\index{function!Erlang type@\Erlang\ type|(}

A \emph{function} of arity $n$ is a term that can be \emph{applied}
to a sequence of $n$ terms.  Evaluating the application may cause
certain effects and may either never complete, complete abruptly
with some associated reason or complete normally with a result.

There are no function literals.  However, \T{fun} expressions, having
functions as their values, are described in
\S\ref{section:fun-exprs}.  Function declarations, described in
\S\ref{section:program-forms}, associate a function name with a
function in a certain module.

\iffalse
% Per talked me out of this.
An \Erlang\ system may represent functions transparently through some
other type, e.g., as tuples, in which case the recognizer BIF for that
type will be true also for tuples.  It is therefore not portable to
assume that functions are distinct from all other
types\index{function!possibly not distinct type}.  If an
implementation uses another type to represent functions, this should
be documented.
\fi

Functions are recognized by the BIF \T{is_function/1}
(\S\ref{section:recognizer-bifs}).

There are no operators acting specifically on functions.

\Erlang\ BIFs relating to functions are described in \S\ref{section:process-bifs}.
\index{function!Erlang type@\Erlang\ type|)}
\fi

\section{Process identifiers}

\label{section:pids}
\index{PID!Erlang type@\Erlang\ type|(}
\index{process|(}

An \Erlang\ \emph{process} is an entity that carries out the
evaluation of an application.  That particular evaluation is
identified by a distinct \emph{process identifier}, usually called
simply a \emph{PID}.  The PID must be used in order to send messages
to the process and when manipulating it (e.g., linking with it or
attempting to kill it).

PIDs are elementary terms and a PID can be created only by spawning a
process.  Spawning a process always yields a PID that is distinct from
all accessible PIDs.  (PIDs and refs are obviously similar and an
inefficient implementation of refs could indeed be obtained by letting
\T{make_ref/0} spawn a new process and use its PID.)

When a process completes, its PID is still a PID but it no longer
refers to a process so BIFs cannot use it.  The result or effect when
a BIF is given the PID of a completed process varies,
cf.~Section\ref{section:process-bifs}.

Processes are further described in Chapter~\ref{chapter:processes}.

When we describe operations (such as transformation to the external
term format, cf.~\S\ref{chapter:external-format}) we shall assume that
the internal representation of a PID \TZ{P} consists of three parts:
\begin{itemize}
\item \T{node[\Z{P}]}, which is the node on which \TZ{P} was spawned,
represented by an atom;
\item \T{creation[\Z{P}]}, a nonnegative integer that is the value of
\T{creation[\Z{N}]} for the node \TZ{N} on which \TZ{P} was spawned;
\item \T{ID[\Z{P}]}, a nonnegative integer which is a
``serial number'' for \TZ{P} on the node on which it was spawned.
\end{itemize}
Two PIDs are equal if all these parts are equal.
\ifOld
In \OldErlang\ \T{ID[\Z{P}]} is limited to XXX.  Thus the BIFs \T{spawn/3}, etc.,
may eventually produce duplicate values.
\fi
\ifStd
If in an implementation there is an upper limit to \T{ID[\Z{P}]}, then
the number of possible PIDs created on a node is bounded and the BIFs
\T{spawn/3}, etc., may eventually produce duplicate values.  If this
is the case, then \I{pids\_bounded} is \B{true} for that
implementation, otherwise \I{pids\_bounded} is \B{false}.  If
\I{pids\_bounded} is \B{true}, then \I{numpids} is the number of
distinct values that \T{ID[\Z{P}]} can have.  The values of
\I{pids\_bounded} and \I{maxpids} (if applicable) are available
through the BIFs \T{proc:bounded/0} and \T{proc:max/0}.
\fi

There are no PID literals.

PIDs are recognized by the BIF \ifStd\T{is_pid/1}\else\T{pid/1}\fi
(\S\ref{section:recognizer-bifs}).

There are no operators acting specifically on processes or PIDs.

\Erlang\ BIFs relating to processes are described in \S\ref{section:process-bifs}.
\index{PID!Erlang type@\Erlang\ type|)}
\index{process|)}

\section{Ports}

\label{section:ports}
\index{port!Erlang type@\Erlang\ type|(}

An \Erlang\ node\index{node!communication} communicates with resources
in the outside world (including the rest of the computer on which it
resides) through \emph{ports}.  Examples of such external resources
are files and non-\Erlang\ programs running on the same host.  An
external resource behaves much like an \Erlang\ process, although
interaction with it causes or is caused by events in the outside
world.

Each external resource is identified by an \Erlang\ term that is
referred to as a port.  When a port is created, it is connected
externally to an entity, which is either
\begin{itemize}
\item a recently spawned external process or recently opened driver
      (\S\ref{section:drivers});
\item a file.
\end{itemize}
Internally the port is connected to a process, which is originally the
process that opened the port.

A process communicates with an external resource through messages sent
to a port.  Any process can send messages to an external resource.
The process connected with a port will receive messages from the
resource.

When the external resource is depleted (i.e., the end of the file has
been reached, the external process has completed or the driver is
closed), the port is closed, corresponding to the termination of a
process.  A process can be linked to a port and it will then be
notified when the port is closed.

Ports are obviously similar to PIDs but do not allow all operations
that PIDs allow.

Ports are further described in \S\ref{chapter:more-about-ports}.

When we describe operations (such as transformation to the external
term format, cf.~\S\ref{chapter:external-format}) we shall assume that
the internal representation of a port \TZ{Q} consists of three parts:
\begin{itemize}
\item \T{node[\Z{Q}]}, which is the node on which \TZ{Q} was opened, represented by
an atom;
\item \T{creation[\Z{Q}]}, a nonnegative integer that is the value of
\T{creation[\Z{N}]} for the node \TZ{N} on which \TZ{Q} was opened;
\item \T{ID[\Z{Q}]}, a nonnegative integer which is a
``serial number'' for \TZ{Q} on the node on which it was opened.
\end{itemize}
Two ports are equal if all these parts are equal.
\ifOld
In \OldErlang\ \T{ID[\Z{Q}]} is limited to 256.  Thus the BIF \T{open_port/2}
may eventually produce duplicate values.  However, \T{open_port/2} will never
return a duplicate open port and the number of simultaneously open ports is
limited to 256.
\fi
\ifStd
If in an implementation there is an upper limit to \T{ID[\Z{Q}]}, then the number
of possible PIDs created on a node is bounded and the BIF \T{port:open/2}, etc.,
may eventually produce duplicate values.  If this is the case,
then \I{ports\_bounded} is \B{true} for that implementation, otherwise
\I{ports\_bounded} is \B{false}.  If \I{ports\_bounded} is \B{true}, then
\I{numports} is the number of distinct values that \T{ID[\Z{Q}]} can have.
The values of \I{ports\_bounded} and \I{maxports} (if applicable) are available
through the BIFs \T{port:bounded/0} and \T{port:max/0}.
\fi

There are no port literals.

Ports are recognized by the BIF \ifStd\T{is_port/1}\else\T{port/1}\fi
(\S\ref{section:recognizer-bifs}).

There are no operators acting specifically on ports.

\Erlang\ BIFs relating to ports are described in \S\ref{section:port-bifs}.
\index{port!Erlang type@\Erlang\ type|)}

\section{Tuples}

\label{section:tuples}
\index{tuple!Erlang type@\Erlang\ type|(}

A $k$-tuple, where $k\geq0$, is a mapping from the integers $1$,
\ldots, $k$ to \Erlang\ terms, which are its
immediate subterms\index{subterm!immediate}, or elements.
(There is exactly one $0$-tuple,
which is a void mapping.)  We say that the \emph{size} of such
a tuple is $k$.  The types of the $k$ terms are
independent.  A $k$-tuple can be used as a sequence of $k$ terms
where each term can be accessed through its index.

\index{maxtuplesize@\I{maxtuplesize}|(}
A tuple must have at most \I{maxtuplesize} elements.
\ifOld
In \OldErlang, \I{maxtuplesize} is 65535.
% 65535?
% 2^27?
% 2^28?
\fi
\ifStd
For any implementation of \StdErlang,
\I{maxtuplesize} must be at least $2^{16}-1 = 65\,535$ and at most
$2^{32}-1 = 4\,294\,967\,296$.
(The reason for the upper limit is a restriction in the packed term format,
cf.~\S\ref{chapter:external-format}.)
This parameters is available through the BIF
\T{tuple:max_size/0} (\S\ref{section:tuple:maxsize0}).
\fi
\index{maxtuplesize@\I{maxtuplesize}|)}

Tuple skeletons are described in \S\ref{section:tuple-skeletons}.
\index{tuple!literal|(}
A tuple literal is a tuple skeleton where all subexpressions are
themselves literals.
\index{tuple!literal|)}

The time for accessing a tuple element given the tuple and an index
(i.e., what is computed by the BIF \T{element/2}\index{element/2
BIF@\T{element/2} BIF})
\ifStd should be \else is \fi
$O(1)$, i.e., a constant-time operation.  The element update operation
--- obtaining a tuple that differs from a given one in exactly one
element (i.e., what is computed by the BIF
\T{setelement/3}\index{setelement/3 BIF@\T{setelement/3} BIF}) ---
\ifStd should be \else is \fi
$O(n)$, where $n$ is the number of elements of the tuple.
\ifOld
(A future version of \Erlang\ may have a different trade-off between
element access and element update.
\fi
\ifStd
(Implementations are not discouraged to explore internal
representations of tuples that make element update more efficient.
\fi
For example, reducing the time for element update to
$O(\log n)$ may justify increasing the time for element access to
$O(\log n)$.)

Tuples are recognized by the BIF \ifStd\T{is_tuple/1}\else\T{tuple/1}\fi
(\S\ref{section:recognizer-bifs}).

There are no operators acting specifically on tuples.

\Erlang\ BIFs relating to tuples are described in \S\ref{section:tuple-bifs}.

\subsection{Records}

\label{section:records}
\index{record!Erlang type@\Erlang\ type|(}

A record type \TZ{R} has a number of \emph{field names}\index{record!field name}.  A term of record
type \TZ{R} has a value for each of these fields.

A term of record type \TZ{R} is a tuple\index{record!is a tuple} which has one more element than the
number of fields of \TZ{R} and having the atom \TZ{R} as its first element.

Terms of a record type \TZ{R} are recognized by record guard tests
(\S\ref{section:record-guards}).

There are no operators acting specifically on records.

Record declarations are described in
\S\ref{section:record-declarations} and record expressions are
described in \S\ref{section:record-exprs}.
\index{record!Erlang type@\Erlang\ type|)}

\ifOld
\subsection{Functions}

\label{section:functions}
\index{function!Erlang type@\Erlang\ type|(}

A \emph{function} of arity $n$ is a term that can be \emph{applied}
to a sequence of $n$ terms.  Evaluating the application may cause
certain effects and may either never complete, complete abruptly
with some associated reason or complete normally with a result.

There are no function literals.  However, \T{fun} expressions, having
functions as their values, are described in
\S\ref{section:fun-exprs}.  Function declarations, described in
\S\ref{section:program-forms}, associate a function name with a
function in a certain module.

\index{function!not a distinct type|(}
In \OldErlang\ a function (i.e., the value of a \T{fun} expression)
is represented by a tuple, hence
the recognizer \T{tuple/1} returns \T{true} for a function.
It is not recommended to exploit this representation. 
\index{function!not a distinct type|)}

\Erlang\ BIFs relating to functions are described in \S\ref{section:process-bifs}.
\index{function!Erlang type@\Erlang\ type|)}
\fi
\index{tuple!Erlang type@\Erlang\ type|)}

\section{Lists and conses}

\label{section:lists}
\index{list|(}
\index{nil!Erlang type@\Erlang\ type|(}
\index{cons!Erlang type@\Erlang\ type|(}

\Erlang\ has a constant \T{[]}, which is called \emph{nil}.
\Erlang\ also has a term-forming binary operator \T{[$\cdots$|$\cdots$]}
called \emph{cons}.  The operands of cons are usually referred
to as the \emph{head} and the \emph{tail} of the resulting term and
are its immediate subterms\index{subterm!immediate}.

The arguments of cons can be any terms but the intended use of cons is
for forming \emph{lists}. (In any use of cons as a general pairing operator, a 2-tuple
[\S\ref{section:tuples}] could be used instead.)

Let us define which terms are \emph{lists}.
\begin{itemize}
\item Nil is an \emph{empty list} (thus having zero elements).
\item Cons applied to an arbitrary term and a list (with $k$ elements) is a list
(with $k+1$ elements).
\item There are no other lists than those constructed by
a finite number of applications of the two preceding rules.
\end{itemize}
A list thus represents a finite sequence.  As suggested by the use of
the cons operator, the properties of a linked representation should be
assumed.  Computing the cons operator takes $O(1)$ time and so does
obtaining the head and/or the tail of a consed term.  However,
obtaining an element at an arbitrary position of a list
\ifStd (e.g., through the BIF \T{list:nth/2} [\S\ref{section:list:nth2}]) \fi
takes $O(n)$ time, where $n$ is the index of the element to retrieve.
%(This is because a linked representation is expected.)

In addition to the nil constant and the cons operator, there are
additional list skeletons, described in
\S\ref{section:list-skeletons}, although for every list skeleton, there is an
equal term that is a composition of cons operators and nil constants.
\index{list!literal|(}
A list literal is a list skeleton in which all subexpressions are
themselves literals.
\index{list!literal|)}

\ifOld
Nil and conses are both recognized by the BIF \T{list/1} (\S\ref{section:recognizer-bifs}),
although the name is highly misleading.
\fi
\ifStd
Nil, conses and lists are recognized by the BIFs \T{is_null/1},
\T{is_cons/1} and \T{is_list/1}, respectively (\S\ref{section:recognizer-bifs}).
\fi

\Erlang\ provides the following operators acting on lists and conses:
\begin{itemize}
\item The list addition operator \T{++} (\S\ref{section:list-addition}).
\item The list subtraction operator \T{--} (\S\ref{section:list-subtraction}).
\end{itemize}
\Erlang\ BIFs relating to lists and conses are described in \S\ref{section:list-bifs}.
\index{nil!Erlang type@\Erlang\ type|)}
\index{cons!Erlang type@\Erlang\ type|)}

\subsection{Strings}

\label{section:strings}
\index{string|(}

A \emph{string} is a list of characters\index{character}
(\S\ref{section:chars}) and can be seen as representing a text.  Note
that a list is a string only if all its elements are characters.  It
follows that a cons is a string only if its head is a character and
its tail is a string.

String literals are described in \S\ref{section:string-literals}
(but note that list literals with character literals also denote strings).

\ifStd
Strings are recognized by the BIF \T{is_string/1}
(\S\ref{section:recognizer-bifs}).
\fi

There are no operators acting specifically on strings (but note the list
operators above).

\ifOld \Erlang\ BIFs converting from and to strings are described in
various sections of \S\ref{chapter:bifs}. \fi
\ifStd \Erlang\ BIFs relating to strings are described in
\S\ref{section:str-module}. \fi 

As strings are lists\index{string!is a list}, note that a string can
be used anywhere a list is expected (for example, as operand of a list
operator or as argument of a list BIF).
\index{string|)}

\subsection{Association lists}

\label{section:assocationlists}
\index{list!association|(}

An \emph{association list} is a list of 2-tuples.  For each
2-tuple we say that the first element is the
key\index{key!in an association list} and the second element is the
value\index{value!in an association list}.

Let \TZ{lst} be an association list
\begin{alltt}
[\{\(\Z{k}\sb{1}\),\(\Z{v}\sb{1}\)\},\{\(\Z{k}\sb{2}\),\(\Z{T}\sb{v}\)\},\tdots,\{\(\Z{k}\sb{n}\),\(\Z{v}\sb{n}\)\}]
\end{alltt}
and let $K$ be the set of keys in \TZ{lst}.  \TZ{lst} represents a mapping
which for each key $\TZ{k}\in K$ contains a pair $(\TZ{k},\TZ{v}_j)$ such that
$\TZ{k}=\TZ{k}_j$ and for all $i$, $1\leq i<j$, $\TZ{k}\neq\TZ{k}_i$.

When we write that a BIF returns an association list, the first element of
each 2-tuple in the returned list is always distinct.
\index{list!association|)}
\index{list|)}

\section{Relational and equality operators on terms}

\Erlang\ provides the following relational and equality operators, acting on a pair of terms,
each of any type.
\begin{itemize}
\item The comparison operators \T{<}, \T{=<}, \T{>} and
\T{>=} (\S\ref{section:relationalops}).
\item The (exact) equality operators \T{=:=} and \T{=/=}
(\S\ref{section:exactequationalops}).
\item The arithmetic equality operators \T{==} and \T{/=}
(\S\ref{section:coercingequationalops}).
\end{itemize}

\subsection{Coercion}

\label{section:coercion}
\index{coercion!to float|(}
\index{conversion!arithmetic|(}

Coercion is applied when computing some arithmetic operators (including
the arithmetic equality operators).

\index{coerce@$\mathit{toFloat}$|(}
The function $\I{toFloat}$ maps a number to a float as follows:
\iftrue
\begin{alignat*}{2}
\mathit{toFloat}(a) &= a && \qquad\text{if $a$ is a float;} \\
                    &= \mathit{cvt}_{\mathtt{integer}\rightarrow\mathtt{float}}(a) && \qquad\text{if $a$ is an integer.}
\end{alignat*}
\else
\[\I{toFloat}(a)=\begin{cases} a & \text{if $a$ is a float;} \\
\I{cvt}_{\mathtt{integer}\rightarrow\mathtt{float}}(a) & \text{if $a$ is an integer.}
\end{cases}\]
\fi
(${cvt}_{\mathtt{integer}\rightarrow\mathtt{float}}$ is as defined in LIA-1
\cite{lia-1}.)
\index{coerce@$\mathit{toFloat}$|)}

\index{coerce@$\mathit{coerce}$|(}
The function $\I{coerce}$ maps a pair of terms to a pair of terms as follows:
\iftrue
\begin{alignat*}{2}
\mathit{coerce}(a,b) &= (a,b) && \qquad\text{if $a$ or $b$ is not a number;} \\
                     &= (\I{toFloat}(a),\I{toFloat}(b)) && \qquad\text{if $a$ or $b$ is a float;} \\
                     &= (a,b) && \qquad\text{otherwise.}
\end{alignat*}
\else
\[\I{coerce}(a,b)=\begin{cases}
(a,b) & \text{if $a$ or $b$ is not a number;} \\
(\I{toFloat}(a),\I{toFloat}(b)) & \text{if $a$ or $b$ is a float;} \\
(a,b) & \text{otherwise.}
\end{cases}\]
\fi
\index{coerce@$\mathit{coerce}$|)}
\index{coercion!to float|)}
\index{conversion!arithmetic|)}

\section{Size of data structures}

\index{term!size|(}
\index{size@$\mathit{size}$|(}
The function $\mathit{size}$
gives a measure of the size of an \Erlang\ term \TZ{T} as
an integer.
\ifStd
It is expected that the memory needed for representing \TZ{T} in
an implementation
\fi
\ifOld
The memory needed for representing \TZ{T} in \OldErlang\
\fi
is $O(\mathit{size}(\TZ{T}))$ (this excludes shared information
such as the printname of an atom).
The measure is also used in this document
for expressing the rate of growth of operations such
as comparisons.
\begin{itemize}
\item If \TZ{T} is an atom, a fixnum (\S\ref{section:integer-type}),
a float, a ref, a PID or a port, then
\[\I{size}(\TZ{T}) = O(1).\]
\item If \TZ{T} is a bignum (\S\ref{section:integer-type}), then
\[\I{size}(\TZ{T}) = O(\log \Er[\TZ{T}]).\]
\item If \TZ{T} is a binary of $k$ bytes, then
\[\I{size}(\TZ{T}) = O(k).\]
\item If \TZ{T} is a cons with head $\TZ{T}_h$ and tail $\TZ{T}_t$, then
\[\I{size}(\TZ{T}) = O(1)+\mathit{size}(\TZ{T}_h)+\mathit{size}(\TZ{T}_h).\]
\item If \TZ{T} is a tuple with elements $\TZ{T}_1$, \ldots, $\TZ{T}_k$, or
a function with values $\TZ{T}_1$, \ldots, $\TZ{T}_k$
for the free variables, then
\[\I{size}(\TZ{T}) = O(1)+O(k)+\sum_{i=1}^k\mathit{size}(\TZ{T}_i).\]
\item If \TZ{T} is a function with values $\TZ{T}_1$, \ldots, $\TZ{T}_k$
for free variables, then
\[\I{size}(\TZ{T}) = O(1)+O(k)+\sum_{i=1}^k\mathit{size}(\TZ{T}_i).\]
\end{itemize}
We will allow ourselves to apply $\mathit{size}$ also to sets of terms
and sets of pairs of terms.
\begin{itemize}
\item If $t$ is a set of items $t_1$, \ldots, $t_k$, then
\[\I{size}(t) = O(1)+O(k)+\sum_{i=1}^k\mathit{size}(t_i).\]
\item If $t$ is a pair of items $a$ and $b$, then
\[\I{size}(t) = O(1)+\mathit{size}(a)+\mathit{size}(b).\]
\end{itemize}
\index{term!size|)}
\index{size@$\mathit{size}$|)}

\subsection{Equality between terms}

\label{section:equality}

\index{equality!exact|(}
(Exact) equality between \Erlang\ terms $a$ and $b$ is defined as follows:
\begin{itemize}
\item If $a$ and $b$ were the result of the same evaluation of an expression, then
they are equal.
\item Otherwise, if $a$ and $b$ are of different type, then they are not equal.
\item Otherwise, equality of $a$ and $b$ depends on the type of $a$ and $b$:
\begin{itemize}
\item \B{Atom}:  $a$ and $b$ are equal if and only
if they have the same printname.
\ifStd
\item \B{Character}:  $a$ and $b$ are equal if and only if they are identical.
\fi
\item \B{Integer}: $a$ and $b$ are equal if and only if $\Er[a]=\Er[b]$.
\item \B{Float}: $a$ and $b$ are equal if and only if $\Er[a]=\Er[b]$.
(As in all programming languages, it is unwise to trust equality for
floats, as imprecision due to rounding may lead to unexpected inequalities.)
\item \B{Ref}: $a$ and $b$ are not equal.
\item \B{Binary}: $a$ and $b$ are equal if and only
if they consist of identical sequences of bytes.
\ifStd
\item \B{Function}: if $a$ and $b$ were the results of (different occurrences of)
identical expressions, then equality is not defined; otherwise they are not equal.
\fi
\item \B{PID}: $a$ and $b$ are not equal.
\item \B{Port}: $a$ and $b$ are not equal.
\item \B{Cons}:
\begin{itemize}
\item If $a$ and $b$ are both empty, then they are equal.
\item Otherwise, if $a$ and $b$ are both nonempty, then they are equal if and only if
the heads of $a$ and $b$ are equal and the tails of $a$ and $b$ are equal.
\item Otherwise, $a$ and $b$ are not equal.
\end{itemize}
\item \B{Tuple}:
\begin{itemize}
\item If $a$ and $b$ have different size, then they are not equal.
\item Otherwise, if all corresponding elements of $a$ and $b$ are pairwise
equal, then $a$ and $b$ are equal.
\item Otherwise, $a$ and $b$ are not equal.
\end{itemize}
It follows that two records are equal
if, and only if, they are of the same type and all corresponding elements are
pairwise equal.

\ifOld
For tuples that represent functions the representation is such that
if $a$ and $b$ were the results of (different occurrences of)
identical expressions, then equality is not defined; otherwise they are not equal.
\fi
% \item \B{Struct}
\end{itemize}
\end{itemize}
\index{equality!exact|)}
\ifStd
As there is no portable way to represent functions in the external term format
(\S\ref{chapter:external-format}), equality is not at all defined for functions that
have been transformed to the external format and back again or sent as messages.
\fi
Due to how the representation of floats in the external term format\ifStd\else\
(\S\ref{chapter:external-format})\fi, equality is not at all defined for floats that
have been transformed to the external format and back again, or have been
sent as messages.

The time required for determining exact equality of two terms $\TZ{T}_1$ and
$\TZ{T}_2$ should be $\min(\mathit{size}(\TZ{T}_1),\mathit{size}(\TZ{T}_2))$.

\index{equality!arithmetic|(}
Arithmetic equality between \Erlang\ terms $a$ and $b$ is defined as follows:
\begin{itemize}
\item If $a$ and $b$ are numbers, then they are arithmetically equal
if and only if $a'$ is (exactly) equal to $b'$, where $(a',b')=\I{coerce}(a,b)$
(\S\ref{section:coercion}).
\item Otherwise, if $a$ and $b$ are both of elementary types, then they
are arithmetically equal if and only if they are (exactly) equal.
\item Otherwise, if $a$ and $b$ are both lists, then:
\begin{itemize}
\item If $a$ and $b$ are both empty, then they are arithmetically equal.
\item Otherwise, if $a$ and $b$ are both nonempty, then they are arithmetically
equal if and only if the heads of $a$ and $b$ are arithmetically equal and the
tails of $a$ and $b$ are arithmetically equal.
\item Otherwise, $a$ and $b$ are not arithmetically equal.
\end{itemize}
\item Otherwise, if $a$ and $b$ are both tuples of the same size, then
$a$ and $b$ are arithmetically equal if and only if all corresponding elements
of $a$ and $b$ are pairwise arithmetically equal.
\item Otherwise $a$ and $b$ are not arithmetically equal.
\end{itemize}
\index{equality!arithmetic|)}

\subsection{The term order}

\label{section:term-order}
\index{term!comparison|(}

The \emph{term order} of \Erlang\ terms, which we will write here as
\T{<}, is a order relation that \ifStd must satisfy \fi \ifOld satisfies \fi
the following criteria:

\begin{itemize}
\item It is transitive, i.e., if \T{$\Z{t}_1$ < $\Z{t}_2$} and
\T{$\Z{t}_2$ < $\Z{t}_3$}, then it must be the case that
\T{$\Z{t}_1$ < $\Z{t}_3$}.
%\item It is irreflexive, i.e., there can be no term \TZ{t} such that
%\T{\Z{t} < \Z{t}}.
\item It is asymmetric, i.e., there can be no terms $\TZ{t}_1$ and $\TZ{t}_2$
such that \mbox{\T{$\Z{t}_1$ < $\Z{t}_2$}} and \mbox{\T{\Z{t}$_2$ < \Z{t}$_1$}}.
(This implies that it is irreflexive, i.e., that
there can be no term \TZ{t} such that \T{\Z{t} < \Z{t}}.)
\item \ifStd With the exception of functions, it \fi
\ifOld It \fi
is an arithmetic total order relation, i.e., if
$\TZ{t}_1$ is not arithmetically equal to
$\TZ{t}_2$, then exactly one of \T{$\Z{t}_1$ < $\Z{t}_2$} and
\T{$\Z{t}_2$ < $\Z{t}_1$} holds, unless $\TZ{t}_1$ and $\TZ{t}_2$ are functions.

\item The terms are primarily ordered according to their type, in the following order:
numbers \T{<} \ifStd characters \T{<} \fi
atoms \T{<} refs \T{<} \ifStd functions \T{<} \fi
ports \T{<} PIDs \T{<} tuples \T{<}
empty list \T{<} conses \T{<} binaries.

\item Numbers are ordered arithmetically (so there is no distinction between integers
and floats in this ordering).  For example, \T{4.5 < 5 < 5.3}.

\ifStd
\item Characters are ordered according to their character codes.
For example, \T{\char`\$5 < \char`\$@ < \char`\$J < \char`\$b}.
\fi

\item Atoms are ordered lexicographically according to the codes of the characters in
the printnames.  For example, \T{'' < a < aaa < ab < b}.

\item If $\TZ{t}_1$ and $\TZ{t}_2$ are both refs, both PIDs or both ports, then
$\TZ{t}_1$ precedes $\TZ{t}_2$ if and only if
either
\begin{itemize}
\item \T{node($\Z{t}_1$)} precedes \T{node($\Z{t}_2$)}, or
\item \T{node($\Z{t}_1$)} equals \T{node($\Z{t}_2$)} and $\TZ{t}_1$ was created before $\TZ{t}_2$.
\end{itemize}

\ifStd
\item Functions are not ordered, except for equality as described above.
\fi

\item Tuples are ordered first by their size, then according to their elements lexicographically.
For example, \T{\{\} < \{a\} < \{aaa\} < \{xxx\} < \{aaa,xxx\} < \{xxx,aaa\}}.
(It follows that records are ordered first by their number of elements, then according to
their type, then according to their elements with fields compared in the order given by
$\mathit{record\_field}_{\TZm{R}}^{-1}$, where \TZ{R} is the record type.)
\ifOld
Functions are not ordered, except for equality as described above.
\fi

\item An empty list precedes a cons (and thus a nonempty list) and
conses are ordered first by their heads, then by their tails.
(Thus a longer list may precede a shorter list even though a shorter tuple
always precedes a longer tuple.)  For example,
\T{[] < [a|2] < [a|b] < [a] < [a,a] < [b]}.

\item Binaries are ordered first by their size, then according to their elements lexicographically.
(That is, the same as the order between tuples of integers.)
\end{itemize}
\index{term!comparison|)}

\section{Lifetime of data structures}

\label{section:life-time}
\index{term!life time of|(}

\index{term!identity|(}
We say that a term \emph{has identity} if it is
\begin{itemize}
\item a ref,
\ifStd\item a function,\fi
\item a PID,
\item a port, or 
\item a compound term (i.e., a tuple or a list) in which
some immediate subterm has identity.
\end{itemize}
An elementary term with identity is created by evaluating an
expression on a certain form. (For example, a ref is created by
evaluating an application of the BIF \T{make_ref/0}).  In this case,
each evaluation of such an expression creates a new term that can be
distinguished from all other such terms.  If an elementary term with
identity is embedded in a compound term, that compound term also has
identity.
\ifStd An implementation is not permitted to \fi
\ifOld \OldErlang\ will not \fi
share or copy terms
with identity except when such sharing or copying is implied by the
language semantics.

For a term with identity there is thus a definite moment of creation.
There is, however, no corresponding moment of destruction: the
lifetime of a term with identity is unbounded.
\index{term!identity|)}

\index{term!generic|(}
A term that does not have identity is said to be \emph{generic}.

For generic terms, the concept of lifetime is not meaningful at all
as there is no way in which two equal specimen of a generic term could
be distinguished.  If the value of an expression is a generic term, it
would be impossible to tell from two such terms whether they were the
result of the same evaluation or two separate evaluations.
(Evaluation of the expression might have side effects that would be
different for one or more evaluations but that is not the point here.)
An implementation is permitted to share or copy generic terms.

For example, when a literal \T{\{1,2,3\}} is evaluated more than once,
an implementation may let all evaluations return references to the
same (generic) tuple or let each evaluation return a new specimen of
such a term.
\index{term!generic|)}
\index{term!life time of|)}

\section{Memory management}

\label{section:memory-management}
\index{memory management|(}

In the previous section we noted that for generic terms, there is no
concept of lifetime and for terms with identity, the lifetime is
unbounded.

However,
\ifStd an implementation must \fi
\ifOld \OldErlang\ will \fi
keep track of all references to terms
(both generic terms and terms with identity) and when no references to
a specimen of a term remain, the memory occupied by the specimen must
eventually be reused.  There
\ifStd must be \fi
\ifOld are \fi
no ``memory leaks,'' i.e., memory
that is not part of the representation of terms that can be referenced
but is never reclaimed.

Reclamation of memory in a process (garbage collection) could occur
incrementally or in batches but
\ifStd must \fi
\ifOld will \fi
not cause violation of the scheduling policies described in
\S\ref{section:scheduling}.
\index{memory management|)}


%
% %CopyrightBegin%
%
% Copyright Ericsson AB 2017. All Rights Reserved.
%
% Licensed under the Apache License, Version 2.0 (the "License");
% you may not use this file except in compliance with the License.
% You may obtain a copy of the License at
%
%     http://www.apache.org/licenses/LICENSE-2.0
%
% Unless required by applicable law or agreed to in writing, software
% distributed under the License is distributed on an "AS IS" BASIS,
% WITHOUT WARRANTIES OR CONDITIONS OF ANY KIND, either express or implied.
% See the License for the specific language governing permissions and
% limitations under the License.
%
% %CopyrightEnd%
%

\chapter{Arithmetics}

\label{chapter:arithmetics}

\emph{We define the mathematical functions in terms of which the arithmetics of
\Erlang\ are defined.  This chapter depends significantly on the international
standard document ISO/IEC 10967-1 \cite{lia-1}, referenced in this
text as LIA-1\index{LIA-1}.}

\ifStd
\emph{This chapter also contains the information needed for a
LIA-1 language binding for \StdErlang.}
\fi

\section{Notation}

\label{section:notation-arith}

Let $\INTS$\index{Z@$\INTS$ (the integers)} be the set of mathematical integers,
$\REALS$\index{R@$\REALS$ (the reals)} the set of
real numbers and $\BOOLEANS$\index{B@$\BOOLEANS$ (the Booleans)} the set of Booleans, denoted by
\B{true}\index{true@\B{true}}
and \B{false}\index{false@\B{false}}.

There are four exceptional values\index{arithmetic!exceptional values}
that are not numbers but may be the
results of the LIA-1 functions defined in this chapter:
\B{integer\_overflow}\index{integer_overflow@\B{integer\_overflow}},
\B{floating\_overflow}\index{floating_overflow@\B{floating\_overflow}},
\B{underflow}\index{underflow@\B{underflow}} and
\B{undefined}\index{undefined@\B{undefined}}.

\ifStd
An implementation conforming to IEC 559\index{IEC 559} \cite{iec559} has three additional values
that are not numbers but may appear as input to floating-point arithmetic operations
and can be returned from such operations:
\B{not_a_number}\index{not_a_number@\B{not_a_number}},
\B{positive_infinity}\index{positive_infinity@\B{positive_infinity}}
and \B{negative_infinity}\index{negative_infinity@\B{negative_infinity}}.
\fi

The following definitions are restated from LIA-1\index{LIA-1}.
For $x\in\REALS$, the notation $\lfloor x\rfloor$ stands for the largest integer not greater than $x$:
\[\lfloor x\rfloor\in\INTS\text{\quad and\quad}x-1<\lfloor x\rfloor\leq x\]
and $\mathit{tr}(x)$ stands for the integer part of $x$ (truncated towards 0):
\begin{alignat*}{2}
\mathit{tr}(x) &= \lfloor x\rfloor && \qquad\text{if $x\geq0$;} \\
               &= -\lfloor-x\rfloor && \qquad\text{if $x<0$.}
\end{alignat*}

The following definitions are restated from the 1995 working draft of
the international standard document ISO/IEC 10967-2 \cite{lia-2},
referenced in this text as LIA-2\index{LIA-2}.

Let $S$ be a subset of $\REALS$, closed under (arithmetic) negation.
The following are four rounding functions\index{rounding function}
for mapping values of $\REALS$ into $S$.
\index{  floor@$\lfloor\cdot\rfloor_S$|(}
\index{  ceiling@$\lceil\cdot\rceil_S$|(}
\index{floor@$\mathit{floor}_S$|(}
\index{ceiling@$\mathit{ceiling}_S$|(}
\index{truncate@$\mathit{truncate}_S$|(}
\index{nearest@$\mathit{nearest}_S$|(}
Given any $x\in\REALS$,
\begin{alignat*}{2}
\lfloor x\rfloor_S &= \max \{\,z\in S \mid z \leq x\,\} \displaybreak[0]\\[\smallskipamount]
\lceil x\rceil_S &= \min \{\,z\in S \mid z \geq x\,\} \displaybreak[0]\\[\smallskipamount]
\mathit{truncate}_S(x) &= \lfloor x\rfloor_S && \qquad\text{if $x\geq0$;} \\
                       &= \lceil x\rceil_S && \qquad\text{if $x<0$.} \displaybreak[0]\\[\smallskipamount]
\mathit{nearest}_S(x) &= \lfloor x\rfloor_S && \qquad\text{if $|\lfloor x\rfloor_S - x| < |x - \lceil x\rceil_S|$;} \\
                      &= \lceil x\rceil_S && \qquad\text{if $|\lfloor x\rfloor_S - x| > |x - \lceil x\rceil_S|$;} \\
                      &= \text{$\lfloor x\rfloor_S$ or $\lceil x\rceil_S$} && \qquad\text{if $|\lfloor x\rfloor_S - x| = |x - \lceil x\rceil_S|$.}
\end{alignat*}
In addition it must hold that $\mathit{nearest}_S(-x)=-\mathit{nearest}_S(x)$.

When the subscript $S$ is omitted, $\INTS$ is assumed.

We may write $\mathit{floor}_S(x)$ for $\lfloor x\rfloor_S$ and
$\mathit{ceiling}_S(x)$ for $\lceil x\rceil_S$.

Note that
\begin{itemize}
\item $\mathit{floor}_S(x)$ rounds $x$ towards negative infinity,
\item $\mathit{ceiling}_S(x)$ rounds $x$ towards positive infinity,
\item $\mathit{truncate}_S(x)$ rounds $x$ towards zero and
\item $\mathit{nearest}_S(x)$ rounds $x$ to the nearest value in $S$.
\end{itemize}
\index{  floor@$\lfloor\cdot\rfloor_S$|)}
\index{  ceiling@$\lceil\cdot\rceil_S$|)}
\index{floor@$\mathit{floor}_S$|)}
\index{ceiling@$\mathit{ceiling}_S$|)}
\index{truncate@$\mathit{truncate}_S$|)}
\index{nearest@$\mathit{nearest}_S$|)}
When we write $[i,j]$\index{ interval closed@$[\cdot,\cdot]$},
where $i$ and $j$ are integers, we mean the
set $\{\,x\in\INTS \mid i\leq x\leq j\,\}$.
When we write $[i,j)$\index{ interval open@$[\cdot,\cdot)$},
where $i$ and $j$ are integers, we mean the
set $\{\,x\in\INTS \mid i\leq x< j\,\}$.

\section{The integer type}

\label{section:integer-type}
\index{integer!properties|(}

\ifStd
A \StdErlang\ implementation must provide at least one integer type that
conforms with LIA-1.  In this document that type is assumed to be the
only integer type.
\fi
\index{I@$I$|(}
The set of numbers that can be represented by the integer type is
called $I$ and is a subset of $\INTS$\index{Z@$\INTS$ (the integers)}.
LIA-1 requires $I$ to be characterized by four parameters:
\index{bounded@\I{bounded}|(}
\index{modulo@\I{modulo}|(}
\index{minint@\I{minint}|(}
\index{maxint@\I{maxint}|(}
\begin{textdisplay}
\begin{tabular}{@{}ll@{}}
$\I{bounded}\in\BOOLEANS$ & (whether the set $I$ is finite) \\
$\I{modulo}\in\BOOLEANS$ & (whether out-of-bounds results ``wrap'') \\
$\I{minint}\in I$ & (the smallest integer in $I$) \\
$\I{maxint}\in I$ & (the largest integer in $I$)
\end{tabular}
\end{textdisplay}
\ifStd
For the integer type of a \StdErlang\ implementation it is required
that \I{modulo} is \B{false}, while \I{bounded} may be either \B{true}
or \B{false}.
\begin{itemize}
\item If \I{bounded} is \B{false}, then $I=\INTS$ and the values of \I{minint} and
\I{maxint} are not meaningful.
\item If \I{bounded} is \B{true}, then
$I = \{\,x \in \INTS \mid \I{minint} \leq x \leq \I{maxint}\,\}$
where
$\I{maxint} \geq 2^{59}-1$ and either
$\I{minint} = -(\I{maxint}+1)$ or $\I{minint} = -\I{maxint}$.
\end{itemize}
\fi
\ifOld
For the integer type of \Erlang, \I{modulo} and \I{bounded} are \B{false}.
As \I{bounded} is \B{false}, $I=\INTS$ and the values of \I{minint} and
\I{maxint} are not meaningful.
\fi
\index{bounded@\I{bounded}|)}
\index{modulo@\I{modulo}|)}
\index{minint@\I{minint}|)}
\index{maxint@\I{maxint}|)}
\Erlang\ has three additional parameters:
\index{fixnum@\I{fixnum}|(}
\index{minfixnum@\I{minfixnum}|(}
\index{maxfixnum@\I{maxfixnum}|(}
\begin{textdisplay}
\begin{tabular}{@{}ll@{}}
$\I{fixnum}\in\BOOLEANS$ & (whether there are ``fixnums'') \\
$\I{minfixnum}\in I$ & (the smallest fixnum in $I$) \\
$\I{maxfixnum}\in I$ & (the largest fixnum in $I$)
\end{tabular}
\end{textdisplay}
\ifStd
If \I{fixnum} is \B{true}, then it must hold that
\[\I{minint} \leq \I{minfixnum} \leq \I{maxfixnum} \leq \I{maxint}.\]
\fi
\ifOld
\I{fixnum} is \B{true}, \I{minfixnum} is $-2^{27}$ and
\I{maxfixnum} is $2^{27}-1$.
\fi
\index{I@$I$|)}

\index{integer!fixnum|(}
\index{I f@$I_f$|(}
\ifStd If \I{fixnum} is \B{true}, let \fi
\ifOld Let \fi
$I_f = \{\,x\in I \mid \I{minfixnum} \leq x \leq \I{maxfixnum}\,\}$.
\ifStd
Otherwise, let $I_f = \emptyset$.
\fi
$I_f$ is
\ifStd meant to be \fi
the set of ``fixnums'', the representation of which can utilize the
most efficient representation of integers in the machine, typically occupying
one word of memory.
\index{I f@$I_f$|)}
\ifStd
The values of \I{minfixnum} and \I{maxfixnum} for an implementation in which
\I{fixnum} is \B{true} should be chosen
so that when both the operands and the result of an integer addition,
subtraction, multiplication or division are in $I_f$, it should be
possible to utilize the most efficient machine instructions for
computing the operation.  Typically one would expect $\I{minfixnum} =
-(\I{maxfixnum}+1)$ but this is not required. For example, an
implementation may have ``unsigned'' fixnums in which case
\I{minfixnum} and \I{maxfixnum} could be $0$ and $2^{28}-1$,
respectively.
\fi
\index{integer!fixnum|)}
\index{fixnum@\I{fixnum}|)}
\index{minfixnum@\I{minfixnum}|)}
\index{maxfixnum@\I{maxfixnum}|)}

\index{integer!bignum|(}
Let $I_b = I \setminus I_f$\index{I b@$I_b$}.
$I_b$ is the set of ``bignums'', the representation of which
may require arbitrary amounts of memory.
An implementation for which $I_b\neq\emptyset$
\ifOld, such as \Erlang,\fi\ is said to have bignums.
\index{integer!bignum|)}

\ifStd
As the parameter \I{modulo} is always \B{false} for the integer type,
it is not made available to programs.  The other parameters are
available through the BIFs
\T{integer:bounded/0}, \T{integer:min_fixnum/0},
\T{integer:max_fix\-num/0},
\T{integer:min_int/0} and
\T{integer:max_int/0} (\S\ref{section:integer-module}).
\fi
\index{integer!properties|)}

\section{Integer operations}

\label{section:integer-operations}
\index{integer!arithmetic operations|(}

Elsewhere in this specification we express the integer arithmetic
operations of \Erlang\ in terms of the following functions from LIA-1:
\begin{xxalignat}{2}
&\lefteqn{\mathit{add}_I : I\times I\to I\cup\{\B{integer\_overflow}\}} \\
&&&(x,y)\mapsto\text{the sum of $x$ and $y$} \displaybreak[0]\\[\smallskipamount]
&\lefteqn{\mathit{sub}_I : I\times I\to I\cup\{\B{integer\_overflow}\}} \\
&&& (x,y)\mapsto\text{the difference of $x$ and $y$} \displaybreak[0]\\[\smallskipamount]
&\lefteqn{\mathit{mul}_I : I\times I\to I\cup\{\B{integer\_overflow}\}} \\
&&& (x,y)\mapsto\text{the product of $x$ and $y$} \displaybreak[0]\\[\smallskipamount]
&\lefteqn{\mathit{div}_I : I\times I\to I\cup\{\B{integer\_overflow},\B{undefined}\}} \\
&&& (x,y)\mapsto\text{the quotient of $x$ and $y$} \displaybreak[0]\\[\smallskipamount]
&\mathit{rem}_I : I\times I\to I\cup\{\B{undefined}\} && (x,y)\mapsto\text{the remainder of $x$ and $y$} \displaybreak[0]\\[\smallskipamount]
&\mathit{mod}_I : I\times I\to I\cup\{\B{undefined}\} && (x,y)\mapsto\text{$x$ modulo $y$} \displaybreak[0]\\[\smallskipamount]
&\mathit{neg}_I : I\to I\cup\{\B{integer\_overflow}\} && (x)\mapsto\text{the (arithmetic) negation of $x$} \displaybreak[0]\\[\smallskipamount]
&\mathit{abs}_I : I\to I\cup\{\B{integer\_overflow}\} && (x)\mapsto\text{absolute value of $x$} \displaybreak[0]\\[\smallskipamount]
\ifStd
&\mathit{sign}_I : I\to I  && (x)\mapsto\text{the sign of $x$} \displaybreak[0]\\[\smallskipamount]
\fi
&\mathit{eq}_I : I\times I\to\BOOLEANS  && (x,y)\mapsto\text{$x$ equals $y$} \displaybreak[0]\\[\smallskipamount]
&\mathit{neq}_I : I\times I\to\BOOLEANS && (x,y)\mapsto\text{$x$ does not equal $y$} \displaybreak[0]\\[\smallskipamount]
&\mathit{lss}_I : I\times I\to\BOOLEANS && (x,y)\mapsto\text{$x$ is less than $y$} \displaybreak[0]\\[\smallskipamount]
&\mathit{leq}_I : I\times I\to\BOOLEANS && (x,y)\mapsto\text{$x$ is not greater than $y$} \displaybreak[0]\\[\smallskipamount]
&\mathit{gtr}_I : I\times I\to\BOOLEANS && (x,y)\mapsto\text{$x$ is greater than $y$} \displaybreak[0]\\[\smallskipamount]
&\mathit{geq}_I : I\times I\to\BOOLEANS && (x,y)\mapsto\text{$x$ is not less than $y$}
\end{xxalignat}
For each function, LIA-1 states a number of axioms.
\ifStd
A \StdErlang\ implementation must satisfy all of these with the added
restriction that $\mathit{modulo}=\B{false}$\index{modulo@\I{modulo}}
(\S\ref{section:integer-type}),
\index{minint@\I{minint}|(}
\index{maxint@\I{maxint}|(}
either $\mathit{minint}=-\mathit{maxint}$ or $\mathit{minint}=-(\mathit{maxint}+1)$
(when $\mathit{bounded}=\B{true}$\index{bounded@\I{bounded}}),
\index{minint@\I{minint}|)}
\index{maxint@\I{maxint}|)} and
$\mathit{mod}_I=\mathit{mod}_I^a$.
\StdErlang\ provides operators for both the pairs $\mathit{div}_I^f/\mathit{rem}_I^f$ and
$\mathit{div}_I^t/\mathit{rem}_I^t$.
\fi
\ifOld
In \OldErlang, $\mathit{modulo}=\B{false}$\index{modulo@\I{modulo}} (\S\ref{section:integer-type}),
$\mathit{bounded}=\B{false}$\index{bounded@\I{bounded}} and $\mathit{mod}_I=\mathit{mod}_I^a$.
For $\mathit{div}$ and $\mathit{rem}$ the pair $\mathit{div}_I^t/\mathit{rem}_I^t$ is provided.
\fi

For convenience we reproduce the
strengthened axioms of Section~5.1.3 of LIA-1 here:
\begin{alignat*}{2}
\mathit{add}_I(x,y) &= x+y && \qquad\text{if $x+y\in I$;} \\
                    &= \B{integer\_overflow} && \qquad\text{if $x+y\notin I$.} \displaybreak[0]\\[\smallskipamount]
\mathit{sub}_I(x,y) &= x-y && \qquad\text{if $x-y\in I$;} \\
                    &= \B{integer\_overflow} && \qquad\text{if $x-y\notin I$.} \displaybreak[0]\\[\smallskipamount]
\mathit{mul}_I(x,y) &= x*y && \qquad\text{if $x*y\in I$;} \\
                    &= \B{integer\_overflow} && \qquad\text{if $x*y\notin I$.} \displaybreak[0]\\[\smallskipamount]
\mathit{div}_I^f(x,y) &= \lfloor x/y\rfloor && \qquad\text{if $y\neq0$ and $\lfloor x/y\rfloor\in I$;} \\
                    &= \B{integer\_overflow} && \qquad\text{if $y\neq0$ and $\lfloor x/y\rfloor\notin I$;} \\
                    &= \B{undefined} && \qquad\text{if $y=0$.} \displaybreak[0]\\[\smallskipamount]
\mathit{rem}_I^f(x,y) &= x-(\lfloor x/y\rfloor*y) && \qquad\text{if $y\neq0$;} \\
                    &= \B{undefined} && \qquad\text{if $y=0$.} \displaybreak[0]\\[\smallskipamount]
\mathit{div}_I^t(x,y) &= \mathit{tr}(x/y) && \qquad\text{if $y\neq0$ and $\mathit{tr}(x/y)\in I$;} \\
                    &= \B{integer\_overflow} && \qquad\text{if $y\neq0$ and $\mathit{tr}(x/y)\notin I$;} \\
                    &= \B{undefined} && \qquad\text{if $y=0$.} \displaybreak[0]\\[\smallskipamount]
\mathit{rem}_I^t(x,y) &= x-(\mathit{tr}(x/y)*y) && \qquad\text{if $y\neq0$;} \\
                    &= \B{undefined} && \qquad\text{if $y=0$.} \displaybreak[0]\\[\smallskipamount]
\mathit{mod}_I^a(x,y) &= x-(\lfloor x/y\rfloor*y) && \qquad\text{if $y\neq0$;} \\
                    &= \B{undefined} && \qquad\text{if $y=0$.} \displaybreak[0]\\[\smallskipamount]
\mathit{neg}_I(x)   &= -x && \qquad\text{if $-x\in I$;} \\
                    &= \B{integer\_overflow} && \qquad\text{if $-x\notin I$.} \displaybreak[0]\\[\smallskipamount]
\mathit{abs}_I(x)   &= |x| && \qquad\text{if $|x|\in I$;} \\
                    &= \B{integer\_overflow} && \qquad\text{if $|x|\notin I$.} \displaybreak[0]\\[\smallskipamount]
\ifStd
\mathit{sign}_I(x)  &= 1 && \qquad\text{if $x>0$;} \\
                    &= 0 && \qquad\text{if $x=0$;} \\
                    &= -1 && \qquad\text{if $x<0$.} \displaybreak[0]\\[\smallskipamount]
\fi
\mathit{eq}_I(x,y)  &= \B{true} && \qquad\text{if $x=y$;} \\
                    &= \B{false} && \qquad\text{if $x\neq y$.} \displaybreak[0]\\[\smallskipamount]
\mathit{neq}_I(x,y) &= \B{true} && \qquad\text{if $x\neq y$;} \\
                    &= \B{false} && \qquad\text{if $x=y$.} \displaybreak[0]\\[\smallskipamount]
\mathit{lss}_I(x,y) &= \B{true} && \qquad\text{if $x<y$;} \\
                    &= \B{false} && \qquad\text{if $x\geq y$.} \displaybreak[0]\\[\smallskipamount]
\mathit{leq}_I(x,y) &= \B{true} && \qquad\text{if $x\leq y$;} \\
                    &= \B{false} && \qquad\text{if $x>y$.} \displaybreak[0]\\[\smallskipamount]
\mathit{gtr}_I(x,y) &= \B{true} && \qquad\text{if $x>y$;} \\
                    &= \B{false} && \qquad\text{if $x\leq y$.} \displaybreak[0]\\[\smallskipamount]
\mathit{geq}_I(x,y) &= \B{true} && \qquad\text{if $x\geq y$;} \\
                    &= \B{false} && \qquad\text{if $x<y$.}
\end{alignat*}

\iffalse
When these functions are used in other chapters, it will sometimes be
convenient to write applications of them to \Erlang\ integer terms,
rather than to the integers that these terms denote.  Similarly we
will sometimes use the result of one of the mathematical functions as
if it were an \Erlang\ integer term.
\fi
\index{integer!arithmetic operations|)}

\section{The floating-point type}

\label{section:float-type}
\index{float!properties|(}

\ifStd
A \StdErlang\ implementation must provide at least one floating-point
type that conforms with LIA-1.  In this document that type is assumed
to be the only floating-point type.
\fi
\index{F@$F$|(}
The set of numbers that can be
represented by the float type is called $F$ and is a finite subset of
$\REALS$\index{R@$\REALS$ (the reals)}.  $F$ may contain both
normalized and denormalized values (cf.~Section~5.2 of LIA-1);
$F_N$ stands for the set of normalized values in $F$.

LIA-1 requires $F$ to be characterized by five parameters:
\index{r@$r$|(}
\index{p@$p$|(}
\index{emin@$\mathit{emin}$|(}
\index{emax@$\mathit{emax}$|(}
\index{denorm@$\mathit{denorm}$|(}
\begin{textdisplay}
\begin{tabular}{@{}ll@{}}
$p\in\INTS$ & (the precision of $F$) \\
$r\in\INTS$ & (the radix of $F$) \\
$\I{emin}\in\INTS$ & (the smallest exponent of $F$) \\
$\I{emax}\in\INTS$ & (the largest exponent of $F$) \\
$\I{denorm}\in\BOOLEANS$ & (whether $F$ contains denormalized values)
\end{tabular}
\end{textdisplay}

\ifOld
\OldErlang\ directly uses the float representation of the underlying
processor so these parameters are not defined.  It is guaranteed,
however, that the size of a float is at least 64 bits.
\iffalse
$r$ is $2$,
$p$ is XXX,
$\mathit{emin}$ is XXX,
$\mathit{emax}$ is XXX and
$\mathit{denorm}$ is \B{true}.
\fi\fi

\ifStd
These parameters are available through the BIFs
\T{float:precision/0}, \T{float:radix/0},
\T{float:e_min/0},
\T{float:e_max/0} and \T{float:de\-norm/0}, respectively (\S\ref{section:float-module}).

In addition to the requirements of Section~5.2 of LIA-1, the following must
hold for the floating-point type of a \StdErlang\ implementation
(from Section~A.5.2.0.2 of LIA-1):
\begin{itemize}
\item $r$ should be even,
\item $r^{p-1}\geq 10^6$,
\item $\I{emin}-1 \leq k*(p-1)$ with $k\geq 2$ and $k$ as large an integer as practical,
\item $\I{emax} > k*(p-1)$, and
\item $-2 \leq (emin-1) + emax \leq 2$.
\end{itemize}

\index{r@$r$|)}
\index{p@$p$|)}
\index{emin@$\mathit{emin}$|)}
\index{emax@$\mathit{emax}$|)}
\index{denorm@$\mathit{denorm}$|)}

The range and granularity of $F$ are characterized by four derived
constants:
\index{fmax@$\mathit{fmax}$|(}
\index{fmin@$\mathit{fmin}$|(}
\index{fminN@$\mathit{fmin}_N$|(}
\index{epsilon@$\mathit{epsilon}$|(}
\begin{textdisplay}
\begin{tabular}{@{}ll@{}}
$\mathit{fmax}\in F$ & (the value of largest magnitude in $F$) \\
$\mathit{fmin}\in F$ & (the value of smallest magnitude in $F$) \\
$\mathit{fmin}_N\in F$ & (the smallest normalized value in $F$) \\
$\mathit{epsilon}\in F$ & (the largest relative representation error in $F_N$)
\end{tabular}
\end{textdisplay}
\index{fmax@$\mathit{fmax}$|)}
\index{fmin@$\mathit{fmin}$|)}
\index{fminN@$\mathit{fmin}_N$|)}
\index{epsilon@$\mathit{epsilon}$|)}
\index{F@$F$|)}

\iffalse
\ifOld
For \OldErlang,
$\mathit{fmax}$ is XXX,
$\mathit{fmin}$ is XXX,
$\mathit{fmin}_N$ is XXX and
$\mathit{epsilon}$ is XXX.
\fi\fi

These derived constants are available through the BIFs
\T{float:f_max/0}, \T{float:f_min/0},
\T{float:f_min_norm/0} and
\T{float:epsilon/0}, respectively (\S\ref{section:float-module}).
\fi
\index{float!properties|)}

\section{Floating-point operations}

\label{section:float-operations}
\index{float!arithmetic operations|(}

Elsewhere in this specification we express the floating-point
arithmetic operations of \Erlang\ in terms of the following functions
from LIA-1:
\begin{xxalignat}{2}
&\lefteqn{\mathit{add}_F : F\times F\to F\cup\{\B{floating\_overflow},\B{underflow}\}} \\
 &&& (x,y)\mapsto\text{the sum of $x$ and $y$} \displaybreak[0]\\[\smallskipamount]
&\lefteqn{\mathit{sub}_F : F\times F\to F\cup\{\B{floating\_overflow},\B{underflow}\}} \\
 &&& (x,y)\mapsto\text{the difference of $x$ and $y$} \displaybreak[0]\\[\smallskipamount]
&\lefteqn{\mathit{mul}_F : F\times F\to F\cup\{\B{floating\_overflow},\B{underflow}\}} \\
 &&& (x,y)\mapsto\text{the product of $x$ and $y$} \displaybreak[0]\\[\smallskipamount]
&\lefteqn{\mathit{div}_F : F\times F\to F\cup\{\B{floating\_overflow},\B{underflow},\B{undefined}\}} \\
 &&& (x,y)\mapsto\text{the quotient of $x$ and $y$} \displaybreak[0]\\[\smallskipamount]
&\mathit{neg}_F : F\to F && (x)\mapsto\text{the (arithmetic) negation of $x$} \displaybreak[0]\\[\smallskipamount]
&\mathit{abs}_F : F\to F && (x)\mapsto\text{absolute value of $x$} \displaybreak[0]\\[\smallskipamount]
&\mathit{sign}_F : F\to I && (x)\mapsto\text{the sign of $x$}\displaybreak[0]\\[\smallskipamount]
&\lefteqn{\mathit{exponent}_F : F\to F\cup\{\B{undefined}\}} \\
 &&& (x)\mapsto\text{the exponent of $x$} \displaybreak[0]\\[\smallskipamount]
&\mathit{fraction}_F : F\to F && (x)\mapsto\text{$x$ scaled by a power of $r$ to the range $[1/r,1)$} \displaybreak[0]\\[\smallskipamount]
&\lefteqn{\mathit{scale}_F : F\times I\times F\to F\cup\{\B{floating\_overflow},\B{underflow}\}} \\
 &&& (x,n)\mapsto\text{the product of $x$ and $r^n$} \displaybreak[0]\\[\smallskipamount]
&\lefteqn{\mathit{succ}_F : F\to F\cup\{\B{floating\_overflow}\}} \\
 &&& (x)\mapsto\text{the least float greater than $x$} \displaybreak[0]\\[\smallskipamount]
&\lefteqn{\mathit{pred}_F : F\to F\cup\{\B{floating\_overflow}\}} \\
 &&& (x)\mapsto\text{the greatest float less than $x$} \displaybreak[0]\\[\smallskipamount]
&\lefteqn{\mathit{ulp}_F : F\to F\cup\{\B{underflow},\B{undefined}\}} \\
 &&& (x)\mapsto\text{the value of one unit in the last place of $x$} \displaybreak[0]\\[\smallskipamount]
&\mathit{trunc}_F : F\times I\to F && (x)\mapsto\text{$x$ with the low $p-n$ digits zeroed} \displaybreak[0]\\[\smallskipamount]
&\lefteqn{\mathit{round}_F : F\times I\to F\cup\{\B{floating\_overflow}\}} \\
 &&& (x)\mapsto\text{$x$ rounded to $n$ significant digits} \displaybreak[0]\\[\smallskipamount]
&\mathit{intpart}_F : F\to F && (x)\mapsto\text{the integer part of $x$} \displaybreak[0]\\[\smallskipamount]
&\mathit{fractpart}_F : F\to F && (x)\mapsto\text{$x$ minus the integer part of $x$} \displaybreak[0]\\[\smallskipamount]
&\mathit{eq}_F : F\times F\to\BOOLEANS && (x,y)\mapsto\text{$x$ equals $y$} \displaybreak[0]\\[\smallskipamount]
&\mathit{neq}_F : F\times F\to\BOOLEANS && (x,y)\mapsto\text{$x$ does not equal $y$} \displaybreak[0]\\[\smallskipamount]
&\mathit{lss}_F : F\times F\to\BOOLEANS && (x,y)\mapsto\text{$x$ is less than $y$} \displaybreak[0]\\[\smallskipamount]
&\mathit{leq}_F : F\times F\to\BOOLEANS && (x,y)\mapsto\text{$x$ is not greater than $y$} \displaybreak[0]\\[\smallskipamount]
&\mathit{gtr}_F : F\times F\to\BOOLEANS && (x,y)\mapsto\text{$x$ is greater than $y$} \displaybreak[0]\\[\smallskipamount]
&\mathit{geq}_F : F\times F\to\BOOLEANS && (x,y)\mapsto\text{$x$ is not less than $y$}
\end{xxalignat}
\ifStd
(LIA-1 specifies that the type of $\mathit{sign}_F$ should be $F\to F$ but in \Erlang\
the resulting integer can be automatically coerced to a float and an integer is more
useful than a float for dispatching upon.)
\fi
For each function, LIA-1 states a number of axioms.
\ifStd A \StdErlang\ implementation must satisfy all of these axioms. \fi
For convenience we reproduce the axioms of Section~5.2.7 of LIA-1 here:
\begin{alignat*}{2}
\mathit{add}_F(x,y) &= \mathit{result}_F(\mathit{add}_F^*(x+y),\mathit{rnd}_F) && \displaybreak[0]\\[\smallskipamount]
\mathit{sub}_F(x,y) &= \mathit{add}_F(x,-y) && \displaybreak[0]\\[\smallskipamount]
\mathit{mul}_F(x,y) &= \mathit{result}_F(x*y,\mathit{rnd}_F) && \displaybreak[0]\\[\smallskipamount]
\mathit{div}_F(x,y) &= \mathit{result}_F(x/y,\mathit{rnd}_F) && \qquad\text{if $y\neq0$;} \\
                    &= \B{undefined} && \qquad\text{if $y=0$.} \displaybreak[0]\\[\smallskipamount]
\mathit{neg}_F(x)   &= -x &&  \displaybreak[0]\\[\smallskipamount]
\mathit{abs}_F(x)   &= |x| && \displaybreak[0]\\[\smallskipamount]
\mathit{sign}_F(x)  &= 1 && \qquad\text{if $x>0$;} \\
                    &= 0 && \qquad\text{if $x=0$;} \\
                    &= -1 && \qquad\text{if $x<0$.} \displaybreak[0]\\[\smallskipamount]
\mathit{exponent}_F(x) &= \lfloor(\log_r|x|\rfloor+1 && \qquad\text{if $x\neq0$;} \\
                    &= \B{undefined} && \qquad\text{if $x=0$.} \displaybreak[0]\\[\smallskipamount]
\mathit{fraction}_F(x) &= x/r^{\mathit{exponent}_F(x)} && \qquad\text{if $x\neq0$;} \\
                    &= \B{undefined} && \qquad\text{if $x=0$.} \displaybreak[0]\\[\smallskipamount]
\mathit{scale}_F(x,n) &= \mathit{result}_F(x*r^n,\mathit{rnd}_F) && \displaybreak[0]\\[\smallskipamount]
\mathit{succ}_F(x)  &= \min\{\,z\in F \mid z > x\,\} && \qquad\text{if $x\neq\mathit{fmax}$;} \\
                    &= \B{floating\_overflow} && \qquad\text{if $x=\mathit{fmax}$.} \displaybreak[0]\\[\smallskipamount]
\mathit{pred}_F(x)  &= \max\{\,z\in F \mid z < x\,\} && \qquad\text{if $x\neq-\mathit{fmax}$;} \\
                    &= \B{floating\_overflow} && \qquad\text{if $x=-\mathit{fmax}$.} \displaybreak[0]\\[\smallskipamount]
\mathit{ulp}_F(x)   &= r^{e_F(x)-p} && \qquad\text{if $x\neq0$ and $r^{e_F(x)-p}\in F$;} \\
                    &= \B{underflow} && \qquad\text{if $x\neq0$ and $r^{e_F(x)-p}\notin F$;} \\
                    &= \B{undefined} && \qquad\text{if $x=0$.} \displaybreak[0]\\[\smallskipamount]
\mathit{trunc}_F(x) &= \lfloor x/r^{e_F(x)-n}\rfloor*r^{e_F(x)-n} && \qquad\text{if $x\geq0$;} \\
                    &= -\mathit{trunc}_F(-x,n) && \qquad\text{if $x<0$.} \displaybreak[0]\\[\smallskipamount]
\mathit{round}_F(x) &= \mathit{rn}_F(x,n) && \qquad\text{if $|\mathit{rn}_F(x,n)|\leq\mathit{fmax}$;} \\
                    &= \B{floating\_overflow} && \qquad\text{if $|\mathit{rn}_F(x,n)|>\mathit{fmax}$.} \displaybreak[0]\\[\smallskipamount]
\mathit{intpart}_F(x) &= \mathit{sign}_F(x)*\lfloor|x|\rfloor && \displaybreak[0]\\[\smallskipamount]
\mathit{fractpart}_F(x) &= x-\mathit{intpart}_F(x) && \displaybreak[0]\\[\smallskipamount]
\mathit{eq}_F(x,y)  &= \B{true} && \qquad\text{if $x=y$;} \\
                    &= \B{false} && \qquad\text{if $x\neq y$.} \displaybreak[0]\\[\smallskipamount]
\mathit{neq}_F(x,y) &= \B{true} && \qquad\text{if $x\neq y$;} \\
                    &= \B{false} && \qquad\text{if $x=y$.} \displaybreak[0]\\[\smallskipamount]
\mathit{lss}_F(x,y) &= \B{true} && \qquad\text{if $x<y$;} \\
                    &= \B{false} && \qquad\text{if $x\geq y$.} \displaybreak[0]\\[\smallskipamount]
\mathit{leq}_F(x,y) &= \B{true} && \qquad\text{if $x\leq y$;} \\
                    &= \B{false} && \qquad\text{if $x>y$.} \displaybreak[0]\\[\smallskipamount]
\mathit{gtr}_F(x,y) &= \B{true} && \qquad\text{if $x>y$;} \\
                    &= \B{false} && \qquad\text{if $x\leq y$.} \displaybreak[0]\\[\smallskipamount]
\mathit{geq}_F(x,y) &= \B{true} && \qquad\text{if $x\geq y$;} \\
                    &= \B{false} && \qquad\text{if $x<y$.}
\end{alignat*}
The functions are expressed in terms of a number of helper functions and sets:
\begin{itemize}
\item The set $F^*$\index{F*@$F^*$} is $F$ extended with all numbers having the same precision as numbers
in $F_N$ but larger magnitude.
\item The approximate addition function $\mathit{add}_F^* : F\times F\to\REALS$ is as
described in Section~5.2.4 of LIA-1, ideally but not necessarily such that
$\mathit{add}_F^*(x,y)=x+y$.
\item The functions $e_F : \REALS\to\INTS$ and $\mathit{rn}_F : F\times\INTS\to F^*$
are as described in Section~5.2.7 of LIA-1, i.e., they are defined such that
\begin{alignat*}{2}
e_F(x) &= \lfloor\log_r|x|\rfloor+1 && \qquad\text{if $|x|\geq\mathit{fmin}_N$;} \\
       &= \mathit{emin} && \qquad\text{if $|x|<\mathit{fmin}_N$.} \displaybreak[0]
\end{alignat*}
and
\[\mathit{rn}_F(x,n) = \mathit{sign}_F(x)*\lfloor|x|/r^{e_F(x)-n}+1/2\rfloor*r^{e_F(x)-n}\]
\item $\mathit{rnd}_F : \REALS\to F^*$ is the rounding function\index{rounding function}
used when taking an exact
result in $\REALS$ to a $p$-digit approximation.  It must satisfy the requirements
stated in Sections~5.2.5 and~5.2.8 of LIA-1.  There are two derived constants characterizing
$\mathit{rnd}_F$:
\begin{itemize}
\item $\I{rnd\_error}\in\REALS$ is the maximum rounding error in ulps;
\item $\I{rnd\_style}\in\{\B{nearest},\B{truncate},\B{other}\}$ is the rounding style.
\end{itemize}
\ifOld
For \OldErlang, $\mathit{rnd\_error}$ is XXX and $\mathit{rnd\_style}$
is XXX.
\fi
\ifStd
They are available at run time through the BIFs
\T{float:rnd_error/0} and
\T{float:rnd_style/0} (\S\ref{section:float-module}).
\fi
\item $\mathit{result}_F : \REALS\times(\REALS\to F^*)\to
F\cup\{\B{floating\_overflow},\B{underflow}\}$ is the function described in Section~5.2.6
of LIA-1.  The value of $\mathit{result}_F(x,\mathit{rnd})$, where $x\in\REALS$ and
\I{rnd} is a rounding function in $\REALS\to F^*$, is the result of applying the
rounding function to $x$, provided that the result is in $F$.
\ifStd
If $|x|$ is greater
than zero but less than \I{fmin}, $\mathit{result}_F(x,\mathit{rnd})$ can always be
\B{underflow} but may be $\mathit{rnd}(x)$ if \I{denorm} is \B{true} and no
denormalization loss occurs at $x$.
A \StdErlang\ implementation for which
\I{denorm} is \B{true} shall document how this choice is made.
\fi
\ifOld
If $|x|$ is greater
than zero but less than \I{fmin}, then XXX???
\fi
\end{itemize}
\index{float!arithmetic operations|)}

\section{Conversions}

\label{section:conversions}
\index{conversion!arithmetic|(}

\ifStd
In a \StdErlang\ implementation with more than one integer type or more
than one floating-point type, conversion functions between integer
types and between floating-point types shall be provided that satisfy
the requirements in Section~5.3 of LIA-1.
\fi

Let $\mathit{nearest}_{I\to F} : I\to F\cup\{\B{floating_overflow}\}$
be defined as
\[\mathit{nearest}_{I\to F}(x) = \mathit{result}_F(x,\mathit{nearest}_F),\]
where $\mathit{result}_F$ is as in \S\ref{section:float-operations}
(cf.~Section~5.2.6 of LIA-1) and $\mathit{nearest}_F$ is a
rounding-to-nearest function for $F$ (\S\ref{section:notation-arith}).

Define the following four functions:
\begin{alignat*}{2}
\mathit{floor}_{F\to I}(x) &= \mathit{floor}_Z(x) && \qquad\text{if $\mathit{floor}_Z(x)\in I$;} \\
       &= \B{integer\_overflow} && \qquad\text{if $\mathit{floor}_Z(x)\notin I$.} \displaybreak[0]\\[\smallskipamount]
\mathit{ceiling}_{F\to I}(x) &= \mathit{ceiling}_Z(x) && \qquad\text{if $\mathit{ceiling}_Z(x)\in I$;} \\
       &= \B{integer\_overflow} && \qquad\text{if $\mathit{ceiling}_Z(x)\notin I$.} \displaybreak[0]\\[\smallskipamount]
\mathit{truncate}_{F\to I}(x) &= \mathit{truncate}_Z(x) && \qquad\text{if $\mathit{truncate}_Z(x)\in I$;} \\
       &= \B{integer\_overflow} && \qquad\text{if $\mathit{truncate}_Z(x)\notin I$.} \displaybreak[0]\\[\smallskipamount]
\mathit{nearest}_{F\to I}(x) &= \mathit{nearest}_Z(x) && \qquad\text{if $\mathit{nearest}_Z(x)\in I$;} \\
       &= \B{integer\_overflow} && \qquad\text{if $\mathit{nearest}_Z(x)\notin I$.}
\end{alignat*}
Note that the four functions $\mathit{floor}_Z$, $\mathit{ceiling}_Z$,
$\mathit{truncate}_Z$ and $\mathit{nearest}_Z$ meet the requirements
in Section~5.3 of LIA-1 for being used as the rounding function
$\mathit{rnd}_{F\to I}$ in a conversion function $\mathit{cvt}_{F\to
I}$.
\index{conversion!arithmetic|)}

\section{Representation and evaluation}

\label{section:eval-notation}

The purpose of this section is to define notation and terminology that is
used in the subsequent chapters.

\index{Re@$\Re[\cdot]$|(}
\begin{itemize}
\item If $i\in I$, then $\Re[i]$ is the \Erlang\ integer representing $i$.
\item If $f\in F$, then $\Re[f]$ is the \Erlang\ float representing $f$. 
\item If $b\in\BOOLEANS$, i.e., \B{true} or \B{false}, then $\Re[b]$
is the \Erlang\ Boolean atom representing $b$.  That is, 
$\Re[\B{true}]=\T{true}$ and $\Re[\B{false}]=\T{false}$.
\item If $x$ is one of %the exceptional values
\B{integer\_overflow},
\B{floating\_overflow}, \B{underflow} and \B{undefined}, then
$\Re[x]$ is
\ifStd
the \Erlang\ atom given by Table~\ref{table:arith-exits}.
\fi
\ifOld
the \Erlang\ atom \T{badarith}.
\fi
\end{itemize}
\index{Re@$\Re[\cdot]$|)}

\ifStd
\begin{table}
\begin{center}
\index{integer_overflow@\B{integer\_overflow}|(}
\index{floating_overflow@\B{floating\_overflow}|(}
\index{underflow@\B{underflow}|(}
\index{undefined@\B{undefined}|(}
\index{integer_overflow exit signal@\T{integer\_overflow} exit signal|(}
\index{float_overflow exit signal@\T{float\_overflow} exit signal|(}
\index{float_underflow exit signal@\T{float\_underflow} exit signal|(}
\index{undefined_arith exit signal@\T{undefined_arith} exit signal|(}
\begin{tabular}{@{}ll@{}}
\hline
Exceptional value & Exit reason \\
\hline
\B{integer\_overflow} & \T{integer_overflow} \\
\B{floating\_overflow} & \T{float_overflow} \\
\B{underflow} & \T{float_underflow} \\
\B{undefined} & \T{undefined_arith} \\
\hline
\end{tabular}
\caption{Exit reasons for exceptional values\index{arithmetic!exceptional values}.}
\label{table:arith-exits}
\index{integer_overflow@\B{integer\_overflow}|)}
\index{floating_overflow@\B{floating\_overflow}|)}
\index{underflow@\B{underflow}|)}
\index{undefined@\B{undefined}|)}
\index{integer_overflow exit signal@\T{integer\_overflow} exit signal|)}
\index{float_overflow exit signal@\T{float\_overflow} exit signal|)}
\index{float_underflow exit signal@\T{float\_underflow} exit signal|)}
\index{undefined_arith exit signal@\T{undefined_arith} exit signal|)}
\end{center}
\end{table}
\fi

\index{Re-1@$\Er[\cdot]$|(}
Similarly,
\begin{itemize}
\item If \TZ{I} is an Erlang\ integer, then $\Er[\TZ{I}]\in I$ is the
integer it represents.
\item If \TZ{F} is an Erlang\ float, then $\Er[\TZ{F}]\in F$ is the
real number it represents.
\item If \TZ{B} is an Erlang\ Boolean atom, then $\Er[\TZ{B}]\in\BOOLEANS$ is the
Boolean it represents.
That is, $\Er[\T{true}]=\B{true}$ and $\Er[\T{false}]=\B{false}$.
\end{itemize}
\index{Re-1@$\Er[\cdot]$|)}

We have a notation for writing the result of evaluating an expression:
\begin{itemize}
\item When we write $\TZ{E}\RETURNS\TZ{T}$\index{  returns@$\RETURNS$}
we state that evaluating the expression
\TZ{E} completes normally and that its value is the term \TZ{T}.  (If the
environment is relevant, it is stated elsewhere.)
\item When we write $\TZ{E}\EXITSWITH\TZ{R}$\index{  exitswith@$\EXITSWITH$}
we state that evaluating the expression
\TZ{E} exits with reason \TZ{R}.
\end{itemize}

\iffalse
\label{section:arith-shorthand}

When describing the evaluation of arithmetic expressions in
\S\ref{chapter:expressions-evaluation} and the BIFs in \S\ref{chapter:bifs},
it will be convenient to use also the following shorthand.

\index{apply!arithmetic operation|(}

When we write ``apply $f_I$ to $\TZ{v}_1$'' where
$f_I$ is one of the integer operations in
\S\ref{section:integer-operations} and $\TZ{v}_1$ is an \Erlang\ term, we mean
\begin{itemize}
\item If $\TZ{v}_1$ is an \Erlang\ integer, then
\begin{itemize}
\item If $f_I(\Er[\TZ{v}_1])\in I$, then $\Re[f_I(\Er[\TZ{v}_1])]$
is the result.
\item If $f_I(\Er[\TZ{v}_1])=\B{integer\_overflow}$, then exit with \T{integer_overflow}.
\item If $f_I(\Er[\TZ{v}_1])=\B{undefined}$, then exit with \T{undefined}.
\end{itemize}
\item If $\TZ{v}_1$ is not an \Erlang\ integer, exit with \T{\badarith}.
\end{itemize}
Similarly for ``apply $f_I$ to $\TZ{v}_1$ and $\TZ{v}_2$'', where both
$\TZ{v}_1$ and $\TZ{v}_2$ must be \Erlang\ integers.

When we write ``apply $f_F$ to $\TZ{v}_1$'' where
$f_F$ is one of the floating-point operations in
\S\ref{section:float-operations} and $\TZ{v}_1$ is an \Erlang\ term, we mean
\begin{itemize}
\item If $\TZ{v}_1$ is an \Erlang\ float, then
\begin{itemize}
\item If $f_F(\Er[\TZ{v}_1])\in F$, then $\Re[f_F(\Er[\TZ{v}_1])]$
is the result.
\item If $f_F(\Er[\TZ{v}_1])=\B{floating\_overflow}$, then exit with \T{floating_overflow}.
\item If $f_F(\Er[\TZ{v}_1])=\B{undefined}$, then exit with \T{undefined}.
\end{itemize}
\item If $\TZ{v}_1$ is not an \Erlang\ float, exit with \T{\badarith}.
\end{itemize}
Similarly for ``apply $f_F$ to $\TZ{v}_1$ and $\TZ{v}_2$'', where both
$\TZ{v}_1$ and $\TZ{v}_2$ must be \Erlang\ floats.

When we write ``apply $f$ to $\TZ{v}_1$''\ where $f_I$ is one of the
integer operations in
\S\ref{section:integer-operations}, $f_F$ is one of the floating-point operations in
\S\ref{section:float-operations} and $\TZ{v}_1$ is an \Erlang\ term, we mean
\begin{itemize}
\item If $\TZ{v}_1$ is an \Erlang\ integer, then apply $f_I$ to
$\TZ{v}_1$.
\item If $\TZ{v}_1$ is an \Erlang\ float, then apply $f_F$ to
$\TZ{v}_1$.
\item Otherwise, exit with \T{badarg}.
\end{itemize}
Similarly for ``apply $f$ to $\TZ{v}_1$ and $\TZ{v}_2$'', where either both
$\TZ{v}_1$ and $\TZ{v}_2$ must be \Erlang\ integers or both must be
\Erlang\ floats.

When we write ``apply $f_{I\to F}$ to $\TZ{v}_1$''\ where
$f_{I\to F}$ is one of the integer to float conversion operations in
\S\ref{section:conversions} and $\TZ{v}_1$ is an
\Erlang\ term, we mean
\begin{itemize}
\item If $\TZ{v}_1$ is an \Erlang\ integer, then
\begin{itemize}
\item If $f_{I\to F}(\Er[\TZ{v}_1])\in F$, then $\Re[f_{I\to F}(\Er[\TZ{v}_1])]$ is the result.
\item If $f_{I\to F}(\Er[\TZ{v}_1])=\B{floating\_overflow}$, then exit with \T{floating_overflow}.
\end{itemize}
\item If $\TZ{v}_1$ is an \Erlang\ float, then $\TZ{v}_1$ is the result.
\item If $\TZ{v}_1$ is neither an \Erlang\ integer, nor a float, exit with \TZ{badarg}.
\end{itemize}

When we write ``apply $f_{F\to I}$ to $\TZ{v}_1$''\ where
$f_{F\to I}$ is one of the float to integer conversion operations in
\S\ref{section:conversions} and $\TZ{v}_1$ is an
\Erlang\ term, we mean
\begin{itemize}
\item If $\TZ{v}_1$ is an \Erlang\ float, then
\begin{itemize}
\item If $f_{F\to I}(\Er[\TZ{v}_1])\in I$, then $\Re[f_{F\to I}(\Er[\TZ{v}_1])]$ is the result.
\item If $f_{F\to I}(\Er[\TZ{v}_1])=\B{integer\_overflow}$, then exit with \T{integer_overflow}.
\end{itemize}
\item If $\TZ{v}_1$ is an \Erlang\ integer, then $\TZ{v}_1$ is the result.
\item If $\TZ{v}_1$ is neither an \Erlang\ integer, nor a float, exit with \TZ{badarg}.
\end{itemize}

The result and exit refer to the result and exit of the expression
being described or the application of the BIF being described.
\index{apply!arithmetic operation|)}
\fi % iffalse

\section{Notification}

\index{arithmetic!notification|(}

Whenever the evaluation of the translated \Erlang\ expressions causes
one of the functions defined in the preceding sections of this chapter
to return an exceptional value, the evaluation of the translated
\Erlang\ expression exits with
\ifStd
a reason that depends on the
exceptional value, cf.~Table~\ref{table:arith-exits}.
\fi
\ifOld
reason \T{badarith}.
\fi

A \ifStd\T{try} expression (\S\ref{section:try-expr}) \else
\T{catch} expression (\S\ref{section:catch}) \fi
can be used for handling
the exception in accordance with Section~6.1.1 of LIA-1.

The usual mechanisms for handling of abnormal completion ensure that
in absence of a \ifStd\T{try} \else\T{catch} \fi expression that catches the arithmetic
exception, the process will complete abruptly; any exit signals sent
to linked processes will propagate information about the arithmetic
exception (\S\ref{section:exit-signals}).

\index{arithmetic!notification|)}

\iffalse
% !!! I would like to have this one included.
\section{Translation}

\index{arithmetic!translation|(}
\emph{In this section I will tell how arithmetic expressions are expected to
be translated into combinations of LIA-1 operations.  Pretty easy since there
is only one integer and one float type.  The dynamic typing might make it
a little more messy.}
\index{arithmetic!translation|)}
\fi

\ifStd
\section{Conformity with IEC 559}

\label{section:arith-iec559}

\index{IEC 559|(}
\index{iec_559@\I{iec\_559}|(}
This specification does not specify a language binding for that part
of IEC~559 (a.\,k.\,a.\ ANSI/IEEE Std.\ 754-1985) \cite{iec559} that
is not covered by LIA-1, except that there is a parameter
$\mathit{iec\_559}\in\BOOLEANS$ that should be \B{true} in an
implementation that fully conforms to IEC~559 and \B{false} elsewhere.

The parameter $\mathit{iec\_559}$ is available to programs through the BIF \linebreak
\T{float:iec_559/0} (\S\ref{section:float:iec5590}).
\index{IEC 559|)}
\index{iec_559@\I{iec\_559}|)}
\fi

\section{Conversion to and from numerals}

We will define conversions from $I$ and $F$ to canonical decimal numerals.
Below we will only discuss decimal numerals and thus omit ``decimal''.
We will also define conversions from decimal numerals to $I$ or $F$.

\subsection{Integer to decimal numeral}

\label{section:integer-to-numeral}
\index{integer!conversion to numeral|(}
Given an integer $i\in I$, the canonical numeral is defined recursively as follows.
\begin{itemize}
\item If $0\leq i<10$, then the canonical numeral for $i$ is the decimal digit with value $i$.
\item If $i<0$, then the canonical numeral for $i$ is a minus sign (`$-$') followed by the
canonical numeral for $-i$.
\item If $i\geq10$, then the canonical numeral for $i$ is the canonical numeral
for $\lfloor i/10\rfloor$ followed by the decimal digit with value $i\bmod10$.
\end{itemize}
\index{integer!conversion to numeral|)}

\subsection{Decimal numeral to integer}

\label{section:numeral-to-integer}
\index{integer!conversion from numeral|(}
Given a sequence of characters, its interpretation as a decimal integer numeral
(if any) is defined as follows:
\begin{itemize}
\item If the sequence consists of a minus sign followed by decimal digits $d_1$, \ldots, $d_k$,
then it denotes $-i$, where $i$ is the integer denoted by the digits $d_1$, \ldots, $d_k$.
\item If the sequence consists of a plus sign followed by decimal digits $d_1$, \ldots, $d_k$,
then it denotes the same integer as that denoted by the digits $d_1$, \ldots, $d_k$.
\item If the sequence consists only of decimal digits $d_1$, \ldots, $d_k$, then it
denotes the integer $\sum_{j=1}^k d_j\cdot10^{k-j}$.
\item Otherwise, it does not denote any integer.
\end{itemize}
Note that this also defines the meaning of a
\NT{DecimalLiteral}\index{DecimalLiteral@\NT{DecimalLiteral}}:
if the sequence of characters
that it constitutes denotes $i\in I$, then the \NT{DecimalLiteral} denotes $\Re[i]$.
\index{integer!conversion from numeral|)}

\subsection{Numeral with radix to integer}

\label{section:radix-numeral-to-integer}
\index{integer!conversion from numeral|(}
In this context, `A' and `a' are digits
with value 10, `B' and `b' are digits with value 11, etc., up to `F' and `f' which are digits
with value 15.
Given a radix $r$ and a sequence of characters, its interpretation as an
integer numeral in radix $r$ (if any) is defined as follows:
\begin{itemize}
\item If the sequence consists of a minus sign followed by digits $d_1$, \ldots, $d_k$,
then it denotes $-i$, where $i$ is the integer denoted by the digits $d_1$, \ldots, $d_k$.
\item If the sequence consists of a plus sign followed by decimal digits $d_1$, \ldots, $d_k$,
then it denotes the same integer as that denoted by the digits $d_1$, \ldots, $d_k$.
\item If the sequence consists only of digits $d_1$, \ldots, $d_k$ where each digit $d_j$,
$1\leq j\leq k$, has a value that is less than $r$, then it
denotes the integer $\sum_{j=1}^k d_j\cdot r^{k-j}$.  
\item Otherwise, it does not denote any integer.
\end{itemize}
Note that this also defines the meaning of a
\NT{ExplicitRadixLiteral}\index{ExplicitRadixLiteral@\NT{ExplicitRadixLiteral}}:
consider the sequence of characters that it constitutes.
Let $r$ be the integer denoted by the digits before the `\#' character.  If
the concatenation of
the sign (if any) with the digits following the `\#' character denotes
$i\in I$ in radix $r$, then the \NT{ExplicitRadixLiteral} denotes $\Re[i]$.
\index{integer!conversion from numeral|)}

\subsection{Float to numeral}

\label{section:float-to-numeral}
\index{float!conversion to numeral|(}
The canonical numeral for a float $f\in F$ is defined recursively as follows.
\begin{itemize}
\item If $f<0$, then the canonical numeral for $f$ is a minus sign (`$-$') followed by the
canonical numeral for $-f$.
\item If $f=0$, then the canonical numeral for $f$ is the digit `0' followed by a decimal
point (`$.$'), the digit `0', the letter `e' and the digit `0'.
\item If $f>0$, then let
$w$ and $e$ be the unique integers such that $f=w\cdot10^e$ and $w \bmod 10\neq 0$.
The canonical numeral for $f$ is the canonical numeral for $w$ with a decimal point
 (`$.$') inserted after the first digit, followed by the letter `e', followed
by the canonical numeral for $e+\lfloor\log_{10}w\rfloor$.
(Obviously $w>0$ and the canonical numeral for $w$ thus begins with a digit.)
\end{itemize}
\index{float!conversion to numeral|)}

\subsection{Numeral to float}

\label{section:numeral-to-float}
\index{float!conversion from numeral|(}
Given a sequence of characters, the number it denotes (if any) is defined as follows.
The sequence of characters should consist of
\begin{itemize}
\item a (possibly signed) decimal numeral, which we will call the whole number part;
\item a decimal point;
\item an unsigned decimal numeral, which we will call the fractional part;
\item optionally an `E' or `e' followed by a (possibly signed) decimal numeral,
which we will call the exponent.
\end{itemize}
If it does not, then the sequence of characters does not denote any number.

Let $e'$ be the number of digits in the fractional part,
let $w$ be the integer denoted by the concatenation of the whole number part and
the fractional part, and let $e$ be the integer denoted by the exponent, or
zero if there was no exponent part
(\S\ref{section:numeral-to-integer}).

The number in $F$ denoted by the sequence of characters is then
\[\mathit{result}_F(w\cdot10^{e-e'},\mathit{rnd}_F).\]

Note that this also defines the meaning of a
\NT{FloatLiteral}\index{FloatLiteral@\NT{FloatLiteral}}: if the sequence of characters
that it constitutes denotes $f\in F$, then the \NT{FloatLiteral} denotes $\Re[f]$.
\index{float!conversion from numeral|)}


%
% %CopyrightBegin%
%
% Copyright Ericsson AB 2017. All Rights Reserved.
%
% Licensed under the Apache License, Version 2.0 (the "License");
% you may not use this file except in compliance with the License.
% You may obtain a copy of the License at
%
%     http://www.apache.org/licenses/LICENSE-2.0
%
% Unless required by applicable law or agreed to in writing, software
% distributed under the License is distributed on an "AS IS" BASIS,
% WITHOUT WARRANTIES OR CONDITIONS OF ANY KIND, either express or implied.
% See the License for the specific language governing permissions and
% limitations under the License.
%
% %CopyrightEnd%
%

\chapter{Expressions and Evaluation}

\label{chapter:expressions-evaluation}

\Erlang\ is on one hand a functional programming
language and on the other hand a
language with concurrency.

\index{programming!functional|(}
That \Erlang\ is a functional language means that the central syntactic
concept is that of an \emph{expression} which is \emph{evaluated} in
order to obtain its \emph{value}, which is the result of the evaluation.
\index{programming!functional|)}

\index{programming!concurrent|(}
\index{process!communication|(}
That \Erlang\ has explicit concurrency means that there is the concept
of a \emph{process} and
\emph{communication}\index{communication!between processes} between processes as an
action.  A process is a dynamic entity with state that carries out the
evaluation of an expression.  During its lifetime, it can exchange
messages with other processes and create new processes.

Communication is commanded by evaluating an expression (of the form
\T{\Z{P} !\ \Z{E}}), which means that there are expressions for which
evaluation has a \emph{side effect}\index{effect}.\footnote{As achieving the effect is
often the sole reason for evaluating the expression, calling it a
``side'' effect is sometimes misleading.}
\index{programming!concurrent|)}
\index{process!communication|)}

The presence of (side) effects means that some expression do not have
a unique value.  Two evaluations in the same context might produce
different results.  Good programmers avoid confusing use of such
possibilities.

\section{Environments}

\label{section:environments}
\index{environment|(}
A \emph{binding}\index{binding} is a pair of a variable and a term.
An \emph{environment} is a mapping (\S\ref{section:mappings}) from
variables to terms, i.e., a set of bindings such that no two bindings
have the same variable in their left halves.
\index{environment|)}

\section{Binding, effect and result}

\label{section:REB}

\Erlang\ is different from most other programming languages ---
including other functional programming languages --- in that
expressions constitute the \emph{only} major syntactical category.  It
is customary to make no distinction between expressions (evaluated for
their result) and commands (executed for their effect) --- cf.~C
\cite{iso-c}, Scheme \cite{scheme-r5rs}, Standard ML
\cite{milner+tofte+harper:revised-definition}, etc.\ --- but all those
languages have declarations as a separate category.

\index{variable!binding|(}
An \Erlang\ expression is always evaluated in an environment, which
we refer to as the \emph{input environment}\index{environment!input}
of the expression.  The
expression may provide bindings for variables not in
its input environment.  The \emph{output environment}\index{environment!output}
of the expression
is then the extension of the input environment with the variable bindings
it provides.
\index{variable!binding|)}

Each occurrence of an \Erlang\ expression has a lexical location.
It may be evaluated several times during the execution of
a program (for example, if it is located in the body of a function) and
the environments in which it will be evaluated may differ.  However,
the domains of all these environments will be the same.  The
\emph{input context}\index{context!input} of the expression is the
set of variables that is the common domain of these input
environments; similarly the \emph{output context}\index{context!output} is the common domain
of the output environments.  Note that the output context of an
expression always contains the input context; there is no shadowing
of variables.

\Erlang\ has been designed so that the domain of the output
environment for an expression is a function of the input domain.  This
allows the input and output context of every expression occurrence in
an \Erlang\ program to be determined at compile-time.  An applied occurrence
of a variable that does not belong to the input context where it occurs
--- usually called an \emph{unbound variable}\index{variable!unbound} ---
can therefore be detected at compile-time and a compiler must do so and give a
compile-time error when an unbound variable is detected.

In \Erlang\ an occurrence of an expression in some environment thus
has three roles:
\begin{itemize}
\item It provides a (possibly trivial) extension of the environment to
other subexpressions of the expression or body in which it occurs.
Given the input context of the expression, it
is possible to determine at compile time its output context.
\item Its evaluation produces effects\index{effect}, i.e., it may cause the
process evaluating it to send or receive messages (which could be
either interprocess communication or I/O through ports).
\item Its evaluation has a result, which is the value of the
expression\index{value!of an expression},
provided that the evaluation of the expression completes normally.
How this value is used depends on the surrounding expression.
\end{itemize}

In this chapter we go through all \Erlang\ expressions and explain
their syntax, their effects, their results and how they extend the
environment at run time (which implies how they extend the context at
compile time).

\section{Variables and their scope}

\label{section:scope}
\index{variable!scope|(}

\ifStd
For each occurrence of a variable there will always be an occurrence
of the same variable that is its \emph{binding
occurrence}\index{variable!binding occurrence}, which can be
determined at compile time.  If a variable occurrence is not a binding
occurrence, then it is called an \emph{applied
occurrence}\index{variable!applied occurrence}.

If a variable is in the output context of an expression occurrence but
not in its input context, then the variable must have a binding
occurrence inside the expression occurrence.  As we will see, the
binding occurrence of a variable will always be in a pattern but a
pattern may also contain applied occurrences of variables.

The scope of a variable binding consists of the maximal set of expressions in
which applied occurrences of a variable have the same binding occurrence.

The \emph{order of evaluation}\index{evaluation!order of}
(\S\ref{section:evorder}) chosen for
\StdErlang\ restricts the scope of a variable: the expression providing
the value for the variable binding must have been evaluated before any
expression in the scope of the variable is evaluated.  In order to
discourage a programming style that heavily depends on the particular
order of evaluation that has been established for \StdErlang, scopes
of variable bindings have not been made as large as possible.  For the
following kinds of compound expressions, evaluation of the immediate
subexpressions is strictly left-to-right, but the scope of a variable
binding having its binding occurrence in one immediate subexpression
does not include any other immediate subexpressions, not even those to
the right of it:
\begin{itemize}
\item The arguments of a function application
(\S\ref{section:application-exprs}).
\item The operands of a binary operator (except for the logical
operators [\S\ref{section:logical}]). 
\item The element expressions of a tuple skeleton
(\S\ref{section:tuple-skeletons}).
\item The element expressions of a list skeleton
(\S\ref{section:list-skeletons}).
\end{itemize}

As a matter of programming style, it is recommended that programmers
only exploit the scope of variable bindings in bodies: the scope of a
variable binding having its binding occurrence in one immediate
subexpression of a body includes the following immediate
subexpressions of the body.
\fi
\ifOld
For each occurrence of a variable there will always be an occurrence
of the same variable that is its \emph{binding
occurrence}\index{variable!binding occurrence}.  However, in general
it cannot be determined until run time which is the binding
occurrence.  This is due to the lack of a defined order of evaluation.
If a variable occurrence is not a binding occurrence, then it is
called an \emph{applied occurrence}\index{variable!applied
occurrence}.

If a variable is in the output context of an expression occurrence but
not in its input context, then the variable must have a binding
occurrence inside the expression occurrence.  As we will see, the
binding occurrence of a variable will always be in a pattern but a
pattern may also contain applied occurrences of variables.
\fi
\index{variable!scope|)}

\section{Normal and Abrupt Completion of Evaluation}

\label{section:completion}

For any expression there is a \emph{normal mode} of
evaluation\index{evaluation!normal mode of} in which the execution is
carried out according to the rules laid out in the following sections.
If the evaluation of an expression is carried out according to these
rules until the computation is finished and the result available, then
the expression is said to \emph{complete
normally}\index{completion!normal}.

The evaluation of an expression may alternatively \emph{complete
abruptly}\index{completion!abrupt}, always with an associated
\emph{reason}\index{reason (for abrupt completion)} which is an
\Erlang\ term.  The abrupt completion will have one of the following
causes:
\begin{itemize}
\item The BIF \T{throw/1}\index{throw/1 BIF@\T{throw/1} BIF} has been applied to a term \TZ{T}.
The reason for the abrupt completion is then the term
\T{\char`\{'THROW',\Z{T}\char`\}}.

\item The BIF \T{exit/1}\index{exit/1 BIF@\T{exit/1} BIF} has been applied
to a term \TZ{T}.  The reason for the abrupt completion is then the
term \T{\char`\{'EXIT',\Z{T}\char`\}}.

\item A run-time error\index{error!run-time} has occurred (for example, in
the evaluation of a BIF application), which is then described by a
term \TZ{T}.
%(For each BIF it is documented which errors it may encounter and the format
%of each term describing the error.)
The reason for the abrupt completion is then the term \T{\char`\{'EXIT',\Z{T}\char`\}}.
% [971105] Refine!
When we write that \emph{evaluation exits with reason} \TZ{R}, this is
short for writing that \emph{evaluation completes abruptly with reason}
\T{\char`\{'EXIT',\Z{R}\char`\}}.
\end{itemize}
It follows that abrupt completion due to an error in a BIF is
indistinguishable from abrupt completion due to evaluation of the BIF
\T{exit/1}.  Indeed the BIF \T{exit/1} is intended to be used to
signal that an error has occurred.

When the evaluation of an expression has completed abruptly, the steps
of the normal mode of evaluation of the expression are no longer
followed and there is no output environment.  Abrupt completion is
discussed separately for each kind of expression but in general,
abrupt completion of a subexpression causes abrupt completion of the
whole expression with the same reason.  The exceptions are
\ifStd\T{try}\index{try expression@\T{try} expression} (and
\T{catch}\index{catch expression@\T{catch} expression}) \else
\T{catch}\index{catch expression@\T{catch} expression} \fi
expressions, which are intended to be used for catching an abrupt
completion and go back to normal mode of evaluation
(\S\ref{section:catch}\ifStd, \S\ref{section:try-expr}\fi).

The BIF \T{throw/1}\index{throw/1 BIF@\T{throw/1} BIF} is intended for abrupt completion as a form of non-local
control and abrupt completion caused by its evaluation should always
be caught.

\section{Order of evaluation}

\label{section:evorder}
\index{evaluation!order of|(}

\ifStd
The order of evaluation in \StdErlang\ is defined (with one exception,
cf.~\S\ref{section:list-comprehensions}).
\ifDiff\footnote{The evaluation order was
\emph{not} defined in \OldErlang, cf.~\S\ref{section:new-evaluation-order}.}\fi
In short, one can say that subexpressions are evaluated from left to
right.

In order to simplify the presentation of the expressions of \Erlang,
we will adopt a convention:
\index{evaluation!left-to-right|(}
When we say that a sequence of expressions
$\TZ{E}_1$, $\TZ{E}_2$, \ldots, $\TZ{E}_k$, $k\geq0$, is evaluated
\emph{left-to-right}
in an environment $\epsilon$, we mean that:
\begin{itemize}
\item In the normal mode of evaluation, first $\TZ{E}_1$ is evaluated, then
$\TZ{E}_2$, and so on, until finally $\TZ{E}_k$ is evaluated.
\item If the evaluation of some expression $\TZ{E}_i$, where $1\leq i\leq k$, completes
abruptly with some reason \TZ{R}, the expressions $\TZ{E}_{i+1}$, \ldots, $\TZ{E}_k$
are not evaluated and evaluation of the whole sequence completes abruptly with reason
\TZ{R}.
\item For each $i$, $1\leq i\leq k$, $\epsilon$ is the input environment of
expression $\TZ{E}_i$.
\item For each $i$, $1\leq i\leq k$, let $\epsilon_i$ be the output environment
of expression $\TZ{E}_i$.  It must hold that for each pair of $i$ and $j$, $1\leq i,j\leq k$
and $i\neq j$, $(\epsilon_i\setminus\epsilon)\cap(\epsilon_j\setminus\epsilon) =
\emptyset$,  i.e., that no variable has a binding occurrence in two distinct
expressions.
\item The output environment of the sequence is
$\epsilon\cup\bigcup_{i=1}^k\epsilon_i\setminus\epsilon$ where $\epsilon_i$ is as above.
\end{itemize}
Note that sequence of expressions being evaluated left-to-right does
not imply anything about how the values of these expressions are used.
This will be described separately for each kind of expression.
\index{evaluation!left-to-right|)}

Note also that what is described here is the evaluation order as
\emph{perceived} by the programmer.  That is, the results obtained,
the effects observed and the reasons for abrupt completion given when
evaluating an expression must cohere with what is described here, both
in debugging systems and deployment systems, in interpreter-based
systems and compiler-based systems.  However, when it is \emph{not}
observable, expressions may be evaluated in any order or in
parallel. A compiler thus has freedom to rearrange the order of
evaluation as long as it cannot be observed.
\fi %ifStd
\ifOld
The order in which the subexpressions of an expression are evaluated
is not defined, with one exception:
\begin{textdisplay}
In a body, the expressions are evaluated strictly from left to right.
\end{textdisplay}
In order to simplify the presentation of the expressions of \Erlang,
we will adopt a convention:
\index{evaluation!in some order|(}
When we say that a sequence of expressions
$\TZ{E}_1$, $\TZ{E}_2$, \ldots, $\TZ{E}_k$, $k\geq0$, is evaluated
\emph{in some order}
in an environment $\epsilon$, we mean that:
\begin{itemize}
\item In the normal mode of evaluation, all expressions are evaluated in some
order, say $\TZ{E}_{o_1}$, \ldots, $\TZ{E}_{o_k}$.
\item If the evaluation of some expression $\TZ{E}_{o_i}$, where $1\leq i\leq k$,
completes abruptly with some reason, the expressions
$\TZ{E}_{o_{i+1}}$, \ldots, $\TZ{E}_{o_k}$
are not evaluated and evaluation of the whole sequence completes abruptly with
the same reason.
\item $\epsilon$ is the input environment of expression $\TZ{E}_{o_1}$.
\item For each $i$, $1< i\leq k$, the output environment of
expression $\TZ{E}_{o_{i-1}}$. is the input environment of expression $\TZ{E}_{o_i}$.
\item The output environment of the sequence is the output environment of
expression $\TZ{E}_{o_k}$.
\end{itemize}
Note that sequence of expressions being evaluated in some order does not
imply anything about how the values of these expressions are used.  This
will be described separately for each kind of expression.
\index{evaluation!in some order|)}

\iffalse
% not used
\index{evaluation!left-to-right|(}
When we say that a sequence of expressions
$\TZ{E}_1$, $\TZ{E}_2$, \ldots, $\TZ{E}_k$, $k\geq0$, is evaluated
\emph{left-to-right}, we mean the same except that it is required that for each $i$,
$1\leq i\leq k$, $o_i=i$.
\index{evaluation!left-to-right|)}
\fi

The uncertainty about the order of evaluation together with the
requirement of \S\ref{section:REB} that the compiler must give a
compile-time error for an applied occurrence of an unbound variable
implies that the compiler must only accept a program if for any
evaluation order, there will not be an applied occurrence of an
unbound variable.  The effect of this is that when a sequence of
expressions will be evaluated in some order, the compiler should
assume for each expression that it will be the first to be evaluated,
so its input context will be $\epsilon$.

For example, in a context where \T{X} is unbound, the expression
\T{(X=8) + X} should give a compile-time error.  If the left operand
of \T{+} is evaluated first, then the occurrence of \T{X} in the right
operand will have the value \T{8}.  However, if the right operand is
evaluated first, then the occurrence of \T{X} in it will be unbound.

In the same context, the expression \T{(X=8) + (X=9)} should be
accepted by the compiler, because regardless of the order in which the
operands are evaluated, the applied occurrence of \T{X} will be bound.
(However, there will be a run-time error because either \T{X} will be
bound to \T{8} and then matched against \T{9}, or it will be bound to
\T{9} and then matched against \T{8}.)
\fi
\index{evaluation!order of|)}

\section{Pattern matching}

\label{section:pattern-matching}

Pattern matching occurs as part of the evaluation of several \Erlang\
language constructs so we describe it separately.

\subsection{Patterns}

\label{section:patterns}
\index{pattern!definition of|(}
\begin{rules}
\grrule{Pattern}
       {\NT{Pattern} = \NT{SimplePattern} \OR
        \NT{SimplePattern}}

\grrule{SimplePattern}
       {\NT{AtomicLiteral} & (\S\ref{section:atomic-literals}) \OR
        \NT{Variable} & (\S\ref{section:variables}) \OR
        \NT{UniversalPattern} & (\S\ref{section:universal-pattern}) \OR
        \NT{TuplePattern} \OR
	\NT{RecordPattern} \OR
        \NT{ListPattern}}

\grrule{TuplePattern}
       {\TXT{\char`\{} \OPT{Patterns} \TXT{\char`\}}}

\grrule{ListPattern}
       {\TXT{[} \TXT{]} \OR
        \TXT{[} \NT{Patterns} \OPT{ListPatternTail} \TXT{]}}

\grrule{ListPatternTail}
       {\TXT{|} \NT{Pattern}}

\grrule{Patterns}
       {\NT{Pattern} \OR
        \NT{Patterns} \TXT{,} \NT{Pattern}}

\grrule{RecordPattern}
       {\TXT{\char`\#} \NT{RecordType} \NT{RecordPatternTuple}}

\grrule{RecordType}
       {\NT{AtomLiteral}}

\grrule{RecordPatternTuple}
       {\TXT{\char`\{} \OPT{RecordFieldPatterns} \TXT{\char`\}}}

\grrule{RecordFieldPatterns}
       {\NT{RecordFieldPattern} \OR
        \NT{RecordFieldPatterns} \TXT{,}\ \NT{RecordFieldPattern}}

\grrule{RecordFieldPattern}
       {\NT{RecordFieldName} \TXT{=} \NT{Pattern}}

\grrule{RecordFieldName}
       {\NT{AtomLiteral}}
\end{rules}
(Strictly speaking ``cons pattern'' would be a more appropriate name
for what we call a list pattern.)

We say that two patterns are equal (and thus exchangeable) if they
match exactly the same terms resulting in exactly the same bindings.

Part of the idea with pattern matching is to verify that a term has a
certain (nested) structure with respect to lists or tuples.  It is
then obvious that:
\begin{itemize}
\item \T{[$\Z{P}_1$]} equals \T{[$\Z{P}_1$|[]]}.
\item \T{[$\Z{P}_1$,$\Z{P}_2$,\tdots,$\Z{P}_k$]}, where $k>1$, equals
\T{[$\Z{P}_1$|[$\Z{P}_2$,\tdots,$\Z{P}_k$]]}.
\item \T{[$\Z{P}_1$,$\Z{P}_2$,\tdots,$\Z{P}_k$|$\Z{P}_{k+1}$]}, where $k>1$, equals
\T{[$\Z{P}_1$|[$\Z{P}_2$,\tdots,$\Z{P}_k$|\allowbreak$\Z{P}_{k+1}$]]}.
\end{itemize}
We can therefore describe pattern matching as if each \NT{ListPattern} is either \TXT{[]}
or \TXT{[} \NT{Pattern} \TXT{|} \NT{Pattern} \TXT{]}.

In the scope of a record declaration
(\S\ref{section:record-declarations}) that establishes \TZ{R} as a
record type with $n$ fields, a record pattern
\T{\char`\#\Z{R}\{$\Z{F}_1$=$\Z{P}_1$,\tdots,$\Z{F}_k$=$\Z{P}_k$\}},
where $\TZ{F}_1$, \ldots, $\TZ{F}_k$ are distinct names of fields in
\TZ{R}, is syntactic sugar for a tuple pattern
\T{\{\Z{R},$\Z{Q}_2$,\tdots,$\Z{Q}_{n+1}$\}}
where for each $i$, $2\leq i\leq n+1$,
\begin{itemize}
\item If there is an integer $j$, $1\leq j\leq k$, such that
$\mathit{record\_field}_{\TZm{R}}(\TZ{F}_j)=i$, then
$\TZ{Q}_i$ is $\TZ{P}_j$.
\item Otherwise, $\TZ{Q}_i$ is \T{_}.
\end{itemize}
It is a compile-time error if a record pattern is not in the scope of
an appropriate record declaration.  As record patterns are syntactic
sugar, we can describe pattern matching as if they did not exist.

A pattern on the form \T{$\Z{P}_1$ = $\Z{P}_2$} allows matching a term
against more than one pattern.  The most useful special case is
perhaps for matching a pattern against a compound term while at the
same time binding a variable to the whole term.  For example, the
pattern \T{Lst = [Hd|Tl]} matches a cons and binds the variables
\T{Hd} and \T{Tl} to the head and the tail of the cons but also binds
the variable \T{Lst} to the whole cons.
\index{pattern!definition of|)}

\subsection{Definition of the pattern matching problem}

\index{pattern matching!definition of|(}
A pattern matching problem takes as input a \emph{pattern} \TZ{P}, a
term \TZ{T} and an (input) environment $\epsilon$ and results in
either \emph{failure} or \emph{success}, in the latter case together
with an (output) environment $\epsilon'$ that extends $\epsilon$.

The domain of $\epsilon'$ must include the domain of $\epsilon$ and
all variables occurring in \TZ{P}.  We say that $\epsilon'$ is minimal
if its domain is exactly that.

Informally we can say that pattern matching succeeds if the structure
of the pattern \TZ{P} is the same as that of the term \TZ{T} and there
exists an environment $\epsilon'$ extending $\epsilon$ such that for
each occurrence of a variable in \TZ{P}, its value in $\epsilon'$ is
the term in the corresponding position of \TZ{T}.  $\epsilon'$ is then
the output environment if it is minimal.

More precisely, the pattern matching succeeds with an output environment $\epsilon'$
if $\epsilon'$ is a minimal extension of $\epsilon$ and \TZ{P} matches \TZ{T},
which means that exactly one of the following hold:
\begin{itemize}
\item \TZ{P} is on the form \T{$\Z{P}_1$ = $\Z{P}_2$} and
both $\TZ{P}_1$ and $\TZ{P}_2$ match \TZ{T};
\item \TZ{P} is an atomic literal which denotes \TZ{T};
\item \TZ{P} is a variable which $\epsilon'$ maps to \TZ{T};
\item \TZ{P} is a universal pattern;
\item there exists a $k\geq0$ such that \TZ{P} is a tuple pattern \T{\char`\{$\Z{P}_1$,\tdots,
$\Z{P}_k$\char`\}},
\TZ{T} is a tuple with size $k$ and elements $\TZ{T}_1$, \ldots, $\TZ{T}_k$, and for
each $i$, $1\leq i\leq k$, $\TZ{P}_i$ matches $\TZ{T}_i$;
\item \TZ{P} is a list pattern \T{[]} and \TZ{T} is an empty list;
\item \TZ{P} is a list pattern \T{[$\Z{P}_h$|$\Z{P}_t$]}, \TZ{T} is a cons
with head $\TZ{T}_h$ and tail $\TZ{T}_t$, $\TZ{P}_h$ matches $\TZ{T}_h$ and
$\TZ{P}_t$ matches $\TZ{T}_t$.
\end{itemize}

\noindent If pattern matching succeeds with an environment $\epsilon'$, then
$\epsilon'$ is unique (as can be shown).
\index{pattern matching!definition of|)}

\subsection{Coding pattern matching}

\label{section:coding-pattern-matching}
\index{pattern matching!coding of|(}

As will be obvious below where pattern matching is used, the pattern
is available at compile-time and so is the context, as was noted in
\S\ref{section:REB}.  Therefore the pattern matching can be computed
by code that traverses the term and verifies that the structure is the
same as in the pattern, filling in values for variables not in the
input context as they are encountered in the pattern.  When a variable
not in the input context of the pattern occurs more than once in the
pattern, any occurrence can be proclaimed the binding occurrence.  It
should be the one actually visited first by the pattern matching
algorithm being used.

Here is an example of how code for matching \TZ{P} against \TZ{T}
could be generated.  The generated code examines the term \TZ{T},
provided at run time.  We assume that the representation of
environments is such that for each variable there is a \emph{location}
with undefined initial contents that can be written once with a value
for the variable.  We assume that when the code is run, $\epsilon'$
has been obtained by extending $\epsilon$ with locations for the
variables that occur in \TZ{P} but not in $\epsilon$.  If execution
passes through all the code, the pattern matching has succeeded and
all locations in $\epsilon'$ have been written with values.

We describe recursively the code generation for a pattern $p$ with
\TZ{P} as initial value.  At run time, $t$ should be the term against
which $p$ is to be matched.  The initial value of $t$ will be \TZ{T}.

\begin{itemize}
\item If $p$ is a pattern \T{$p_1$ = $p_2$}, generate code that:
\begin{itemize}
\item Match $p_1$ against $t$.
\item Match $p_2$ against $t$.
\end{itemize}
\item If $p$ is an atomic literal, generate code that finishes the matching with failure
if $t$ is not exactly that literal.
\item If $p$ is the binding occurrence of a variable, generate code that
writes $t$ in its location in $\epsilon'$.
\item If $p$ is a variable but not the binding occurrence, generate code that
finishes the matching with failure if the contents of its location in $\epsilon'$ is
not (exactly) equal to $t$.
\item If $p$ is the universal pattern, generate no code.
\item If $p$ is a tuple pattern \T{\char`\{$p_1$,\tdots,$p_k$\char`\}} (where $k\geq 0$),
generate code that:
\begin{itemize}
\item If $t$ is not a tuple of size $k$, complete with failure.
\item Match $p_1$ against element $1$ of $t$.
\item[] \ldots
\item Match $p_k$ against element $k$ of $t$.
\end{itemize}
\item If $p$ is a list pattern \T{[]}, generate code that finishes the matching
with failure if $t$ is not an empty list.
\item If $p$ is a list pattern \T{[$p_h$|$p_t$]} (or some list pattern that
is equal to such a pattern), generate code that:
\begin{itemize}
\item If $t$ is not a cons, finish the matching with failure.
\item Match $p_h$ against the head of $t$.
\item Match $p_t$ against the tail of $t$.
\end{itemize}
\end{itemize}

As there are no loops in the generated code (and assuming that testing
for equality always completes) the matching must either finish with
failure or reach the end and thus finish successfully.
\index{pattern matching!coding of|)}

\section{Functions, function applications and calls}

\label{section:function-application}
\index{function!application|(}

Function application is part of the evaluation of several kinds of
\Erlang\ expressions, so we describe it once and for all here. The
syntax of these expressions is described elsewhere
(\S\ref{section:application-exprs},
\ifOld\S\ref{section:process-bifs}\fi
\ifStd\S\ref{section:process-module}\fi), as is the
syntax of the expressions that name or denote functions
(\S\ref{section:fun-exprs},
\S\ref{section:program-forms}).

\index{function!call|(}
Evaluation of a function application consists of two parts: evaluation
of the arguments\index{evaluation!of arguments} of the function and a
\emph{function call}.  How arguments are evaluated is described
separately for each form of function application and the function call
never begins until all arguments have been evaluated so here will be
described only how the actual function call is evaluated.

The input to a function call is some specification of which function
is to be applied and the values of the arguments as a sequence of
terms $\TZ{v}_1$, \ldots, $\TZ{v}_{\TZm{n}}$, for some $\TZ{n}\geq 0$.

The function to be applied is always specified in one of the following four
ways:
\begin{enumerate}

\item \label{item:explicit-mod-fun}
A \emph{remote application}\index{function!application!remote}: two atoms \TZ{Mod} and \TZ{Fun}.
Let \TZ{P} be the process evaluating the application.
\begin{itemize}

\item If a row with key $(\TZ{Mod},\TZ{Fun},\TZ{n})$ is in
\T{entry_points[node[\Z{P}]]} (\S\ref{section:node-state-dynamic}),
then the function to be applied is \T{\Z{Fun}/\Z{n}} in
the module named \Z{Mod} and the value of the row is a pointer to the executable code
(\S\ref{section:exported-functions}).

\item \index{undefined_function/3 function@\T{undefined_function/3} function|(}
\index{error_handler@\T{error_handler}!module|(}
Otherwise, if there is a row with key
$(\TZ{E},\T{undefined_function},\T{3})$ in
\T{entry_points[node[\Z{P}]]} (\S\ref{section:process-state-dynamic},
\S\ref{section:node-state-dynamic}), where \TZ{E} is the value of \T{error_handler[\Z{P}]},
then the result of the function application is obtained by instead
evaluating an application
\begin{alltt}
\Z{E}:undefined_function(\Z{Mod},\Z{Fun},[\(\Z{v}\sb{1}\),\tdots,\(\Z{v}\sb{\TZm{n}}\)])\R{.}
\end{alltt}
The initial value of \T{error_handler[\Z{P}]} is \T{error_handler} (cf.~below).

\item Otherwise, the result of the
function application is obtained by evaluating an application
\begin{alltt}
error_handler:undefined_function(\Z{Mod},\Z{Fun},[\(\Z{v}\sb{1}\),\tdots,\(\Z{v}\sb{\TZm{n}}\)])
\end{alltt}
The exported function \T{undefined_function/3} in the
preloaded module \T{error_handler}
exits with
\T{\{undef,\{\Z{Mod},\Z{Fun},[$\Z{v}_1$,\tdots,\linebreak[0]$\Z{v}_{\Zm{n}}$]\}\}}.
\index{undefined_function/3 function@\T{undefined_function/3} function|)}
\index{error_handler@\T{error_handler}!module|)}

\end{itemize}

\item \label{item:function-application-atom}
An atom \TZ{Fun}.
The only kind of expression that specifies the
function in this way is an \NT{ApplicationExpr} on the form
\NT{AtomLiteral} \TXT{(} \OPT{Exprs} \TXT{)}.
There are three possibilities:
%%It is proposed to change this in Standard Erlang.
\begin{itemize}

\item If there is a BIF with an unqualified name
\T{\Z{F}/\Z{n}}, then it is the function to be applied.

\item Otherwise, if there is an attribute
\index{import attribute@\T{import} attribute}
\begin{alltt}
-import(\Z{Mod},[\tdots,\Z{F}/\Z{n},\tdots])
\end{alltt}
then the function to be applied is to be obtained exactly as in
case~\ref{item:explicit-mod-fun} from the atoms \TZ{Mod} and \TZ{Fun}.
(It is thus a remote application.)

\item Otherwise, if there is a definition of a function named
\T{\Z{Fun}/\Z{n}} in the lexically enclosing module (which we may
assume to be named \TZ{Mod}), then the function to be applied is the
one so defined.  This is called a \emph{local
application}\index{function!application!local}.

\item Otherwise it is a compile-time error.
\end{itemize}

\item \label{item:function-application-impl-fun}
\index{function!application!of \T{fun} expression|(}
\index{fun expression@\T{fun} expression|(}
An \emph{implicit \T{fun} application}: a function term that is the
value of an expression \T{fun \Z{Fun}/\Z{k}}.  If there is a
definition of a function named \T{\Z{Fun}/\Z{k}} in the lexically
enclosing module (which we may assume to be named \TZ{Mod}), then the
function to be applied is the one so defined; otherwies it is a
compile-time error.  The function named \T{\Z{Fun}/\Z{k}} need not be
exported from the module.

\item \label{item:function-application-expl-fun}
An \emph{explicit \T{fun} application}:
a function term that is the value of an explicit \T{fun} expression
(\S\ref{section:fun-exprs}).
Let \TZ{Mod} be the module in which the \T{fun} expression lexically occurred.
\index{function!application!of \T{fun} expression|)}
\index{fun expression@\T{fun} expression|)}
\end{enumerate}
In cases \ref{item:function-application-impl-fun} and
\ref{item:function-application-expl-fun} above, it may be that
the arity of the function is not the same as the number of arguments
to which it is being applied.  In this case, evaluation of the
function application exits with
\T{\{badarity,\{\Z{Mod},\Z{Fun},[$\Z{v}_1$,\tdots,\linebreak[0]$\Z{v}_{\Zm{n}}$]\}\}}.

From the four cases above we see that
the function to be applied is either one named \T{\Z{Fun}/\Z{n}}
defined through
a \NT{FunctionDeclaration} (\S\ref{section:program-forms}) in some module named
\TZ{Mod}, or it is denoted by an explicit \T{fun} expression (\S\ref{section:fun-exprs}).
We now describe how the evaluation proceeds in these two cases.

\subsection{Call of a named function}

\label{section:appl-named-function}
\index{function!application!of named function|(}

We shall describe evaluation of a call by a process \TZ{P} of the function named
\T{\Z{Fun}/\Z{n}} in module
\TZ{Mod} to the values $\TZ{v}_1$, \ldots, $\TZ{v}_{\TZm{n}}$ of the arguments.
First, \T{current_function[\Z{P}]} (\S\ref{section:process-state-dynamic})
should be set to
\T{\{\Z{Mod},\Z{Fun},[$\Z{v}_1$,\tdots,$\Z{v}_{\Zm{n}}$]\}}.\footnote{This is
necessary only in order to support the function \T{process_info/2}
(\S\ref{section:processinfo2}).}
Next, let the \NT{FunctionDeclaration} defining \T{\Z{Fun}/\Z{n}} in \T{Mod} be
\begin{alltt}
\Z{F}(\(\Z{P}\sb{1,1}\),\tdots,\(\Z{P}\sb{1,\TZm{n}}\)) \([\mbox{\T{when \(\Z{G}\sb{1}\)}}]\) -> \(\Z{B}\sb{1}\) ;
\(\vdots\) ;
\Z{F}(\(\Z{P}\sb{k,1}\),\tdots,\(\Z{P}\sb{k,\TZm{n}}\)) \([\mbox{\T{when \(\Z{G}\sb{k}\)}}]\) -> \(\Z{B}\sb{k}\).
\end{alltt}
where $k$ is a natural number, each $\TZ{P}_{i,j}$ ($1\leq i\leq k$ and
$1\leq j\leq \TZ{n}$) is a \NT{Pattern}, each (optional) $\TZ{G}_i$ is a \NT{Guard}
and each \T{$\Z{B}_i$} is a \NT{Body}.

Carry out the following step for each function clause $i$, where $i$ goes from $1$ to $k$ in that order,
until a clause $s$ is found for which both pattern matching and guard evaluation succeeds or all clauses
have been tried.
\begin{itemize}
\item Match the terms $\TZ{v}_1$, \ldots, $\TZ{v}_{\TZm{n}}$
against the patterns $\TZ{P}_{i,1}$, \ldots, $\TZ{P}_{i,\TZm{n}}$ in an empty environment.  If the
matching succeeds, evaluate the guard $\TZ{G}_i$ in the output environment of the pattern matching (an omitted
guard trivially succeeds).
\end{itemize}
If there is no clause for which both pattern matching and guard evaluation succeeds,
then the evaluation of the function application exits with
\T{\{function_clause,\{\Z{Mod},\Z{Fun},[$\Z{v}_1$,\tdots,$\Z{v}_{\Zm{n}}$]\}\}}.
Otherwise, the evaluation of the function application continues by evaluating the body
$\TZ{B}_s$ in the output environment of guard $\TZ{G}_s$.
\index{function!application!of named function|)}

\subsection{Call of an unnamed function}

\label{section:appl-unnamed-function}
\index{function!application!of \T{fun} expression|(}
\index{fun expression@\T{fun} expression|(}
\index{clause!of fun expression@of \T{fun} expression|(}

We shall describe evaluation of a call of a function term to the values
$\TZ{v}_1$, \ldots, $\TZ{v}_{\Zm{n}}$ of the arguments.
Suppose that the function term was obtained by evaluating in an environment $\epsilon$ a
\NT{FunExpr}
\begin{alltt}
fun (\(\Z{P}\sb{1,1}\),\tdots,\(\Z{P}\sb{1,\TZm{n}}\)) \([\)when \(\Z{G}\sb{1}]\) -> \(\Z{B}\sb{1}\) ;
    \(\vdots\) ;
    (\(\Z{P}\sb{k,1}\),\tdots,\(\Z{P}\sb{k,\TZm{n}}\)) \([\)when \(\Z{G}\sb{k}]\) -> \(\Z{B}\sb{k}\)
end
\end{alltt}
where $k$ is a natural number, each $\TZ{P}_{i,j}$ ($1\leq i\leq k$ and
$1\leq j\leq \TZ{n}$) is a \NT{Pattern}, each (optional) $\TZ{G}_i$ is a \NT{Guard}
and each \T{$\Z{B}_i$} is a \NT{Body}.  Suppose also that the \T{fun} expression
occurred lexically in the module named \TZ{Mod}.

Carry out the following step for each function clause $i$, where $i$ goes from $1$ to $k$ in that order,
until a clause $s$ is found for which both pattern matching and guard evaluation succeeds or all clauses
have been tried.
\begin{itemize}
\item Match the terms $\TZ{v}_1$, \ldots, $\TZ{v}_{\TZm{n}}$
against the patterns $\TZ{P}_{i,1}$, \ldots, $\TZ{P}_{i,\TZm{n}}$ in an empty environment
(\emph{not} $\epsilon$).  If
the matching succeeds, let $\epsilon'_i$ be the output environment.
Evaluate the guard $\TZ{G}_i$ in $\epsilon\oplus\epsilon'_i$ (an omitted
guard trivially succeeds).
\end{itemize}
If there is no clause for which both pattern matching and guard evaluation succeeds,
then the evaluation of the function application exits with
\ifStd
\T{\{lambda_clause,\{\Z{Mod},\Z{T},[$\Z{v}_1$,\tdots,$\Z{v}_{\Zm{n}}$]\}\}},
where \TZ{T} is an implementation-specific term that may help in
identifying the \T{fun} expression.
\else
\T{\{lambda_clause,\Z{Mod}\}}.
\fi
Otherwise, evaluation of the function application continues by
evaluating the body $\TZ{B}_s$ in the output environment of guard
$\TZ{G}_s$.

Note that matching the formal parameters (i.e., the patterns of the
clauses) against actual parameters
(i.e., the terms $\TZ{v}_1$, \ldots,
$\TZ{v}_{\TZm{n}}$) in an empty environment implies that
variables in the patterns of a clause shadow variables
in the input environment of the \T{fun} expression. It is recommended that
the compiler issues a warning when such shadowing takes place (i.e., when
there is a variable in a pattern of a \T{fun} clause that is bound in the
input environment of the \T{fun} expression).
\index{function!application!of \T{fun} expression|)}
\index{fun expression@\T{fun} expression|)}
\index{clause!of fun expression@of \T{fun} expression|)}

\subsection{Extent of function calls and last call optimization}

\label{section:extent-function-calls}
\index{function!extent of call|(}

We shall state precisely when a function call begins and ends.
Consider a function application where a function is specified in either
of the four ways described in \S\ref{section:function-application}.
The function call begins when matching of the patterns of the function clauses
(either as identified through a module name, a function symbol and an arity,
or as given in an explicit \T{fun} expression) against the values of the
arguments begins.  (In \ifStd an \fi \ifOld the \fi
actual implementation there is \ifStd typically \fi an
entry point of the code for the function, cf.~\T{entry_points[\Z{N}]} of
a node \TZ{N}, and the beginning of the function call corresponds to the
moment when execution reaches that entry point.)

Note that argument evaluation in the evaluation of a function application thus
always occurs before the function call begins.

Note also that in the case of a remote function application (i.e., case
\ref{item:explicit-mod-fun} of \S\ref{section:function-application}) when there
is no exported function with the given module name, function symbol and arity,
there is no function call, so there is no beginning, nor an end.

In order to state when the function call ends we must consider several cases
(cf.~\S\ref{section:appl-named-function}):
\begin{itemize}
\item If there is no clause of the function declaration for which both pattern matching
and guard evaluation succeeds, then the function call ends when the evaluation
of the function application completes (abruptly).
\item Otherwise, let \TZ{B} be the body
of the selected clause and let $\TZ{E}'$ be the final expression in the evaluation
of \TZ{B} as defined below.
\begin{itemize}
\item If $\TZ{E}'$ is a
\iffalse remote \fi
function application, and
evaluation of \TZ{B} does not complete abruptly before $\TZ{E}'$ is evaluated, then
the original function call ends when the function call in $\TZ{E}'$ begins.
The function call in $\TZ{E}'$ is said to be the \emph{last call}\index{function!last call}
of \TZ{B}.
\item Otherwise, the function call ends when the evaluation
of the original function application completes (normally or abruptly).
\end{itemize}
\end{itemize}

We shall define the final expression of a body and of an expression through mutual recursion.
This is well-defined only when evaluation of the body or the expression completes
normally.

\begin{itemize}
\item The final expression in the evaluation of a body \T{$\Z{E}_1$, \tdots, $\Z{E}_k$}, where
$k\geq1$, is the expression $\TZ{E}_k$.
\item The final expression in the evaluation of an expression \TZ{E} is defined case by case:
\begin{itemize}
\item If \TZ{E} is a block expression \T{begin \Z{B} end}, then the final expression of
\TZ{E} is the final expression of the body \TZ{B}.
\item If \TZ{E} is \ifStd a \T{cond}, \else an \fi
\T{if} or \T{case} expression (\ifStd\S\ref{section:cond-expr},\fi
\S\ref{section:if-expr}, \S\ref{section:case-expr}), then
the final expression of
\TZ{E} is the final expression of the body of the selected clause of \TZ{E}.
\item If \TZ{E} is a \T{receive} expression (\S\ref{section:receive-expr}), then:
\begin{itemize}
\item If the expiry time was reached, then 
the final expression of \TZ{E} is the final expression of the expiry body of \TZ{E}.
\item Otherwise,
the final expression of
\TZ{E} is the final expression of the body of the selected clause of \TZ{E}.
\end{itemize}
\ifStd
\item If \TZ{E} is a \T{try} expression (\S\ref{section:try-expr}), then:
\begin{itemize}
\item If evaluation of the protected body of the \T{try} expression completed normally, then 
the final expression of
\TZ{E} is \TZ{E} itself.
\item Otherwise,
the final expression of
\TZ{E} is the final expression of the body of the selected clause of \TZ{E}.
\end{itemize}
\fi
\item If \TZ{E} is a parenthesized expression \T{($\Z{E}'$)}, then the final expression of
\TZ{E} is $\TZ{E}'$.
\item Otherwise, the final expression of \TZ{E} is \TZ{E} itself.
\end{itemize}
\end{itemize}

When a function call ends,
\ifStd a \StdErlang\ implementation must ensure \fi
\ifOld \Erlang\ ensures \fi
that any resources that are not recycled through garbage collection
have been restored.  In particular this means that if memory for
function calls is allocated on a stack, the size of the stack
\ifStd must be \fi
\ifOld is \fi
the same when a function call begins and when it ends.

\label{section:lco}

\index{last call optimization|(}
A consequence of this requirement and the definition of when a
function call ends is that an \Erlang\ implementation \ifStd must
provide \else provides \fi \emph{last call optimization} when the
final expression in the body of a called function is a function
application.
\index{last call optimization|)}

\label{section:function-use}

\index{function!use of|(}
Consider a process \TZ{P} that is evaluating an application of the
exported function \T{\Z{Fun}/\Z{n}} in the module named \TZ{M}.  Let
\TZ{B} be the binary that contains the compiled code
(\S\ref{section:code-generation}) for the version of \TZ{M} that is
current when the function call begins
(\S\ref{section:current-version}).  Process \TZ{P} is then
\emph{using} function \T{\Z{Fun}/\Z{n}} in \TZ{B} from the time that
the function call begins until the function call ends.
\index{function!use of|)}
\index{function!extent of call|)}
\index{function!application|)}
\index{function!call|)}

\section{Bodies}

\label{section:bodies}

\index{body|(}
A \emph{body} is a nonempty sequence of expressions.

\SYNTAX

\begin{rules}
\grrule{Body}
       {\NT{Exprs}}

\grrule{Exprs}
       {\NT{Expr} \OR
        \NT{Exprs} \TXT{,} \NT{Expr}}
\end{rules}

\EVALUATION

Evaluation of a body \T{$\Z{E}_1$, \tdots, $\Z{E}_k$},
where $k\geq 1$, with an input environment $\epsilon$
is carried out as follows:

\begin{itemize}
\item First $\TZ{E}_1$ is evaluated, then
$\TZ{E}_2$, and so on, until finally $\TZ{E}_k$ is evaluated.  The
values of expressions $\TZ{E}_1$, \ldots, $\TZ{E}_{k-1}$ are
completely ignored.  If the evaluations of all these expressions
complete normally, then the evaluation of the body also completes
normally and its value is the value of expression $\TZ{E}_k$.
\item If the evaluation of some expression $\TZ{E}_i$, where $1\leq
i\leq k$, completes abruptly with some reason \TZ{R}, the expressions
$\TZ{E}_{i+1}$, \ldots, $\TZ{E}_k$ are not evaluated and evaluation of
the body completes abruptly with reason \TZ{R}.
\end{itemize}
(Recall that in \S\ref{section:evorder} we stated that this is the
\emph{perceivable} evaluation order.  If advantageous from an
efficiency point of view, an implementation can often change the order
of evaluation of expressions that have no side effects.)

\ENVIRONMENTS

\begin{itemize}
\item $\epsilon$ is used as input environment of expression $\TZ{E}_1$.
\item For each $i$, $1< i\leq k$, the output environment of
expression $\TZ{E}_{i-1}$ is used as input environment of expression
$\TZ{E}_i$.
\item The output environment of expression $\TZ{E}_k$ is
used as output environment of the body.
\end{itemize}
\index{body|)}

\section{\T{catch} expressions}

\label{section:catch}
\index{catch expression@\T{catch} expression|(}

\ifOld
A \T{catch} expression is used for restoring normal mode of
evaluation\index{evaluation!normal mode of}.
\else
A \T{catch} expression is an obsolete form of expression used for
restoring normal mode of evaluation\index{completion!restoring
normal}\index{evaluation!normal mode of}.  In new code a \T{try}
expression (\S\ref{section:try-expr}) should be used instead of a
\T{catch} expression\ifDiff (\S\ref{section:new-try})\fi.
\fi

\SYNTAX

\begin{rules}
\grrule{Expr}
       {\TXT{catch} \NT{Expr} \OR
        \NT{MatchExpr}}
\end{rules}

\EVALUATION

Evaluating an expression \T{catch \Z{E}} begins by evaluating \TZ{E}.
\begin{itemize}
\item If evaluation of \TZ{E} completes normally and its result is \TZ{v},
then evaluation of \T{catch \Z{E}} also completes normally with result
\TZ{v}.
\item If evaluation of \TZ{E} completes abruptly with reason
\T{\char`\{'THROW',\Z{T}\char`\}}, for some term \TZ{T}, then evaluation
of \T{catch \Z{E}} completes normally with result \TZ{T}.
\item If evaluation of \TZ{E} completes abruptly with reason
\T{\char`\{'EXIT',\Z{T}\char`\}},
for some term \TZ{T}, then evaluation of \T{catch \Z{E}} completes
normally with result \T{\char`\{'EXIT',\Z{T}\char`\}}.
\end{itemize}

\ENVIRONMENTS

\begin{itemize}
\item The input environment of \T{catch \Z{E}} is used as input environment of \TZ{E}.
\item The output environment of \TZ{E} is not used;\footnote{If evaluation of
\TZ{E} completes abruptly, then the values of some variables with binding occurrences
in \TZ{E} may not have been computed.  Therefore any bindings in
\TZ{E} must not be visible outside it.}
the output environment of \T{catch \Z{E}} is the same as its input environment.
\end{itemize}
\index{catch expression@\T{catch} expression|)}

\section{Match expressions}

\label{section:match-expr}
\index{match expression|(}
\index{= operator@\T{=} operator|(}

A match expression consists of a pattern and an expression.  Its
purpose is to match the pattern against the value of the expression,
providing bindings for variables having their binding occurrence in
the pattern.

\SYNTAX

\begin{rules}
\grrule{MatchExpr}
       {\NT{Pattern} \TXT{=} \ifStd\NT{SendExpr}\fi\ifOld\NT{MatchExpr}\fi \OR
        \NT{SendExpr}}
\end{rules}

\EVALUATION

The evaluation of an expression \T{\Z{P} = \Z{E}}, where \TZ{P} is a
pattern and
\TZ{E} is an expression, begins with evaluating \TZ{E}.
\begin{itemize}
\item If the evaluation of \TZ{E} completes
abruptly with reason \TZ{R}, then the evaluation of the match expression also
completes abruptly with reason \TZ{R}.
\item If the evaluation of \TZ{E} completes normally with the term
\TZ{T} as result, then what remains is matching \TZ{P} against \TZ{T}.
\begin{itemize}
\item If the matching succeeds, then the computation of
the match expression completes normally with result \TZ{T}.
\item If the matching fails, the computation of the
match expression exits with reason \T{\char`\{badmatch,$\Z{T}'$\char`\}}, where
$\TZ{T}'$ is some term that is a (not necessarily strict) subterm of \TZ{T} such
that its top level does not match the corresponding subpattern of \TZ{P}.
\end{itemize}
\end{itemize}

\ENVIRONMENTS

\begin{itemize}
\item The input environment of the match expression is used as input environment of \TZ{E}.
\item The output environment of \TZ{E} is used as input environment of the pattern matching.
\item The output environment of the pattern matching is used as output environment of
the match expression.
\end{itemize}

\iffalse
A match expression has only one proper subexpression so it is not meaningful to
talk about an order of evaluation.  However, note that in an expression
\begin{alltt}
\(\Z{P}\sb{1}\) = \(\Z{P}\sb{2}\) = \(\cdots\) = \(\Z{P}\sb{k}\) ! \Z{E}
\end{alltt}
matching against pattern $\TZ{P}_i$ will be completed before
matching against pattern $\TZ{P}_{i-1}$ begins and variables
having their binding occurrence in pattern $\TZ{P}_i$ can have
applied occurrences in pattern $\TZ{P}_{i-1}$.
\fi
\index{match expression|)}
\index{= operator@\T{=} operator|)}

\section{Send expressions}

\label{section:send-expr}
\index{send expression|(}
\index{"! operator@\T{"!}~operator|(}

A send expression has two operands. The value of the leftmost operand
should identify a process or port to which the value of the rightmost
operand will be sent.

\SYNTAX

\begin{rules}
\grrule{SendExpr}
       {\NT{CompareExpr} \TXT{!}\ \NT{SendExpr} \OR
        \NT{CompareExpr}}
\end{rules}

\EVALUATION

The evaluation of a send expression \T{$\Z{E}_1$ !\ $\Z{E}_2$} begins
with evaluating the operands $\TZ{E}_1$ and $\TZ{E}_2$
\ifStd left-to-right \fi \ifOld in some order \fi
in the input environment of the send expression.

Let $\TZ{v}_1$ and $\TZ{v}_2$ be the values of $\TZ{E}_1$ and
$\TZ{E}_2$ respectively.
\begin{itemize}
\item If $\TZ{v}_1$ is a PID\index{PID} or a port\index{port},
then $\TZ{v}_2$ is dispatched as a message to $\TZ{v}_1$
(\S\ref{section:messages}, \S\ref{section:send-port}).
\item If $\TZ{v}_1$ is an atom, then $\TZ{v}_1$ is looked up in
\T{registry[\Z{N}]}, where \TZ{N} is the node on which the current
process is executing.
\begin{itemize}
\item If there is a process with some PID \TZ{P}
registered\index{process!registry} under the name $\TZ{v}_1$ on node
\TZ{N} (\S\ref{section:process-registry}), then $\TZ{v}_2$ is
dispatched as a message to \TZ{P}.
\item If there is no process registered under the name $\TZ{v}_1$ on
node \TZ{N}, then
\ifOld evaluation of  the send expression exits with \T{badarg}. \fi
\ifStd the send expression has no effect. \fi
\end{itemize}
\item If $\TZ{v}_1$ is a 2-tuple of atoms \TZ{A} and \TZ{N}, then
\TZ{A} is looked up in \T{registry[\Z{N}]} (although this lookup is
performed on node \TZ{N}).
\begin{itemize}
\item If there is a process with some PID \TZ{P} registered under the name \TZ{A}
on node \TZ{N}, then $\TZ{v}_2$ is dispatched as a message to \TZ{P}
(\S\ref{section:messages}).
\item If there is no process registered under the name \TZ{A} on node \TZ{N}, then
the send expression has no effect.
\end{itemize}
\item If $\TZ{v}_1$ is not a PID, neither a port, nor an atom, nor a
2-tuple of atoms, then
evaluation of the send expression exits with \T{badarg}.
\end{itemize}

If evaluation completes normally, the value of the send expression is
$\TZ{v}_2$.

\ENVIRONMENTS

The output environment of the operands is used as output environment
of the send expression.
\index{send expression|)}
\index{"! operator@\T{"!}~operator|)}

\section{Relational and equational operators}

\label{section:relational}
\index{term!comparison|(}
\index{relational operators|(}
\index{equational operators|(}
\index{< operator@\T{<} operator|(}
\index{=< operator@\T{=<} operator|(}
\index{> operator@\T{>} operator|(}
\index{>= operator@\T{>=} operator|(}
\index{=:= operator@\T{=:=} operator|(}
\index{=/= operator@\T{=/=} operator|(}
\index{== operator@\T{==} operator|(}
\index{/= operator@\T{/=} operator|(}

The relational and equality operators do not associate neither to left,
nor to the right.

These operators can be applied to any pair of values and will always
return \T{true} or \T{false}.

\SYNTAX

\begin{rules}
\grrule{CompareExpr}
       {\NT{ListConcExpr} \NT{RelationalOp} \NT{ListConcExpr} \OR
        \NT{ListConcExpr} \NT{EqualityOp} \NT{ListConcExpr} \OR
        \NT{ListConcExpr}}

\grruleoneof{RelationalOp}{\TXT{<~~~~=<~~~>~~~~>=}}

\grruleoneof{EqualityOp}{\TXT{=:=~~=/=~~==~~~/=}}
\end{rules}

\EVALUATION

Evaluation of an expression \T{$\Z{E}_1$ \Z{O} $\Z{E}_2$}, where \TZ{O} is
one of the eight relational and equality operators, begins with evaluating
the operands $\TZ{E}_1$ and $\TZ{E}_2$
\ifStd left-to-right \fi \ifOld in some order \fi
in the input environment
of \T{$\Z{E}_1$ \Z{O} $\Z{E}_2$}.  Let the values of
the operands be $\TZ{v}_1$ and $\TZ{v}_2$, respectively.
See the following sections for how the result is computed for each
operator.

\ENVIRONMENTS

The output environment of the operands is used as output environment of
\T{$\Z{E}_1$ \Z{O} $\Z{E}_2$}.

\subsection{Relational operators \T{<}, \T{=<}, \T{>}, and \T{>=}}

\label{section:relationalops}

%First the terms $\TZ{v}_1$ and $\TZ{v}_2$ are coerced (\S\ref{section:coercion});
%let $(\TZ{w}_1,\TZ{w}_2)=\I{coerce}(\TZ{v}_1,\TZ{v}_2)$.
The terms are compared according to the term order (\S\ref{section:term-order})
and equality (\S\ref{section:equality}).  If the comparison succeeds, the value
of the expression is \T{true}; otherwise it is \T{false}.
\begin{itemize}
\item The operator \T{<} succeeds if $\TZ{v}_1$ precedes $\TZ{v}_2$ in the term
order.
\item The operator \T{=<} succeeds if $\TZ{v}_1$ precedes $\TZ{v}_2$ in the term
order, or $\TZ{v}_1$ is equal to $\TZ{v}_2$.
\item The operator \T{>} succeeds if $\TZ{v}_2$ precedes $\TZ{v}_1$ in the term
order.
\item The operator \T{>=} succeeds if $\TZ{v}_2$ precedes $\TZ{v}_1$ in the term
order, or $\TZ{v}_1$ is equal to $\TZ{v}_2$.
\end{itemize}

\subsection{Exact equational operators \T{=:=}, \T{=/=}}

\label{section:exactequationalops}
\index{equality!exact|(}

If the operator is \T{=:=}, then the value of the expression is \T{true} if
$\TZ{v}_1$ is (exactly) equal to $\TZ{v}_2$ and \T{false} otherwise.

If the operator is \T{=/=}, then the value of the expression is \T{false} if
$\TZ{v}_1$ is (exactly) equal to $\TZ{v}_2$ and \T{true} otherwise.
\index{equality!exact|)}

\subsection{Arithmetic equational operators \T{==}, \T{/=}}

\label{section:coercingequationalops}
\index{equality!arithmetic|(}

If the operator is \T{==}, then the value of the expression is \T{true} if
$\TZ{v}_1$ is arithmetically equal to $\TZ{v}_2$ and \T{false} otherwise.

If the operator is \T{/=}, then the value of the expression is \T{false} if
$\TZ{v}_1$ is arithmetically equal to $\TZ{v}_2$ and \T{true} otherwise.
\index{equality!arithmetic|)}

\index{term!comparison|)}
\index{relational operators|)}
\index{equational operators|)}
\index{< operator@\T{<} operator|)}
\index{=< operator@\T{=<} operator|)}
\index{> operator@\T{>} operator|)}
\index{>= operator@\T{>=} operator|)}
\index{=:= operator@\T{=:=} operator|)}
\index{=/= operator@\T{=/=} operator|)}
\index{== operator@\T{==} operator|)}
\index{/= operator@\T{/=} operator|)}

\section{List concatenation operators}

\label{section:listconc-exprs}
\index{list!concatenation operators|(}
\index{++ operator@\T{++} operator|(}
\index{-- operator@\T{--} operator|(}

The list concatenation operators associate to the right.  This results in the most
efficient computation of expressions on the form \T{$\Z{E}_1$
++ $\Z{E}_2$ ++ $\Z{E}_3$}.  However, note that an expression \T{$\Z{E}_1$
-- $\Z{E}_2$ -- $\Z{E}_3$} is equivalent to \T{$\Z{E}_1$
-- ($\Z{E}_2$ -- $\Z{E}_3$)}, which may be counterintuitive.

\SYNTAX

\begin{rules}
\grrule{ListConcExpr}
       {\NT{AdditionShiftExpr} \NT{ListConcOp} \NT{ListConcExpr} \OR
        \NT{AdditionShiftExpr}}

\grruleoneof{ListConcOp}{\TXT{++~~~--}}
\end{rules}

\EVALUATION

Evaluation of an expression \T{$\Z{E}_1$ ++ $\Z{E}_2$} or
\T{$\Z{E}_1$ -- $\Z{E}_2$} begins by evaluating
the operands $\TZ{E}_1$ and $\TZ{E}_2$
\ifStd left-to-right \fi \ifOld in some order \fi
in the input
environment of the whole expression.  Let the values of the operands
be $\TZ{v}_1$ and $\TZ{v}_2$, respectively.
See the following sections for how the result is computed for each
operator.

\ENVIRONMENTS

The output environment of the operands is used as output environment of
the list concatenation expression.

\subsection{List addition operator \T{++}}

\label{section:list-addition}
\index{list!addition|(}

\begin{itemize}
\item If $\TZ{v}_1$ is not a list,
the evaluation of \T{$\Z{E}_1$
++ $\Z{E}_2$} exits with reason \T{badarg}.
\item Otherwise, suppose that the list $\TZ{v}_1$ has the $k$ elements
$\TZ{x}_1$, \ldots, $\TZ{x}_k$.  The value of  \T{$\Z{E}_1$ ++ $\Z{E}_2$}
is then a list with $\TZ{x}_1$, \ldots, $\TZ{x}_k$ as its first
$k$ elements and $\TZ{v}_2$ as its $k$th tail.  This means that if
$\TZ{v}_2$ is a list with the $l$ elements
$\TZ{y}_1$, \ldots, $\TZ{y}_l$, then the value of \T{$\Z{E}_1$
++ $\Z{E}_2$} is a list with the $k+l$ elements
$\TZ{x}_1$, \ldots, $\TZ{x}_k$, $\TZ{y}_1$, \ldots, $\TZ{y}_l$.
(If $\TZ{v}_2$ is not a list, then neither is the value of
\T{$\Z{E}_1$ ++ $\Z{E}_2$}.)
\end{itemize}

The time required for computing the result from $\TZ{v}_1$ and
$\TZ{v}_2$ should be % at most [this is understood?]
$O(k)$ (where $k$ is the number of elements in $\TZ{v}_1$).

Informally, this operator computes the result of concatenating two lists.
\index{list!addition|)}

\subsection{List difference operator \T{--}}

\label{section:list-subtraction}
\index{list!difference|(}

\begin{itemize}
\item If $\TZ{v}_1$ or $\TZ{v}_2$ is not a list,
the evaluation of \T{$\Z{E}_1$
-- $\Z{E}_2$} exits with reason \T{badarg}.
\item Otherwise, suppose that $\TZ{v}_2$ has the $l$ elements
\T{$\Z{y}_1$}, \ldots, \T{$\Z{y}_l$}.  Let us define inductively a sequence $s_0$, \ldots,
$s_l$, each of which is a list.  Let $s_0$ be $\TZ{v}_1$ and let $s_{i+1}$ (for $0\leq i<l$)
be:
\begin{itemize}
\item If $\TZ{y}_{i+1}$ is an element in $s_i$, then let $s_{i+1}$ be $s_i$ without the
first occurrence of $\TZ{y}_{i+1}$.
\item Otherwise, let $s_{i+1}$ be $s_i$.
\end{itemize}
The value of \T{$\Z{E}_1$ -- $\Z{E}_2$} is then $s_l$.
\end{itemize}

The time required for computing the result from $\TZ{v}_1$ and
$\TZ{v}_2$ should be % at most [this is understood?]
$O(kl)$ (where $k$ and $l$ are the number of elements in
$\TZ{v}_1$ and $\TZ{v}_2$, respectively).

Informally, this operator computes the result of removing the elements of one list
from another, a form of ``list difference''.
\index{list!difference|)}

\index{list!concatenation operators|)}
\index{++ operator@\T{++} operator|)}
\index{-- operator@\T{--} operator|)}

\section{Additive and shift operators}

\label{section:additive}
\label{section:shift}
\index{additive operators|(}
\index{shift operators|(}
\index{+ operator@\T{+} operator|(}
\index{- operator@\T{-} operator|(}
\index{bor operator@\T{bor} operator|(}
\index{bxor operator@\T{bxor} operator|(}
\index{bsl operator@\T{bsl} operator|(}
\index{bsr operator@\T{bsr} operator|(}
\index{or operator@\T{or} operator|(}
\index{xor operator@\T{xor} operator|(}

The additive and shift operators associate to the left.

\SYNTAX

\begin{rules}
\grrule{AdditionShiftExpr}
       {\NT{AdditionShiftExpr} \NT{AdditionOp} \NT{MultiplicationExpr} \OR
        \NT{AdditionShiftExpr} \NT{ShiftOp} \NT{MultiplicationExpr} \OR
        \NT{AdditionShiftExpr} \TXT{or} \NT{MultiplicationExpr} \OR
        \NT{AdditionShiftExpr} \TXT{xor} \NT{MultiplicationExpr} \OR
        \NT{MultiplicationExpr}}

\grruleoneof{AdditionOp}{\TXT{+~~~~-}\\
                         \TXT{bor~~bxor}}

\grruleoneof{ShiftOp}{\TXT{bsl~~bsr}}
\end{rules}

\EVALUATION

Evaluation of an expression \T{$\Z{E}_1$ \Z{O} $\Z{E}_2$}, where
\TZ{O} is one of the additive operators \TXT{+}, \TXT{-}, \TXT{bor}
and \TXT{bxor}, shift operators \TXT{bsl} and \TXT{bsr} or logical
operators \TXT{or} and \TXT{xor} begins with evaluating the operands
$\TZ{E}_1$ and $\TZ{E}_2$
\ifStd left-to-right \fi \ifOld in some order \fi
in the input environment of the whole expression.  Let the values
of the operands be $\TZ{v}_1$ and $\TZ{v}_2$, respectively.
See the following sections for how the result is computed for each
operator.

\ENVIRONMENTS

The output environment of the operands is used as output environment of
\T{$\Z{E}_1$ \Z{O} $\Z{E}_2$}.

\subsection{Numeric addition operators \T{+} and \T{-}}

\label{section:additionops}
\index{integer!addition|(}
\index{float!addition|(}
\index{integer!subtraction|(}
\index{float!subtraction|(}

\begin{itemize}
\item If $\TZ{v}_1$ or $\TZ{v}_2$ is not a number, then
evaluation of \T{$\Z{E}_1$ \Z{O} $\Z{E}_2$} exits with \T{\badarith}.
\item Otherwise, if both $\TZ{v}_1$ and $\TZ{v}_2$ are integers,
$\mathit{add}_I(\Er[\TZ{v}_1],\Er[\TZ{v}_2])$ (if \TZ{O} is \T{+}) or
$\mathit{sub}_I(\Er[\TZ{v}_1],\Er[\TZ{v}_2])$ (if \TZ{O} is \T{-}) is
computed; let the result be $r$.
\item Otherwise, the terms $\TZ{v}_1$ and $\TZ{v}_2$ are coerced to floats
(\S\ref{section:coercion});
let $(\TZ{w}_1,\TZ{w}_2)=\mathit{coerce}(\TZ{v}_1,\TZ{v}_2)$.
Next $\mathit{add}_F(\Er[\TZ{w}_1],\Er[\TZ{w}_2])$ (if \TZ{O} is \T{+}) or
$\mathit{sub}_F(\Er[\TZ{w}_1],\Er[\TZ{w}_2])$ (if \TZ{O} is \T{-}) is
computed; let the result be $r$.
\end{itemize}
If $r$ is a number, the value of \T{$\Z{E}_1$ \Z{O} $\Z{E}_2$} is $\Re[r]$;
otherwise, evaluation of \T{$\Z{E}_1$ \Z{O} $\Z{E}_2$}
exits with $\Re[r]$.
\index{integer!addition|)}
\index{float!addition|)}
\index{integer!subtraction|)}
\index{float!subtraction|)}

\subsection{Integer bitwise operator \T{bor}}

\label{section:bitwiseor}
\index{integer!bitwise or|(}

\begin{itemize}
\item If $\TZ{v}_1$ or $\TZ{v}_2$ is not an integer, then
      evaluation of \T{$\Z{E}_1$ bor $\Z{E}_2$} exits with \T{\badarith}.
\item Otherwise, the value of \T{$\Z{E}_1$ or $\Z{E}_2$} is the integer
      that is the bitwise OR of $\TZ{v}_1$ and $\TZ{v}_2$,
      i.e., the integer that in binary two's-com\-ple\-ment
      representation has a zero in those positions where the binary
      two's-com\-ple\-ment representations of both
      $\TZ{v}_1$ and $\TZ{v}_2$ have a zero, and a
      one in the other positions.
\end{itemize}
\index{integer!bitwise or|)}

\subsection{Integer bitwise operator \T{bxor}}

\label{section:bitwisexor}
\index{integer!bitwise xor|(}

\begin{itemize}
\item If $\TZ{v}_1$ or $\TZ{v}_2$ is not an integer, then
      evaluation of \T{$\Z{E}_1$ bxor $\Z{E}_2$} exits with \T{\badarith}.
\item Otherwise, the value of \T{$\Z{E}_1$ xor $\Z{E}_2$} is the integer
      that is the bitwise XOR of $\TZ{v}_1$ and $\TZ{v}_2$,
      i.e., the integer that in binary two's-com\-ple\-ment
      representation has a one in those positions where the binary
      two's-com\-ple\-ment representation of exactly one of
      $\TZ{v}_1$ and $\TZ{v}_2$ has a one,
      and a zero in the other positions.
\end{itemize}
\index{integer!bitwise xor|)}

\subsection{Shift operators \T{bsl} and \T{bsr}}

\label{section:shift-ops}
\index{integer!bitwise shift|(}

These operators compute bitwise shifts in the binary two's-com\-ple\-ment
representation of integers.
The left-hand operator gives the integer to be shifted and the right-hand
operator gives the number of positions to shift.

The \T{bsl} operator computes bitwise shift to the left with zeroes in
the lowest-order bits of the result.  The \T{bsr} operator computes
arithmetic shift to the right, i.e., with ones in
the highest-order bits of the result if the left-hand operand was negative
and zeros otherwise.

\begin{itemize}
\item If $\TZ{v}_1$ or $\TZ{v}_2$ is not an integer, evaluation of
the shift expression exits with \T{\badarith}.
\ifStd It is also permitted for an implementation to exit with
\T{integer_overflow} \fi
\ifOld \OldErlang\ also exits with \T{badarg} \fi
if $\TZ{v}_2$ is not a fixnum (\S\ref{section:integer-type}).
\item Otherwise, if the operator is \T{bsl}:
\begin{itemize}
\item If $\TZ{v}_2$ is negative, the value of the shift expression is
the same as \T{bsr} for $\TZ{v}_1$ and $-\TZ{v}_2$.
\item Otherwise, compute $\Er[\TZ{v}_1]\cdot2^{\Er[\TZ{v}_2]}$ and let
the result be $r$.  (This is the result of extending the lowest-order
bits in the binary two's-complement representation of $\TZ{v}_1$ with
$\TZ{v}_2$ zeroes.)  If $r\in I$, then the value of \T{$\Z{E}_1$ bsl
$\Z{E}_2$} is $\Re[r]$; otherwise evaluation of the shift expression
exits with $\Re[\B{integer\_overflow}]$.
\end{itemize}
\item Otherwise, the operator is \T{bsr}:
\begin{itemize}
\item If $\TZ{v}_2$ is negative, the value of the shift expression is
the same as \T{bsl} for $\TZ{v}_1$ and $-\TZ{v}_2$.
\item Otherwise, compute
$\lfloor\Er[\TZ{v}_1]\cdot2^{-\Er[\TZ{v}_2]}\rfloor$ and let the
result be $r$.  (This is the result of removing the $\TZ{v}_2$
lowest-order bits in the binary two's-complement representation of
$\TZ{v}_1$.)  The result is always in $I$ when $\TZ{v}_1$ is in $I$;
the value of \T{$\Z{E}_1$ bsr $\Z{E}_2$} is $\Re[r]$.
\end{itemize}
\end{itemize}
\index{integer!bitwise shift|)}

\subsection{Disjunction operator \T{or}}

\label{section:booleanor}
\index{Boolean!OR|(}

\begin{itemize}
\item If $\TZ{v}_1$ or $\TZ{v}_2$ is not a Boolean, then
      evaluation of \T{$\Z{E}_1$ or $\Z{E}_2$} exits with
      \ifStd\T{badbool}\else\T{badarg}\fi.
\item Otherwise, if at least one of $\TZ{v}_1$ and $\TZ{v}_2$ is \T{true},
      the result is \T{true}.
\item Otherwise, the result is \T{false}.
\end{itemize}
\index{Boolean!OR|)}

\subsection{Exclusion operator \T{xor}}

\label{section:booleanxor}
\index{Boolean!XOR|(}

\begin{itemize}
\item If $\TZ{v}_1$ or $\TZ{v}_2$ is not a Boolean, then
      evaluation of \T{$\Z{E}_1$ xor $\Z{E}_2$} exits with
      \ifStd\T{badbool}\else\T{badarg}\fi.
\item Otherwise, if exactly one of $\TZ{v}_1$ and $\TZ{v}_2$ is \T{true},
      the result is \T{true}.
\item Otherwise, the result is \T{false}.
\end{itemize}
\index{Boolean!XOR|)}

\index{additive operators|)}
\index{shift operators|)}
\index{+ operator@\T{+} operator|)}
\index{- operator@\T{-} operator|)}
\index{bor operator@\T{bor} operator|)}
\index{bxor operator@\T{bxor} operator|)}
\index{bsl operator@\T{bsl} operator|)}
\index{bsr operator@\T{bsr} operator|)}
\index{or operator@\T{or} operator|)}
\index{xor operator@\T{xor} operator|)}

\section{Multiplicative operators}

\label{section:multiplicative}
\index{multiplicative operators|(}
\index{* operator@\T{*} operator|(}
\index{/ operator@\T{/} operator|(}
\ifStd\index{// operator@\T{//} operator|(}\fi
\index{div operator@\T{div} operator|(}
\ifStd\index{mod operator@\T{mod} operator|(}\fi
\index{rem operator@\T{rem} operator|(}
\index{band operator@\T{band} operator|(}
\index{and operator@\T{and} operator|(}

The multiplicative operators associate to the left.

\SYNTAX

\begin{rules}
\grrule{MultiplicationExpr}
       {\NT{MultiplicationExpr} \NT{MultiplicationOp} \NT{PrefixOpExpr} \OR
        \NT{MultiplicationExpr} \TXT{and} \NT{PrefixOpExpr} \OR
        \NT{PrefixOpExpr}}

\grruleoneof{MultiplicationOp}{\TXT{*~~~~/}\\
                               \TXT{\ifStd//~~~\fi div~~\ifStd mod~~\fi rem}\\
                               \TXT{band}}
\end{rules}

\EVALUATION

Evaluation of an expression \T{$\Z{E}_1$ \Z{O} $\Z{E}_2$}, where
\TZ{O} is one of the \ifStd seven \else five \fi multiplicative
operators \TXT{*}, \TXT{/},
\ifStd\TXT{//}, \fi\TXT{div}, \ifStd\TXT{mod}, \fi\TXT{rem} and
\TXT{band} or logical operator \TXT{and} begins with evaluating
the operands $\TZ{E}_1$ and $\TZ{E}_2$
\ifStd left-to-right \fi \ifOld in some order \fi
in the input environment of \T{$\Z{E}_1$ \Z{O} $\Z{E}_2$}.  Let the values
of the operands be $\TZ{v}_1$ and $\TZ{v}_2$, respectively.
See the following sections for how the result is computed for each
operator.

\ENVIRONMENTS

The output environment of the operands is the output environment of
\T{$\Z{E}_1$ \Z{O} $\Z{E}_2$}.

\subsection{Numeric multiplication operator \T{*}}

\label{section:multiplication}
\index{integer!multiplication|(}
\index{float!multiplication|(}

\begin{itemize}
\item If $\TZ{v}_1$ or $\TZ{v}_2$ is not a number, then
evaluation of \T{$\Z{E}_1$ * $\Z{E}_2$} exits with \T{\badarith}.
\item Otherwise, if both $\TZ{v}_1$ and $\TZ{v}_2$ are integers,
$\mathit{mul}_I(\Er[\TZ{v}_1],\Er[\TZ{v}_2])$ is
computed; let the result be $r$.
\item Otherwise, the terms $\TZ{v}_1$ and $\TZ{v}_2$ are coerced to floats
(\S\ref{section:coercion});
let $(\TZ{w}_1,\TZ{w}_2)=\mathit{coerce}(\TZ{v}_1,\TZ{v}_2)$.
Next $\mathit{mul}_F(\Er[\TZ{w}_1],\Er[\TZ{w}_2])$ is
computed; let the result be $r$.
\end{itemize}
If $r$ is a number, the value of \T{$\Z{E}_1$ * $\Z{E}_2$} is $\Re[r]$;
otherwise, evaluation of \T{$\Z{E}_1$ * $\Z{E}_2$} exits with $\Re[r]$.
\index{integer!multiplication|)}
\index{float!multiplication|)}

\subsection{Float division operator \T{/}}

\label{section:floatdiv}
\index{float!division|(}

\begin{itemize}
\item If $\TZ{v}_1$ or $\TZ{v}_2$ is not a number, then
evaluation of \T{$\Z{E}_1$ / $\Z{E}_2$} exits with \T{\badarith}.
\item Otherwise, the terms $\TZ{v}_1$ and $\TZ{v}_2$ are both coerced to floats
(\S\ref{section:coercion}); let $(\TZ{w}_1,\TZ{w}_2)=
(\mathit{toFloat}(\TZ{v}_1),\mathit{toFloat}(\TZ{v}_2))$.
Next $\mathit{div}_F(\Er[\TZ{w}_1],\Er[\TZ{w}_2])$ is computed; let the result be $r$.
\end{itemize}
If $r$ is a number, the value of \T{$\Z{E}_1$ / $\Z{E}_2$} is $\Re[r]$;
otherwise, evaluation of \T{$\Z{E}_1$ / $\Z{E}_2$} exits with $\Re[r]$.
\index{float!division|)}

\ifStd
\subsection{Integer division operator \T{//}}

\label{section:intdiv-f}
\index{integer!division|(}

\begin{itemize}
\item If $\TZ{v}_1$ or $\TZ{v}_2$ is not an integer, then
evaluation of \T{$\Z{E}_1$ // $\Z{E}_2$} exits with \T{\badarith}.
\item Otherwise, $\mathit{div}_I^f(\Er[\TZ{v}_1],\Er[\TZ{v}_2])$ is computed;
let the result be $r$.
\end{itemize}
If $r$ is a number, the value of \T{$\Z{E}_1$ // $\Z{E}_2$} is $\Re[r]$;
otherwise, evaluation of \T{$\Z{E}_1$ // $\Z{E}_2$} exits with $\Re[r]$.
\index{integer!division|)}
\fi

\subsection{Integer division operator \T{div}}

\label{section:intdiv}
\index{integer!division|(}

\begin{itemize}
\item If $\TZ{v}_1$ or $\TZ{v}_2$ is not an integer, then
evaluation of \T{$\Z{E}_1$ div $\Z{E}_2$} exits with \T{\badarith}.
\item Otherwise, $\mathit{div}_I^t(\Er[\TZ{v}_1],\Er[\TZ{v}_2])$ is computed;
let the result be $r$.
\end{itemize}
If $r$ is a number, the value of \T{$\Z{E}_1$ div $\Z{E}_2$} is $\Re[r]$;
otherwise, evaluation of \T{$\Z{E}_1$ div $\Z{E}_2$} exits with $\Re[r]$.
\index{integer!division|)}

\ifStd
\subsection{Integer modulo operator \T{mod}}

\label{section:intmod}
\index{integer!modulo|(}

\begin{itemize}
\item If $\TZ{v}_1$ or $\TZ{v}_2$ is not an integer, then
evaluation of \T{$\Z{E}_1$ mod $\Z{E}_2$} exits with \T{\badarith}.
\item Otherwise, $\mathit{rem}_I^f(\Er[\TZ{v}_1],\Er[\TZ{v}_2])$ is computed;
let the result be $r$.
\end{itemize}
If $r$ is a number, the value of \T{$\Z{E}_1$ mod $\Z{E}_2$} is $\Re[r]$;
otherwise, evaluation of \T{$\Z{E}_1$ mod $\Z{E}_2$} exits with $\Re[r]$.
\index{integer!modulo|)}
\fi

\subsection{Integer remainder operator \T{rem}}

\label{section:intrem}
\index{integer!remainder|(}

\begin{itemize}
\item If $\TZ{v}_1$ or $\TZ{v}_2$ is not an integer, then
evaluation of \T{$\Z{E}_1$ rem $\Z{E}_2$} exits with \T{\badarith}.
\item Otherwise, $\mathit{rem}_I^t(\Er[\TZ{v}_1],\Er[\TZ{v}_2])$ is computed;
let the result be $r$.
\end{itemize}
If $r$ is a number, the value of \T{$\Z{E}_1$ rem $\Z{E}_2$} is $\Re[r]$;
otherwise, evaluation of \T{$\Z{E}_1$ rem $\Z{E}_2$} exits with $\Re[r]$.
\index{integer!remainder|)}

\subsection{Integer bitwise operator \T{band}}

\label{section:bitwiseand}
\index{integer!bitwise and|(}

\begin{itemize}
\item If $\TZ{v}_1$ or $\TZ{v}_2$ is not an integer, then
      evaluation of \T{$\Z{E}_1$ band $\Z{E}_2$} exits with \T{\badarith}.
\item Otherwise, the value of \T{$\Z{E}_1$ band $\Z{E}_2$} is the integer
      that is the bitwise AND of $\TZ{v}_1$ and $\TZ{v}_2$,
      i.e., the integer that in binary two's-com\-ple\-ment
      representation has a one in those positions where the binary
      two's-com\-ple\-ment representations of both $\TZ{v}_1$ and $\TZ{v}_2$
      have a one, and a zero in the other positions.
\end{itemize}
\index{integer!bitwise and|)}

\subsection{Conjunction operator \T{and}}

\label{section:booleanand}
\index{Boolean!AND|(}

\begin{itemize}
\item If $\TZ{v}_1$ or $\TZ{v}_2$ is not a Boolean, then
      evaluation of \T{$\Z{E}_1$ and $\Z{E}_2$} exits with
      \ifStd\T{badbool}\else\T{badarg}\fi.
\item Otherwise, if both $\TZ{v}_1$ and $\TZ{v}_2$ are \T{true},
      the result is \T{true}.
\item Otherwise, the result is \T{false}.
\end{itemize}
\index{Boolean!AND|)}

\index{integer!bitwise and|)}
\index{multiplicative operators|)}
\index{* operator@\T{*} operator|)}
\index{/ operator@\T{/} operator|)}
\ifStd\index{// operator@\T{//} operator|)}\fi
\index{div operator@\T{div} operator|)}
\ifStd\index{mod operator@\T{mod} operator|)}\fi
\index{rem operator@\T{rem} operator|)}
\index{band operator@\T{band} operator|)}
\index{and operator@\T{and} operator|)}

\section{Unary operators}

\label{section:unary}
\index{unary operators|(}
\index{+ operator@\T{+} operator|(}
\index{- operator@\T{-} operator|(}
\index{bnot operator@\T{bnot} operator|(}
\index{not operator@\T{not} operator|(}

The unary operators are \T{+}, \T{-}, \T{bnot} and \T{not}.
The unary operators do not associate.  For example, the
parentheses are necessary in \T{bnot(-X)}.

\SYNTAX

\begin{rules}
\grrule{PrefixOpExpr}
       {\NT{PrefixOp} \NT{RecordExpr} \OR
        \NT{RecordExpr}}

\grruleoneof{PrefixOp}{\TXT{+~~~~-}\\
                       \TXT{bnot~not}}
\end{rules}

\EVALUATION

Evaluation of an expression \T{\Z{O} \Z{E}}, where \TZ{O} is
one of the four unary operators \TXT{+}, \TXT{-},
\TXT{bnot} and \TXT{not}, in an environment $\epsilon$ begins with evaluating
the operand \TZ{E} in $\epsilon$.  Let its value be \TZ{v}.
See the following sections for how the result is computed for each
operator.

\ENVIRONMENTS

The output environment of \TZ{E} is used as output environment of
\T{\Z{O} \Z{E}}.

\subsection{Unary plus operator \T{+}}

\label{section:unaryplus}
\index{integer!identity|(}
\index{integer!unary plus|(}
\index{float!identity|(}
\index{float!unary plus|(}

\ifStd
\begin{itemize}
\item If \TZ{v} is a number, then the value of \T{+ \Z{E}} is \TZ{v}.
\item Otherwise, the evaluation of \T{+ \Z{E}} exits with \T{\badarith}.
\end{itemize}
\fi
\ifOld
\TZ{v} is returned.\footnote{It is not intended that the unary \T{+} operator
should be applied to anything but numbers.}
\fi
\index{integer!identity|)}
\index{integer!unary plus|)}
\index{float!identity|)}
\index{float!unary plus|)}

\subsection{Unary minus operator \T{-}}

\label{section:unaryminus}
\index{integer!negation|(}
\index{integer!unary minus|(}
\index{float!negation|(}
\index{float!unary minus|(}

The type of the result depends on the type of $\TZ{v}$.
\begin{itemize}
\item If $\TZ{v}$ is not a number,
evaluation of \T{- \Z{E}} exits with \T{badarith}.
%$\mathit{neg}_I(\Er[\TZ{v}])$ is computed; let the result be $r$.
\item Otherwise, if $\TZ{v}$ is an integer,
$\mathit{neg}_I(\Er[\TZ{v}])$ is computed; let the result be $r$.
\item Otherwise $\TZ{v}$ is a float,
$\mathit{neg}_F(\Er[\TZ{v}])$ is computed; let the result be $r$.
\end{itemize}
If $r$ is a number, then the value of \T{- \Z{E}} is $\Re[r]$;
otherwise, evaluation of \T{- \Z{E}} exits with $\Re[r]$.
\index{integer!negation|)}
\index{integer!unary minus|)}
\index{float!negation|)}
\index{float!unary minus|)}

\subsection{Bitwise complement operator \T{bnot}}

\label{section:bitwisecomp}
\index{integer!bitwise complement|(}
\index{integer!bitwise negation|(}

\begin{itemize}
\item If \TZ{v} is an integer, then the value of \T{bnot \Z{E}} is $(-\TZ{v})-1$
(Note that in two's-com\-ple\-ment representation this numeral is the bitwise
complement of \TZ{v} and that $(-\TZ{v})-1$ may be representable even when $-\TZ{v}$ is not)
\item Otherwise, the evaluation of \T{bnot \Z{E}} exits with \T{\badarith}.
\end{itemize}
\index{integer!bitwise complement|)}
\index{integer!bitwise negation|)}

\subsection{Boolean complement operator \T{not}}

\label{section:booleannot}
\index{Boolean!complement|(}
\index{Boolean!negation|(}

\begin{itemize}
\item If \TZ{v} is the atom \T{true}, then the value of \T{not \Z{E}}
      is \T{false}.
\item If \TZ{v} is the atom \T{false}, then the value of \T{not \Z{E}}
      is \T{true}.
\item Otherwise, the evaluation of \T{not \Z{E}} exits with
      \ifStd\T{badbool}\else\T{badarg}\fi.
\end{itemize}
\index{Boolean!complement|)}
\index{Boolean!negation|)}

\index{unary operators|)}
\index{+ operator@\T{+} operator|)}
\index{- operator@\T{-} operator|)}
\index{bnot operator@\T{bnot} operator|)}
\index{not operator@\T{not} operator|)}

\section{Record expressions}

\label{section:record-exprs}
\index{record!expression|(}
\index{# separator@\T{\char`\#} separator|(}

A record declaration (\S\ref{section:record-declarations})
\begin{alltt}
-record(\Z{R},\{\(\Z{F}\sb{1}[\mbox{\T{=\(\Z{E}\sb{1}\)}}],\tdots,\Z{F}\sb{n}[\mbox{\T{=\(\Z{E}\sb{n}\)}}]\)\})
\end{alltt}
establishes \TZ{R} as a record type with $n$ named fields $\TZ{F}_1$,
\ldots, $\TZ{F}_n$.  The compiler decides an invertible mapping for
\TZ{R} from the field names $\TZ{F}_1$, \ldots, $\TZ{F}_n$ to the
integers $2$, \ldots, $n+1$.  Let us call this mapping
$\mathit{record\_field}_{\TZm{R}}$\index{record_field@$\mathit{record\_field}$}
and its inverse
$\mathit{record\_field}_{\TZm{R}}^{-1}$\index{record_field
-1@$\mathit{record\_field}^{-1}$}.
\index{record!is a tuple|(}
A record term of type \TZ{R} is then represented by a tuple with $n+1$
elements where the first element is the atom
\TZ{R} and element $i$ contains the value for field
$\mathit{record\_field}_{\TZm{R}}^{-1}(i)$, where $2\leq i\leq n+1$.
\index{record!is a tuple|)}

The value of $\mathit{record\_field}_{\TZm{R}}(\TZ{F})$ is available as
\T{\char`\#\Z{R}.\Z{F}}. For most purposes it is sufficient and
appropriate to use this mapping only indirectly through record element
access and record update expressions (see below).

\SYNTAX

\begin{rules}
\grrule{RecordExpr}
       {\OPT{RecordExpr} \TXT{\char`\#} \NT{RecordType} \TXT{.}\ \NT{RecordFieldName} \OR
        \OPT{RecordExpr} \TXT{\char`\#} \NT{RecordType} \NT{RecordUpdateTuple} \OR
        \NT{ApplicationExpr}}

\grrule{RecordUpdateTuple}
       {\TXT{\char`\{} \OPT{RecordFieldUpdates} \TXT{\char`\}}}

\grrule{RecordFieldUpdates}
       {\NT{RecordFieldUpdate} \OR
        \NT{RecordFieldUpdates} \TXT{,}\ \NT{RecordFieldUpdate}}

\grrule{RecordFieldUpdate}
       {\NT{RecordFieldName} \NT{RecordFieldValue}}

\grrule{RecordFieldValue}
       {\TXT{=} \NT{Expr}}
\end{rules}
(The rules for \NT{RecordType} and \NT{RecordFieldName} appear
in \S\ref{section:pattern-matching}.)

There are four kinds of record expression and we will describe them one by one.

\subsection{Record field index}

\label{section:field-index}
\index{record!field index expression|(}
A record field index expression gives the position of a field in the tuple
that is the record.

\EVALUATION

The value of an expression \T{\char`\#\Z{R}.\Z{F}}, where \TZ{R} is a
record name and \TZ{F} is the name of a field in \TZ{R}, is the
integer $\mathit{record\_field}_{\TZm{R}}(\TZ{F})$.

It is a compile-time error if the expression is not in the scope of
a declaration of a record type \TZ{R} having a field \TZ{F}.

\ENVIRONMENTS

The output environment is the same as the input environment.

\NOTE

\ifStd An implementation may treat \fi
\ifOld \OldErlang\ treats \fi
a record field index expression \T{\char`\#\Z{R}.\Z{F}}
as syntactic sugar for an integer literal $\mathit{record\_field}_{\TZm{R}}(\TZ{F})$.
\index{record!field index expression|)}

\subsection{Record field access}

\label{section:field-access}
\index{record!field access expression|(}
A record field access expression extracts one field of a record.

\EVALUATION

In the scope of a record declaration that establishes \TZ{R} as
a record type with $n$ fields, the evaluation of an expression
\T{\Z{E}\char`\#\Z{R}.\Z{F}}, where \TZ{F} is the name of a field in \TZ{R},
begins with
evaluating the expression \TZ{E}; let its value be \TZ{v}.
\begin{itemize}
\item If \TZ{v} is a tuple with $n+1$ elements and element $1$
of \TZ{v} is \TZ{R}, the value of the whole expression is
element $\mathit{record\_field}_{\TZm{R}}(\TZ{F})$ of \TZ{v}.
\item Otherwise, evaluation of the whole expression exits
with \T{badarg}.
\end{itemize}
It is a compile-time error if the expression is not in the scope of
a declaration of a record type \TZ{R} having a field \TZ{F}.

\ENVIRONMENTS

The input environment of the whole expression is used as input environment
of \TZ{E} and the output environment of \TZ{E} is used as output environment of the
whole expression.

\NOTE

\ifStd An implementation may treat \fi
\ifOld \OldErlang\ treats \fi
\T{\Z{E}\char`\#\Z{R}.\Z{F}}
as syntactic sugar for a function application
\T{element($\mathit{record\_field}_{\TZm{R}}(\TZ{F})$,\Z{E})}.
\index{record!field access expression|)}

\subsection{Record creation}

\label{section:record-creation}
\index{record!creation expression|(}

A record creation expression creates a new record with field values as specified
or according to the record declaration.

\EVALUATION

In the scope of a record declaration that establishes \TZ{R} as
a record type with $n$ fields, an expression
\T{\char`\#\Z{R}\{$\Z{F}_1$=$\Z{E}_1$,\tdots,$\Z{F}_k$=$\Z{E}_k$\}},
where \TZ{R} is a record name and $\TZ{F}_1$, \ldots, $\TZ{F}_k$ are
distinct names of fields in \TZ{R}, is evaluated exactly as if it
were a tuple skeleton
\begin{textdisplay}
\T{\{\Z{R},$\Z{E}'_1$,\tdots,$\Z{E}'_n$\}}
\end{textdisplay}
where each expression $\TZ{E}'_i$, $1\leq i\leq n$, is as follows:
\begin{itemize}
\item If there is an integer
$j$, $1\leq j\leq k$, such that $\mathit{record\_field}_{\TZm{R}}(\TZ{F}_j)=i$,
then $\TZ{E}'_i$ is $\TZ{E}_j$.
\item Otherwise, if there is a default initializer expression for
the field named $\mathit{record\_field}_{\TZm{R}}^{-1}(i)$, then
$\TZ{E}'_i$ is that expression.
\item Otherwise, $\TZ{E}'_i$ is the atom literal \T{undefined}.
\end{itemize}
It is a compile-time error if the expression is not in the scope of
a declaration of a record type \TZ{R} having at least the fields
$\TZ{F}_1$, \ldots, $\TZ{F}_k$, or if $\TZ{F}_1$, \ldots, $\TZ{F}_k$
are not distinct.

Expressions $\TZ{E}'_1$, \ldots, $\TZ{E}'_n$ are evaluated
\ifStd left-to-right\fi \ifOld in some order\fi,
let their values be $\TZ{v}'_1$, \ldots, $\TZ{v}'_n$.  The value
of the record creation expression is then a $n+1$-tuple with
the atom \TZ{R} as its first element and $\TZ{v}'_1$, \ldots, $\TZ{v}'_n$
as the remaining elements.

\ENVIRONMENTS

The input environment of the whole expression is used as input environment
of $\TZ{E}_1$, \ldots, $\TZ{E}_k$ and the output environment of
$\TZ{E}_1$, \ldots, $\TZ{E}_k$ is used as output environment of the
whole expression.  However, any default initializer expressions
must be evaluated in an empty environment and must have an empty
output environment.

\NOTE

\ifStd An implementation may treat \fi
\ifOld \OldErlang\ treats \fi
\T{\char`\#\Z{R}\char`\{$\Z{F}_1$=$\Z{E}_1$,\tdots,$\Z{F}_k$=$\Z{E}_k$\char`\}}
as syntactic sugar for the tuple skeleton described above.
\index{record!creation expression|)}

\subsection{Record update}

\label{section:record-update}
\index{record!update expression|(}
A record update expression creates a new record with field values as specified
or according to a given record.

\EVALUATION

In the scope of a record declaration that establishes \TZ{R} as
a record type with $n$ fields, the evaluation of an expression
\T{$\Z{E}_0$\char`\#\Z{R}\{$\Z{F}_1$=$\Z{E}_1$,\tdots,$\Z{F}_k$=$\Z{E}_k$\}},
where \TZ{R} is a record name and $\TZ{F}_1$, \ldots, $\TZ{F}_k$ are
distinct names of fields in \TZ{R}, begins with
evaluating the expressions $\TZ{E}_0$, $\TZ{E}_1$, \ldots,
$\TZ{E}_k$ \ifStd left-to-right\fi \ifOld in some order\fi.  Let their values be
$\TZ{v}_0$, $\TZ{v}_1$, \ldots, $\TZ{v}_k$.
\begin{itemize}
\item If $\TZ{v}_0$ is not a tuple with $n+1$ elements, or its first element is
not the atom \TZ{R}, then evaluation of the record update expression exits
with \T{badarg}.
\item Otherwise, the value of the whole expression is
an $n+1$-tuple where element $1$ is the atom \TZ{R} and
where for each $i$, $2\leq i\leq n+1$,
element $i$ is obtained as follows:
\begin{itemize}
\item If there is an integer
$j$, $1\leq j\leq k$, such that $\mathit{record\_field}_{\TZm{R}}(\TZ{F}_j)=i$,
then element $i$ is $\TZ{v}_j$.
\item Otherwise, element $i$ is element $i$ of $\TZ{v}_0$.
\end{itemize}
\end{itemize}
It is a compile-time error if the expression is not in the scope of
a declaration of a record type \TZ{R} having at least the fields
$\TZ{F}_1$, \ldots, $\TZ{F}_k$, or if $\TZ{F}_1$, \ldots, $\TZ{F}_k$
are not distinct.

\ENVIRONMENTS

The input environment of the whole expression is used as input environment
of $\TZ{E}_0$, $\TZ{E}_1$, \ldots, $\TZ{E}_k$ and the output environment of
$\TZ{E}_0$, $\TZ{E}_1$, \ldots, $\TZ{E}_k$ is used as output environment of the
whole expression.

\NOTE

\ifStd An implementation may treat \fi
\ifOld \OldErlang\ treats \fi
\T{$\Z{E}_0$\char`\#\Z{R}\{$\Z{F}_1$=$\Z{E}_1$,\tdots,$\Z{F}_k$=$\Z{E}_k$\}}
as syntactic sugar for
\iftrue
an expression such as
\begin{alltt}
case \(\Z{E}\sb{0}\) of
    \{\Z{R},\(\Z{V}\sb{1}\),\tdots,\(\Z{V}\sb{n}\)\} ->
        \{\Z{R},\(\Z{E}\sp{\prime}\sb{1}\),\tdots,\(\Z{E}\sp{\prime}\sb{n}\)\}
    _ -> exit(badarg)
end
\end{alltt}
where each $\TZ{V}_i$ and $\TZ{E}'_i$, $1\leq i\leq n$, is as follows:
\begin{itemize}
\item If there is an integer
$j$, $1\leq j\leq k$, such that $\mathit{record\_field}_{\TZm{R}}(\TZ{F}_j)=i$,
then $\TZ{V}_i$ is a universal pattern and $\TZ{E}'_i$ is $\TZ{E}_j$.
\item Otherwise, $\TZ{V}_i$ is a new variable and $\TZ{E}'_i$ is $\TZ{V}_i$.
\end{itemize}
\else
a function application
\begin{alltt}
fun (\(\Z{V}\sb{0}\),\(\Z{V}\sb{1}\),\tdots,\(\Z{V}\sb{k}\)) ->
    case \(\Z{V}\sb{0}\) of
        \{\Z{R},\(\Z{W}\sb{1}\),\tdots,\(\Z{W}\sb{n}\)\} ->
            \{\Z{R},\(\Z{E}\sp{\prime}\sb{1}\),\tdots,\(\Z{E}\sp{\prime}\sb{n}\)\}
        _ -> exit(badarg)
    end (\(\Z{E}\sb{0}\),\(\Z{E}\sb{1}\),\tdots,\(\Z{E}\sb{k}\))
\end{alltt}
where each $\TZ{V}_i$, $0\leq i\leq k$, is a new variable
and each $\TZ{W}_i$ and $\TZ{E}'_i$, $1\leq i\leq n$, is as follows:
\begin{itemize}
\item If there is an integer
$j$, $1\leq j\leq k$, such that $\mathit{record\_field}_{\TZm{R}}(\TZ{F}_j)=i$,
then $\TZ{W}_i$ is a universal pattern and $\TZ{E}'_i$ is $\TZ{V}_j$.
\item Otherwise, $\TZ{W}_i$ is a new variable and $\TZ{E}'_i$ is $\TZ{W}_i$.
\end{itemize}
\fi
A \emph{new variable} is a variable not in the input domain.
Moreover, all new variables are distinct.
\index{record!update expression|)}
\index{record!expression|)}
\index{# separator@\T{\char`\#} separator|)}

\section{Function application expressions}

\label{section:application-exprs}
\index{function!application!expression|(}

There are three forms of function application expressions: for applying a
named function in the same module, for applying a named and exported function
in a different module, and for applying the value of a function expression.
There are also BIFs for function application, cf.\
\ifOld\S\ref{section:process-bifs}\fi
\ifStd\S\ref{section:process-module}\fi.  Syntactically the first form
is included in the third form.

\SYNTAX

\begin{rules}
\grrule{ApplicationExpr}
       {\NT{PrimaryExpr} \TXT{(} \OPT{Exprs} \TXT{)} \OR
        \NT{PrimaryExpr} \TXT{:} \NT{PrimaryExpr} \TXT{(} \OPT{Exprs} \TXT{)} \OR
        \NT{PrimaryExpr}}
\end{rules}

\EVALUATION

There are three forms of application expressions and we will describe
them one by one.
\begin{itemize}

\item An expression on the form
\T{\Z{Fun}($\Z{E}_1$,\tdots,$\Z{E}_{\Zm{k}}$)}, where \TZ{Fun} is a
function name, is a
\emph{local function application}\index{function!application!local}.
It is evaluated by evaluating the expressions $\TZ{E}_1$, \ldots,
$\TZ{E}_{\TZm{k}}$
\ifStd left-to-right \fi \ifOld in some order \fi and then
as described in case~\ref{item:function-application-atom} of
\S\ref{section:function-application}.  It is a compile-time error if
there is no function named \T{\Z{Fun}/\Z{k}} in the lexically
enclosing module declaration.

\item An expression on the form
\T{$\Z{E}_m$:$\Z{E}_0$($\Z{E}_1$,\tdots,$\Z{E}_{\TZm{k}}$)} where
$\TZ{E}_m$ and $\TZ{E}_0$ evaluate to atoms is a
\emph{remote function application}\index{function!application!remote}.
It is evaluated by evaluating the expressions $\TZ{E}_1$, \ldots,
$\TZ{E}_{\TZm{k}}$
\ifStd left-to-right \fi \ifOld in some order \fi and then
as described in case~\ref{item:explicit-mod-fun} of
\S\ref{section:function-application}.

\item The evaluation of an expression on the form \T{$\Z{E}_0$($\Z{E}_1$,\tdots,$\Z{E}_{\Zm{k}}$)} (where
$\TZ{E}_0$ is not an atomic literal) begins with
evaluating the expressions $\TZ{E}_0$, $\TZ{E}_1$, \ldots, $\TZ{E}_{\TZm{k}}$
\ifStd left-to-right\fi \ifOld in some order\fi.
Then it depends on the value of $\TZ{E}_0$:
\begin{itemize}

\item If the value of $\TZ{E}_0$ is the result of evaluating an
implicit \T{fun} expression (\S\ref{section:fun-exprs}), then
evaluation proceeds as described in
case~\ref{item:function-application-impl-fun} of
\S\ref{section:function-application}.

\item If the value of $\TZ{E}_0$ is the result of evaluating an
explicit \T{fun} expression (\S\ref{section:fun-exprs}), then
evaluation proceeds as described in
case~\ref{item:function-application-expl-fun} of
\S\ref{section:function-application}.

\item If the value of $\TZ{E}_0$ is a 2-tuple \T{\{\Z{Mod},\Z{Fun}\}},
where \TZ{Mod} and \TZ{Fun} are atoms, then the expression is a remote
function application and is evaluated exactly as if it were an
application expression
\T{\Z{Mod}:\Z{Fun}($\Z{E}_1$,\tdots,$\Z{E}_{\Zm{k}}$)}.
(see above).  Note that this is form is obsolete and should not be
used.  It is only kept for backwards compatibility may not be
supported in the future.

\item Otherwise, evaluation of the application expression exits with
\T{badfun}.
\end{itemize}

\end{itemize}

\ENVIRONMENTS

The input environment of the whole expression is used as input environment
of [[$\TZ{E}_m$,] $\TZ{E}_0$,] $\TZ{E}_1$, \ldots, $\TZ{E}_k$ and the output
environment of
[[$\TZ{E}_m$,] $\TZ{E}_0$,] $\TZ{E}_1$, \ldots, $\TZ{E}_k$
is used as output environment of the whole expression.
\index{function!application!expression|)}

\section{Primary Expressions}

\label{section:primary-exprs}
\index{primary expressions|(}
The primary expressions are the building blocks from which other
expressions are constructed.  Note that many of them are themselves
compound but the enclosed expressions can be thought of as being
parenthesized so primary expressions have infinite precedence.

\begin{rules}
\grrule{PrimaryExpr}
       {\NT{Variable} \OR
        \NT{AtomicLiteral} \OR
        \NT{TupleSkeleton} \OR
        \NT{ListSkeleton} \OR
        \NT{ListComprehension} \OR
        \NT{BlockExpr} \OR
        \ifStd\NT{AllTrueExpr} \OR
        \NT{SomeTrueExpr} \OR\fi
        \ifStd\NT{CondExpr} \OR\fi
        \NT{IfExpr} \OR
        \NT{CaseExpr} \OR
        \NT{ReceiveExpr} \OR
        \ifStd\NT{TryExpr} \OR\fi
        \NT{FunExpr} \OR
        \ifOld\NT{QueryExpr} \OR\fi
        \NT{ParenthesizedExpr}}
\end{rules}

\subsection{Variables}

\label{section:variables-eval}

\index{variable!expression|(}
An applied occurrence of a variable\index{variable!applied occurrence}
has its value given by a variable
binding in the current environment\index{environment}.
The requirement to verify at compile time
that every applied variable occurrence is in the input context\index{context!input}
at the occurrence (\S\ref{section:REB})
guarantees that there must be such a binding at run time.

\SYNTAX

A \NT{Variable} is a \NT{Token} and its syntax is described in \S\ref{section:variables}.

\EVALUATION

Evaluation of an expression \TZ{V}, where \TZ{V} is a variable,
consists of looking up the value for \TZ{V} in the input environment.
If there is no binding for \TZ{V} in the input environment
(\S\ref{section:environments}), then it is
a compile-time error (\S\ref{section:REB}, \S\ref{section:scope}).

\ENVIRONMENTS

The input and output environments of a variable are the same.
\index{variable!expression|)}

\subsection{Atomic literals}

\label{section:atomic-literals}
\index{atomic literal|(}

An \emph{atomic literal} always denotes the same value regardless of
the context so no evaluation is necessary to determine which term an
atomic literal denotes.  All atomic literals except string literals
are elementary terms.

\SYNTAX

\begin{rules}
\grrule{AtomicLiteral}
       {\NT{IntegerLiteral} \OR
        \NT{FloatLiteral} \OR
        \NT{CharLiteral} \OR
        \NT{StringLiterals} \OR
        \NT{AtomLiteral}}

\grrule{StringLiterals}
       {\NT{StringLiteral} \OR
	\NT{StringLiterals} \NT{StringLiteral}}
\end{rules}

\EVALUATION

The compiler can be expected to compile a literal into code that directly
creates the internal representation of the term.

\begin{itemize}
\item An atom literal always denotes an atom, cf.~\S\ref{section:atoms}.
\item An integer literal always denotes an integer, cf.~\S\ref{section:integers}.
\item A float literal always denotes a float, cf.~\S\ref{section:floats}.
\item A character literal always denotes a character, cf.~\S\ref{section:chars}.
\item A string literal always denotes a string, cf.~\S\ref{section:strings}.
\end{itemize}

Note that a string literal may be written as several subsequent stubs,
which are nevertheless thought of as constituting a single string
literal.  For example,
\begin{verbatim}
"this " "is o" "ne string"
\end{verbatim}
is indistinguishable from the
string literal
\begin{alltt}
"this is one string"\R{.}
\end{alltt}
This suggests one way to write a string literal on more than one line:
dividing it into several parts and breaking the lines (and adding
indentation) between them.

\ENVIRONMENTS

The input and output environments of an atomic literal are the same.
\index{atomic literal|)}

\subsection{Tuple skeletons}

\label{section:tuple-skeletons}
\index{tuple!skeleton|(}

For a tuple skeleton, the ``surface structure'' is obvious and no
evaluation is needed to obtain it.  If all immediate subexpressions of
a tuple skeleton are literals, the tuple skeleton is itself a (tuple)
literal.

\SYNTAX

\begin{rules}
\grrule{TupleSkeleton}
       {\TXT{\char`\{} \OPT{Exprs} \TXT{\char`\}}}
\end{rules}

\ifOld
A tuple skeleton may have at most 255 elements so tuples with more elements
must be created in some other way than through syntax (e.g., by calling the BIF
\T{list_to_tuple/1}\index{list_to_tuple/1@\T{list_to_tuple/1}}
[\S\ref{section:listtotuple1}]).
\fi

\EVALUATION

For a tuple skeleton it is always obvious that it denotes a tuple with
$k$ elements, for some natural number $k$.

Evaluation of an expression \T{\char`\{$\Z{E}_1$,\tdots,$\Z{E}_k$\char`\}}
begins with
evaluating the immediate subexpressions $\TZ{E}_1$, \ldots, $\TZ{E}_k$
\ifStd left-to-right \fi \ifOld in some order \fi
in the input environment of \T{\char`\{$\Z{E}_1$,\tdots,$\Z{E}_k$\char`\}}.
Let the values of the immediate subexpressions be $\TZ{v}_1$, \ldots, $\TZ{v}_k$.
The value of the tuple skeleton is then a $k$-tuple mapping $i$ to $\TZ{v}_i$,
$1\leq i\leq k$.

\ENVIRONMENTS

The output environment of the immediate subexpressions is used as
output environment of
\T{\char`\{$\Z{E}_1$,\tdots,$\Z{E}_k$\char`\}}.
\index{tuple!skeleton|)}

\subsection{List skeletons}

\label{section:list-skeletons}
\index{cons|(}
\index{list!skeleton|(}

(As for list patterns, ``cons skeletons'' may be a more accurate name but
the main use is for lists.)

For a list skeleton, the ``surface structure'' is obvious and no
evaluation is needed to obtain it.  If all immediate subexpressions of
a list skeleton are literals, the list skeleton is itself a (list)
literal.

\SYNTAX

\begin{rules}
\grrule{ListSkeleton}
       {\TXT{[} \TXT{]} \OR
        \TXT{[} \NT{Exprs} \OPT{ListSkeletonTail} \TXT{]}}

\grrule{ListSkeletonTail}
       {\TXT{|} \NT{Expr}}
\end{rules}

\EVALUATION

For a list skeleton it is always obvious that for some natural number
$k$, it denotes either
\begin{itemize}
\item a list with exactly $k$ elements (when on the
form \T{[$\Z{E}_1$,\tdots,$\Z{E}_k$]}),
\item a list with $k+l$ elements (when on the form
\T{[$\Z{E}_1$,\tdots,$\Z{E}_k$|$\Z{E}_{k+1}$]} and the value
of $\TZ{E}_{k+1}$ is a list with $l$ elements), or
\item a term that is not a list at all (when on the form
\T{[$\Z{E}_1$,\tdots,$\Z{E}_k$|$\Z{E}_{k+1}$]} and the value
of $\TZ{E}_{k+1}$ is not a list).
\end{itemize}

As for list patterns we should note that certain list skeletons
are equal:
\begin{itemize}
\item \T{[$\Z{E}_1$]} equals \T{[$\Z{E}_1$|[]]}.
\item \T{[$\Z{E}_1$,$\Z{E}_2$,\tdots,$\Z{E}_k$]} (where $k>1$) equals \T{[$\Z{E}_1$|[$\Z{E}_2$,\tdots,$\Z{E}_k$]]}.
\item \T{[$\Z{E}_1$,$\Z{E}_2$,\tdots,$\Z{E}_k$|$\Z{E}_{k+1}$]} (where $k>1$) equals \T{[$\Z{E}_1$|[$\Z{E}_2$,\tdots,\allowbreak$\Z{E}_k$|\allowbreak$\Z{E}_{k+1}$]]}.
\end{itemize}
We can therefore describe evaluation and other properties of list skeletons as if each
\NT{ListSkeleton} is either \TXT{[]} or \TXT{[} \NT{Expr} \TXT{|}
\NT{Expr} \TXT{]}.

A list skeleton \T{[]} is a literal and its value is an empty list.

An expression \T{[$\Z{E}_1$|$\Z{E}_2$]} can be thought of as if it
were an application of a cons operator.\footnote{Indeed the
corresponding language element in some related languages is a proper
(right-associative) operator.  For example, in Standard ML
\cite{milner+tofte+harper:revised-definition} the corresponding
expression is written as \T{$\Z{E}_1$ ::\ $\Z{E}_2$}.} We shall
therefore refer to $\TZ{E}_1$ and $\TZ{E}_2$ as the operands of the
expression.

Evaluation of an expression \T{[$\Z{E}_1$|$\Z{E}_2$]} begins with
evaluating the operands $\TZ{E}_1$ and $\TZ{E}_2$
\ifStd left-to-right \fi \ifOld in some order \fi
in the
input environment of \T{[$\Z{E}_1$|$\Z{E}_2$]}.  Let the values of the
operands be $\TZ{v}_1$ and $\TZ{v}_2$, respectively.  The value of the
\T{[$\Z{E}_1$|$\Z{E}_2$]} is then a cons with $\TZ{v}_1$ as its head
and $\TZ{v}_2$ as its tail.  (If $\TZ{v}_2$ is a list, then so is the
value of \T{[$\Z{E}_1$|$\Z{E}_2$]}.)

\ENVIRONMENTS

The output environment of the operands is used as output environment
of \T{[$\Z{E}_1$|$\Z{E}_2$]}.
\index{cons|)}
\index{list!skeleton|)}

\subsection{List comprehensions}

\label{section:list-comprehensions}
\index{list!comprehension|(}

A list comprehension always denotes a list, produced by evaluating an
expression for each collection of values of its variables.  These
collections of values are produced by some generators and are those
that in addition satisfy certain filters.

\SYNTAX

\begin{rules}
\grrule{ListComprehension}
       {\TXT{[} \NT{Expr} \TXT{||} \NT{ListComprehensionExprs} \TXT{]}}

\grrule{ListComprehensionExprs}
       {\NT{ListComprehensionExpr} \OR
        \NT{ListComprehensionExprs} \TXT{,} \NT{ListComprehensionExpr}}

\grrule{ListComprehensionExpr}
       {\NT{Generator} \OR
        \NT{Filter}}

\grrule{Generator}
       {\NT{Pattern} \TXT{<-} \NT{Expr}}

\grrule{Filter}
       {\NT{Expr}}
\end{rules}
The syntactic constituents of a list comprehension is a \emph{template
expression} (to the left of \T{||}) and a collection of generators and
filters that we will call the \emph{body} (to the right of \T{||}).

A nontrivial generator has at least one variable in its pattern.  Each
such variable can be expected to appear either in the template
expression or in the right-hand side of some later generator or in
some later filter.  The list is generated by substituting in the
template expression all combinations of values that are yielded by the
generators, such that the values satisfy all filters.

Each variable occurrence in the pattern of a generator is a binding
occurrence so it is a compile-time error if a variable occurs twice in such a pattern.
To the right of a generator, all variables occurring in its pattern shadow variables in
the input environment (or bound by generators to the left  of it).
The template expression of a generator is used in the environment in which all
generators have contributed their bindings, as we explain in more detail below.

\EVALUATION

We will explain the result of evaluating a list comprehension
\begin{alltt}
[ \Z{E} || \(\Z{W}\sb{1}\),\tdots,\(\Z{W}\sb{k}\) ]
\end{alltt}
in terms of a sequence of sequences of environments $\Phi_0$,
$\Phi_1$, \ldots, $\Phi_k$.

\index{generator (in list comprehension)|(}
\index{filter (in list comprehension)|(}
The initial sequence of environments, $\Phi_0$, contains exactly one
environment, which is the input environment of the list comprehension.
Each other sequence of environments $\Phi_i$ ($1\leq i\leq k$) consists
of all extensions of $\Phi_0$ that are generated by the generators in
$\TZ{W}_1$, \ldots, $\TZ{W}_i$ and that satisfy the filters in
$\TZ{W}_1$, \ldots, $\TZ{W}_i$.  The value of the list comprehension
is a list $l$ of the same length as the number $n$ of environments in
$\Phi_k$ and for each $i$, $1\leq i\leq n$, the $i$th element of $l$
is the value of the template \TZ{E} when evaluated in the $i$th environment of
$\Phi_k$.

We shall now explain how given a sequence of environments $\Phi_i$,
we obtain the next sequence of environments $\Phi_{i+1}$, where
$0\leq i<k$.

There are two cases depending on whether $\TZ{W}_{i+1}$ is a generator
or a filter.

\begin{itemize}
\item $\TZ{W}_{i+1}$ is a generator \T{$\Z{P}_{i+1}$ <- $\Z{E}_{i+1}$}.
Suppose that $\Phi_i$ consists of the environments $\epsilon_{i,1}$,
\ldots, $\epsilon_{i,m_i}$ for some natural number $m_i$.  For each
$j$ ($1\leq j\leq m_i$), let $\TZ{v}_{i,j}$ be the value of
$\TZ{E}_{i+1}$ when evaluated in $\epsilon_{i,j}$.
If there is a $j$ ($1\leq j\leq m_i$) for which $\TZ{v}_{i,j}$ is not a list,
then the computation of the list comprehension exits
with reason \T{badmatch}.  Otherwise, for each $j$ ($1\leq j\leq m_i$),
let $\Psi_{i,j}$ be the sequence of environments such that each
element of $\Psi_{i,j}$ consists of the environment $\epsilon_{i,j}$ extended
with the bindings resulting from matching the pattern $\TZ{P}_i$
against the corresponding element of the list $\TZ{v}_{i,j}$.
If $\TZ{P}_i$ does not match an element of $\TZ{v}_{i,j}$, that
element is discarded, so
$\Psi_{i,j}$ may contain fewer elements than $\TZ{v}_{i,j}$.  The order between
elements in $\TZ{v}_{i,j}$ must be preserved in $\Psi_{i,j}$.  Finally, 
$\Phi_{i+1}$ is the result of concatenating the sequences
$\Psi_{i,1}$, \ldots, $\Psi_{i,m_i}$, in order.

\item $\TZ{W}_{i+1}$ is a filter.
$\Phi_{i+1}$ is obtained by evaluating $\TZ{W}_{i+1}$ in each
environment of $\Phi_i$, keeping those environments in which the
result is \T{true} and discarding those environments in which the
result is \T{false}.  If for some environment the value of
$\TZ{W}_{i+1}$ is not a Boolean, the computation of the list
comprehension completes abruptly with reason
\ifStd\T{badbool}\else\T{badarg}\fi.  The order between
environments in $\Phi_i$ must be preserved in $\Phi_{i+1}$.
\end{itemize}

\ifStd
\index{evaluation!order of|(}
The order in which generators and filters are evaluated is not
defined.
(The description above of the result of evaluating a list comprehension
suggests a ``breadth-first'' evaluation where each $\Phi_i$ is computed
to completion before computing any part of $\Phi_{i+1}$.  An implementation
may just as well use, for example, a ``depth-first'' evaluation strategy.)
\index{evaluation!order of|)}
\fi
\index{generator (in list comprehension)|)}
\index{filter (in list comprehension)|)}

\ENVIRONMENTS

The input and output environments of a list comprehension are the same.

\EXAMPLES

The value of the list comprehension in the body
\begin{verbatim}
Y = [1,2],
Z = 42,
[{X,Y,Z,W} || X <- Y, W <- [a,b], Y <- [4,5], W <- [c,d]]
\end{verbatim}
is
\begin{verbatim}
[{1,4,42,c},{1,4,42,d},{1,5,42,c},{1,5,42,d},
 {1,4,42,c},{1,4,42,d},{1,5,42,c},{1,5,42,d},
 {2,4,42,c},{2,4,42,d},{2,5,42,c},{2,5,42,d},
 {2,4,42,c},{2,4,42,d},{2,5,42,c},{2,5,42,d}]
\end{verbatim}
Note that the generator for \T{X} is evaluated in an expression in which \T{Y} is \T{[1,2]},
that the value for \T{Z} in the input environment is never shadowed by any generator so it is
the same for every tuple in the result, and that although the leftmost generator for \T{W}
produces values for \T{W} that are not accessible in the template pattern, they cause the
results to be duplicated (as there are two possible values for the leftmost binding of \T{W}).

\index{list!comprehension|)}

\subsection{Block expressions}

\label{section:block-exprs}
\index{block expression|(}
\index{begin expression@\T{begin} expression|(}
A block expression has no effect on evaluation and is merely a way to
parenthesize and delimit a sequence of expressions, i.e., a body.  This allows
using a body where otherwise only a single expression would
be allowed.

\SYNTAX

\begin{rules}
\grrule{BlockExpr}
       {\TXT{begin} \NT{Body} \TXT{end}}
\end{rules}

\EVALUATION

Evaluating an expression \T{begin \Z{B} end} in an environment
$\epsilon$ means to evaluate the body \TZ{B} in $\epsilon$
(\S\ref{section:bodies}).

If the evaluation of \TZ{B} completes abruptly with reason \TZ{R},
evaluation of the block expression also completes abruptly with reason
\TZ{R}.  If it completes normally, the value of \T{begin \Z{B} end} is
the value of \TZ{B}.

\ENVIRONMENTS

The output environment of \TZ{B} is used as output environment of
\T{begin \Z{B} end}.
\index{block expression|)}
\index{begin expression@\T{begin} expression|)}

\ifStd
\subsection{\T{all_true} expressions}

\label{section:alltrue-exprs}
\index{all_true expression@\T{all_true} expression|(}

An \T{all_true} expression evaluates a sequence of expressions until
an expression has been found for which the value was
\T{false}, or all expressions have been evaluated.  The value of
the \T{all_true} expression is \T{false} in the former case and
\T{true} in the latter.

\SYNTAX

\begin{rules}
\ifStd
\grrule{AndTrueExpr}
       {\TXT{all_true} \NT{Exprs} \TXT{end}}
\fi
\end{rules}

\EVALUATION

Evaluating an expression
\begin{alltt}
all_true \(\Z{E}\sb{1}\) ; \ldots ; \(\Z{E}\sb{k}\) end
\end{alltt}
in an environment $\epsilon$ means to evaluate the expressions
$\TZ{E}_1$, \ldots, $\TZ{E}_k$ left-to-right with the following
additional control:
If the value $\TZ{v}_i$ of an expression $\TZ{E}_i$, $1\leq i\leq k$, is
anything but \T{true}, then
expressions $\TZ{E}_{i+1}$, \ldots, $\TZ{E}_k$ will not be evaluated and:
\begin{itemize}
\item if $\TZ{v}_i$ is \T{false}, evaluation of the \T{all_true} expression
completes normally with result \T{false};
\item otherwise, evaluation of the \T{all_true} expression exits
with reason \T{badbool}.
\end{itemize}
If all expressions are evaluated with result \T{true}, then
evaluation of the \T{all_true} expression
completes normally with result \T{true}.

\ENVIRONMENTS

The output environment of expression $\TZ{E}_1$ is used as output environment of
the \T{all_true} expression.
\index{all_true expression@\T{all_true} expression|)}

\subsection{\T{some_true} expressions}

\label{section:sometrue-exprs}
\index{some_true expression@\T{some_true} expression|(}

A \T{some_true} expression evaluates a sequence of expressions until
an expression has been found for which the value was
\T{true}, or all expressions have been evaluated.  The value of
the \T{some_true} expression is \T{true} in the former case and
\T{false} in the latter.

\SYNTAX

\begin{rules}
\ifStd
\grrule{SomeTrueExpr}
       {\TXT{some_true} \NT{Exprs} \TXT{end}}
\fi
\end{rules}

\EVALUATION

Evaluating an expression
\begin{alltt}
some_true \(\Z{E}\sb{1}\) ; \ldots ; \(\Z{E}\sb{k}\) end
\end{alltt}
in an environment $\epsilon$ means to evaluate the expressions
$\TZ{E}_1$, \ldots, $\TZ{E}_k$ left-to-right with the following
additional control:
If the value $\TZ{v}_i$ of an expression $\TZ{E}_i$, $1\leq i\leq k$, is
anything but \T{false}, then
expressions $\TZ{E}_{i+1}$, \ldots, $\TZ{E}_k$ will not be evaluated and:
\begin{itemize}
\item if $\TZ{v}_i$ is \T{true}, evaluation of the \T{some_true} expression
completes normally with result \T{true};
\item otherwise, evaluation of the \T{some_true} expression exits
with reason \T{badbool}.
\end{itemize}
If all expressions are evaluated with result \T{false}, then
evaluation of the \T{some_true} expression
completes normally with result \T{false}.

\ENVIRONMENTS

The output environment of expression $\TZ{E}_1$ is used as output environment of
the \T{some_true} expression.
\index{some_true expression@\T{some_true} expression|)}
\fi % ifStd

\subsection{If expressions}

\label{section:if-expr}
\index{if@\T{if}!expression|(}
\index{clause!of if expression@of \T{if} expression|(}

An \T{if} expression goes through a sequence of clauses, each
consisting of a guard and a body, and its value is the value of the
first body having a successful corresponding guard.

\SYNTAX

\begin{rules}
\grrule{IfExpr}
       {\TXT{if} \NT{IfClauses} \TXT{end}}

\grrule{IfClauses}
       {\NT{IfClause} \OR
        \NT{IfClauses} \TXT{;} \NT{IfClause}}

\grrule{IfClause}
       {\NT{Guard} \NT{ClauseBody}}

\grrule{ClauseBody}
       {\TXT{->} \NT{Body}}
\end{rules}
(The rule for \NT{Guard} appears in \S\ref{section:guards}.)

Each clause consists of a \emph{guard} and a \emph{body}.

\EVALUATION

Evaluation of an expression
\begin{alltt}
if
    \(\Z{G}\sb{1}\) -> \(\Z{B}\sb{1}\) ;
    \(\vdots\) ;
    \(\Z{G}\sb{k}\) -> \(\Z{B}\sb{k}\)
end
\end{alltt}
in an environment $\epsilon$ is carried out as follows.

Each guard $\TZ{G}_i$ ($1\leq i\leq k$) is evaluated
(in \ifStd that \fi \ifOld some \fi order)
in $\epsilon$, as described in \S\ref{section:guards}, until one
succeeds or all guards have failed.
Let $s$ be the smallest number such that $\TZ{G}_s$
succeeds, if such a number exists; otherwise evaluation of the
\T{if} expression exits with reason \T{if_clause}.

The value of the \T{if} expression is obtained by evaluating the body
$\TZ{B}_s$ with the output environment of $\TZ{G}_s$ as input
environment.

\ENVIRONMENTS

For each clause \T{$\Z{G}_i$ -> $\Z{B}_i$}, $1\leq i\leq k$, let $d_i$
be the domain of the output environment of $\TZ{G}_i$ when $\epsilon$
is the input environment, and let $d'_i$ be the domain of the output
environment of $\TZ{B}_i$ when $d_i$ is the domain of its input
environment.  The output environment of the \T{if}
expression is obtained as the output environment of $\TZ{B}_s$
restricted to the intersection of all $d'_i$, $1\leq i\leq k$.

\NOTE

The \T{if} expression above is equal to the following
\T{case} expression (\S\ref{section:case-expr}):
\begin{alltt}
case \Z{L} of
    _ when \(\Z{G}\sb{1}\) -> \(\Z{B}\sb{1}\) ;
    \(\vdots\) ;
    _ when \(\Z{G}\sb{k}\) -> \(\Z{B}\sb{k}\)
end
\end{alltt}
where \TZ{L} is any literal, such as \T{42}.  An \T{if} expression
is thus like a \T{case} expression but with trivial matching.
\index{if@\T{if}!expression|)}
\index{clause!of if expression@of \T{if} expression|)}

\ifStd
\subsection{Cond expressions}

\label{section:cond-expr}
\index{cond expression@\T{cond} expression|(}
\index{clause!of cond expression@of \T{cond} expression|(}

A \T{cond} expression goes through a sequence of clauses, each
consisting of a Boolean expression and a body, and its value is the
value of the first body having a true test.

\SYNTAX

\begin{rules}
\ifStd
\grrule{CondExpr}
       {\TXT{cond} \NT{CondClauses} \TXT{end}}

\grrule{CondClauses}
       {\NT{CondClause} \OR
        \NT{CondClauses} \TXT{;} \NT{CondClause}}

\grrule{CondClause}
       {\NT{Expr} \NT{ClauseBody}}
\fi
\end{rules}
(The rule for \NT{ClauseBody} appears in \S\ref{section:if-expr}.)
% % Groucho Marx would love this one...
% We refer to the sequence of clauses as the \emph{clauses} of the
% \T{cond} expression.
Each clause consists of a \emph{test} (which is a Boolean expression)
and a \emph{body}.

\EVALUATION

Evaluation of an expression
\begin{alltt}
cond
    \(\Z{E}\sb{1}\) -> \(\Z{B}\sb{1}\) ;
    \(\vdots\) ;
    \(\Z{E}\sb{k}\) -> \(\Z{B}\sb{k}\)
end
\end{alltt}
in an environment $\epsilon$ is carried out as follows.

Each test $\TZ{E}_i$ ($1\leq i\leq k$) is evaluated (in that
order) with
$\epsilon$ as input environment until one is found that
does not complete normally with \T{false} as result or
all tests have been evaluated.

\begin{itemize}
\item If evaluation of every test $\TZ{E}_i$, $1\leq i\leq k$,
completed normally with \T{false} as result, then evaluation
of the \T{cond} expression exits with reason \T{cond_clause}.
\item Otherwise, let $\TZ{E}_s$ be the first test not to complete
normally with \T{false}.  There are three possibilities:
\begin{itemize}
\item If evaluation of $\TZ{E}_s$ completed abruptly with some
reason \TZ{R}, then evaluation of the \T{cond} expression
completes abruptly with reason \TZ{R}.
\item If evaluation of $\TZ{E}_s$ completed normally and the
result was not \T{true}, then evaluation of the \T{cond} expression
completes abruptly with reason \TZ{badbool}.
\item If evaluation of $\TZ{E}_s$ completed normally with
result \T{true}, then the value of the \T{cond} expression
is obtained by evaluating the body $\TZ{B}_s$ with the
output environment of $\TZ{E}_s$ as input environment.
\end{itemize}
\end{itemize}

\ENVIRONMENTS

For each clause \T{$\Z{E}_i$ -> $\Z{B}_i$}, $1\leq i\leq k$, let $d_i$
be the domain of the output environment of $\TZ{E}_i$ when $\epsilon$
is the input environment, and let $d'_i$ be the domain of the output
environment of $\TZ{B}_i$ when $d_i$ is the domain of its input
environment.  The output environment of the \T{cond}
expression is obtained as the output environment of $\TZ{B}_s$
restricted to the intersection of all $d'_i$, $1\leq i\leq k$.
\index{cond expression@\T{cond} expression|)}
\index{clause!of cond expression@of \T{cond} expression|)}
\fi	%\ifStd

\subsection{Case expressions}

\label{section:case-expr}
\index{case expression@\T{case} expression|(}
\index{clause!of case expression@of \T{case} expression|(}

A \T{case} expression chooses between sequences of expressions to
evaluate, depending on the value of some expression.

\SYNTAX

\begin{rules}
\grrule{CaseExpr}
       {\TXT{case} \NT{Expr} \TXT{of} \NT{\CrtClauses} \TXT{end}}

\grrule{\CrtClauses}
       {\NT{\CrtClause} \OR
        \NT{\CrtClauses} \TXT{;} \NT{\CrtClause}}

\grrule{\CrtClause}
       {\NT{Pattern} \OPT{ClauseGuard} \NT{ClauseBody}}

\grrule{ClauseGuard}
       {\TXT{when} \NT{Guard}}

\grrule{Guard}
       {\NT{Body}}
\end{rules}
(The rule for \NT{ClauseBody} appears in \S\ref{section:if-expr}.)

We refer to the expression between \T{case} and \T{of} as the \emph{switch
expression} and to the sequence of clauses to the right of \T{of} as
the \emph{clauses} of the \T{case} expression.  Each clause consists
of a \emph{pattern}, an optional \emph{guard} and a \emph{body}.
An omitted guard is equivalent with a trivially satisfied \T{true} guard.

\EVALUATION

Evaluation of an expression
\begin{alltt}
case \Z{E} of
    \(\Z{P}\sb{1}\) \([\)when \(\Z{G}\sb{1}]\) -> \(\Z{B}\sb{1}\) ;
    \(\vdots\) ;
    \(\Z{P}\sb{k}\) \([\)when \(\Z{G}\sb{k}]\) -> \(\Z{B}\sb{k}\)
end
\end{alltt}
in an environment $\epsilon$ is carried out as follows.

First the switch expression \TZ{E} is evaluated in $\epsilon$.  If
that evaluation completes abruptly with reason \TZ{R}, then evaluation
of the \T{case} expression also completes abruptly with reason \TZ{R}.
If the evaluation of the switch expression completes normally, let us
call its value \TZ{v} and its output environment $\epsilon'$.

Next each pattern $\TZ{P}_i$ ($1\leq i\leq k$) is matched
(in \ifStd that \fi \ifOld some \fi order)
against \TZ{v} in $\epsilon'$ and (in case of a successful
match) the corresponding guard $\TZ{G}_i$ is evaluated in the output
environment of $\TZ{P}_i$.  Let $s$ be the smallest number such that
$\TZ{P}_s$ matches \TZ{v} and $\TZ{G}_s$ succeeds, if such a number
exists; otherwise evaluation of the \T{case} expression exits with
reason \T{case_clause}.

The value of the \T{case} expression is obtained by evaluating the
body $\TZ{B}_s$ in the output environment of $\TZ{G}_s$.

\ENVIRONMENTS

For each clause \T{$\Z{P}_i$ when $\Z{G}_i$ -> $\Z{B}_i$}, $1\leq
i\leq k$, let $d_i$ be the domain of the output environment of
$\TZ{P}_i$ when $\epsilon'$ is its input environment, let $d'_i$ be
the domain of the output environment of $\TZ{G}_i$ when $d_i$ is the
domain of its input environment, and let $d''_i$ be the domain of the
output environment of $\TZ{B}_i$ when $d'_i$ is the domain of its
input environment.  The output environment of the \T{case} expression
is obtained as the output environment of $\TZ{B}_s$ restricted to the
intersection of all $d''_i$, $1\leq i\leq k$.

\NOTE

The name \NT{\CrtClause} refers to the fact that \NT{CaseExpr}%
\ifStd, \NT{ReceiveExpr} and \NT{TryExpr} all \else\ and \NT{ReceiveExpr} both \fi
have the same kind of clauses.)
\index{case expression@\T{case} expression|)}
\index{clause!of case expression@of \T{case} expression|)}

\subsection{Receive expressions}

\label{section:receive-expr}
\index{receive expression@\T{receive} expression|(}
\index{message!reception of|(}
\index{expiry!of \T{receive} expression|(}
\index{clause!of receive expression@of \T{receive} expression|(}

A \T{receive} expression normally consumes one message from the
message queue of the process evaluating it.  The exception is when the
\T{receive} expression specifies an expiry time, there is no suitable
message waiting when the processing of the \T{receive} expression
begins and the specified amount of time passes before such a message
arrives. \T{receive} expressions are similar to \T{case} expressions
(S\ref{section:case-expr}) both in syntax and in semantics but cannot
be defined in terms of them in a useful way (because a \T{receive}
expression matches expressions against the message queue without
removing them unless there is a matching clause).

\SYNTAX

\begin{rules}
\grrule{ReceiveExpr}
       {\TXT{receive} \NT{\CrtClauses} \TXT{end} \OR
        \TXT{receive} \OPT{\CrtClauses} \TXT{after} \NT{Expr} \NT{ClauseBody} \TXT{end}}
\end{rules}
(The rules for \NT{\CrtClauses} and \NT{ClauseBody} appear
in \S\ref{section:case-expr} and \S\ref{section:if-expr}, respectively.)

We refer to the sequence of clauses as the \emph{clauses} of the
\T{receive} expression.  Each clause consists of a \emph{pattern}, an optional
\emph{guard} and a \emph{body}.
An omitted guard is equivalent with a trivially satisfied \T{true} guard.
A \T{receive} expression containing
a part \T{after \Z{E} -> \Z{B}} is said to have an \emph{expiry part}.
Then \TZ{E} is called the \emph{expiry expression} of the
\T{receive} expression and \TZ{B} is called the \emph{expiry body}.
A clause must either have at least one clause or an expiry part, or both.

\EVALUATION

Evaluation of an expression
\begin{alltt}
receive
    \(\Z{P}\sb{1}\) \([\)when \(\Z{G}\sb{1}]\) -> \(\Z{B}\sb{1}\) ;
    \(\vdots\) ;
    \(\Z{P}\sb{k}\) \([\)when \(\Z{G}\sb{k}]\) -> \(\Z{B}\sb{k}\)
\([\)after
    \Z{E} -> \(\Z{B}\sb{k+1}]\)
end
\end{alltt}
(where $k$ may be zero, in which case there must be an expiry part) in
an environment $\epsilon$ is carried out as follows by a process
\TZ{Q}.

The evaluation of a \T{receive} expression has three parts.  The first
part only takes place if the \T{receive} expression has an expiry part.
The second part is the same for all \T{receive} expressions.
The third part is somewhat different if the \T{receive} expression
has an expiry part.

\begin{itemize}
\item \B{Part 1.}
The purpose of this part is to determine the expiry time of the
\T{receive} expression, if it has an expiry part.

If the \T{receive} expression has an expiry part, the expiry
expression \TZ{E} is evaluated in $\epsilon$.  If that evaluation
completes abruptly with reason \TZ{R}, then evaluation of the
\T{receive} expression also completes abruptly with reason \TZ{R}.  If
the evaluation of the expiry expression completes normally, let us
call its value \TZ{WaitingTime}.  If \TZ{WaitingTime} is neither the
atom \T{infinity}, nor a nonnegative integer, the evaluation of the
\T{receive} expression completes abruptly with reason \T{badarg}.
\TZ{WaitingTime} will be used in part 3, if evaluation reaches that part.
\ifOld In \OldErlang\ \TZ{WaitingTime} must be a fixnum. \fi
% should we state a lowest limit on WaitingTime???
Next the system clock of \T{node[\Z{Q}]} is read before anything else
happens.  Let its value be \TZ{Start}.

Note that the output environment of the expiry expression is not used.
Its variable bindings are therefore local.\footnote{From the syntax it
might seem as if bindings of the expiry expression ought to be visible
in the expiry body but if the bindings of the expiry expression were
visible at all, they could then just as well be visible for all
clauses.  A careful approach has therefore been to make all its
bindings local.}

\item \B{Part 2.}
The purpose of this part is to process an existing message if there
is one in the queue.

Suppose that \T{message_queue[\Z{Q}]} contains
the $n$ terms $\TZ{M}_1$, \ldots, $\TZ{M}_n$, in that order.
\begin{textdisplay}
For each term $\TZ{M}_j$ ($1\leq j\leq n$, in that order):
\begin{itemize}
\item[] for each clause $i$ ($1\leq j\leq k$, in \ifStd that \fi \ifOld some \fi order):
\begin{itemize}
\item[] match $\TZ{P}_i$ against $\TZ{M}_j$ in $\epsilon$ and if that
succeeds, evaluate $\TZ{G}_i$ in the output environment of $\TZ{P}_i$,
\end{itemize}
\end{itemize}
\end{textdisplay}
until a (first) term $\TZ{M}_t$ has been found for which there is a
\ifStd (first) \fi matching pattern $\TZ{P}_s$ with successful guard $\TZ{G}_s$,
if they exist at all; otherwise evaluation of the
\T{receive} expression continues with part~3 below.

\iffalse
Each term $\TZ{M}_j$ ($1\leq j\leq n$) is matched (in that order)
against each pattern $\TZ{P}_i$ ($1\leq i\leq k$, in that
order) and (in case of a successful match) the corresponding
guard $\TZ{G}_i$ is evaluated.  The input environment of each pattern matching
is $\epsilon$ and the output environment of each matching is the
input environment of the corresponding guard.
Let $t$ be the smallest number
for which there exists a smallest number $s$ such that
$\TZ{P}_s$ matches $\TZ{M}_t$ and $\TZ{G}_s$ succeeds, if such a number $t$ exists;
otherwise evaluation of the
\T{receive} expression continues with part~3 below.
\fi

The term $\TZ{M}_t$ is removed from \T{message_queue[\Z{Q}]}.

The value of the \T{receive} expression is then obtained by evaluating
the body $\TZ{B}_s$ in the output environment of $\TZ{G}_s$.

(Note that if necessary, every message in the message queue will be
tried against every clause, even if there is an expiry.)

\item \B{Part 3.}
The purpose of this part is to wait for a receivable term to appear in
the message queue, or to expire.
\iffalse
We define two actions:
\begin{Lentry}
\item[Maybe new message] If there is some message in
\T{message_queue[\Z{Q}]} that has not previously been examined,
process the first such message as in part~2. (This may complete the
evaluation of the \T{receive} expression.)
\item[Maybe timeout] The system clock of \T{node[\Z{Q}]} is read.
If its value is greater than or equal to
\T{\Z{Start}+\T{WaitingTime}}, the evaluation of the \T{receive}
expression finishes by letting $s$ be $k+1$ and evaluating the expiry body $\TZ{B}_s$ with
$\epsilon$ as input environment.  If its evaluation completes abruptly with reason \TZ{R},
then evaluation of the
\T{receive} expression also completes abruptly with reason \TZ{R}.  If
the evaluation of the expiry body completes normally, its value is also the
value of the \T{receive} expression.
\end{Lentry}
\fi
There are three alternatives:
\begin{enumerate}
\item If the \T{receive} expression has no expiry part, or \TZ{WaitingTime}
is the atom \T{infinity}, then:
\begin{dinglist}{230}
\item \T{status[\Z{Q}]} is changed to \T{waiting}
and \T{timer[\Z{Q}]} is set to \T{0}.
When \T{status[\Z{Q}]} becomes \T{running} again,
if there are messages in \T{message_queue[\Z{Q}]}
which have not previously been examined, process them as in part~2.
If that does not complete the evaluation of the \T{receive} expression,
execution continues at \ding{230} above.
\end{dinglist}
\item \label{case:zero-waiting}
If the \T{receive} expression has an expiry part and \TZ{WaitingTime}
is the integer \T{0}, then:
\begin{itemize}
\item[] The evaluation of the \T{receive} expression
finishes by letting $s$ be $k+1$ and evaluating the expiry body $\TZ{B}_s$ with
$\epsilon$ as input environment.  If its evaluation completes abruptly with
reason \TZ{R}, then evaluation of the \T{receive} expression also completes
abruptly with reason \TZ{R}.  If the evaluation of the expiry body completes
normally, its value is also the value of the \T{receive} expression.
(Note that \T{status[\Z{Q}]} is not affected.)
\end{itemize}
\item Otherwise, the \T{receive} expression has an expiry part for which \TZ{WaitingTime}
is a positive integer.  \T{timer[\Z{Q}]} is set to \TZ{Waiting\-Time}.
\begin{dinglist}{230}
\item \T{status[\Z{Q}]} is changed to \T{waiting}.
If there are messages in \T{message_queue[\Z{Q}]} that have not previously been examined
when \T{status[\Z{Q}]} becomes \T{running} again,
process them as in part~2.
If that does not complete the evaluation of the \T{receive} expression,
the system clock of \T{node[\Z{Q}]} is read and if its value is greater than
or equal to \T{\Z{Start}+\T{WaitingTime}}, evaluation of the \T{receive} expression
finishes as described for case~\ref{case:zero-waiting}.
Otherwise, execution continues at \ding{230} above.
\end{dinglist}
\end{enumerate}
(Note that when there are unexamined messages in the queue,
they will be processed even if the expiry time
has been reached.  This implies that the value of the expiry
expression cannot
be seen as a hard limit on how much time the evaluation of the receive expression
may take.)
\end{itemize}

\ENVIRONMENTS

For each clause \T{$\Z{P}_i$ when $\Z{G}_i$ -> $\Z{B}_i$}, $1\leq
i\leq k$, let $d_i$ be the domain of the output environment of
$\TZ{P}_i$ when $\epsilon$ is its input environment, let $d'_i$ be
the domain of the output environment of $\TZ{G}_i$ when $d_i$ is the
domain of its input environment, and let $d''_i$ be the domain of the
output environment of $\TZ{B}_i$ when $d'_i$ is the domain of its
input environment.  Let $d''_{k+1}$ be the domain of the output
environment of $\TZ{B}_{k+1}$ when its input environment is $\epsilon$.

The output environment of the \T{receive} expression is
the output environment of $\TZ{B}_s$ restricted to the
intersection of all $d''_i$, $1\leq i\leq k+1$.
\index{receive expression@\T{receive} expression|)}
\index{message!reception of|)}
\index{expiry!of \T{receive} expression|)}
\index{clause!of receive expression@of \T{receive} expression|)}

\ifStd
\subsection{Try expressions}

\label{section:try-expr}
\index{try expression@\T{try} expression|(}
\index{clause!of try expression@of \T{try} expression|(}

A \T{try} expression has as purpose to evaluate a sequence of
expressions normally except that if evaluation of the expressions
completes abruptly, alternative sequences of expressions can be
evaluated to achieve a normal
completion\index{evaluation!normal mode of}\index{completion!restoring normal},
depending on the reason for the abrupt completion.

\T{try} expressions generalize the earlier \T{catch} expressions (cf.\
\S\ref{section:catch}) which should not be used in new code.\ifDiff\footnote{The \T{try}
expression is an addition from \OldErlang, cf.~\S\ref{section:new-try}.}\fi

\SYNTAX

\begin{rules}
\ifStd
\grrule{TryExpr}
       {\TXT{try} \NT{Body} \TXT{catch} \NT{\CrtClauses} \TXT{end} \OR
        \TXT{try} \NT{Body} \TXT{end}}
\fi
\end{rules}
(The rules for \NT{\CrtClauses} appear in S\ref{section:case-expr}.)

We refer to the body to the left of \T{catch} as
the \emph{protected body} and to the (optional) sequence of clauses to the
right of \T{catch} as the \emph{clauses} of the
\T{receive} expression.  Each clause consists of a \emph{pattern}, an optional
\emph{guard} and a \emph{body}.
An omitted guard is equivalent with a trivially satisfied \T{true} guard.

\EVALUATION

Evaluation of an expression
\begin{alltt}
try
    \Z{B}
catch
    \(\Z{P}\sb{1}\) \([\)when \(\Z{G}\sb{1}]\) -> \(\Z{B}\sb{1}\) ;
    \(\vdots\) ;
    \(\Z{P}\sb{k}\) \([\)when \(\Z{G}\sb{k}]\) -> \(\Z{B}\sb{k}\)
end
\end{alltt}
in an environment $\epsilon$ is carried out as follows.

First the protected body \TZ{B} is evaluated in $\epsilon$.
\begin{itemize}
\item If that
evaluation completes normally with result \TZ{v}, then evaluation of the
\T{try} expression also completes normally with result \TZ{v}.
\item Otherwise, evaluation of the protected body completes abruptly;
let \TZ{R} be the reason for the abrupt completion.
(Recall that this is a term on the form \T{\char`\{'EXIT',\Z{T}\char`\}}
or \T{\char`\{'THROW',\Z{T}\char`\}}.)

For each clause $i$ ($1\leq j\leq k$, in that order),
\begin{itemize}
\item match $\TZ{P}_i$ against $\TZ{R}$ in $\epsilon$ and if that
succeeds evaluate $\TZ{G}_i$ in the output environment of $\TZ{P}_i$,
\end{itemize}
\end{itemize}
until a (first) clause $s$ has been found for which
$\TZ{P}_s$ matches $\TZ{R}$ and $\TZ{G}_s$ succeeds.
\begin{itemize}
\item If no such clause exists, evaluation of the
\T{try} expression completes abruptly with reason \TZ{R}.
\item Otherwise, the value of the \T{try} expression
is obtained by evaluating the $\TZ{B}_s$ in the output
environment of $\TZ{G}_s$.
\end{itemize}

An expression
\begin{alltt}
try
    \Z{B}
end
\end{alltt}
is syntactic sugar for an expression
\begin{alltt}
try
    \Z{B}
catch
    \Z{V} -> \Z{V}
end
\end{alltt}
where \TZ{V} is a free variable not in the output environment of \TZ{B}.
An expression \T{try \Z{E} end} is thus similar to an
expression \T{catch \Z{E}} although not equivalent (they have different
results when \TZ{E} completes abruptly with a reason \T{\{'THROW',\tdots\}}).
However, both expressions will always complete normally.

\ENVIRONMENTS

For each clause \T{$\Z{P}_i$ when $\Z{G}_i$ -> $\Z{B}_i$}, $1\leq
i\leq k$, let $d_i$ be the domain of the output environment of
$\TZ{P}_i$ when $\epsilon$ is its input environment, let $d'_i$ be
the domain of the output environment of $\TZ{G}_i$ when $d_i$ is the
domain of its input environment, and let $d''_i$ be the domain of the
output environment of $\TZ{B}_i$ when $d'_i$ is the domain of its
input environment.

Let $d''_0$ be the domain of the output environment of \TZ{B}.

Let $d$ be the intersection of all $d''_i$, $0\leq i\leq k$.

\begin{itemize}
\item If evaluation of $B$ completed normally, the output environment of
the \T{try} expression is the output environment of \TZ{B} restricted
to $d$.
\item Otherwise, the output environment of
the \T{try} expression is the output environment of $\TZ{B}_s$ restricted
to $d$.
\end{itemize}
\index{try expression@\T{try} expression|)}
\index{clause!of try expression@of \T{try} expression|)}
\fi

\subsection{\T{fun} expressions}

\label{section:fun-exprs}
\index{fun expression@\T{fun} expression|(}

A \T{fun} expression denotes a function.  Its value can therefore be applied.

A \T{fun} expression is similar to a literal in that it is normal (i.e.,
it cannot be further simplified).  However, it is not expected that the value
of a \T{fun} expression
is represented internally in such a way that the \T{fun} expression can be
reconstructed.

\SYNTAX

\begin{rules}
\grrule{FunExpr}
       {\TXT{fun} \NT{FunctionArity} \OR
        \TXT{fun} \NT{FunClauses} \TXT{end}}

\grrule{FunClauses}
       {\NT{FunClause} \OR
        \NT{FunClauses} \TXT{;} \NT{FunClause}}

\grrule{FunClause}
       {\TXT{(} \OPT{Patterns} \TXT{)} \OPT{ClauseGuard} \NT{ClauseBody}}
\end{rules}

\EVALUATION

A \T{fun} expression is either \emph{implicit} or \emph{explicit}.  In the former
case, it refers to a named
function in the same module and in the latter case, it explicitly describes a function.
\begin{itemize}
\item Consider first an implicit \T{fun} expression on
the form \T{fun \Z{F}/\Z{A}}, where \TZ{F} is an atom and \TZ{A} is a decimal literal.
If there is no function with name \TZ{F} and
arity \TZ{A} in the module in which the \T{fun} expression lexically appears, then
it is a compile-time error.  Otherwise,
the expression \T{fun \Z{F}/\Z{A}} denotes a function term that can be applied.
Evaluation of an application of such a function term is described in
\S\ref{section:function-application}.
\item
\index{clause!of fun expression@of \T{fun} expression|(}
Consider now an explicit \T{fun} expression, i.e., an expression
\begin{alltt}
fun (\(\Z{P}\sb{1,1}\),\tdots,\(\Z{P}\sb{1,n\sb{1}}\)) \([\)when \(\Z{G}\sb{1}]\) -> \(\Z{B}\sb{1}\) ;
    \(\vdots\) ;
    (\(\Z{P}\sb{k,1}\),\tdots,\(\Z{P}\sb{k,n\sb{k}}\)) \([\)when \(\Z{G}\sb{k}]\) -> \(\Z{B}\sb{k}\)
end
\end{alltt}
where $k$ is a natural number, each $\TZ{P}_{i,j}$ ($1\leq i\leq k$ and
$1\leq j\leq n_i$) is a \NT{Pattern}, each (optional) $\TZ{G}_i$ is a \NT{Guard}
and each \T{$\Z{B}_i$} is a \NT{Body}.  It is a compile-time error if
there is no number \TZ{A} such that $\TZ{A}=n_1=\cdots=n_k$.
Otherwise the \T{fun} expression denotes a function term that can be applied.
Evaluation of an application of such a function term is described in
\S\ref{section:function-application}.
\index{clause!of fun expression@of \T{fun} expression|)}
\end{itemize}

\ENVIRONMENTS

The input and output environments of a \T{fun} expression are the same, i.e.,
any variables bound in (an explicit) \T{fun} expression are local to it.
\index{fun expression@\T{fun} expression|)}

\ifOld
\subsection{\T{query} expressions}

\label{section:query-exprs}
\index{query expression@\T{query} expression|(}

Query expressions are only syntactically part of \Erlang\ but is a
query interface to the database system Mnesia, part of OTP
\cite[ch.~6]{otp-mnesia}.  Their semantics will not be described in
detail here.

\SYNTAX

\begin{rules}
\grrule{QueryExpr}
       {\TXT{query} \NT{ListComprehension} \TXT{end}}
\end{rules}
There is a syntactic extension to \NT{RecordExpr} that is only valid inside
a \NT{QueryExpr}:
\begin{rules}
\grrule{RecordExpr}
       {\NT{RecordExpr} \TXT{.}\ \NT{RecordFieldName}}
\end{rules}
\index{query expression@\T{query} expression|)}
\fi % ifOld

\subsection{Parenthesized expressions}

\label{section:paren-expr}
\index{parenthesized expression|(}

An expression may always be enclosed in parentheses without changing its meaning.
Such parentheses may be used in order to express the syntactic structure
of an expression when the grammatical rules would otherwise give another
structure.  

\SYNTAX

\begin{rules}
\grrule{ParenthesizedExpr}
       {\TXT{(} \NT{Expr} \TXT{)}}
\end{rules}

\EVALUATION

Evaluating an expression \T{(\Z{E})} means evaluating \TZ{E}.

\ENVIRONMENTS

The output environment of \TZ{E} is used as the output environment of \T{(\Z{E})}.
\index{primary expressions|)}
\index{parenthesized expression|)}

\section{Guards}

\label{section:guards}
\index{guard|(}

In \Erlang\ a pattern can optionally be augmented with a \emph{guard}
for expressing additional conditions on the term that is to be matched
against the pattern.  A guard consists of a nonempty sequence of
\emph{guard tests}\index{guard!test}.

The guard tests have subexpressions, which are \emph{guard
expressions}\index{guard!expression}.
When compared with expressions, both guard tests and guard
expressions are syntactically restricted.
They are built from a small repertoire of primitives for which it is
guaranteed that:
\begin{itemize}
\item Their evaluation takes bounded (often constant) time.
\item They do not have any effects.
\item There are only ``simple'' errors.
\end{itemize}

\SYNTAX

\begin{rules}
\grrule{Guard}
       {\NT{GuardTest} \OR
        \NT{Guard} \TXT{,} \NT{GuardTest}}
\end{rules}

\EVALUATION

Evaluation of a guard completes with success or with failure; there is
no concept of abrupt completion.

A guard is evaluated by evaluating the guard tests from left to right
until one is found that fails, in which case evaluation of the whole
guard fails without evaluating any more guard tests, or all guard
tests have been evaluated, in which case evaluation of the whole guard
succeeds.

\ENVIRONMENTS

The input and output environments of a guard are the same.

\subsection{Guard tests}

\label{section:record-guards}
\index{guard!test|(}

Guard tests are syntactically identical with certain Boolean
expressions but their semantics are slightly different in the case of
abrupt completion.

There are five kinds of guard tests: trivially true tests,
record tests, recognizers, term comparisons and parenthesized guard tests.

\SYNTAX

\begin{rules}
\grrule{GuardTest}
       {\TXT{true} \OR
        \NT{GuardRecordTest} \OR
\ifstruct        \NT{GuardStructTest} \OR \fi
        \NT{GuardRecognizer} \OR
        \NT{GuardTermComparison} \OR
        \NT{ParenthesizedGuardTest}}

\grrule{GuardRecordTest}
       {\TXT{record} \TXT{(} \NT{GuardExpr} \TXT{,} \NT{RecordType} \TXT{)}}

\ifstruct
\grrule{GuardStructTest}
       {\TXT{struct} \TXT{(} \NT{GuardExpr} \TXT{,} \NT{GuardExpr} \TXT{)}}
\fi

\grrule{GuardRecognizer}
       {\NT{RecognizerBIF} \TXT{(} \NT{GuardExpr} \TXT{)}}

\grrule{RecognizerBIF}
       {\NT{AtomLiteral}}

\grrule{GuardTermComparison}
       {\NT{GuardExpr} \NT{RelationalOp} \NT{GuardExpr} \OR
        \NT{GuardExpr} \NT{EqualityOp} \NT{GuardExpr}}

\grrule{ParenthesizedGuardTest}
       {\TXT{(} \NT{GuardTest} \TXT{)}}
\end{rules}

\EVALUATION

We discuss the kinds of guard tests one by one.

\label{section:record2}

\begin{itemize}
\item \index{true expression@\T{true} expression|(}
A \T{true} test succeeds trivially.
\index{true expression@\T{true} expression|)}
\item \index{record/2@\T{record/2}!recognizer|(}
A \NT{GuardRecordTest} \T{record(\Z{E},\Z{R})} (where \TZ{E} is a guard
expression and \TZ{R} is a record type) is evaluated by evaluating \TZ{E}
and then testing the result.  The test succeeds if the evaluation of \TZ{E}
completes normally with some result \TZ{v} and \TZ{v} is a record of type \TZ{R};
otherwise the
test fails.  Note that there is no BIF \T{record/2}, a \NT{GuardRecordTest}
just happens to have the same syntax as a function application.
\index{record/2@\T{record/2}!recognizer|)}
\item \index{BIF!recognizer|(}
A \NT{GuardRecognizer} is an application of one of the recognizer
BIFs of \S\ref{section:recognizer-bifs} to a guard expression.  It is
evaluated by evaluating the guard expression and applying the BIF to
the result.  The test succeeds if the evaluation of the guard
expression completes normally with some result \TZ{v}
and the BIF returns \T{true} for \TZ{v}; otherwise the test fails.
\index{BIF!recognizer|)}
\item \index{term!comparison|(}
A \NT{GuardTermComparison} is a \NT{RelationalOp} or an
\NT{EqualityOp} applied to a pair of guard expressions.  It is applied
by evaluating both operands and then computing a boolean as described
in \S\ref{section:relational}.  The test succeeds if evaluations of
both operands complete normally with some results $\TZ{v}_1$ and
$\TZ{v}_2$ and the subsequently computed boolean is \T{true};
otherwise the test fails.
\index{term!comparison|)}
\item \index{test!parenthesized|(}
A \NT{ParenthesizedGuardTest} \T{(\Z{G})} succeeds if, and only if,
the guard test \TZ{G} succeeds.
\index{test!parenthesized|)}
\end{itemize}
Note that if any subexpression of a guard test completes abruptly, the
guard test (and thus the guard of which it is a part) fails.  A guard
test can never complete abruptly.

\ENVIRONMENTS

The input and output environments of a guard test are the same.
\index{guard!test|)}

\subsection{Guard expressions}

\index{guard!expression|(}

The guard expressions are syntactically identical with
certain expressions.

\begin{rules}
\grrule{GuardExpr}
       {\NT{GuardAdditionShiftExpr}}

\grrule{GuardAdditionShiftExpr}
       {\NT{GuardAdditionShiftExpr} \NT{AdditionOp} \NT{GuardMultiplicationExpr} \OR
        \NT{GuardAdditionShiftExpr} \NT{ShiftOp} \NT{GuardMultiplicationExpr} \OR
        \NT{GuardMultiplicationExpr}}

\grrule{GuardMultiplicationExpr}
       {\NT{GuardMultiplicationExpr} \NT{MultiplicationOp} \NT{GuardPrefixOpExpr} \OR
        \NT{GuardPrefixOpExpr}}

\grrule{GuardPrefixOpExpr}
       {\NT{PrefixOp} \NT{GuardApplicationExpr} \OR
        \NT{GuardApplicationExpr}}

\grrule{GuardApplicationExpr}
       {\NT{GuardBIF} \TXT{(} \OPT{GuardExprs} \TXT{)} \OR
        \NT{GuardRecordExpr} \OR
        \NT{GuardPrimaryExpr}}

\iffalse
How to get it through BISON:

\grrule{GuardApplicationExpr}
       {\NT{GuardBIF} \TXT{(} \TXT{)} \OR
        \NT{GuardTypeTest} \OR
        \NT{GuardBIF} \TXT{(} \NT{GuardExpr} \TXT{,} \OPT{GuardExprs} \TXT{)} \OR
        \NT{GuardRecordExpr} \OR
        \NT{GuardPrimaryExpr}}
\fi

\grrule{GuardBIF}
       {\NT{AtomLiteral}}

\grrule{GuardExprs}
       {\NT{GuardExpr} \OR
        \NT{GuardExprs} \TXT{,} \NT{GuardExpr}}

\grrule{GuardRecordExpr}
       {\OPT{GuardPrimaryExpr} \TXT{\#} \NT{AtomLiteral} \TXT{.} \NT{AtomLiteral}}

\grrule{GuardPrimaryExpr}
       {\NT{Variable} \OR
        \NT{AtomicLiteral} \OR
        \NT{GuardListSkeleton} \OR
        \NT{GuardTupleSkeleton} \OR
        \NT{ParenthesizedGuardExpr}}

\grrule{GuardListSkeleton}
       {\TXT{[} \TXT{]} \OR
        \TXT{[} \NT{GuardExprs} \OPT{GuardListSkeletonTail} \TXT{]}}

\grrule{GuardListSkeletonTail}
       {\TXT{|} \NT{GuardExpr}}

\grrule{GuardTupleSkeleton}
       {\TXT{\{} \OPT{GuardExprs} \TXT{\}}}

\grrule{ParenthesizedGuardExpr}
       {\TXT{(} \NT{GuardExpr} \TXT{)}}
\end{rules}
It is described in \S\ref{chapter:bifs}
which BIFs are guard BIFs.

\EVALUATION

All guard expressions are expressions and a guard expression is
evaluated exactly as the corresponding expression.  Evaluation of a
guard expression may complete abruptly but a guard expression always
occurs as part of a guard test which will restore normal evaluation
by simply failing.

\ENVIRONMENTS

The input and output environments of a guard expression are the same.
\index{guard!expression|)}
\index{guard|)}


%
% %CopyrightBegin%
%
% Copyright Ericsson AB 2017. All Rights Reserved.
%
% Licensed under the Apache License, Version 2.0 (the "License");
% you may not use this file except in compliance with the License.
% You may obtain a copy of the License at
%
%     http://www.apache.org/licenses/LICENSE-2.0
%
% Unless required by applicable law or agreed to in writing, software
% distributed under the License is distributed on an "AS IS" BASIS,
% WITHOUT WARRANTIES OR CONDITIONS OF ANY KIND, either express or implied.
% See the License for the specific language governing permissions and
% limitations under the License.
%
% %CopyrightEnd%
%

\chapter{Compiling a module}

\label{chapter:compilation}
\index{module!compilation|(}
The unit of compilation in \Erlang\ is a module.
The compilation of a module is carried out in the following steps:
\begin{enumerate}
\item Lexical processing, as described in \S\ref{chapter:lexical}.
This takes a sequence of
\ifNew Unicode characters (\S\ref{section:unicode}) \fi
\ifOld ASCII characters \fi
and produces a sequence
of tokens and full stops, i.e., a sequence of \NT{TokenSequence}
(\S\ref{section:tokenseq}).
\item Preprocessing, as described in \S\ref{section:preprocessing}.
This takes a sequence of \NT{TokenSequence} and produces a new
sequence of \NT{TokenSequence}, after conditional compilation
and macro expansion.
\item Parsing, as described in \S\ref{section:parsing}.
Each \NT{TokenSequence} in the sequence is parsed as an
\Erlang\ form (\S\ref{chapter:programs-modules}).
The result is a list of parse trees, which are represented as
\Erlang\ terms
(\S\ref{chapter:parse-trees}).
\item Parse transformation, as described in \S\ref{section:parse-transform}.
If a parse transform function has been specified, it is applied
to the list of parse trees,
resulting in a new list of parse trees.  Otherwise the result is the
same list of parse trees.
\item Code generation, as described in \S\ref{section:code-generation}.
This takes a list of parse trees and returns a binary representation
of a module, which can be loaded (\S\ref{chapter:module-dynamics}).
\end{enumerate}
The resulting binary can be loaded into a running \Erlang\ system or be saved in a
file and loaded from there (\S\ref{section:loading}).
\index{module!compilation|)}

\section{Preprocessing}

\label{section:preprocessing}
\index{module!preprocessing of|(}

The preprocessing step takes a sequence of \NT{TokenSequence}
as input and produces a new sequence of \NT{TokenSequence}.
(A \NT{TokenSequence} is a sequence of tokens followed by a full stop,
cf.\ \S\ref{section:tokenseq}.)

\index{skipping mode|(}
\index{processing mode|(}
During preprocessing, the compiler is in one of two modes: skipping
or processing.  When the compiler is in skipping mode, a \NT{TokenSequence}
is ignored unless it is one of the directives for conditional compilation
discussed in \S\ref{section:condcomp}.  When the compiler is in processing mode,
the treatment of a \NT{TokenSequence} is as described in this section.
The compiler is originally in processing mode.
\index{skipping mode|)}
\index{processing mode|)}

Each \NT{TokenSequence} that is one of the following directives
is processed and is not part of the output sequence of \NT{TokenSequence}.
\begin{rules}
\grrule{Directive}
       {\NT{MacroDefinition} & (\S\ref{section:macrodef}) \OR
        \NT{MacroUndefinition} & (\S\ref{section:macroundef}) \OR
        \NT{IncludeDirective} & (\S\ref{section:include}) \OR
        \NT{IncludeLibDirective} & (\S\ref{section:include}) \OR
        \NT{IfdefDirective} & (\S\ref{section:condcomp}) \OR
        \NT{IfndefDirective} & (\S\ref{section:condcomp}) \OR
        \NT{ElseDirective} & (\S\ref{section:condcomp}) \OR
        \NT{EndifDirective} & (\S\ref{section:condcomp})}
\end{rules}
\index{define directive@\T{define} directive|(}
\index{undef directive@\T{undef} directive|(}
The directives \T{-define(\Z{M}$[\text{($\NT{V}_1$,\tdots,$\NT{V}_k$)}]$,\Z{Toks})}
and \T{-undef(\Z{M})} maintain the set of macro definitions,
\index{define directive@\T{define} directive|)}
\index{undef directive@\T{undef} directive|)}
\index{include directive@\T{include} directive|(}
\index{include_lib directive@\T{include_lib} directive|(}
the directives \T{-include(\Z{F})} and \T{-include_lib(\Z{F})} control file inclusion,
\index{include directive@\T{include} directive|)}
\index{include_lib directive@\T{include_lib} directive|)}
\index{ifdef directive@\T{ifdef} directive|(}
\index{ifndef directive@\T{ifndef} directive|(}
\index{else directive@\T{else} directive|(}
\index{endif directive@\T{endif} directive|(}
and the directives \T{-ifdef(\Z{M})}, \T{-ifndef(\Z{M})}, \T{-else} and \T{-endif} control
conditional compilation (\S\ref{section:condcomp}).
\index{ifdef directive@\T{ifdef} directive|)}
\index{ifndef directive@\T{ifndef} directive|)}
\index{else directive@\T{else} directive|)}
\index{endif directive@\T{endif} directive|)}

The tokens in a \NT{TokenSequence} that is not a \NT{Directive} are
subject to macro expansion\index{macro!expansion}
(\S\ref{section:macroexp}) and the resulting tokens, followed by a
full stop, form a \NT{TokenSequence} that is part of the output of the
preprocessing.

\section{Macros}

\label{section:macros}

As the preprocessor goes through the sequence of \NT{TokenSequence},
it maintains a set of macro definitions.  A \NT{MacroDefinition} adds
to the set (\S\ref{section:macrodef}), a \NT{MacroUndefinition} may
remove from the set (\S\ref{section:macroundef}), any other
\NT{TokenSequence} leaves it unchanged.

A \NT{TokenSequence} that is not recognized as a \NT{Directive}
(\S\ref{section:preprocessing}) is subject to macro expansion
(\S\ref{section:macroexp}).

The initial set of macro definitions is described in \S\ref{section:initial-macros}.

\subsection{Macro definition}

\label{section:macrodef}
\index{define directive@\T{define} directive|(}
\index{macro!definition|(}

A macro definition is a directive that adds a macro definition.

\begin{rules}
\grrule{MacroDefinition}
       {\TXT{-} \TXT{define} \TXT{(} \NT{MacroName} \OPT{MacroParams}
       \TXT{,} \NT{MacroBody} \TXT{)} \NT{FullStop}}

\grrule{MacroName}
      {\NT{AtomLiteral} \OR
       \NT{Variable}}

\grrule{MacroParams}
       {\TXT{(} \OPT{Variables} \TXT{)}}

\grrule{Variables}
       {\NT{Variable} \OR
        \NT{Variables} \TXT{,} \NT{Variable}}

\grrule{MacroBody}
       {\NT{Tokens}}
\end{rules}

A macro definition \T{-define(\Z{M},\Z{Toks})} associates the macro
name \TZ{M} with no sequence of parameters and an arbitrary (and
possibly empty) sequence of tokens \TZ{Toks}.

A macro definition
\T{-define(\Z{M}($\NT{V}_1$,\tdots,$\NT{V}_k$),\Z{Toks})}, where
$k\geq0$, associates the macro name \TZ{M} with a (possibly empty)
sequence of macro parameters $\NT{V}_1$, \ldots, $\NT{V}_k$ which must
be distinct variables, and a (possibly empty) sequence of tokens
\TZ{Toks}.

Unlike the case for function names, which associate a symbol and an
arity with a function, a macro name is simply a symbol and the macro
named \TZ{M} either takes no parameters at all or obtains the arity
$k$ above.  It is thus not possible to define two macros with the same
name \TZ{M} but different arities.

In either case, the scope of the association for \TZ{M} begins
immediately after the macro definition.  It is a compile-time error if
a macro definition of \TZ{M} occurs in the scope of another macro
definition of \TZ{M}.

Note the difference between the macro definitions \T{-define(abc,123)}
and \T{-define(abc(),123)}.  The former is associated with no
parameters while the latter is associated with an empty sequence of
parameters and an application of it must therefore have an empty
argument sequence (\S\ref{section:macroexp}).

\ifOld
In \OldErlang, a macro name \T{'Foo'} and a macro name \T{Foo} denote
different macros.  This is to be viewed as a bug (not a feature) and
will change in a future version so a
\fi
\ifStd
A
\fi
macro name that is a quoted atom and a macro name that is a variable
denote the same macro if the print name of the atom consists of the
same sequence of characters as the variable.  For example, the macro
names \T{'Foo'} and \T{Foo} \ifOld will then \fi denote the same
macro.

\index{macro!definition|)}
\index{define directive@\T{define} directive|)}

\subsection{Macro undefinition}

\label{section:macroundef}
\index{macro!undefinition|(}
\index{undef directive@\T{undef} directive|(}

The scope of a macro definition is terminated by a macro undefinition or the end of the
module definition.
\begin{rules}
\grrule{MacroUndefinition}
       {\TXT{-} \TXT{undef} \TXT{(} \NT{MacroName} \TXT{)} \NT{FullStop}}
\end{rules}
Beginning immediately following an undefinition \T{-undef(\Z{M})},
the macro name \TZ{M} is undefined.
It is not an error if \TZ{M} was undefined also immediately preceding the undefinition.
\index{macro!undefinition|)}
\index{undef directive@\T{undef} directive|)}

\subsection{Macro expansion}

\label{section:macroexp}
\index{macro!expansion|(}

Each macro application in a \NT{TokenSequence} is expanded, i.e., replaced with
tokens according to the macro definition that is in force at that point.

\begin{rules}
\grrule{MacroApplication}
       {\TXT{?} \NT{MacroName} \OR
	\TXT{?} \NT{MacroName} \TXT{(} \OPT{MacroArguments} \TXT{)}}

\grrule{MacroArguments}
       {\NT{MacroArgument} \OR
	\NT{MacroArguments} \TXT{,} {MacroArgument}}

\grrule{MacroArgument}
       {\NT{BalancedExpr} that is not one of \TXT{,} or \TXT{)}}

\grrule{BalancedExpr}
       {\TXT{(} \NT{BalancedExprs} \TXT{)} \OR
        \TXT{[} \NT{BalancedExprs} \TXT{]} \OR
        \TXT{\{} \NT{BalancedExprs} \TXT{\}} \OR
        \TXT{begin} \NT{BalancedExprs} \TXT{end} \OR
        \ifNew\TXT{all_true} \NT{BalancedExprs} \TXT{end} \OR
        \TXT{some_true} \NT{BalancedExprs} \TXT{end} \OR
        \TXT{cond} \NT{BalancedExprs} \TXT{end} \OR\fi
        \TXT{if} \NT{BalancedExprs} \TXT{end} \OR
        \TXT{case} \NT{BalancedExprs} \TXT{end} \OR
        \TXT{receive} \NT{BalancedExprs} \TXT{end} \OR
        \ifNew\TXT{try} \NT{BalancedExprs} \TXT{end} \OR\fi
        \ifOld\TXT{query} \NT{BalancedExprs} \TXT{end} \OR\fi
	\NT{OtherToken} \OR
	\NT{BalancedExpr} \NT{BalancedExpr}}
\end{rules}

Macro expansion of a \NT{TokenSequence} is defined as follows.  Let us
write $\langle\TZ{T}_1\ \TZ{T}_2\ \ldots\ \TZ{T}_k\rangle$ for the
macro expansion of a \NT{TokenSequence} $\TZ{T}_1\ \TZ{T}_2\ \ldots\
\TZ{T}_k$.  We write $\tau$ or $\tau'$ for an arbitrary sequence of
tokens and $\beta_1$, \ldots, $\beta_k$ for $k$ balanced expressions,
i.e., \NT{BalancedExpr} above.
\begin{itemize}
\item $\langle\text{\T{?}\ \TZ{A} $\tau$ \NT{FullStop}}\rangle = \langle\text{$\tau'$ $\tau$
\NT{FullStop}}\rangle$\ if \TZ{A} is an atom and there is a macro definition for \TZ{A} with no
parameters and a replacement sequence $\tau'$.
\item $\langle\text{\T{?}\ \TZ{A} \T{(} $\beta_1$ \T{,}\ \ldots\ \T{,}\ $\beta_k$ \T{)} $\tau$ \NT{FullStop}}\rangle = \text{}$ \\
\hfil$\langle\text{$\tau'$[$\beta_1$/$v_1$,\ldots,$\beta_1$/$v_k$] $\tau$ \NT{FullStop}}\rangle$ \\
if \TZ{A} is an atom and there is a macro definition for \TZ{A} with $k$ parameters
$v_1$, \ldots, $v_k$ and a replacement sequence $\tau'$.
\item $\langle\text{\T{?}\ $\tau$ \NT{FullStop}}\rangle$\ in any other case is a compile-time error.
\item $\langle\text{\TZ{T} $\tau$ \NT{FullStop}}\rangle = \text{\TZ{T} $\langle\text{$\tau$ \NT{FullStop}}\rangle$}$\
where \TZ{T} is any token but \T{?}.
\item $\langle\text{\NT{FullStop}}\rangle = \NT{FullStop}$.
\end{itemize}
To summarize in words, after \T{?}\ must follow a defined macro name.
Either it is defined to have no parameters, in which case \T{?}\ and
the atom are replaced by the expansion and macro expansion starts over from there,
or it is defined to have a number of parameters in which case \T{?} and the atom
must be followed by that many macro arguments.  In that case
\T{?}, the atom and the macro arguments are replaced by the expansion in which each
occurrence of a macro parameter has been replaced by the corresponding macro argument
and macro expansion starts over from there.  If the first token is not \T{?}, it is
made part of the result of the macro expansion.

For example, suppose the following macro definitions are in force:
\begin{verbatim}
-define(foo,fum(?).
-define(bar(X),(X))).
\end{verbatim}
They associate the macro name \T{foo} with no parameters and the three tokens \T{fum},
\T{(} and \T{?}, and the macro name \T{bar} with one macro parameter \T{X} and an
expansion consisting of \T{(}, the token sequence given as macro argument, \T{)} and \T{)} again.
(This would be an extremely objectionable programming style.)
Then the \NT{TokenSequence}
\begin{verbatim}
foo(X) -> ?foo bar(6*X).
\end{verbatim}
which consists of the thirteen tokens \T{foo}, \T{(}, \T{X}, \T{)}, \T{->},
\T{?}, \T{foo}, \T{bar}, \T{(}, \T{6}, \T{*}, \T{X} and \T{)},
followed by a full stop, is macro expanded to
\begin{verbatim}
foo(X) -> fum((6*X)).
\end{verbatim}
in the following steps:
\begin{enumerate}
\item $\langle\text{\T{foo ( X ) -> ? foo bar ( 6 * X )} \NT{FullStop}}\rangle$ \\
The first five tokens (including the atom \T{foo} even though there happens to be a macro
with that name) are unaffected by macro expansion.
\item \T{foo ( X ) ->} $\langle\text{\T{? foo bar ( 6 * X )} \NT{FullStop}}\rangle$ \\
The next two tokens are the separator \T{?}\ and the atom \T{foo}.  There is a macro definition
of \T{foo} with no parameters and expansion consisting of the tokens \T{fum},
\T{(} and \T{?}; thus \T{?}\ and \T{foo} are replaced by \T{fum}, \T{(} and \T{?}
and macro expansion starts over from there.
\item \T{foo ( X ) ->} $\langle\text{\T{fum ( ? bar ( 6 * X )} \NT{FullStop}}\rangle$ \\
The next two tokens are unaffected by macro expansion.
\item \T{foo ( X ) -> fum (} $\langle\text{\T{? bar ( 6 * X )} \NT{FullStop}}\rangle$ \\
The next two tokens are \T{?} and \T{bar} and there is a macro definition of \T{bar} with
one macro parameter. The macro argument consists of the tokens \T{6}, \T{*} and \T{X}.
The tokens \T{?}, \T{bar}, \T{(}, \T{6}, \T{*}, \T{X} and \T{)} are then replaced with the
expansion \T{(}, \T{6}, \T{*}, \T{X}, \T{)}, \T{)} and macro expansion starts over from there.
\item \T{foo ( X ) -> fum (} $\langle\text{\T{( 6 * X ) )} \NT{FullStop}}\rangle$ \\
The remaining six tokens are unaffected by macro expansion.
\item \T{foo ( X ) -> fum ( ( 6 * X ) )} \NT{FullStop} \\
Macro expansion is complete.
\end{enumerate}

Note that directives are not macro expanded (and their arguments are not evaluated).
For example, it is not
possible to write
\begin{verbatim}
-define(SRCDIR(FN),"/usr/local/src/myproj/" FN ".erl").

-include(?SRCDIR("bliss")).
\end{verbatim}
because the argument of \T{include} must be an \NT{IncludeFileName}, i.e., a \NT{One\-String\-Literal}.
\index{macro!expansion|)}

\subsection{Initial set of macro definitions}

\label{section:initial-macros}
\index{macro!initial set|(}

Initially the following three macros
\ifOld are \fi
\ifNew must be \fi
defined, each one associated with no parameters:
\begin{itemize}
\item \T{?MODULE} is expanded to a single token: an atom literal that
is the name of the module being compiled.
\item \T{?FILE} is expanded to a single token which is a string
literal that is a full path to the file that is being compiled.
\item \T{?LINE} is expanded to a single token which is an integer
literal that is the number of the line on which the \T{LINE} token
appears.  If \T{LINE} occurs within a macro then \T{LINE} is expanded
to the number of the line in which that macro is expanded.  Undefining
or redefining the \T{LINE} macro has no effect. (Note that \T{LINE}
does not have a fixed token sequence and has to be handled as a
special case in the macro expansion.)
\ifOld
\item \T{?MACHINE} is expanded to a single token: an atom literal that
is the name of the abstract \Erlang\ machine on which the compiler is
running. The known abstract \Erlang\ machines are:
\begin{itemize}
\item \T{'JAM'}
\item \T{'BEAM'}
\item \T{'VEE'}
\end{itemize}
% Atoms for new abstract machines should be registered with Ericsson Software Technology AB.
\fi
\iffalse
\item \T{?VERSION} is expanded to an integer token. An implementation supporting
\NewErlang\ should expand \T{?VERSION} to \T{500}.  An implementation supporting
\OldErlang\ should expand \T{?VERSION} to \T{470}.
% Hellre lista???
\fi
% ?DATE ???
\end{itemize}
\ifNew
An implementation may provide additional predefined macros, which
should be documented.
\fi
\index{macro!initial set|)}

\section{File inclusion}

\label{section:include}
\index{include directive@\T{include} directive|(}
\index{include_lib directive@\T{include_lib} directive|(}
\index{file!inclusion of|(}

The \T{include} and \T{include_lib} directives splice in the contents
of a file.

\begin{rules}
\grrule{IncludeDirective}
       {\TXT{-} \TXT{include} \TXT{(} \NT{IncludeFileName} \TXT{)} \NT{FullStop}}

\grrule{IncludeLibDirective}
       {\TXT{-} \TXT{include_lib} \TXT{(} \NT{IncludeFileName} \TXT{)} \NT{FullStop}}

\grrule{IncludeFileName}
       {\NT{OneStringLiteral}}
\end{rules}

The difference between the \T{include} and \T{include_lib} directives
lies in which paths are searched when the given filename is not
absolute.  For \T{-include(\Z{F})}, file \TZ{F} is searched in each
directory for which an include path was given to the compiler (option
\T{\{i,\Z{Dir}\}}), in order (as if using \T{file:path_open/3} \cite[p.~230]{otp-dev-ref}).
Also for \T{-include_lib(\Z{F})}, these paths are tried first.
However, if the file is not found, then if \TZ{F} can be split into a
string \TZ{L} that does not contain the character `\T{/}' and one
string \TZ{N} that begins with `\T{/}', \TZ{L} is looked up as a
library (i.e., as if it was given as argument to \T{code:lib_dir/1}
\cite[p.~164]{otp-dev-ref}).  If that yields a path \TZ{P} to a
library directory then the path which is the concatenation of
\TZ{P} and \TZ{N} is used.

It is a compile-time error if no readable file is found.

The contents of the file is subject to lexical processing
(\S\ref{chapter:lexical}).  It is an error if lexical processing of
the contents of the included file does not yield a
\NT{TerminatedTokens}, i.e., a sequence of \NT{TokenSequence}.

The resulting sequence of \NT{TokenSequence} is processed exactly as
described in this section.  While processing the included file, the
\T{FILE} and \T{LINE} macros (\S\ref{section:initial-macros}) refer to
the path of the included file and line numbers in it, respectively.
The sequence of \NT{TokenSequence} that is the result of the
preprocessing of the included file becomes part of the output of the
preprocessing of the including sequence of \NT{TokenSequence} at the
location of the \T{include} or \T{include_lib} directive.
\index{include directive@\T{include} directive|)}
\index{include_lib directive@\T{include_lib} directive|)}
\index{file!inclusion of|)}

\section{Conditional compilation}

\label{section:condcomp}
\index{conditional compilation|(}
\index{ifdef directive@\T{ifdef} directive|(}
\index{ifndef directive@\T{ifndef} directive|(}
\index{else directive@\T{else} directive|(}
\index{endif directive@\T{endif} directive|(}

There are four directives related to conditional compilation and two additional
directives for which syntax is reserved.

\begin{rules}
\grrule{IfdefDirective}
       {\TXT{-} \TXT{ifdef} \TXT{(} \NT{MacroName} \TXT{)} \NT{FullStop}}
\grrule{IfndefDirective}
       {\TXT{-} \TXT{ifndef} \TXT{(} \NT{MacroName} \TXT{)} \NT{FullStop}}
\grrule{ElseDirective}
       {\TXT{-} \TXT{else} \NT{FullStop}}
\grrule{EndifDirective}
       {\TXT{-} \TXT{endif} \NT{FullStop}}
\end{rules}

It is a compile-time error if the sequence of \NT{TokenSequence} given
to the preprocessing step does not contain matching triples or pairs
of
\begin{itemize}
\item \NT{IfdefDirective}, \NT{ElseDirective} and \NT{EndifDirective};
\item \NT{IfndefDirective}, \NT{ElseDirective} and \NT{EndifDirective};
\item \NT{IfdefDirective} and \NT{EndifDirective};
\item \NT{IfndefDirective} and \NT{EndifDirective}.
\end{itemize}
This must hold for a whole module definition as well as individually for each
included file (\S\ref{section:include}).

\index{skipping mode|(}
\index{processing mode|(}
If the compiler is in processing mode and
encounters an \T{-ifdef(\Z{M})} directive [or \T{-ifndef(\Z{M})} directive],
the compiler tests whether there is a
macro definition for \TZ{M}.  If this is [not] the case, then the compiler continues in
processing mode until it encounters the matching \T{else} or \T{endif} directive.  (Note that this
may involve handling any number of enclosed nested triples or pairs of the kinds above.)
If an \T{else} directive was encountered, the compiler continues in skipping [processing] mode until the
matching \T{endif} directive is encountered.  In either case, after the \T{endif} directive,
the compiler continues in processing mode.

If the compiler is in skipping mode and encounters an \T{-ifdef(\Z{M})}
or \T{-ifndef(\Z{M})} directive, it continues in skipping mode until
it encounters the matching \T{else} or \T{endif} directive.  (Again,
this may involve handling any number of enclosed nested triples or
pairs of the kinds above.)  If an \T{else} directive was encountered,
the compiler continues in skipping mode until the matching \T{endif}
directive is encountered.  In either case, after the \T{endif}
directive, the compiler continues in skipping mode.  Thus, it is
obligatory for the compiler to keep track of the directives for
conditional compilation also when in skipping mode.  For example, the
compiler must report a compile-time error for a module containing the
following directives:
\begin{verbatim}
-define(foo,42).
-ifdef(foo).
-else.
-else
-endif.
\end{verbatim}

The compiler may implement this mechanism in any suitable way, but the
following machinery serves as an example and may clarify the
description above.  Let the compiler have a state consisting of a
register \T{Processing} and a stack.  \T{Processing} is
either \T{true} or \T{false}.  Each stack item is a pair where the
left half is either \T{if} or \T{else} and the right half is either
\T{true} or \T{false}. Initially \T{Processing} is \T{true} and the
stack is empty.  Here is what happens when the compiler encounters a
\NT{TokenSequence}:
\begin{itemize}
\item If the \NT{TokenSequence} is \T{-ifdef(\Z{M})} or \T{-ifndef(\Z{M})}, then
a pair of \T{if} and the value of \T{Processing} is pushed onto the stack.
If \T{Processing} is \T{true} and either
the \NT{TokenSequence} is \T{-ifdef(\Z{M})} and \TZ{M} has no macro definition,
or the \NT{TokenSequence} is \T{-ifndef(\Z{M})} and \TZ{M} has a macro definition,
then \T{Processing} is changed to \T{false}.
\item If the \NT{TokenSequence} is \T{-else} then it is a compile-time error if
the stack is empty or the left half of the top pair is not \T{if}.  The left half
of the top pair is changed to \T{else} and the contents of \T{Processing} is set
to the logical conjunction of the right half of the top pair and
the negation of \T{Processing}.
\item If the \NT{TokenSequence} is \T{-endif} then it is a compile-time error if
the stack is empty.  \T{Processing} is set to the right half of the top stack pair
and that pair is popped off the stack.
\item If the \NT{TokenSequence} is anything else, then if \T{Processing} is \T{true},
it is processed as described in \S\ref{section:preprocessing}, otherwise it is
ignored.
\end{itemize}
\index{skipping mode|)}
\index{processing mode|)}

The directives \T{if}\index{if@\T{if}!(reserved) directive}
and \T{elif}\index{elif (reserved) directive@\T{elif} (reserved) directive}
(with any number of arguments) are
reserved for future extension of conditional compilation.
\index{conditional compilation|)}
\index{ifdef directive@\T{ifdef} directive|)}
\index{ifndef directive@\T{ifndef} directive|)}
\index{else directive@\T{else} directive|)}
\index{endif directive@\T{endif} directive|)}
\index{module!preprocessing of|)}

\section{Parsing}

\label{section:parsing}
\index{module!parsing|(}

The sequence of \NT{TokenSequence} is parsed as a
\NT{ModuleDeclaration} (\S\ref{section:module-declarations}).
\index{parse tree!result of parsing|(}
The result is a list of parse trees
(one for each top-level form), which are
represented as \Erlang\ terms (\S\ref{chapter:parse-trees}).
\index{parse tree!result of parsing|)}
\index{module!parsing|)}

\section{Parse transforms}

\label{section:parse-transform}
\index{parse transform|(}
\index{parse tree!in parse transform|(}

A parse transform is a function mapping lists of parse trees to lists of
parse trees.  Each parse transform is implemented by a separate module
with an exported function named
\T{parse_transform/1}\index{parse_transform/1 function@\T{parse_transform/1} function}.
Suppose that the compiler has been instructed (through compiler options)
to use the $k$
parse transforms implemented by modules $\TZ{M}_1$, \ldots, $\TZ{M}_k$.
(The order is relevant and is the same as the order in which the compiler
options appear.)

Let $\TZ{L}_0$ be the list of parse trees resulting from parsing the
preprocessed sequence
of \NT{TokenSequence}, represented as \Erlang\ terms (\S\ref{chapter:parse-trees}).
For each $i$, $1\leq i\leq k$, let $\TZ{L}_i$ be the result of computing
\begin{alltt}
\(\Z{M}\sb{i}\):parse_transform(\(\Z{L}\sb{i-1}\))
\end{alltt}
if that computation completes normally.  The result must be a list of parse
trees for a \NT{ModuleDeclaration} (just as $\TZ{L}_0$ is).
If for some $\TZ{M}_i$, the computation completes abruptly or the result of the
computation is not a list of parse trees for a \NT{ModuleDeclaration}, then
it is a compile-time error.
Otherwise, $\TZ{L}_k$ is used in the code
generation step (\S\ref{section:code-generation}).
\index{parse transform|)}
\index{parse tree!in parse transform|)}

\section{Code generation}

\label{section:code-generation}
\index{module!code generation|(}

\index{parse tree!input to code generation|(}
\index{binary!representation of compiled module|(}
The code generation step takes a list of parse trees represented as \Erlang\ terms
(\S\ref{chapter:parse-trees}) and produces a binary that contains a loadable
representation of the module.
\index{parse tree!input to code generation|)}
That binary can either be loaded immediately
(\S\ref{section:loading}) or be written to a file for later loading.
The actual format of the binary is implementation-defined.
\index{binary!representation of compiled module|)}
\index{module!code generation|)}

\iffalse
% MID stuff
\section{Loading a module}

\label{section:loading}

A binary is either available immediately, for example, directly from code generation
(\S\ref{section:loading}) or read from a file.

Loading the binary results in a \emph{module term}, as 
described in \S\ref{section:module-terms}.  It is also possible to update
the mapping of module names to module terms on the node onto which the binary is loaded,
as described in \S\ref{section:module-names}.
\fi


%
% %CopyrightBegin%
%
% Copyright Ericsson AB 2017. All Rights Reserved.
%
% Licensed under the Apache License, Version 2.0 (the "License");
% you may not use this file except in compliance with the License.
% You may obtain a copy of the License at
%
%     http://www.apache.org/licenses/LICENSE-2.0
%
% Unless required by applicable law or agreed to in writing, software
% distributed under the License is distributed on an "AS IS" BASIS,
% WITHOUT WARRANTIES OR CONDITIONS OF ANY KIND, either express or implied.
% See the License for the specific language governing permissions and
% limitations under the License.
%
% %CopyrightEnd%
%

\chapter{Programs and modules}

\label{chapter:programs-modules}

\index{module!declaration|(}
An \Erlang\ \emph{module declaration} consists of a collection of
\emph{forms}.  Forms are function declarations together with some
attributes\iftypedecl, type declarations\fi\ and record declarations
for the module.  It is the smallest unit of code that can be
separately compiled and loaded.  An \Erlang\ compiler thus has access
to all the code of a module at compile-time.  Therefore it can
determine at compile-time, for example, the names of all functions
being declared in the module.
\index{module!declaration|)}

\index{program|(}
An \Erlang\ \emph{program} typically consists of a number of modules,
out of which some may be standard \Erlang\ modules or parts of library
applications \cite{otp-dev-ref}.
\index{program|)}

\section{Module declarations}

\label{section:module-declarations}
\index{module!declaration|(}

A module declaration may begin with one or more file
attributes\index{file attribute@\T{file} attribute}, as
described in \S\ref{section:file-attrib}.

\index{module attribute@\T{module} attribute|(}
\index{module!name|(}
After those must follow a module attribute
\T{-module(\Z{Name})}, in which the argument \TZ{Name} must be an atom.  The
module attribute states the name of the module.
\index{module attribute@\T{module} attribute|)}
\index{module!name|)}

The module attribute is followed by a \emph{header part}\index{module!header
part of declaration}, consisting of a
(possibly empty) sequence of
header forms, which are additional attributes and declarations of the module.

Finally there is a \emph{code part}\index{module!code
part of declaration}, consisting of a (possibly empty) sequence of
\emph{program forms}: function
declarations, possibly interspersed with such attributes and declarations that
need not be in the header part.

\begin{rules}
\grrule{ModuleDeclaration}
       {\OPT{FileAttributes} \NT{ModuleAttribute} \OPT{HeaderForms} \OPT{ProgramForms}}

\grrule{FileAttributes}
       {\NT{FileAttribute} \OR
        \NT{FileAttributes} \NT{FileAttribute}}

\grrule{ModuleAttribute}
       {\TXT{-} \TXT{module} \TXT{(} \NT{ModuleName} \TXT{)} \NT{FullStop}}

\grrule{ModuleName}
       {\NT{AtomLiteral}}
\end{rules}
\ifOld
There is a restriction in the file compiler that a file named
\T{\Z{Mod}.erl}\index{.erl file name extension@\T{.erl} file name extension}
must contain the full source code of an \Erlang\ module named \TZ{Mod}.
(This file may however
include other files containing header forms, cf.\
\S\ref{section:header-forms}.)
This restriction may be lifted
or altered in a future version of \Erlang.
\fi
\ifNew
The compiler of a \StdErlang\ implementation may require that a file named
\T{\Z{Mod}.erl}\index{.erl file name extension@\T{.erl} file name extension}
must contain the full source code of an \Erlang\ module named \TZ{Mod}.
\fi

\section{The header part}

\label{section:header-forms}
\index{module!header part of declaration}

\index{header!attribute|(}
\index{anywhere attribute|(}
The header part is a sequence of header forms, which are \emph{header
attributes} or \emph{anywhere attributes}.  The former may only appear
in the header part while the latter may also appear in the code part.
\iffalse
% [980514] Junk, I think:
, type
declarations (cf.\ \S\ref{section:type-declarations}) and record declarations (cf.\
\S\ref{section:record-declarations}), and they all begin with a dash.
There are also attributes that may appear anywhere in a module after
the header forms.
\fi

\begin{rules}
\grrule{HeaderForms}
       {\NT{HeaderForm} \OR
        \NT{HeaderForms} \NT{HeaderForm}}

\grrule{HeaderForm}
       {\NT{HeaderAttribute} \OR
        \NT{AnywhereAttribute}}

\grrule{HeaderAttribute}
       {\NT{ExportAttribute} \OR
        \NT{ImportAttribute} \OR
        \NT{CompileAttribute} \OR
        \NT{WildAttribute}}

\grrule{AnywhereAttribute}
       {\NT{FileAttribute} \OR
        \NT{MacroDefinition} \OR
        \NT{RecordDeclaration}\iftypedecl \OR
        \NT{TypeDeclaration} \OR
        \NT{RuleDeclaration}\fi}
\end{rules}
\index{header!attribute|)}
\index{anywhere attribute|)}

\iffalse Rule declarations are defined in Chapter~\ref{chapter:rules}.\fi

Macro definitions are defined in \S\ref{section:macros}.

The remaining attributes (both header attributes and others)
are described in the following sections.

\subsection{Export attributes}

\label{section:export}
\index{export attribute@\T{export} attribute|(}
\index{function!exported|(}
\index{module!exported functions of|(}

An export attribute is syntactically like an application of
a (hypothetical) function \T{export/1} to a list of
\emph{function names}, preceded by a dash.  Each function name
consists of a function symbol and an arity numeral, separated by
\T{/}.

\begin{rules}
\grrule{ExportAttribute}
       {\TXT{-} \TXT{export} \TXT{(} \NT{FunctionNameList} \TXT{)} \NT{FullStop}}

\grrule{FunctionNameList}
       {\TXT{[} \OPT{FunctionNames} \TXT{]}}

\grrule{FunctionNames}
       {\NT{FunctionName} \OR
        \NT{FunctionNames} \TXT{,} \NT{FunctionName}}

\grrule{FunctionName}
       {\NT{FunctionSymbol} \TXT{/} \NT{Arity}}
       
\grrule{FunctionSymbol}
       {\NT{AtomLiteral}}

\grrule{Arity}
       {\NT{IntegerLiteral}}
\end{rules}

There may be more than one export attribute among the header forms.
The order between functions names in an export attribute is
irrelevant.  The same function name may appear more than once in an
export attribute and in several export attributes.

If there is an export attribute in module \TZ{M} that lists the
function \T{\Z{F}/\Z{A}}, then the function with symbol \TZ{F} and
arity \TZ{A} in module \TZ{M} can be applied using a remote
application (\S\ref{section:function-application}).

Note that if there is \emph{not} an export attribute in module
\TZ{M} that lists the function \T{\Z{F}/\Z{A}}, then the function cannot
be applied through a remote application even in the code for module \TZ{M}.

It is a compile-time error if there is an export attribute listing a function
\T{\Z{F}/\Z{A}} but no declaration of a function \T{\Z{F}/\Z{A}}.
\index{export attribute@\T{export} attribute|)}
\index{function!exported|)}
\index{module!exported functions of|)}

\subsection{Import attributes}

\label{section:import-attribute}
\index{import attribute@\T{import} attribute|(}
\index{function!imported|(}
\index{module!imported functions of|(}

An import attribute is syntactically like an application of a
(hypothetical) function \T{import/1} to a module name and a list of
function names (cf.\ \S\ref{section:export}), preceded by a dash.

\begin{rules}
\grrule{ImportAttribute}
       {\TXT{-} \TXT{import} \TXT{(} \NT{ModuleName} \TXT{,} \NT{FunctionNameList} \TXT{)} \NT{FullStop}}
\end{rules}
Import attributes have the following role.  If there is an import
attribute in module \TZ{M} that lists the function \T{\Z{F}/\Z{A}} of
module $\TZ{M}'$, then a seemingly local function application
\T{\Z{F}(\(\Z{E}_{1}\),\tdots,\(\Z{E}_{\TZm{A}}\))} in module \TZ{M}
is syntactic sugar for a remote function application
\T{$\Z{M}'$:\Z{F}(\(\Z{E}_{1}\),\tdots,\(\Z{E}_{\TZm{A}}\))}
(unless there is a BIF named \T{\Z{F}/\Z{A}}, cf.\
\S\ref{section:application-exprs}).

There may be more than one import attribute among the header forms.
The order between import attributes and between functions names in an
import attribute is irrelevant.  The same function name may appear
more than once in an import attribute and in several import attributes
for the same module.

It is a compile-time error if in the same module there are import attributes
listing the same function name but different modules.

If in some module there is both an import attribute listing a function
\T{\Z{F}/\Z{A}} and a function declaration for \T{\Z{F}/\Z{A}}, then
all applications of the form
\T{\Z{F}(\(\Z{E}_{1}\),\tdots,\(\Z{E}_{\TZm{A}}\))} in the module will
refer to the imported function.  However, if there is an export
attribute listing \T{\Z{F}/\Z{A}}, it refers to the function being
declared.  For example, consider the following module:

\begin{verbatim}
-module(expimp).

-import(lists,[sort/1]).

-export([sort/1]).

sort(X) -> sort(X).
\end{verbatim}

The declaration of the function \T{sort/1} may appear to be recursive
but it is not: the application \T{sort(X)} to the right of `\T{->}'
calls the imported function.  However, the export attribute refers to
the function being declared in module \T{expimp}.  (Here the function
being declared simply returns the result of the imported function but
if the module is later replaced by a new version, the function
\T{sort/1} may be defined differently.)
\index{import attribute@\T{import} attribute|)}
\index{function!imported|)}
\index{module!imported functions of|)}

\subsection{Compile attributes}

\label{section:compile-attrib}
\index{compile attribute@\T{compile} attribute|(}
\index{module!compiler options for|(}

A compile attribute is syntactically like an application of a (hypothetical) unary
function \T{compile} to a list of terms.

\begin{rules}
\grrule{CompileAttribute}
       {\TXT{-} \TXT{compile} \TXT{(} \TXT{[} \OPT{Terms} \TXT{]} \TXT{)} \NT{FullStop}}
\end{rules}
The terms in the list will be appended at the end of the list of
options given to the \T{compile:file/2} function.  The terms must
therefore be acceptable as options for the \Erlang\ compiler being
used, or compilation will fail.

The compiler options are implementation-defined.
% or should they be???
\index{compile attribute@\T{compile} attribute|)}
\index{module!compiler options for|)}

\subsection{File attributes}

\label{section:file-attrib}
\index{file attribute@\T{file} attribute|(}

%Stinks.

A file attribute is syntactically like an application of a
(hypothetical) function \T{file/2} to a string literal (presumably a
file name although it is not an error if no such file exists) and a
line numeral, preceded by a dash.

\begin{rules}
\grrule{FileAttribute}
       {\TXT{-} \TXT{file} \TXT{(} \NT{StringLiteral} \TXT{,}
       \NT{LineNumeral} \TXT{)} \NT{FullStop}}

\grrule{LineNumeral}
       {\NT{IntegerLiteral}}
\end{rules}
The purpose of a file attribute \T{-file(\Z{F},\Z{L})} is to inform
the compiler that the code being compiled originated from line \TZ{L}
in the file \TZ{F} so that error messages can refer to the proper
place in the original file.
\index{file attribute@\T{file} attribute|)}

\iftypedecl
\subsection{Type declarations}

\label{section:type-declarations}

The \T{type} and \T{deftype} attributes are reserved for a possible future
extension.

\begin{rules}
\iftypedecl
\grrule{TypeDeclaration}
       {\TXT{-} \TXT{type} \TXT{(} \OPT{Tokens} \TXT{)} \NT{FullStop} \OR
        \TXT{-} \TXT{deftype} \TXT{(} \OPT{Tokens} \TXT{)} \NT{FullStop}}
\fi
\end{rules}
\fi

\subsection{Wild attributes}

\index{wild attribute|(}

A wild attribute is syntactically like an application of a unary
function other than \T{export/1}, \T{import/1}, \T{file/2},
\T{compile/1}, \T{type/1}, \T{deftype/1} and \T{record/1} to a term.

\begin{rules}
\grrule{WildAttribute}
       {\TXT{-} \NT{AtomLiteral} \TXT{(} \NT{Term} \TXT{)}
       \NT{FullStop}}
\end{rules}

\index{module_info/1@\T{module_info/1}!function|(}
The compiler gathers all wild attributes for a module \TZ{M} and at
runtime, a BIF call \T{\Z{M}:module_info(attributes)} returns a list
of them (cf.\ \S\ref{section:moduleinfo1}).
\index{module_info/1@\T{module_info/1}!function|)}
\index{wild attribute|)}

\section{Program forms}

\label{section:program-forms}
\index{program!form|(}
\index{function!declaration|(}

The program forms are \emph{function declarations} and such attributes
that may appear anywhere in a module declaration. (For unambiguity of
the grammar, it is required that the \NT{ProgramForms} begins with a
\NT{FunctionDeclaration} but there is no significant difference
between an \NT{AnywhereAttribute} that appears as part of the
\NT{HeaderForms} and one that appears as part of the
\NT{ProgramForms}.)

\begin{rules}
\grrule{ProgramForms}
       {\NT{FunctionDeclaration} \OR
        \NT{ProgramForms} \NT{FunctionDeclaration} \OR
        \NT{ProgramForms} \NT{AnywhereAttribute}}

\grrule{FunctionDeclaration}
       {\NT{FunctionClauses} \NT{FullStop}}

\grrule{FunctionClauses}
       {\NT{FunctionClause} \OR
        \NT{FunctionClauses} \TXT{;} \NT{FunctionClause}}

\grrule{FunctionClause}
       {\NT{FunctionSymbol} \NT{FunClause}}

\ifappendix\else
\grrule{FunClause}
       {\TXT{(} \OPT{Patterns} \TXT{)} \OPT{ClauseGuard} \NT{ClauseBody}}
\fi
\end{rules}
(The rule for \NT{FunClause} is repeated from \S\ref{section:fun-exprs} for
convenience.)

\index{function!clause|(}
\index{function!name|(}
\index{function!arity|(}
Each function declaration consists of one or more \emph{function
clauses}, separated by semicolons.  Every clause of a function
declaration must begin with the same \NT{AtomLiteral} and the sequence
of arguments following it must have the same length in every clause.
The \NT{AtomLiteral} is the function symbol and together with the
length of the argument sequences, which is the arity of the function,
it forms the \emph{function name}.
\index{function!name|)}
\index{function!arity|)}
A function declaration for a function named \T{\Z{F}/\Z{A}} is thus on
the form
\begin{alltt}
\Z{F}(\(\Z{P}\sb{1,1}\),\tdots,\(\Z{P}\sb{1,\TZm{A}}\)) \([\mbox{\T{when \(\Z{G}\sb{1}\)}}]\) -> \(\Z{B}\sb{1}\) ;
\(\vdots\) ;
\Z{F}(\(\Z{P}\sb{k,1}\),\tdots,\(\Z{P}\sb{k,\TZm{A}}\)) \([\mbox{\T{when \(\Z{G}\sb{k}\)}}]\) -> \(\Z{B}\sb{k}\).
\end{alltt}
where the guard part of each function clause is optional.
\index{function!clause|)}

It is a compile-time error if in a module there is more than one
declaration for a function name.

Application of functions is defined in \S\ref{section:application-exprs}.
\index{program!form|)}
\index{function!declaration|)}

\section{Record declarations}

\label{section:record-declarations}
\index{record!declaration|(}
\index{record@\T{record}!attribute|(}

\begin{rules}
\grrule{RecordDeclaration}
       {\TXT{-} \TXT{record} \TXT{(} \NT{RecordType} \TXT{,} \NT{RecordDeclTuple} \TXT{)} \NT{FullStop}}

\grrule{RecordDeclTuple}
       {\TXT{\char`\{} \OPT{RecordFieldDecls} \TXT{\char`\}}}

\grrule{RecordFieldDecls}
       {\NT{RecordFieldDecl} \OR
        \NT{RecordFieldDecls} \TXT{,}\ \NT{RecordFieldDecl}}

\grrule{RecordFieldDecl}
       {\NT{RecordFieldName} \OPT{RecordFieldValue}}
\end{rules}
(The rules for \NT{RecordType} and \NT{RecordFieldName} appear
in \S\ref{section:pattern-matching} while
the rules for \NT{RecordFieldValue} appear
in \S\ref{section:record-exprs}.)

In a record declaration
\begin{alltt}
-record(\Z{R},\{\(\Z{F}\sb{1}[\mbox{\T{=\(\Z{E}\sb{1}\)}}],\tdots,\Z{F}\sb{n}[\mbox{\T{=\(\Z{E}\sb{n}\)}}]\)\})
\end{alltt}
the field names $\TZ{F}_1$, \ldots, $\TZ{F}_n$ must all be distinct
and the (optional) default initializer expressions $\TZ{E}_1$, \ldots,
$\TZ{E}_n$ will be evaluated in an empty environment and must have an
empty output environment.  It should be a compile-time error if a
default initializer expression contains a free variable.

The scope of the record declaration begins immediately after its
lexical occurrence and ends at the end of the module declaration.  In
a module there must be at most one record declaration for each record
type.  It is possible for two modules that are part of the same
application to have incoherent record declarations. This could lead to
severe problems. Similar problems may appear, unless caution is taken,
when loading a new version of a module if its record declarations have
changed (because there may be existing records created according to
the old record declarations).

The record declaration establishes \TZ{R} as the name of a record type
having $n$ fields named $\TZ{F}_1$, \ldots, $\TZ{F}_n$.  The optional
default initializer expression $\TZ{E}_i$ is used when a record is
created (\S\ref{section:record-creation}) and no value is specified
for the field named $\TZ{F}_i$.  (If no default initializer expression
is given either, then the atom \T{undefined} is used.)  For example, a
record declaration
\begin{verbatim}
-record(music,{title,artist,medium=cd,number})
\end{verbatim}
declares \T{music} to be a record type and that each record term of type
\T{music} has four
fields named \T{title}, \T{artist}, \T{medium} and \T{number}.  An expression
\begin{verbatim}
#music{}
\end{verbatim}
creates a record in which these fields have the values
\T{undefined}, \T{undefined}, \T{cd} and \T{undefined}, respectively, while
an expression
\begin{verbatim}
#music{title="The Dark Side of the Moon",
       artist="Pink Floyd",
       medium=lp}
\end{verbatim}
creates a record in which the fields have the values
\T{"The Dark side of the Moon"}, \T{"Pink Floyd"}, \T{lp} and \T{undefined}, respectively
(the default initializer expression \T{cd} for \T{medium} was overridden).
\index{record!declaration|)}
\index{record@\T{record}!attribute|)}
\index{module!declaration|)}

\section{The module information functions}

When compiling a module, the compiler automatically adds declarations
for two functions \T{module_info/0} and \T{module_info/1} in it.

\subsection{The function \T{module\char'137info/0}}

\label{section:moduleinfo0}
\index{module_info/0@\T{module_info/0}!function|(}

In any module \TZ{M}, the compiler automatically adds a declaration of a function
\T{module_info/0}, such that an application \T{\Z{M}:module_info()}
returns an association list (cf.\
\S\ref{section:assocationlists}).  This association list is such
that for every key \TZ{K} that the function \T{module_info/1} accepts,
except \T{module}, the list contains a 2-tuple
\T{\{\Z{K},\Z{M}:module_info(\Z{K})\}} and no other 2-tuples.
\index{module_info/0@\T{module_info/0}!function|)}

\subsection{The function \T{module\char'137info/1}}

\label{section:moduleinfo1}
\index{module_info/1@\T{module_info/1}!function|(}

In any module \TZ{M}, the compiler adds a declaration for an exported function
\T{module_info/1} such that for certain terms, the function returns a value.
It must hold that:
\begin{itemize}
\item \index{module!name}
\T{\Z{M}:module_info(module)} returns \TZ{M} (i.e., the name of the module as an atom).
\item \index{module!exported functions of|(}
\T{\Z{M}:module_info(exports)} returns a list of 2-tuples such that
for every function name \T{\Z{F}/\Z{A}} that appears in an export
attribute, the list contains a 2-tuple \T{\{\Z{F},\Z{A}\}}.  That is, the first element
of each 2-tuple is an atom and the second element is an integer.  There are no
other 2-tuples in the list and the order between the 2-tuples is undefined.
\index{module!exported functions of|)}
\item \index{module!imported functions of|(}
\T{\Z{M}:module_info(imports)} returns a list of 2-tuples such that
for every function name \T{\Z{F}/\Z{A}} that appears in an import attribute for
a module $\TZ{M}'$, the list contains a 2-tuple \T{\{\{\Z{F},\Z{A}\},$\TZ{M}'$\}}.
That is, the first element of each 2-tuple is itself a 2-tuple of an atom and an integer,
and the second element of each pair is an atom.
There are no
other 2-tuples in the list and the order between the 2-tuples is undefined.
\index{module!imported functions of|)}
\item \index{module!attributes of|(}
\T{\Z{M}:module_info(attributes)} returns an association list such that
for every wild or \T{compile} attribute \T{-\Z{K}(\Z{T})} of module \TZ{M}, the list
contains a 2-tuple \T{\{\Z{K},\Z{T}\}}.  That is, the first element of each
2-tuple is an atom and the second element is an arbitrary term.
There are no other 2-tuples in the list and the order between the 2-tuples is undefined.
\index{module!attributes of|)}
\item \index{module!information about compilation|(}
\T{\Z{M}:module_info(compile)} returns an association list with information
about the compilation of module \TZ{M}.  The association list must at least
map the atom \T{time} to the time when compilation of module \TZ{M} began
(represented as a six-tuple of integers: year, month, day, hours, minutes and seconds,
like the elements in the result of the BIF \T{date/0} [\S\ref{section:date0}]
followed by the elements
in the result of the BIF \T{time/0} [\S\ref{section:time0}])
% UPPDATERA OM BIF-PROPOSAL G�R IGENOM!!!
% GMT?
and the atom \T{options} to the list of options that were given to the
compiler (including those provided through \T{compile} directives
[\S\ref{section:compile-attrib}]).
\ifNew
An implementation may include additional 2-tuples provided that it is documented
what they are.
\fi
\index{module!information about compilation|)}
\end{itemize}
An implementation may let \T{\Z{M}:module_info/1} accept
additional atoms as argument (in which case the association list
returned by \T{\Z{M}:module_info/0} should be extended accordingly,
cf.\ \S\ref{section:moduleinfo0}).
\index{module_info/1@\T{module_info/1}!function|)}


\grammarindexfalse

%
% %CopyrightBegin%
%
% Copyright Ericsson AB 2017. All Rights Reserved.
%
% Licensed under the Apache License, Version 2.0 (the "License");
% you may not use this file except in compliance with the License.
% You may obtain a copy of the License at
%
%     http://www.apache.org/licenses/LICENSE-2.0
%
% Unless required by applicable law or agreed to in writing, software
% distributed under the License is distributed on an "AS IS" BASIS,
% WITHOUT WARRANTIES OR CONDITIONS OF ANY KIND, either express or implied.
% See the License for the specific language governing permissions and
% limitations under the License.
%
% %CopyrightEnd%
%

\chapter{Dynamics of modules}

\label{chapter:module-dynamics}

\Erlang\ has been designed to make it possible to incorporate functionality for
replacing a version of a module with a new version of that module, even though
at the same time there are processes executing the old version of the module.

\ifOld
An example of such functionality is provided by a collection of BIFs:
\T{load_module/2}, \T{delete_module/1}, \T{purge_module/1}, etc.,
which are described in this chapter.\footnote{These BIFs are, in turn, used for
implementing the \T{code} module of OTP \cite[p.~158--167]{otp-dev-ref}.}
\fi
\ifStd
An example of such functionality is provided by the standard module
\T{codeload} which is described in this chapter.\footnote{The module
\T{codeload} is, in turn, used for
implementing the \T{code} module of OTP \cite[p.~158--167]{otp-dev-ref}.}
Other (nonstandard) modules for code maintenance could behave differently.
\fi

A process that is evaluating
an application of a function in the old version of module can finish the evaluation
of that application even though a new version of the function has been loaded.
However, there can be only one ``old version'' of a module so
it is presumed that the \Erlang\ code of the modules has been written
so that processes will begin using the new version of the module as soon as possible.

As soon as a new version of a module named \TZ{Mod} has been loaded, evaluation of
remote applications of functions in module \TZ{Mod} will use the new version of module
(\S\ref{section:exported-functions}).

\index{function!tail recursive|(}
This implies, for example, that a tail recursive function which is intended to run
for a long time should typically use a remote function application
for the tail recursive call.  The tail recursive call will then use
the most recently loaded version of the module.  (Note also that
\ifStd
the requirement that an \Erlang\ implementation must provide
\fi
\ifOld the \fi
\emph{last call optimization}\index{last call optimization}
[\S\ref{section:lco}] entails that evaluation of such a tail recursive function
can run using stack space bounded only by the need for each iteration.)
\index{function!tail recursive|)}

Obviously, the programmer must be aware of the possibility that modules may be
replaced by new versions and design the code so it will cope gracefully with such
replacements.

\section{Loading or replacing a module}

\label{section:current-version}
\label{section:replacing-module}
\label{section:loading}

\index{module!version of|(}
A \emph{version} of a module named \TZ{Mod} is the result of compiling a
\NT{ModuleDeclaration} with a module attribute \T{-module(\Z{Mod})}, represented
as a binary (\S\ref{section:code-generation}).
(Module declarations are described in \S\ref{chapter:programs-modules}
and their compilation in \S\ref{chapter:compilation}.)

\index{module!version of!current|(}
The \emph{current version} of a module named \TZ{Mod} on a node \TZ{N} is the
one for which there are rows in \T{entry_points[\Z{N}]} (\S\ref{section:exported-functions}).
the current version of \TZ{Mod} on \TZ{N} is accessible as
\T{current_version[module_table[\Z{N}](\Z{Mod})]} (\S\ref{section:loaded-modules}).

\index{module!version of!old|(}
A version of a module named \TZ{Mod}, represented as a binary \TZ{B},
is the \emph{current version} on node \TZ{N} from the time it has been loaded on node \TZ{N}
(\S\ref{section:making-current-version}) until it is made old
on node \TZ{N} (\S\ref{section:making-old-version}).
\index{module!version of!current|)}

The \emph{old version} of a module named \TZ{Mod} on a node \TZ{N} has all its code
intact on \TZ{N} but there are no longer any rows for it in \T{entry_points[\Z{N}]} so
no new remote function calls can reach code in it.  The version is the old version
until it is purged
\index{module!version of!old|)}
(\S\ref{section:purging-old-version}).

When we write that a process \TZ{P} uses some version of a module \TZ{Mod},
represented as a binary \TZ{B}, we mean that \TZ{P} is
using\label{section:process-using-module}
some exported function \T{\Z{F}/\Z{A}} in \TZ{B} (\S\ref{section:function-use},
\S\ref{section:checking-process-module}).
\index{module!version of|)}

\index{module!replacing a version|(}
When a version of a module named \TZ{Mod} is to be replaced with a
new version on an \Erlang\ node
\TZ{N}, the following sequence of steps is intended to be followed:
\begin{itemize}
\item The current version of module \TZ{Mod} on node \TZ{N}, if any,
is made the old version (\S\ref{section:making-old-version}).
\item A version of module \TZ{Mod} is made the current version of \TZ{Mod}
on node \TZ{N} (\S\ref{section:making-current-version}).
\item When it has been ensured that no process on node \TZ{N} is using the old version
of module \TZ{Mod} anymore,
the old version of \TZ{Mod} is purged (\S\ref{section:purging-old-version}).
This reclaims the space used on node \TZ{N} by the old version of the code.
\index{module!replacing a version|)}

Ensuring that no process is using the old version
of a module can be accomplished in several ways, e.g.:
\begin{itemize}
\item Waiting until all such processes complete or no longer use the old version
of the code (cf.\ \S\ref{section:checking-process-module}).
\item Killing such processes.
\item Designing the code of the module so that a process using it can be sent a message that
makes it prepare for a version change by, e.g., completing or evaluating a remote
tail recursive call.
\end{itemize}
\end{itemize}

\section{Loaded modules}

\label{section:loaded-modules}
\index{module!loaded on a node|(}

At any time, a module is \emph{loaded} on node \TZ{N} if there is a current version of
\TZ{Mod} on \TZ{N}, i.e., if \T{current_version[module_table[\Z{N}](\Z{Mod})]} is a binary.

\S\ref{section:making-current-version} and \S\ref{section:making-old-version}
describe how \T{current_version[module_table[\Z{N}](\Z{Mod})]} is set to a binary or
to \T{none}, respectively.

A process residing on node \TZ{N} can find out whether module \TZ{Mod} is loaded on
node \TZ{N} through a BIF call
\ifOld \T{erlang:module_loaded(\Z{Mod})}\fi
\ifStd \T{codeload:module_loaded(\Z{Mod})}\fi,
which inspects \T{current_version[module_table[\Z{N}](\Z{Mod})]} and
returns \T{true} if the value is a binary
and \T{false} if it is \T{none} (\S\ref{section:moduleloaded1}).

% 0.7 removed, there is enugh elsewhere.
\iffalse
\index{module!preloaded on a node|(}
A process residing on node \TZ{N} can obtain a list of the module names in
\T{preloaded[\Z{N}]}, i.e., the names of the modules
that were loaded as part
of starting the node \TZ{N}, through a BIF call
\ifOld \T{erlang:pre_loaded()}\fi
\ifStd \T{codeload:pre_loaded()}\fi.
The order of the elements in the list is not specified.
\index{module!preloaded on a node|)}
\fi
\index{module!loaded on a node|)}

\section{Exported functions of a module}

\label{section:exported-functions}
\index{module!exported functions of|(}
\index{function!exported|(}

Each node maintains a table \T{entry_points[\Z{N}]} (\S\ref{section:node-state-dynamic})
in which the keys are triples consisting of a module name (an atom), a function
symbol (an atom) and an arity (a nonnegative integer) and the values are
\ifStd implementation-defined \fi
entry points to executable code.  The table contains one row for
each exported function of each loaded module.

\S\ref{section:making-current-version} and \S\ref{section:making-old-version}
describe how rows are are added to and removed from
\T{entry_points[\Z{N}]}, respectively.

The table \T{entry_points[\Z{N}]} is used implicitly
when evaluating remote function applications (\S\ref{section:appl-named-function}).
Only while a version \TZ{B} of a module \TZ{Mod} is current on a node \TZ{N} can
a remote call of an exported function \T{\Z{F}/\Z{n}} in \TZ{B} begin.
\index{module!exported functions of|)}
\index{function!exported|)}

\section{Loading a new current version}

\label{section:making-current-version}
\index{module!making version current|(}

When a binary \TZ{B} is to be made the current version of a module \TZ{Mod}
on a node \TZ{N}, there are two preconditions:
\begin{itemize}
\item \TZ{B} must represent the result of compiling a module declaration for
a module \TZ{Mod}, and
\item there must be no current version of module \TZ{Mod} on node \TZ{N}, i.e.,
\T{current_version[module_table[\Z{N}](\Z{Mod})]} should be \T{none}.
\end{itemize}
The actions required to make \TZ{B} the current version are
\begin{enumerate}
\item For each exported function \T{\Z{Name}/\Z{Arity}} of the version
of \TZ{Mod} represented by \TZ{B}, add to \T{entry_points[\Z{N}]} a
row with $(\TZ{Mod},\TZ{Name},\linebreak[0]\TZ{Arity})$ as key and an
entry point of the executable code for that function as value.
\item Set \T{current_version[module_table[\Z{N}](\Z{Mod})]} to \TZ{B}.
\end{enumerate}
\index{module!making version current|)}

\section{Making a current version old}

\label{section:making-old-version}
\index{module!making version old|(}

When the current version of a module \TZ{Mod} on a node \TZ{N} is to become the
old version of that module on the node, the precondition is that there is
not already an old version, i.e., that 
\T{old_version[module_table[\Z{N}](\Z{Mod})]} is \T{none}.

Let the value of
\T{current_version[module_table[\Z{N}](\Z{Mod})]}, i.e.,
the current version of \TZ{Mod} on \TZ{N} be \TZ{B}.
The actions required to make \TZ{B} the old version of \TZ{Mod} on \TZ{N} are
\begin{enumerate}
\item Remove every row from
\T{entry_points[\Z{N}]} that has \TZ{Mod} as key.
\item Set \T{old_version[module_table[\Z{N}](\Z{Mod})]} to \TZ{B}.
\item Set \T{current_version[module_table[\Z{N}](\Z{Mod})]} to \T{none}.
\end{enumerate}
\index{module!making version old|)}

\section{Purging an old version}

\label{section:purging-old-version}
\index{module!purging old version|(}

When the old version of a module \TZ{Mod} on a node \TZ{N} is to be
purged, there had better be no process using it.  If some process
residing on \TZ{N} is using the old version of \TZ{Mod}, the behaviour
of that process is thereafter undefined.

The action required to purge the old version of \TZ{Mod} on \TZ{N} is
setting\linebreak[2] \T{old_version[module_table[\Z{N}](\Z{Mod})]} to
\T{none}.  If that was the last reference to the binary that was the
old version of \TZ{Mod} on \TZ{N}, then the memory management
subsystem (\S\ref{section:memory-management}) will eventually reclaim
the memory occupied by it.  (The reason that processes still using the
old version may behave erratically is that it cannot be expected that
their references through return addresses will prevent the binary from
being ``garbage collected''.)
\index{module!purging old version|)}

\section{Checking a process for module usage}

\label{section:checking-process-module}
\index{module!used by process|(}
\index{process!using a module|(}

A BIF call
\ifOld \T{erlang:check_process_code(\Z{P},\Z{Mod})}
(\S\ref{section:checkprocesscode2}) \fi
\ifStd \T{codeload:check_process_code(\Z{P},\Z{Mod})}
(\S\ref{section:codeload:checkprocesscode2}) \fi
inspects the value of \T{stack_trace[\Z{P}]} and returns \T{true} if there is
a reference to the code of some function in
\T{old_version[module_table[node[\Z{P}]](\Z{Mod})]} and \T{false} otherwise.

A list of the PIDs of all processes residing on node \TZ{N} can be obtained
through a BIF call
\ifOld \T{processes()} (\S\ref{section:processes0}) \fi
\ifStd \T{node:processes()} (\S\ref{section:node:processes0}) \fi
on node \TZ{N}.
\index{module!used by process|)}
\index{process!using a module|)}


%%% OLD STUFF FROM HERE ON!


\iffalse
\section{Terminology}

% This is MID stuff.
%The most recently loaded module named \TZ{Mod} is called the \emph{current version of
%\TZ{Mod}}.  A module named \TZ{Mod} that is not the current version of \TZ{Mod}
%is called an \emph{old version of \TZ{Mod}}.

% This is MID stuff.
%\section{Modules}

% This is MID stuff with structs too.
Each node \TZ{N} has as part of its state a dictionary
\T{modules[\Z{N}]} mapping atoms (module names) to modules.
This dictionary is used at run time for obtaining the
current version of a named module when evaluating
a remote function application or resolving a remote struct name.
It can also be accessed through the BIF \T{get_module/1}.

A module \TZ{M} consists of
\begin{itemize}
\item A module name (an atom).
\item A mapping \T{exported_functions[\Z{M}]} from exported function
names to functions.
\item A mapping \T{internal_functions[\Z{M}]} from internal function
names to functions.
\item A mapping \T{exported_structs[\Z{M}]} from exported struct names
(atoms) to struct descriptors.
\item A mapping \T{internal_structs[\Z{M}]} from internal struct names
(atoms) to struct descriptors.
\end{itemize}

Given a module \TZ{M}, the mappings \T{exported_functions[\Z{M}]} and
\T{exported_structs[\Z{M}]} can be accessed from within any module
either
\begin{itemize}
\item by evaluating
a remote function application or resolving a remote struct name, or
\item through the BIFs \T{get_function/3} and \T{get_struct/2}.
\end{itemize}
The mappings \T{internal_functions[\Z{M}]} and
\T{internal_structs[\Z{M}]} can only be accessed from code occurring
lexically inside the module declaration for \TZ{M} by evaluating a
local function application or resolving a local struct name.

\section{The module dictionary}

Modules having the same name are presumed to provide similar
functionality and a module is thus presumed to supercede any
previously loaded module with the same name.  A remote function
application or remote struct name uses the module with that name most
recently loaded onto the node.  This is accomplished through the table
\T{modules[\Z{N}]} for each node \TZ{N}, which has module names as
keys BLURP...  to the most recently loaded module with that name.

Loading a module \TZ{M} with name \TZ{Mod} onto a node \TZ{N} means
that \T{modules[\Z{N}]} should be updated so that it maps the module
name \TZ{Mod} to \TZ{M}.  This does not have any effect on any
previously loaded module $\TZ{M}'$ named \TZ{Mod}.  A process that is
evaluating an application of a function in such a module $\TZ{M}'$
will continue to use functions in $\TZ{M}'$ for evaluating local
function applications and resolving local struct names.  However,
should it evaluate a remote function application or resolve a remote
struct name, that module name will be looked up in
\T{modules[\Z{N}]} and may yield \TZ{M} but never $\TZ{M}'$.

When there is no longer any reference to a module, the memory it
occupies can be reclaimed, as for any other data structure.

This means that the \emph{language} permits the simultaneous existence
of an unlimited number of versions of a module. For some
\emph{applications} this may not be desirable.  There are thus means
for obtaining at run time information about which processes have
ongoing evaluations of functions in a certain module. Such processes
can then be killed in order to purge all references to a module so
that its memory will eventually be reclaimed.

\section{Loading modules}

\T{codeload:load_module/1} or \T{2}?
\fi


%
% %CopyrightBegin%
%
% Copyright Ericsson AB 2017. All Rights Reserved.
%
% Licensed under the Apache License, Version 2.0 (the "License");
% you may not use this file except in compliance with the License.
% You may obtain a copy of the License at
%
%     http://www.apache.org/licenses/LICENSE-2.0
%
% Unless required by applicable law or agreed to in writing, software
% distributed under the License is distributed on an "AS IS" BASIS,
% WITHOUT WARRANTIES OR CONDITIONS OF ANY KIND, either express or implied.
% See the License for the specific language governing permissions and
% limitations under the License.
%
% %CopyrightEnd%
%

\chapter{Processes and concurrency}

\label{chapter:processes}
\index{process|(}

\section{An overview of \Erlang\ processes}

\label{section:spawning-processes}

An \Erlang\ \emph{process} is an entity that exists for a certain
time, is evaluating a function application, has a state and is able to
communicate with other processes during its lifetime.  It may complete
normally or abruptly.

\index{PID|(}
Each process is associated with a unique term, called its \emph{process
identifier} or \emph{PID}.  The PID of a process is used for referring
to the process, for example, when communicating with it. As there is
a one-to-one correspondence between processes and PIDs, we will often
abuse our terminology and write about the PIDs as if they were the
actual processes. A process can obtain its own PID using the BIF
\T{self/0}\index{self/0 BIF@\T{self/0} BIF} (\S\ref{section:self0}).
\index{PID|)}

\index{process!spawning a|(}
Normally a function application appears in a program through an
expression on the form \T{\Z{F}($\Z{T}_1$,\tdots,$\Z{T}_k$)} or
\T{\Z{M}:\Z{F}($\Z{T}_1$,\tdots,$\Z{T}_k$)} (or through an application
of one of the BIFs \T{apply/2} and \T{apply/3}).
In the normal mode of
evaluation, the expression is then evaluated as part of the same
process until it completes
normally, in which case the result is the value of the expression,
or it completes abruptly, in which case the enclosing expression also
completes abruptly.

Through the BIFs \T{spawn/3}, \T{spawn/4}, \T{spawn_link/3}
and \T{spawn_link/4} (\ifOld\S\ref{section:process-bifs}\fi
\ifNew\S\ref{section:process-module}\fi)
it is possible to request instead that the application is evaluated in a process
of its own.  An application \T{spawn(\Z{M},\T{F},[$\Z{E}_1$,\tdots,$\Z{E}_k$])}
has the effect of spawning a new process that evaluates the application
\T{\Z{M}:\Z{F}($\Z{E}_1$,\tdots,$\Z{E}_k$)}, but does not wait for its
value to be computed.  The value of the expression
\T{spawn(\Z{M},\T{F},[$\Z{E}_1$,\tdots,$\Z{E}_k$])} is instead the
PID of the new process.  The value of
\T{\Z{M}:\Z{F}($\Z{E}_1$,\tdots,$\Z{E}_k$)} cannot be accessed even
though it is computed (when the computation completes normally) so it is
obvious that any result of a process must be made available through
communication.

\index{process!initial call|(}
Let the values of the expressions \TZ{M}, \TZ{F}, $\TZ{E}_1$, \ldots, $\TZ{E}_k$
be \TZ{Mod}, \TZ{Fun}, $\TZ{v}_1$, \ldots, $\TZ{v}_k$.  The call of
the function \T{\Z{Mod}:\Z{Fun}/$k$} to the values  $\TZ{v}_1$, \ldots, $\TZ{v}_k$
is known as the \emph{initial call} of the process.
\index{process!initial call|)}

The BIF \T{spawn/3} spawns the new process on the same node as the
process evaluating the application while
\T{spawn/4} spawns the new process on a specified node (\S\ref{chapter:nodes}).
The BIFs \T{spawn_link/3} and \T{spawn_link/4} are like
\T{spawn/3} and \T{spawn/4}, respectively, but link the spawning and
spawned processes (\S\ref{section:links}).
\index{process!spawning a|)}

\index{process!communication|(}
\index{signal!communication|(}
\index{message!communication|(}
\index{exit!signal|(}
Processes communicate through \emph{signals}.  The two kinds
of signals that are immediately noticeable for \Erlang\ programmers
are \emph{messages} and \emph{exit signals}.
Both messages and exit signals are terms but they are sent and received
differently and with different purposes.

An exit signal is automatically sent upon completion of a process
and it may cause abrupt completion of other processes, as described
in \S\ref{section:exit-signals}.
\index{exit!signal|)}

A message is sent by evaluating a \emph{send expression}, i.e.,
an application of the \T{!}\ operator (\S\ref{section:send-expr}).
\iffalse
The value of the left-hand operand should be the PID of the receiving
process (or an atom that is registered to identify a PID)
and the value of the right-hand operand should be the message to be sent.
\fi
Correspondingly, evaluating a \T{receive}
expression (\S\ref{section:receive-expr}) normally receives one message and
dispatches on its form.
The mechanism for communication by messages is described in
\S\ref{section:messages}.
\index{message!communication|)}
\index{process!communication|)}
\index{signal!communication|)}

\section{Process names}

\label{section:process-names}
\index{process!registry|(}

Each node maintains a registry of names of processes, providing a level
of abstraction when referring to a process.  These names are atoms and
can be used instead of the PID when sending a message to a process.
It is also possible to retrieve the current PID that a name is
registered to stand for.

Process names are described in more detail elsewhere
(\S\ref{section:process-registry}) as are the BIFs
\T{register/2}, \T{whereis/1}, \T{unregister/1} and \T{registered/0}
(\ifOld\S\ref{section:process-registry-bifs}\fi
\ifNew\S\ref{section:process-module}\fi).
\index{process!registry|)}

\section{Linked processes}

\label{section:links}
\index{process!linking|(}

\index{exit!signal|(}
A pair of processes $\TZ{P}_1$ and $\TZ{P}_2$
may be \emph{linked},
which means that when
$\TZ{P}_1$ completes, an exit signal (\S\ref{section:exit-signals}) is sent to
$\TZ{P}_2$, and vice versa. Receiving the exit signal can either cause $\TZ{P}_2$ to
complete abruptly or notify $\TZ{P}_2$ of the completion of $\TZ{P}_1$
in the form of a message (\S\ref{section:messages}). Links are
completely symmetric so there is no distinction between the linked
processes. A process may link to itself but that has no effect.
\index{exit!signal|)}

\iffalse
Although a link is between two processes, the transitive closure of
the links forms groups of processes.  Unless some process in such a
group is trapping exit signals (\S\ref{section:exit-signals}),
completion of one process in the group eventually causes (abrupt)
completion of all processes in the group.
\else
Note that when a process $\TZ{P}_1$ is linked to two or more processes,
say $\TZ{P}_2$ and $\TZ{P}_3$, then if $\TZ{P}_2$ completes abruptly,
the exit signal received by $\TZ{P}_1$ will cause abrupt completion also
of $\TZ{P}_1$ (unless it is trapping exits), which will cause an exit signal
to be sent to $\TZ{P}_3$, etc.  The abrupt completion of $\TZ{P}_2$ thus
causes abrupt completion of $\TZ{P}_3$ even though the processes were not
linked directly.
In this way, abrupt completion of one process in a collection of linked
processes may cause abrupt completion of several or all processes in the group.
\fi

A link can be created between two processes
$\TZ{P}_1$ and $\TZ{P}_2$ in the following ways:
\begin{itemize}
\item If one of the processes spawned the other using the BIF
\T{spawn_link/3}\index{spawn_link/3 BIF@\T{spawn_link/3} BIF}
or \T{spawn_link/4}\index{spawn_link/4 BIF@\T{spawn_link/4} BIF}.
In this case, the link is in effect as soon as the process has been created.
\item \index{link/1 BIF@\T{link/1} BIF|(}
If process $\TZ{P}_1$ calls the BIF \T{link/1}
with $\Z{P}_2$ as argument,
or process $\TZ{P}_1$ calls it with $\Z{P}_2$ as argument.
In this case, setting up the link takes an
unspecified amount of time.
If the two processes are already linked when one of them evaluates an application of
\T{link/1}, it has no effect. If when (say) $\TZ{P}_1$ evaluates \T{link($\Z{P}_2$)} the process
$\TZ{P}_2$ has completed, then no link is created and $\TZ{P}_1$ eventually receives an exit signal
\T{\{'EXIT',$\Z{P}_2$,\Z{noproc}\}} (\S\ref{section:exit-signals}).

For details see the description of the BIF \T{link/1} (\S\ref{section:link1})
and \S\ref{section:signal-arrival}.
\index{link/1 BIF@\T{link/1} BIF|)}
\end{itemize}
\index{unlink/1 BIF@\T{unlink/1} BIF|(}
The link between two processes $\TZ{P}_1$ and $\TZ{P}_2$ is removed if process $\TZ{P}_1$ evaluates
\T{unlink($\Z{P}_2$)} or process $\TZ{P}_2$ evaluates \T{unlink($\Z{P}_1$)}.  It takes an
unspecified amount of time before the link is removed.
If the two processes are not linked when one of them evaluates an application of
\T{unlink/1}, it has no effect.
For details see the description of the BIF \T{unlink/1} (\S\ref{section:unlink1})
and \S\ref{section:signal-arrival}.
\index{unlink/1 BIF@\T{unlink/1} BIF|)}

Obviously, two processes setting up or removing a link between each other may need to
synchronize with messages to ensure the status of the link before further processing.

It is also possible for a process and a port to be linked, as described in
\S\ref{section:port-linking}.
\index{process!linking|)}

\section{Completion of processes and exit signals}

\label{section:exit-signals}

In this section we describe what causes a process to complete and the exit signals and
messages that are dispatched when that happens.  We also describe how exit signals are
sent using the BIF \T{exit/2} and what happens when exit signals are received.

\subsection{Process completion}

\label{section:process-completion}
\index{process!completion|(}

A process may complete for one of the following four reasons:
\begin{itemize}
\item The evaluation of the toplevel application of a process completes normally:
\TZ{Reason} will be \T{normal}.
\item The evaluation of the toplevel application of a process completes abruptly
with exit reason \TZ{R}: \TZ{Reason} will be \TZ{R}.
\item The evaluation of the toplevel application of a process completes abruptly
with a thrown term: \TZ{Reason} will be \T{nocatch}.
\item The process receives an exit signal that causes it to complete abruptly
(\S\ref{section:receiving-exit-signal}),
\TZ{Reason} will then be as specified in \S\ref{section:signal-arrival}.
\end{itemize}

\index{exit/1 BIF@\T{exit/1} BIF|(}
The BIF \T{exit/1} (\S\ref{section:exit1}) is typically used to make a function
call exit with some reason, as some kind of error has been detected.  Unless the
code has been written to restore normal computation, the process executing the
function call will terminate.
That is, evaluation of an application \T{exit(\Z{Reason})} for some term \TZ{Reason} completes
abruptly with the exit term \TZ{Reason} as reason
and unless the evaluation is governed by a \ifNew\T{try} (or \T{catch}) \else \T{catch} \fi
expression, the process completes as described
above.
\index{exit/1 BIF@\T{exit/1} BIF|)}

\index{throw/1 BIF@\T{throw/1} BIF|(}
The BIF \T{throw/1} (\S\ref{section:throw1}) is intended for nonlocal
control, not for signalling an error or making a process terminate.
\index{throw/1 BIF@\T{throw/1} BIF|)}

\index{exit!signal|(}
\ifNew\index{process!monitoring|(}\fi
When a process completes, an \emph{exit signal} is sent to every process
that is linked to it\ifNew\ and a message is sent to every process
monitoring it\fi.
The order in which these exit signals \ifNew and messages \fi are sent
is not defined.
The exit signal contains the PID of the completing process and contains the exit
reason\ifNew, while a message contains only the PID\fi.
\ifNew\index{process!monitoring|)}\fi
\index{exit!signal|)}

\index{process!linking|(}
More precisely, suppose that process \TZ{P} completes for one of the reasons above.
For each process $\TZ{P}'$ for which $\TZ{P}'$ is in \T{linked[\Z{P}]}\ifNew
or for which there is a pair $\langle\TZ{P}',k\rangle$ (for some $k$) in
\T{monitoring_processes[\Z{P}]}\fi,
dispatch an exit signal with \TZ{P} as sender and \TZ{Reason} as reason.

If two processes $\TZ{P}_1$ and $\TZ{P}_2$ residing on different nodes
(\S\ref{chapter:nodes})
are linked and the nodes lose contact, $\TZ{P}_1$ and $\TZ{P}_2$ will receive
exit signals with reason \T{noconnection}, as if the other process had completed.
\index{process!linking|)}

\ifNew
\index{process!monitoring|(}
Similarly a message \T{\{taskdown,$\Z{P}_2$\}} will be sent to $\TZ{P}_1$ if it is
monitoring $\TZ{P}_2$, and vice versa.
\index{process!monitoring|)}
\fi
\index{process!completion|)}

\subsection{Sending exit signals explicitly}

\label{section:sending-exit-signal}

\index{exit!signal!sending|(}
The BIF \T{exit/2} (\S\ref{section:exit2}) can be used to send
exit signals to processes.
If process $\TZ{P}_1$ evaluates the application \T{exit($\Z{P}_2$,\Z{Reason})},
an exit signal with reason \TZ{Reason} is sent to process $\TZ{P}_2$, similarly to
when process $\TZ{P}_1$ is linked with $\TZ{P}_2$ and
completes with exit reason \TZ{Reason}.
However, if \TZ{Reason} is \T{kill}\index{kill@\T{kill}!exit signal},
the receiving process will always
complete (\S\ref{section:signal-arrival}).

Note that if $\TZ{P}_1$ is linked to $\TZ{P}_2$,
and \T{trap_exit[$\Z{P}_2$]}\index{exit!signal!trapping} is \T{false}
(or \TZ{Reason} is \T{kill}\index{kill@\T{kill}!exit signal}),
then $\TZ{P}_2$ will
terminate and an exit signal
will subsequently be sent to $\TZ{P}_1$ itself.

For details, see the description of the BIF
\T{exit/2}\index{exit/2 BIF@\T{exit/2} BIF} (\S\ref{section:exit2}).
\index{exit!signal!sending|)}

\subsection{Receiving an exit signal}

\label{section:receiving-exit-signal}
\index{exit!signal!receiving|(}

\index{exit!signal!trapping|(}
We write that a process \TZ{P} is \emph{trapping exits} if \T{trap_exits[\Z{P}]}
is \T{true}.

When an exit signal is received, one of three things happens:
\begin{itemize}
\item If the receiving process is trapping exits and the exit signal
was not sent using the BIF \T{exit/2}\index{exit/2 BIF@\T{exit/2} BIF} with
\T{kill}\index{kill@\T{kill}!exit signal} as reason,
then the reason is placed in the message queue of the receiving
process.
\item Otherwise, if the reason is
\T{normal}\index{normal@\T{normal}!exit signal}, nothing happens.
\item Otherwise (the reason is not \T{normal} and
either the process is not trapping exits or the exit signal
was sent using the BIF \T{exit/2} with \T{kill} as reason),
the process completes 
(\S\ref{section:signal-arrival}).
\end{itemize}
\index{exit!signal!trapping|)}

For details, cf.~\S\ref{section:signal-arrival}.
\index{exit!signal!receiving|)}

\section{Communication by messages}

\label{section:messages}

\index{message!communication|(}
A message in \Erlang\ can be any \Erlang\ term. A message is sent by a
\emph{sending process} to a specific \emph{receiving process}, identified
by its PID, by a registered name, which is an atom, or by 2-tuple of a node name and
a registered name.

Communication by messages in \Erlang\ is asynchronous, i.e., the process sending
a message does not wait until the message is received.
We shall therefore describe sending of messages and reception of messages in three steps:
\begin{itemize}
\item A message is dispatched from a process with some process as recipient.
\item The message arrives at the message queue of the recipient.
\item The recipient retrieves the message from its message queue.
\end{itemize}
\index{message!communication|)}

\subsection{Sending a message}

\label{section:sending-messages}
\index{message!sending|(}

The only primitive for sending a message in \Erlang\ is the operator \T{!}.
An expression \T{\Z{E}$_1$ !\ \Z{E}$_2$} is called a \emph{send expression}\index{send expression}
(\S\ref{section:send-expr}).

The value of the left operand should either be a PID, an atom that is
registered on the current node as a process name (\S\ref{section:process-names}),
or a 2-tuple of atoms where the first element is a name for some process on the node named
by the second element.
The value of the right
operand is the message to be dispatched. Evaluation of send expressions
is described precisely in \S\ref{section:send-expr}.

There is no direct way for the sending process to find out whether the
message ever arrived at the message queue of the recipient process or was
retrieved\ifOld\ (except that evaluation of a send expression will exit with reason
\T{badarg} when the receiver is specified through an atom that is not a registered name)\fi.

The dispatched message contains no other information than the message as such.
In particular, it does not identify the sending process unless its PID is made part of
the message.
\index{message!sending|)}

\subsection{Message arrival}

\label{section:message-arrival}
\index{message!arrival|(}

When a term \TZ{Msg} arrives at the message queue\index{message!queue}
of the recipient process \TZ{P} it is placed
at the end of \T{message_queue[\Z{P}]}.  If \T{status[\Z{P}]} is \T{waiting}, then
\T{status[\Z{P}]} is changed to \T{runnable} and the process will eventually proceed in
part~3 of the evaluation of a \T{receive} expression (\S\ref{section:receive-expr}).

For details, cf.~\S\ref{section:signal-arrival}.
\index{message!arrival|)}

\subsection{Receiving a message}

\label{section:receiving-messages}
\index{message!reception|(}

The only primitive for receiving a message in \Erlang\
is the \T{receive} expression\index{receive expression@\T{receive} expression}
(\S\ref{section:receive-expr})
which has the syntax
\begin{alltt}
receive
    \Z{P}\(\sb{1}\) \([\)when \Z{G}\(\sb{1}]\) -> \Z{B}\(\sb{1}\) ;
    \tdots
    \Z{P}\(\sb{k}\) \([\)when \Z{G}\(\sb{k}]\) -> \Z{B}\(\sb{k}\)
\([\)after
    \Z{E} -> \Z{B}\(\sb{k+1}]\)
end
\end{alltt}
where the \T{after} part is optional but must be included if $k=0$.

When a receive expression is evaluated, each term in the message queue\index{message!queue}
of the process is matched (in order) against each clause (i.e., a
message is matched against the pattern of the clause and the guard of
the clause is evaluated) until a message is found for which there is a
matching clause. The message is then removed from the message queue of
the process and the body of the first matching clause is evaluated.

If no term in the message queue matched any clause, then the process suspends until
at least one term has arrived at the queue, each such term is then tried as above until one arrives
that matches some clause. However, if the receive expression has an \T{after} part,
then the waiting for a message expires\index{expiry!of \T{receive} expression} after
the specified number of milliseconds have passed
and the expiry body is evaluated.

The evaluation of \T{receive} expressions is described in detail in
\S\ref{section:receive-expr}.
\index{message!reception|)}

\subsection{Order of messages}

\index{message!order|(}
It is assured (through the rules of signals, cf.~\S\ref{section:signal-order})
that if a process $\TZ{P}_1$ dispatches two messages
$\TZ{M}_1$ and
$\TZ{M}_2$ to the same process $\TZ{P}_2$, in that order, then message $\TZ{M}_1$ will never
arrive after $\TZ{M}_2$ at the message queue of $\TZ{P}_2$.

Note that this does not guarantee anything about in which order messages arrive when a process
sends messages to two different processes.  Also note that linking/unlinking requests are
processed as soon as they arrive at a process while messages can be received long after they
arrived at the message queue\index{message!queue} (when the process evaluates a \T{receive} expression).
For example, if a process $\TZ{P}_1$ evaluates
\begin{alltt}
link(\(\Z{P}\sb{2}\)), \(\Z{P}\sb{2}\) ! foo, unlink(\(\Z{P}\sb{2}\))
\end{alltt}
then the link between $\TZ{P}_1$ and $\TZ{P}_2$ may or may not still be in effect when process
$\TZ{P}_2$ actually receives message \T{foo}, depending on whether the unlink request
from process $\TZ{P}_1$ has arrived yet.  In order to guarantee that the link is in effect
when process $\TZ{P}_2$ receives \T{foo}, process $\TZ{P}_2$ could send an answer message
to $\TZ{P}_1$ which would wait to receive that message before unlinking.
\index{message!order|)}

\section{Signals}

\label{section:signals}
\index{signal!communication|(}

In this section we describe a model for how communication between two processes
or between a process and a port takes place in \Erlang.  The model is assumed in
other parts of this specification.  It is important that it is a model:
\ifNew
an implementation is not required to use it for implementation, but communication
should behave according to the model.
\fi
\ifOld
\OldErlang\ does not use it in the implementation,
but communication behaves according to the model.
\fi

All communication between processes and ports takes place through \emph{signals}.
A signal is characterized by a sending process/port, a destination process/port,
a \emph{kind} and additional data depending on the kind of the signal.
The kind is one of the following:
\begin{Lentry}
\item[\I{Message}.] Additional data: a term that is the actual message.
\item[\I{Exit signal}.] Additional data: a Boolean flag saying whether
the sending process just died or not and a reason which is a term.
\item[\I{Link request}.] Additional data: none. See \S\ref{section:links}.
\item[\I{Unlink request}.] Additional data: none. See \S\ref{section:links}.
\item[\I{Group leader request}.] Additional data: the process that is to
be the new group leader of the destination process. See \S\ref{section:group-leader}.
\item[\I{Info request}.] Additional data: a key which is an atom. See \S\ref{section:processinfo2}.
\iffalse
% [980603] I cannot see the need for this one!
\item[\I{Monitor node request}.] Additional data: none. See \S\ref{section:monitor-node}.
\fi
\end{Lentry}
%New signals may be added in the future.
\ifOld
\OldErlang\ has three additional kinds of signals, which are not described in detail here:
\begin{Lentry}
\item[\I{Suspend/resume request}.] For suspending of resuming a process, respectively.
\item[\I{Garbage collection request}.] For requesting that garbage collection be done.
\item[\I{Trace/notrace request}.] For turning tracing of a process on or off, respectively
\end{Lentry}
\fi
\ifNew
An implementation my have additional signals
and at a low level the signals may be different from those reported here (there may
not even be a signal concept) as long as the behaviour with respect to messages,
exit signals, linking/unlinking, etc., coheres with what is described here.
\fi
\index{signal!communication|)}

\subsection{Sending signals}

\index{signal!sending|(}
Sending a signal is completely asynchronous: the sending process has no direct way
to infer when the signal reached its destination or even whether it was ever received.

Sending the various types of signals is described in \S\ref{section:send-expr} (\I{messages}),
\S\ref{section:exit-signals} (\I{exit signals}), \S\ref{section:links} (\I{link and unlink
requests}), \S\ref{section:groupleader2} (\I{group leader requests}) and
\S\ref{section:processinfo2} (\I{info requests}).
% and \S\ref{section:monitornode2} (\I{monitor node requests}).
\index{signal!sending|)}

\subsection{Order of signals}

\label{section:signal-order}
\index{signal!order|(}

The amount of time that passes between the dispatch of a signal $s$ destined
for a process \TZ{P}
and the arrival of $s$ at \TZ{P} is unspecified but positive. If \TZ{P}
has completed, $s$ will never arrive at \TZ{P}.  In that case it is still possible that
$s$ triggers another signal, for example if it is a link request
(cf.~\S\ref{section:signal-arrival}).

\ifOld It is guaranteed that \fi
\ifNew An implementation must guarantee that \fi
if a process $\TZ{P}_1$ dispatches two signals $s_1$ and $s_2$
to the same process $\TZ{P}_2$, in that order, then signal $s_1$ will never
arrive after $s_2$ at $\TZ{P}_2$.
\ifOld It is ensured that \fi
\ifNew An implementation should ensure that \fi
whenever possible, a signal dispatched to a process should eventually arrive at it.
There are
situations when it is not reasonable to require that all signals arrive at their
destination, in particular when a signal is sent to a process on a different node
and communication between the nodes is temporarily lost.
\index{signal!order|)}

\subsection{Arrival of signals}

\label{section:signal-arrival}
\index{signal!arrival|(}

Consider a signal in transit with destination process \TZ{P}.
When the signal arrives at the node on which \TZ{P} resides or resided,
what happens primarily depends on whether \TZ{P} has completed or not.
(If the sender and destination processes reside on the same node, the processing
may nevertheless be subject to delay.)
\begin{itemize}
\item If \TZ{P} has completed, what happens depends on the kind of the signal:
\begin{itemize}
\item \index{process!linking|(}If the signal was a link request from a process $\TZ{P}'$
then an exit signal \T{noproc}\index{noproc exit signal@\T{noproc} exit signal}
is sent to $\TZ{P}'$.\index{process!linking|)}
%\item Anything else???
\item Otherwise, the signal is discarded.
\end{itemize}
\item If \TZ{P} has not completed, what happens also depends on the kind of the signal:
\begin{itemize}
\item \I{Message} with actual message \TZ{M}:
the term \TZ{M} is placed at the end of
\T{message_queue[\Z{P}]}\index{message!queue} (\S\ref{section:message-arrival}).
\item \index{exit!signal|(}
\I{Exit signal} with sender $\TZ{P}'$, flag \TZ{ProcessCompleted} and term \TZ{Cause}.
(\TZ{ProcessCompleted} is \T{true} if $\TZ{P}'$ completed and \T{false} otherwise.)
\index{exit!signal!trapping|(}
First of all it is established whether reception of this exit signal causes \TZ{P}
to complete abruptly or not.
\begin{itemize}
\item If \TZ{ProcessCompleted} is \T{true}, $\TZ{P}'$ is in \T{linked[\Z{P}]},
\T{trap_exit[\Z{P}]} is \T{false} and \TZ{Cause} is something other than
\T{normal}\index{normal@\T{normal}!exit signal},
then \TZ{P} should complete: \TZ{Reason} is \TZ{Cause}.
\item Otherwise, if \TZ{ProcessCompleted} is \T{false} and \TZ{Cause} is
\T{kill}\index{kill@\T{kill}!exit signal},
then \TZ{P} should complete: \TZ{Reason} is
\T{killed}\index{killed exit signal@\T{killed} exit signal}.
\item Otherwise, if \TZ{ProcessCompleted} is \T{false}, \T{trap_exit[\Z{P}]} is \T{false}
and \TZ{Cause} is something other than \T{normal}
then \TZ{P} should complete: \TZ{Reason} is \TZ{Cause}.
\item Otherwise, \TZ{P} should not complete.
\end{itemize}
\index{exit!signal!trapping|)}

What happens next depends on \TZ{ProcessCompleted}:
\begin{itemize}
\item If \TZ{ProcessCompleted} is \T{true}: \TZ{P} must have been linked
with $\TZ{P}'$\ifNew\ or is monitoring $\TZ{P}'$\fi.
% SHOULD THIS HAPPEN OR WILL A SEPARATE UNLINK SIGNAL BE SENT?
If $\TZ{P}'$ is in \T{linked[\Z{P}]}, then
remove it.

\ifNew
\index{process!monitoring|(}
If there is a pair $\langle\TZ{P}',k\rangle$ in
\T{monitoring_processes[\Z{P}]}, then remove it and put $k$ terms \T{\{taskdown,$\Z{P}'$\}}
at the end of
\T{message_queue[\Z{P}]} (\S\ref{section:message-arrival}). (This is not necessary if
the exit signal causes \TZ{P} to complete.)
\index{process!monitoring|)}
\fi

\item If \TZ{ProcessCompleted} is \T{false}: $\TZ{P}'$ must have called the BIF
\T{exit/2}\index{exit/2 BIF@\T{exit/2} BIF}
with \TZ{P} and \TZ{Cause} as arguments.
\end{itemize}

What finally happens depends on whether \TZ{P} should complete abruptly or not.
\begin{itemize}
\item If \TZ{P} should complete, the final processing is as described in
\S\ref{section:process-completion}.
\item If \TZ{P} should not complete, then
unless \TZ{Cause} is \T{normal},
a message \T{\{'EXIT',$\Z{P}'$,\Z{Cause}\}} is placed at the end of
\T{message_queue[\Z{P}]} (\S\ref{section:message-arrival}).
\end{itemize}
\index{exit!signal|)}

\item \index{process!linking|(}
\I{Link request} with sender $\TZ{P}'$.
If $\TZ{P}'$ is not in
\T{linked[\Z{P}]}, then $\TZ{P}'$ is added to \T{linked[\Z{P}]}.

\item \I{Unlink request} with sender $\TZ{P}'$.
If $\TZ{P}'$ is in
\T{linked[\Z{P}]}, then $\TZ{P}'$ is removed from \T{linked[\Z{P}]}.
\index{process!linking|)}

\item \index{process!group|(}
\I{Group leader request} with new group leader $\TZ{P}'$.
\T{group_leader[\Z{P}]} is set to $\TZ{P}'$.
\index{process!group|)}

\item \I{Info request} with sender $\TZ{P}'$ and key \TZ{K}.  Reply to the sender
with a message containing the requested information.
\ifNew (The format of the message is implementation-defined.) \fi

%\item MONITOR NODE REQUEST???
\end{itemize}
\end{itemize}
\index{signal!arrival|)}

\section{Scheduling of processes}

\label{section:scheduling}
\index{process!scheduling|(}

If a process \TZ{P} is suspended in part 3 of the evaluation of a
\T{receive} expression\index{receive expression@\T{receive} expression}
(\S\ref{section:receive-expr}), then \T{status[\Z{P}]} is
\T{waiting}\index{waiting process status@\T{waiting} process status},
otherwise it is either \T{runnable}\index{runnable process status@\T{runnable} process status}
or \T{running}\index{running process status@\T{running} process status}.
Depending on \T{status[\Z{P}]}, we say that
\TZ{P} is waiting, runnable or running.
At any time, at most one of the processes residing on a node
is running, if the node is on a uniprocessor system.  On a node on a
multiprocessor system there may be multiple processes running simultaneously.
A newly spawned process is initially runnable.

In \S\ref{section:process-state-dynamic} it is described how \T{status[\Z{P}]} may change
to \T{waiting} or to \T{runnable}.  In addition, the scheduler of an \Erlang\ node
has as its task to repeatedly choose which runnable process gets to run.
\index{cycle!scheduling|(}
When the
scheduler changes the \T{status[\Z{P}]} to \T{running}, process \TZ{P} will run for
one \emph{cycle}, unless during that cycle \T{status[\Z{P}]} changes to \T{waiting},
in which case another runnable process will be scheduled to run.  At the end of the
cycle, if \TZ{B} is still running, the scheduler changes \T{status[\Z{P}]} back to \T{runnable}.

\ifOld \OldErlang\ attempts \fi
\ifNew An implementation should attempt \fi
to make cycles be of equal and short duration, the latter to favour interactive
processes that do a small amount of work between waiting states.
\index{cycle!scheduling|)}

When we write that
\ifOld \OldErlang\ schedules a set of processes \emph{fairly}, \fi
\ifNew an implementation should schedule a set of processes \emph{fairly}, \fi
we mean that each runnable process in the set should eventually be scheduled and preferably
in the same order that they became runnable.  (This is a very weak requirement.)

\index{process!priority|(}
\index{high process priority@\T{high} process priority|(}
\index{normal@\T{normal}!process priority|(}
\index{low process priority@\T{low} process priority|(}
Each process \TZ{Q} has an associated priority \T{priority[\Z{Q}]}, which
is an \Erlang\ term. Three priorities
\ifOld are \fi
\ifNew must be \fi
recognized: \T{high}, \T{normal} and \T{low}.

\begin{itemize}
\item While there are runnable processes with priority \T{high},
\ifOld \OldErlang\ schedules \fi
\ifNew the implementation should schedule \fi
the processes with priority \T{high} fairly,
ignoring processes with priority \T{low} or \T{normal}.

\item While there are no runnable processes with priority \T{high} or \T{low},
\ifOld \OldErlang\ schedules \fi
\ifNew the implementation should schedule \fi
the processes with priority \T{normal} fairly.

\item While there are no runnable processes with priority \T{high} or \T{normal},
\ifOld \OldErlang\ schedules \fi
\ifNew the implementation should schedule \fi
the processes with priority \T{low} fairly.

\item While there are no runnable processes with priority \T{high} but
there are runnable processes with \T{normal} or \T{low},
\ifOld \OldErlang\ schedules \fi
\ifNew the implementation should schedule \fi
the processes with priority \T{normal}
and \T{low} fairly.  It
\ifOld also attempts \fi
\ifNew should also attempt \fi
to schedule processes with
priority \T{normal}
\I{normal\_advantage}
times between each scheduling of a process with
priority \T{low}.
\end{itemize}
\ifOld
For \OldErlang, \I{normal\_advantage} is $8$.
\fi
\ifNew
An implementation should document what the value \I{normal\_advantage} is and what
additional scheduling policies are used.
\fi
\index{high process priority@\T{high} process priority|)}
\index{normal@\T{normal}!process priority|)}
\index{low process priority@\T{low} process priority|)}
\index{process!priority|)}
\index{process!scheduling|)}

\section{Process group leaders}

\label{section:group-leader}
\index{process!group|(}

Each process \TZ{P} has a \emph{group leader}, which is a process, possibly itself,
referred to in this specification as \T{group_leader[\Z{P}]}.
The set of processes that have the same group leader may be thought of as a
\emph{process group}, hence the term group leader.

The group leader of a process is the default process for handling
in- and output of the process.

A process can retrieve its own group leader (using the BIF
\T{group_leader/0}\index{group_leader/0 BIF@\T{group_leader/0} BIF},
\S\ref{section:groupleader0})
and any process can set the group leader
of any process (using the BIF \T{group_leader/2}\index{group_leader/2 BIF@\T{group_leader/2} BIF},
\S\ref{section:groupleader2}).
When a new process is created, its group leader will be the same 
as that of the process that spawned it.

In- and output (except for direct communication with a port,
cf.~\S\ref{chapter:more-about-ports}) is otherwise not covered by this specification.
\index{process!group|)}

\section{Static and dynamic properties of a process}

\label{section:process-state}
\index{process!state|(}

When a process is created, some properties of the process are determined and will be
in effect until the process terminates.  During that time also a dynamic state
is maintained, consisting of properties of the process that change as time passes.
This state is affected by and affects the computation of the process.

We refer to these as the \emph{static} and \emph{dynamic properties} of a process.  We
refer collectively to the values of the latter at a certain time as the \emph{state}
of the process at that time.

\subsection{Static properties}

\label{section:process-state-static}

\begin{Lentry}
\item[\T{creation[\Z{P}]}]
\index{creation@\T{creation}!process property|(}
The value of \T{creation[\Z{P}]} is the value of \T{creation[\Z{N}]} for the
node \TZ{N} on which \TZ{P} was created.
\index{creation@\T{creation}!process property|)}

\item[\T{ID[\Z{P}]}]
\index{ID@\T{ID}!process property|(}
The value of \T{ID[\Z{P}]} is a nonnegative integer that is a serial number for \TZ{P}
on the node on which it was
created.  The value cannot be obtained directly but is used in the transformation
to the external term format (\S\ref{chapter:external-format}).
\index{ID@\T{ID}!process property|)}

\item[\T{initial_call[\Z{P}]}]
\index{initial_call process property@\T{initial_call} process property|(}
When a process \TZ{P} is spawned it is evaluating a remote application
\T{\Z{Mod}:\Z{Fun}($\Z{Arg}_1$,\tdots,$\Z{Arg}_{\TZm{k}}$)} (the initial call).
The value of \T{initial_call[\Z{P}]} is then
the 3-tuple \T{\{\Z{Mod},\Z{Fun},\Z{k}]\}}.
\index{initial_call process property@\T{initial_call} process property|)}

\item[\T{node[\Z{P}]}]
\index{node@\T{node}!process property|(}
When a process is spawned it is created on some node \T{node[\Z{P}]}.
This node never changes. The process \TZ{P} itself can access
\T{node[\Z{P}]} by evaluating an expression \T{node()} (\S\ref{section:node0}).
Any process can access \T{node[\Z{Q}]} for a process \TZ{Q} by evaluating an expression
\T{node(\Z{Q})} (\S\ref{section:node1}).
\index{node@\T{node}!process property|)}
\end{Lentry}

\subsection{Dynamic properties}

\label{section:process-state-dynamic}

\begin{Lentry}
\item[\T{current_function[\Z{P}]}]
\index{current_function process property@\T{current_function} process property|(}
The value is either the atom \T{undefined}, or a
3-tuple \T{\{\Z{Mod},\Z{Fun},\Z{k}\}} if the
most recently begin function call was to the function \T{\Z{Mod}:\Z{Fun}/\Z{k}}
(\S\ref{section:extent-function-calls}).
It should be updated each time a named function is entered (\S\ref{section:appl-named-function}).
It can be accessed as \T{process_info(\Z{P},current_function)}
(\S\ref{section:processinfo2}).
\ifNew
(An optimized implementation may wish to not keep this part of the state up to date.)
\fi
\index{current_function process property@\T{current_function} process property|)}

\item[\T{dictionary[\Z{P}]}]
\index{dictionary process property@\T{dictionary} process property|(}
\index{process!dictionary|(}
The value is a table (\S\ref{section:tables}).
A process can access and modify \T{dictionary[\Z{P}]} using the six BIFs
\T{get/0}, \T{get/1}, \T{get_keys/1}, \T{put/2}, \T{erase/0} and \T{erase/1}
(\ifOld\S\ref{section:dictionary-bifs}\fi
\ifNew\S\ref{section:process-module}\fi).
The table is initially empty.
\index{process!dictionary|)}
\index{dictionary process property@\T{dictionary} process property|)}

\item[\T{error_handler[\Z{P}]}]
\index{error_handler@\T{error_handler}!process property|(}
The value is a module name (an atom) and it affects the evaluation of applications of
undefined function names (\S\ref{section:application-exprs})\iffalse and send expressions with
undefined process names (\S\ref{section:send-expr})\fi.
A process \TZ{P} can set \T{error_handler[\Z{P}]} to a module name
\TZ{M} by evaluating an expression
\T{process_flag(error_handler,\Z{M})}\index{process_flag/2 BIF@\T{process_flag/2} BIF}
(\S\ref{section:processflag2}).
The value is initially \T{error_handler}.
\index{error_handler@\T{error_handler}!process property|)}

\item[\T{group_leader[\Z{P}]}]
\index{group_leader process property@\T{group_leader} process property|(}
\index{process!group|(}
The value is a PID which identifies the group leader of process \TZ{P}
(\S\ref{section:group-leader}).
A process \TZ{P} can access \T{group_leader[\Z{P}]} by calling the BIF
\T{group_leader/0}\index{group_leader/0 BIF@\T{group_leader/0} BIF} (\S\ref{section:groupleader0})
and any process can set the group leader of any
process by calling the BIF \T{group_leader/2}\index{group_leader/2 BIF@\T{group_leader/2}
BIF} \S\ref{section:groupleader2}.
(Any process can also access \T{group_leader[\Z{P}]} as \T{process_info(\Z{P},group_leader)}.)
When a process \TZ{P} spawns a new process \TZ{Q}, the initial value
of \T{group_leader[\Z{Q}]} is \T{group_leader[\Z{P}]}.
\index{process!group|)}
\index{group_leader process property@\T{group_leader} process property|)}

\item[\T{heap_size[\Z{P}]}]
\index{heap_size process property@\T{heap_size} process property|(}
The value should reflect the amount of memory presently used for storing the terms
allocated by process \Z{P}.
\T{heap_size[\Z{P}]}
can be accessed through the BIF \T{process_info/2} (\S\ref{section:processinfo2}).
\index{heap_size process property@\T{heap_size} process property|)}

\item[\T{linked[\Z{P}]}]
\index{linked@\T{linked}!process property|(}
\index{process!linking|(}
The value is a representation of the set of PIDs and ports that identify
the processes and ports to which \TZ{P} is linked
(\S\ref{section:links}).
It cannot be modified directly but a PID or port \TZ{Q}
will be added to \T{linked[\Z{P}]}
if it is not already in it and either
\begin{itemize}
\item \TZ{P} evaluates an application \T{link(\Z{Q})}\index{link/1 BIF@\T{link/1} BIF}
(\S\ref{section:link1}), or
\item \TZ{P} receives a link request signal from \TZ{Q}.
\end{itemize}
A PID \TZ{Q} will be removed from \T{linked[\Z{P}]} if it is in the set and
\begin{itemize}
\item \TZ{P} evaluates an application \T{unlink(\Z{Q})}\index{unlink/1 BIF@\T{unlink/1} BIF}
(\S\ref{section:unlink1}),
\item \TZ{P} receives an unlink request signal from \TZ{Q}, or
\item \TZ{P} receives an exit signal\index{exit!signal}
\T{\{'EXIT',\Z{Q},\Z{Reason}\}} for some term
\TZ{Reason}, and the exit signal was sent due to process completion
(\S\ref{section:exit-signals}).
\end{itemize}
\index{process!linking|)}
\index{linked@\T{linked}!process property|)}

\item[\T{memory_in_use[\Z{P}]}]
\index{memory_in_use process property@\T{memory_in_use} process property|(}
The value should reflect the total amount of memory presently used by process \Z{P}.
\T{memory_in_use[\Z{P}]}
can be accessed through the BIF \T{process_info/2} with \T{memory} as
second argument (\S\ref{section:processinfo2}).
\index{memory_in_use process property@\T{memory_in_use} process property|)}

\item[\T{message_queue[\Z{P}]}]
\index{message_queue process property@\T{message_queue} process property|(}
\index{message!queue|(}
The value is a representation of a queue of terms, which are the
messages that have arrived at the process
(\S\ref{section:message-arrival}) but that have not yet been received
(\S\ref{section:receiving-messages}).  When a \T{receive} expression
is evaluated by a process \TZ{P}, it will first try to match the
messages in \T{message_queue[\Z{P}]} against the clauses, in the order
that they appear in the queue (\S\ref{section:receive-expr}).  New
messages will be added at the end of the queue.  There is no direct way
of accessing or modifying \T{message_queue[\Z{P}]} for a process
\TZ{P}.
\index{message!queue|)}
\index{message_queue process property@\T{message_queue} process property|)}

\ifNew
\item[\T{monitored_processes[\Z{P}]}]
\index{process!monitoring|(}
\index{monitored_processes process property@\T{monitored_processes} process property|(}
The value is a representation of a set of pairs of a PID and a nonnegative integer,
identifying the processes that \TZ{P} is monitoring and how many times \TZ{P} has requested
to monitor them (\S\ref{section:monitor2}).
\index{process!monitoring|)}
\index{monitored_processes process property@\T{monitored_processes} process property|)}

\item[\T{monitoring_processes[\Z{P}]}]
\index{process!monitoring|(}
\index{monitoring_processes process property@\T{monitoring_processes} process property|(}
The value is a representation of a set of pairs of a PID and a nonnegative integer,
identifying the processes that are monitoring \TZ{P} and how many times they have requested
to monitor \TZ{P} (\S\ref{section:monitor2}).
\index{process!monitoring|)}
\index{monitoring_processes process property@\T{monitoring_processes} process property|)}
\fi

\item[\T{priority[\Z{P}]}]
\index{priority process property@\T{priority} process property|(}
\index{process!priority|(}
\index{low process priority@\T{low} process priority|(}
\index{normal@\T{normal}!process priority|(}
\index{high process priority@\T{high} process priority|(}
The value is one of the atoms \T{low},
\T{normal} and
\T{high} and it affects the
scheduling priority of process \TZ{P} (\S\ref{section:scheduling}).
A process \TZ{P} can set \T{priority[\Z{P}]} to priority atom
\TZ{R} by evaluating an expression
\T{process_flag(priority,\Z{R})} (\S\ref{section:processflag2}).
The value is initially \T{normal}.
\index{low process priority@\T{low} process priority|)}
\index{normal@\T{normal}!process priority|)}
\index{high process priority@\T{high} process priority|)}
\index{process!priority|)}
\index{priority process property@\T{priority} process property|)}

\item[\T{reductions[\Z{P}]}]
\index{reductions@\T{reductions}!process property|(}
The value is an integer that should reflect the number of function calls that the process has begun
(\S\ref{section:extent-function-calls}) since it was spawned.
\ifNew
An implementation could define a scheduling cycle (\S\ref{section:scheduling})
as a certain number of reductions, in which case
it would be necessary to make sure that reductions are counted accurately.
\fi
\ifOld
\OldErlang\ defines a scheduling cycle (\S\ref{section:scheduling})
as a certain (implementation-defined) number of reductions.
\fi
\T{reductions[\Z{P}]}
can be accessed through the BIF \T{process_info/2} (\S\ref{section:processinfo2}).
\index{reductions@\T{reductions}!process property|)}

\item[\T{registered_name[\Z{P}]}]
\index{registered_name process property@\T{registered_name} process property|(}
\index{process!registry|(}
The value is an atom \TZ{A} if \TZ{P} is registered with the name TZ{A} 
or \T{[]} if \TZ{P} is not registered under any name.  \T{registered_name[\Z{P}]}
must be coherent with the contents of \T{registry[node[\Z{P}]]}.
The value is initially \TZ{[]}.  It can be set to an atom by the BIF
\T{register/2}\index{register/2 BIF@\T{register/2} BIF} (\S\ref{section:register2})
and to \T{[]} by the BIF
\T{unregister/1}\index{unregister/1 BIF@\T{unregister/1} BIF} (\S\ref{section:unregister1}).  Given the atom
\T{registered_name[\Z{P}]}, the PID
\TZ{P} can be obtained through the BIF
\T{whereis/1}\index{whereis/1 BIF@\T{whereis/1} BIF} (\S\ref{section:whereis1},
cf.\ \S\ref{section:process-registry})
whereas given \TZ{P}, \T{registered_name[\Z{P}]} can be obtained through the BIF
\T{process_info/2}\index{process_info/2 BIF@\T{process_info/2} BIF}
(\S\ref{section:processinfo2}).
\index{process!registry|)}
\index{registered_name process property@\T{registered_name} process property|)}

\item[\T{stack_trace[\Z{P}]}]
\index{stack_trace process property@\T{stack_trace} process property|(}
\T{stack_trace[\Z{P}]} is a dynamic representation of the evaluation that \TZ{P}
is carrying out.
It should contain sufficient information that a BIF call
\ifOld \T{erlang:check_process_code(\Z{P},\Z{Mod})} \fi
\ifNew \T{codeload:check_process_code(\Z{P},\Z{Mod})} \fi
should be able to return \T{true} if \TZ{P} is using
\T{old_version[module_table[node[\Z{P}]](\Z{Mod})]} and \T{false} otherwise.
It cannot be accessed or updated explicitly.
\index{stack_trace process property@\T{stack_trace} process property|)}

\item[\T{status[\Z{P}]}]
\index{status process property@\T{status} process property|(}
\index{process!status|(}
\index{waiting process status@\T{waiting} process status|(}
\index{runnable process status@\T{runnable} process status|(}
\index{running process status@\T{running} process status|(}
\T{status[\Z{P}]} reflects the scheduling status of \TZ{P}: waiting, runnable or running.
\T{status[\Z{P}]} is initially \T{runnable}.
\T{status[\Z{P}]} is changed from \T{runnable} to
\T{running} or from \T{running} to \T{runnable} by the scheduler.
If \T{status[\Z{P}]} is \T{waiting} when a message is added to the end of
\T{message_queue[\Z{P}]},
it is changed to \T{runnable}.  If \T{status[\Z{P}]} is \T{waiting} when
\T{timer[\Z{P}]} changes from \T{1} to \T{0}, it is changed to \T{runnable}.
Evaluation of a \T{receive} expression (\S\ref{section:receive-expr}) may set
\T{status[\Z{P}]} to \T{waiting}.
\T{status[\Z{P}]} can be accessed through the BIF \T{process_info/2} but
cannot be changed except as described above.
\index{waiting process status@\T{waiting} process status|)}
\index{runnable process status@\T{runnable} process status|)}
\index{running process status@\T{running} process status|)}
\index{process!status|)}
\index{status process property@\T{status} process property|)}

\item[\T{timer[\Z{P}]}]
\index{timer process property@\T{timer} process property|(}
If the value of \T{timer[\Z{P}]} is positive, it reflects the number of milliseconds
that remain until the status of process \TZ{P} should change from
\T{waiting}\index{waiting process status@\T{waiting} process status}
to \T{runnable}\index{runnable process status@\T{runnable} process status}.

If \T{timer[\Z{P}]} is positive, it is decremented automatically once every millisecond.
When it changes from \T{1} to \T{0}, if \T{status[\Z{P}]} is
\T{waiting}, \T{status[\Z{P}]} is changed to \T{runnable}.

\T{timer[\Z{P}]} cannot be accessed directly and is set only as part of the evaluation of
a \T{receive} expression (\S\ref{section:receive-expr}).
\index{timer process property@\T{timer} process property|)}

\item[\T{trap_exit[\Z{P}]}]
\index{exit!signal!trapping|(}
\index{trap_exit process property@\T{trap_exit} process property|(}
The value is a Boolean atom and it affects how exit signals arriving at process \TZ{P} are
processed (\S\ref{section:exit-signals}).
A process \TZ{P} can set \T{trap_exit[\Z{P}]} to a Boolean
atom \TZ{B} by evaluating an expression
\T{process_flag(trap_exit,\Z{B})} (\S\ref{section:processflag2}).
The value is initially \T{false}.
\T{trap_exit[\Z{P}]} can be accessed as \T{process_info(\Z{P},trap_exit)}
(\S\ref{section:processinfo2}).
\index{exit!signal!trapping|)}
\index{trap_exit process property@\T{trap_exit} process property|)}
\end{Lentry}
\index{process!state|)}
\index{process|)}


%
% %CopyrightBegin%
%
% Copyright Ericsson AB 2017. All Rights Reserved.
%
% Licensed under the Apache License, Version 2.0 (the "License");
% you may not use this file except in compliance with the License.
% You may obtain a copy of the License at
%
%     http://www.apache.org/licenses/LICENSE-2.0
%
% Unless required by applicable law or agreed to in writing, software
% distributed under the License is distributed on an "AS IS" BASIS,
% WITHOUT WARRANTIES OR CONDITIONS OF ANY KIND, either express or implied.
% See the License for the specific language governing permissions and
% limitations under the License.
%
% %CopyrightEnd%
%

\chapter{Nodes}

\label{chapter:nodes}

\section{Single-node and multi-node systems}

\label{section:single-multi-node}

An \Erlang\ node is the operating system for \Erlang\ processes.  
Every process runs on some specific node and once a process has been created
it remains on that node.

\index{node!communicating|(}
\index{node!isolated|(}
By default an \Erlang\ node is \emph{isolated} and constitutes a single-node system.
In order to become a \emph{communicating} node that can be part of a cluster
of \Erlang\ nodes, it
must have a process registered under the name
\T{net_kernel}\index{net_kernel process name@\T{net_kernel} process name} that has
opened a port through which it communicates
with other nodes using the protocol described in \S\ref{chapter:external-format}.
\index{node!communicating|)}
\index{node!isolated|)}

\index{node!name|(}
Every \Erlang\ node has a name, which is an atom.  The printname of
the atom always contains exactly one \T{@} character\index{"@
character@\T{"@} character} (\T{\char`\\\ifStd u0040\fi\ifOld
100\fi}).  An isolated node always has the name
\T{nonode@nohost}\index{nonode"@nohost node name@\T{nonode"@nohost} node name}.
In a communicating node, the part of the printname after the \T{@}\
must be the network name of the host computer on which the node
resides (typically an Internet domain name).  We call an atom with
such a printname a \emph{valid node name}.  The name of each
communicating node must be unique.  As the node name contains the
network name of the host computer (which is assumed to be unique), it
is sufficient to ensure that all nodes running on the same host
computer have unique names.  This is ensured by the use of EPMD, as
described in \S\ref{section:registering-nodes}.  Because node names
are unique and that nodes are always referred to through their names
in the language, we will identify nodes with their names.
\index{node!name|)}

\index{node!communicating|(}
A node with a \T{net_kernel}\index{net_kernel process name@\T{net_kernel} process name}
process that has opened a port \TZ{R} as described above
becomes a communicating node with name \TZ{A} by some process making a BIF call
\ifOld \T{erlang:set_node(\TZ{A},\TZ{R})}\index{set_node/2 BIF@\T{set_node/2} BIF}
(\S\ref{section:setnode2}).\fi
\ifStd \T{node:set_node(\TZ{A},\TZ{R})}\index{set_node/2 BIF@\T{set_node/2} BIF}
(\S\ref{section:node:setnode2}).\fi
CHECK THE ARGUMENTS!!!
\ifOld It is also possible to make a node communicating from the beginning
\cite[pp.~5--9]{otp-dev-ref}.\fi
\ifStd An implementation should provide a
way to make a node communicating from the beginning.\fi
\index{node!communicating|)}

\index{node!isolated|(}
On an isolated node the effects and result is undefined when
any other node name than
\T{nonode@nohost}\index{nonode"@nohost node name@\T{nonode"@nohost} node name}
is used in a BIF that have an explicit node argument
or when sending messages.  (This is not stated again in the descriptions
of BIFs and send expressions.)
In the remainder of this chapter, \Erlang\ nodes are assumed to be
communicating unless
it is stated explicitly that a node is or may be isolated.  Moreover, as
every communicating node has a unique name and there is no way to refer to a node except
through its name, we will identify nodes with their names.  Thus, when
we write ``node \TZ{A}'', where \TZ{A} is an atom, we mean the node with
name \TZ{A}, provided that such a node exists.
\index{node!isolated|)}

\index{node!magic cookie|(}
Each node has a term associated with it, called the \emph{magic cookie} of the node.
In order for a processes on some node $\TZ{N}_1$ to communicate successfully with a
process on another node $\TZ{N}_2$, the \emph{magic cookie} of node
$\TZ{N}_2$ must be provided
in the communication from node $\TZ{N}_1$.
A group of nodes can be protected from communication
originating on other nodes by sharing a magic cookie and not disclosing it
to other nodes.  Magic cookies are discussed in more detail in
\S\ref{section:magic-cookie}.
\index{node!magic cookie|)}

\index{friend (of a node)!see{node, friendship}}
\index{node!friendship|(}
At any given time, each \Erlang\ node has a number of other nodes as
\emph{friends}.  The set of such friends may change dynamically over time.
When a process running on some node $\TZ{N}_1$ attempts to communicate with
a process
on another node $\TZ{N}_2$, node $\TZ{N}_1$
attempts to set up a friendship with node $\TZ{N}_2$ through negotiation
between the \T{net_kernel}\index{net_kernel process name@\T{net_kernel} process name}
processes of the two nodes.  Friendship is a
symmetric property so in the process, $\TZ{N}_1$
also becomes a friend of node $\TZ{N}_2$.  The friendship is terminated
when the nodes can no longer communicate or when a process residing on one
of the nodes calls the BIF
\ifOld \T{erlang:disconnect_node/1}\index{erlang:disconnect_node/1
BIF@\T{erlang:disconnect_node/1} BIF} (\S\ref{section:disconnectnode1}) \fi
\ifStd \T{node:disconnect/1}\index{disconnect/1 BIF@\T{disconnect/1} BIF}
(\S\ref{section:node:disconnect1}) \fi
with the name of the other node as argument.
Friendship is described in more detail in \S\ref{section:friendship}.
\index{node!friendship|)}

\section{Registering a node}

\label{section:registering-nodes}
\index{EPMD|(}
\index{net_kernel process name@\T{net_kernel} process name|(}

MAKE COHERENT WITH ABOVE AND FINISH!!!

Any computer that can be the host of \Erlang\ nodes must provide the service
\emph{EPMD}\index{EPMD} (\Erlang\ Port Mapper Daemon).
The EPMD service on the host computer
stores the names of all \Erlang\ nodes running on the computer, ensuring that
the names are unique.  It will provide a handle to the port of the \T{net_kernel}
process of any node residing on the host computer, given the name of the node
(cf.\ \S\ref{section:friendship}).

MUST OBTAIN INFO ABOUT SET_NODE/2-3 BEFORE FIXING NEXT PARAGRAPH!!!

An \Erlang\ node becomes a communicating node by spawning a process registered
with the name \T{net_kernel}.  It should
open a port through which other nodes can communicate with it.  It then
contacts the
EPMD service on its host computer, requesting to register the node with a certain
name and the opened port.  The protocol for this registration is described
in \S\ref{section:epmd-register}.
\index{net_kernel process name@\T{net_kernel} process name|)}
\index{EPMD|)}

\section{Initializing and terminating friendship}

\label{section:friendship}
\index{node!friendship|(}

\index{net_kernel process name@\T{net_kernel} process name|(}
When a signal is to be delivered from a process residing on some node $\TZ{N}_1$
to a process residing on a different node $\TZ{N}_2$, $\TZ{N}_2$ must be a friend
of $\TZ{N}_1$.  NOT CORRECT!!!:
Concretely, this means that the \T{net_kernel} process of
node $\TZ{N}_1$ has opened a port to the \T{net_kernel} process of
node $\TZ{N}_2$.  Associated with this port is an
\emph{atom table}\index{node!atom table}, as
described in \S\ref{section:atom-tables}.
\index{net_kernel process name@\T{net_kernel} process name|)}

Making node $\TZ{N}_2$ a friend of node $\TZ{N}_1$ is thus initiated when
a process on node $\TZ{N}_1$ needs to communicate with a process on node $\TZ{N}_2$
and node $\TZ{N}_2$ is not yet a friend of node $\TZ{N}_1$.  The friendship
remains until node $\TZ{N}_1$ detects that it cannot communicate with node
$\TZ{N}_2$ or either node calls the BIF
\ifOld \T{erlang:disconnect_node/1}\index{erlang:disconnect_node/1
BIF@\T{erlang:disconnect_node/1} BIF} (\S\ref{section:disconnectnode1}) \fi
\ifStd \T{node:disconnect/1}\index{disconnect/1 BIF@\T{disconnect/1} BIF}
(\S\ref{section:node:disconnect1}) \fi
with the name of the other
node as argument.

When node $\TZ{N}_1$ makes node $\TZ{N}_2$ a friend, node $\TZ{N}_1$ should also
become a friend of node $\TZ{N}_2$.  Under normal circumstances, friendship is thus
a reflexive relation.  When the means of communication is broken
between two nodes that are friends, they may not detect this simultaneously
and friendship may then temporarily be unilateral.  Should communications be restored
in a situation where node $\TZ{N}_1$ is still a friend of node $\TZ{N}_2$ but
node $\TZ{N}_2$ is no longer a friend of node $\TZ{N}_1$, then
$\TZ{N}_1$ should cease to be a friend of node $\TZ{N}_2$ before (bilateral)
friendship is resumed.  (This is to ensure that both processes on $\TZ{N}_1$ monitoring
friendship with $\TZ{N}_2$ and processes on $\TZ{N}_2$ monitoring
friendship with $\TZ{N}_1$ will be notified that friendship has been broken,
cf.\ \S\ref{section:monitor-node}.)

NEXT PARAGRAPH IS HIGHLY UNCERTAIN!!!

\index{net_kernel process name@\T{net_kernel} process name|(}
When a node $\TZ{N}_1$ wishes to initiate friendship with a node $\TZ{N}_2$ the
\T{net_kernel} process of node $\TZ{N}_1$
first contacts the EPMD\index{EPMD} on the host computer of node $\TZ{N}_2$ and requests a
handle of the port of the \T{net_kernel} process of node $\TZ{N}_2$, as described
in \S\ref{section:epmd-find}.  If this succeeds, then the \T{net_kernel}
process of node $\TZ{N}_1$
contacts the \T{net_kernel} process of node $\TZ{N}_2$ and completes the setup of
the friendship as described in \S\ref{chapter:external-format}.  That communication
also establishes node $\TZ{N}_2$ as a friend of node $\TZ{N}_1$.
\index{net_kernel process name@\T{net_kernel} process name|)}

The current set of friends of a node \TZ{N} is referred to in this document as
\T{friends[\Z{N}]}\index{friends node property@\T{friends} node property}.
\index{node!friendship|)}

\section{Remote communication and magic cookies}

\label{section:magic-cookie}
\index{node!magic cookie|(}

All forms of communication between two
processes residing on different nodes go through the
\T{net_kernel}\index{net_kernel process name@\T{net_kernel} process name}
processes on those nodes.  Note that all communication between processes
is in the form of signals\index{signal!communication} (\S\ref{section:signals}).

Each node \TZ{N} has associated with it an atom
\T{magic_cookie[\Z{N}]}\index{magic_cookie node property@\T{magic_cookie} node property}
that is called the \emph{magic cookie} of the node.  Any process residing on node \TZ{N}
can obtain the magic cookie of node \TZ{N} through a BIF call
\ifOld \T{erlang:get_cookie()}\index{get_cookie/0 BIF@\T{get_cookie/0} BIF}
(\S\ref{section:getcookie0}).\fi
\ifStd \T{node:get_cookie()}\index{get_cookie/0 BIF@\T{get_cookie/0} BIF}
(\S\ref{section:node:getcookie0}).\fi
The magic cookie of node \TZ{N} can be set
by any process residing on node \TZ{N}
to some atom \TZ{A} through a BIF call
\ifOld \T{erlang:set_cookie(\Z{N},\Z{A})}\index{set_cookie/2 BIF@\T{set_cookie/2} BIF}
(\S\ref{section:setcookie2}).\fi
\ifStd \T{node:set_cookie(\Z{N},\Z{A})}\index{set_cookie/2 BIF@\T{set_cookie/2} BIF}
(\S\ref{section:node:setcookie2}).\fi

Each node also has a table
\T{magic_cookies[\Z{N}]}\index{magic_cookies node property@\T{magic_cookies} node property}
in which the keys are node names and the values are
the presumed magic cookies of those nodes.  The presumed magic
cookie on node $\TZ{N}_1$ of a node $\TZ{N}_2$ cannot be retrieved but can
be set to some atom \TZ{A} by any process residing on node $\TZ{N}_1$ through a BIF
call
\ifOld \T{erlang:set_cookie($\Z{N}_2$,\Z{A})}\index{set_cookie/2 BIF@\T{set_cookie/2} BIF}
(\S\ref{section:setcookie2}).\fi
\ifStd \T{node:set_cookie($\Z{N}_2$,\Z{A})}\index{set_cookie/2 BIF@\T{set_cookie/2} BIF}
(\S\ref{section:node:setcookie2}).\fi

\index{exit!signal!sending|(}
When a process $\TZ{P}_1$ residing on a node $\TZ{N}_1$ attempts
to send a signal \TZ{S} to a process $\TZ{P}_2$ on a different node $\TZ{N}_2$,
then the following happens:

\begin{itemize}
\item \index{net_kernel process name@\T{net_kernel} process name|(}
The signal \TZ{S}, source $\TZ{P}_1$ and destination $\TZ{P}_2$ are
passed to the \T{net_kernel} process on node $\TZ{N}_1$.
\item If $\TZ{N}_2$ is not in
\T{friends[\Z{N}]}\index{friends node property@\T{friends} node property}, then the \T{net_kernel}
attempts to make it a friend of $\TZ{N}_1$.  Should this fail, the
signal $s$ is simply discarded.
\item The \T{net_kernel} process on $\TZ{N}_1$ looks up
the atom $\TZ{N}_2$ in
\T{magic_cookies[$\Z{N}_1$]}\index{magic_cookies node property@\T{magic_cookies} node property} and 
if there is no value for $\TZ{N}_2$, then the atom
\T{nocookie}\index{nocookie magic cookie@\T{nocookie} magic cookie}
is inserted for it.  Let \TZ{C} be the result of the lookup
(possibly after insertion of \T{nocookie}).
\item The \T{net_kernel} process on $\TZ{N}_1$ passes
$\TZ{P}_1$, \TZ{C}, \TZ{S} and $\TZ{P}_2$ to the \T{net_kernel}
process on $\TZ{N}_2$
using the protocol described in \S\ref{chapter:external-format}.
\item The \T{net_kernel} process on $\TZ{N}_2$ compares
\TZ{C} with \T{magic_cookie[$\Z{N}_2$]}\index{magic_cookie node property@\T{magic_cookie} node property}.
\begin{itemize}
\item If \TZ{C} and \T{magic_cookie[$\Z{N}_2$]} are (exactly) equal
(\S\ref{section:equality}), then the signal \TZ{S} is passed on to
process $\TZ{P}_2$ as described in \S\ref{section:signal-arrival}.
\item Otherwise, if the signal \TZ{S} is a message with \TZ{T} as body, a message
\T{\{$\Z{P}_1$,badcookie,$\Z{P}_2$,\Z{T}\}} is placed at the end of the
message queue of the \T{net_kernel} process of node $\TZ{N}_2$.
\item Otherwise,
\ifStd
the implementation should carry out one of the following actions:
\begin{itemize}
\item pass the signal \TZ{S}
on to process $\TZ{P}_2$ as described in \S\ref{section:signal-arrival},
\item do nothing (i.e., silently drop the signal),
\item place an informative message at the end of the
message queue of the \T{net_kernel} process on node $\TZ{N}_2$.
\end{itemize}
\fi
\ifOld
the signal is passed on to
process $\TZ{P}_2$ as described in \S\ref{section:signal-arrival}.
\fi
\end{itemize}
\end{itemize}
\index{exit!signal!sending|)}
\index{net_kernel process name@\T{net_kernel} process name|)}

\index{security|(}
The magic cookie mechanism is built into the communication mechanism
of \Erlang\ in order to ensure that processes on a node cannot
disrupt processes running on an arbitrary node.
The security model of \Erlang\ is such that if a process $\TZ{P}_1$ resides
on the same node as on a process $\TZ{P}_2$, or $\TZ{P}_1$ resides on a node
that has the correct magic cookie for the node on which $\TZ{P}_2$ resides,
then process $\TZ{P}_1$ has complete access to process $\TZ{P}_2$.  Process
$\TZ{P}_1$ can send arbitrary signals (including exit signals
with \T{kill} as reason, cf.\ \S\ref{section:signal-arrival})
to process $\TZ{P}_2$.  It is thus up to the programmer to ensure that
processes running on the same node or on nodes that have each others' correct
magic cookies can trust each other. (Typically but not necessarily one will
make sure that whenever
a node $\TZ{N}_1$ has the correct magic cookie of node $\TZ{N}_2$, node
$\TZ{N}_2$ has the correct magic cookie of node $\TZ{N}_1$.  In particular,
a cluster of nodes may share a magic cookie.)
\index{security|)}
\index{node!magic cookie|)}

\section{Process registry}

\label{section:process-registry}
\index{process!registry|(}

For each node there is a table
\T{registry[\Z{N}]}\index{registry node property@\T{registry} node property} in
which the keys are atoms naming processes residing on that node and the values
are the PIDs of the processes.
The process registry makes it possible to send messages to registered processes
on a node without knowing their PIDs.

Some processes, such as the \T{net_kernel}
of a communicating node, are spawned and registered automatically by the
\Erlang\ system.
It is also possible for any process on a node \TZ{N} to register a name \TZ{A}
for a live process \TZ{P}
residing on \TZ{N} through a BIF call
\T{register(\Z{A},\Z{P})}\index{register/2 BIF@\T{register/2} BIF}
(\S\ref{section:register2}).
There can be at most one name registered for a process.
A process name is
removed from the registry automatically when the process completes.
It is also possible for any process on a node \TZ{N} to remove the name \TZ{A}
from the registry
of \TZ{N} through a BIF call
\T{unregister(\Z{A})}\index{unregister/1 BIF@\T{unregister/1} BIF} (\S\ref{section:unregister1}).
A process \TZ{P} can look up a name
\TZ{A} in the registry on \T{node[\Z{P}]} and obtain the PID through a BIF call
\T{whereis(\Z{A})}\index{whereis/1 BIF@\T{whereis/1} BIF}
(\S\ref{section:whereis1})
or obtain a list of all currently registered names on \T{node[\Z{P}]} through
a BIF call
\T{registered()}\index{registered/0 BIF@\T{registered/0} BIF}
(\S\ref{section:registered0}).

The BIF \T{register/2} must ensure that
a process cannot be registered on node \TZ{N} unless it resides on
that node and is alive.  Also it must be ensured that when a process
completes, any name for it in the registry of the node on which it resides
is removed.
\index{process!registry|)}

\section{Monitoring of nodes}

\label{section:monitor-node}
\index{node!monitoring|(}
\ifOld \index{monitor_node/2 BIF@\T{monitor_node/2} BIF|(} \fi
\ifStd \index{node:monitor/2 BIF@\T{node:monitor/2} BIF|(} \fi

THERE ARE THINGS MISSING ABOUT THIS, ESPECIALLY WHEN DISCUSSING SIGNALS AND
THE DYNAMIC STATE.

Any process on a node $\TZ{N}_1$ can be notified when some node $\TZ{N}_2$ ceases to
be a friend of $\TZ{N}_1$.

After a process \TZ{P} on a node $\TZ{N}_1$ has made a BIF call
\ifOld \T{monitor_node($\TZ{N}_2$,true)} (\S\ref{section:monitornode2}),\fi
\ifStd \T{node:monitor($\TZ{N}_2$,true)} (\S\ref{section:node:monitor2}),\fi
where $\TZ{N}_2$ is an atom, a message \T{\{nodedown,$\TZ{N}_2$\}} will be sent to \TZ{P}
if $\TZ{N}_2$ ceases to be a friend of $\TZ{N}_1$.  If $\TZ{N}_2$ is not a friend of $\TZ{N}_1$
when the call is made, the message \T{\{nodedown,$\TZ{N}_2$\}} is sent to \TZ{P} immediately.
A BIF call
\ifOld \T{monitor_node($\TZ{N}_2$,false)} \fi
\ifStd \T{node:monitor($\TZ{N}_2$,false)} \fi
cancels the effect of a call
\ifOld \T{monitor_node($\TZ{N}_2$,true)}. \fi
\ifStd \T{node:monitor($\TZ{N}_2$,true)}. \fi
The effect of a BIF call
\ifOld \T{monitor_node($\TZ{N}_2$,false)} \fi
\ifStd \T{node:monitor($\TZ{N}_2$,false)} \fi
when there is no call
\ifOld \T{monitor_node($\TZ{N}_2$,true)} \fi
\ifStd \T{node:monitor($\TZ{N}_2$,true)} \fi
to cancel is undefined.  There can be
more than one call
\ifOld \T{monitor_node($\TZ{N}_2$,true)} \fi
\ifStd \T{node:monitor($\TZ{N}_2$,true)} \fi
in effect and \TZ{P} will receive one message
for each call
\ifOld \T{monitor_node($\TZ{N}_2$,true)} \fi
\ifStd \T{node:monitor($\TZ{N}_2$,true)} \fi
that has not been canceled when $\TZ{N}_2$ ceases
to be a friend of  $\TZ{N}_1$.
\ifOld \index{monitor_node/2 BIF@\T{monitor_node/2} BIF|)} \fi
\ifStd \index{node:monitor/2 BIF@\T{node:monitor/2} BIF|)} \fi
\index{node!monitoring|)}

\section{The state of a node}

\index{node!state|(}

When a node is started, some properties of the node are determined and will be
in effect until the node is terminated.  During that time also a dynamic state
is maintained, consisting of properties of the node that change as time passes.
This state is affected by and affects the behaviour of the node.

We refer to these as the \emph{static} and \emph{dynamic properties} of a node.  We
refer collectively to the values of the latter at a certain time as the \emph{state}
of the node at that time.

\subsection{Static properties}

\label{section:node-state-static}

\begin{Lentry}
\item[\T{creation[\Z{N}]}]
\index{creation@\T{creation}!node property|(}
When a node is started, the value of \T{creation[\Z{N}]} --- a nonnegative integer ---
is obtained from EPMD\index{EPMD}.  Its purpose is to
to distinguish the current instance of the node from previous instances
that had the same name.  The value is used when creating new refs, PIDs
and ports.
\index{creation@\T{creation}!node property|)}

\item[\T{communicating[\Z{N}]}]
\index{communicating node property@\T{communicating} node property|(}
\index{node!communicating|(}
\index{node!isolated|(}
This is a Boolean atom which is \T{false} if the node is isolated
and \T{true} if it is communicating.  When a node becomes communicating, the
property is changed from \T{false} to \T{true} but after that it cannot be changed.
(The property is
thus not strictly static as it can change once but in practice we can
treat it as being static.)
It can be accessed on the node itself through the BIF
\ifOld \T{erlang:is_alive/0}\index{is_alive/0 BIF@\T{is_alive/0} BIF}
(\S\ref{section:isalive0}).\fi
\ifStd \T{node:is_alive/0}\index{is_alive/0 BIF@\T{is_alive/0} BIF}
(\S\ref{section:node:isalive0}).\fi
\index{node!communicating|)}
\index{node!isolated|)}
\index{communicating node property@\T{communicating} node property|)}

\item[\T{name[\Z{N}]}]
\index{name node property@\T{name} node property|(}
\index{node!name|(}
This is the name of the node, which is an atom.  For an isolated node it
is always \T{nonode@nohost}\index{nonode"@nohost node name@\T{nonode"@nohost} node name}.
When a node becomes communicating, the
name is changed but after that it cannot be changed.  (The property is
thus not strictly static as it can change once but in practice we can
treat it as being static.)
It can be accessed on the node itself through the BIF
\T{node/0}\index{node/0 BIF@\T{node/0} BIF} (\S\ref{section:node0}).
\index{node!name|)}
\index{name node property@\T{name} node property|)}

\item[\T{preloaded[\Z{N}]}]
\index{preloaded node property@\T{preloaded} node property|(}
\index{module!preloaded on a node|(}
This is a set of the names of the modules that were loaded as part of
starting the node.  Although there should normally never be
any reason to delete these modules, their presence in the (static)
set does not imply that
they are still loaded.
The set can be accessed on the node itself through the BIF
\ifOld \T{erlang:preloaded/0}\index{preloaded/0 BIF@\T{preloaded/0} BIF}
(\S\ref{section:preloaded0}).\fi
\ifStd \T{node:preloaded/0}\index{preloaded/0 BIF@\T{preloaded/0} BIF}
(\S\ref{section:node:preloaded0}).\fi
\index{module!preloaded on a node|)}
\index{preloaded node property@\T{preloaded} node property|)}
\end{Lentry}

\subsection{Dynamic properties}

\label{section:node-state-dynamic}

\begin{Lentry}
\item[\T{atom_tables[\Z{N}]}]
\index{atom_tables node property@\T{atom_tables} node property|(}
\index{node!atom table|(}
This is a table that for each friend of \TZ{N} contains a row with the name
and the atom table (\S\ref{section:atom-tables}) for that friend as key and value,
respectively.
When \TZ{N} gets a new friend, a row should be added to \T{atom_tables[\Z{N}]} with
the name of the friend as key and an empty atom table as value.
When a node ceases to be a friend of \TZ{N}, its row in
\T{atom_tables[\Z{N}]} must be removed.
\index{node!atom table|)}
\index{atom_tables node property@\T{atom_tables} node property|)}

\item[\T{distribution_port[\Z{N}]}]
\index{distribution_port node property@\T{distribution_port} node property|(}
On a communicating node, this is the port that was given as second argument
to \T{node:alive/2} ???.
The EPMD\index{EPMD} server for the computer on which the node resides
will communicate with the \T{net_kernel}\index{net_kernel process name@\T{net_kernel} process name}
process of \TZ{N} through that port.
\index{distribution_port node property@\T{distribution_port} node property|)}

\item[\T{entry_points[\Z{N}]}]
\index{entry_points node property@\T{entry_points} node property|(}
This is a table mapping triples consisting of a module name (an atom), a function
symbol (an atom) and an arity (a nonnegative integer) to
\ifStd implementation-defined values that denote \fi
entry points to executable code.  The table cannot be accessed
directly but is used when evaluating a function application where the function is specified
through a module name, a function symbol and an arity (\S\ref{section:appl-named-function}).
The table is updated by BIFs for module
dynamics (\S\ref{chapter:module-dynamics},
\ifOld\S\ref{section:module-bifs}\fi
\ifStd\S\ref{section:module-module}\fi).
\index{entry_points node property@\T{entry_points} node property|)}

\item[\T{friends[\Z{N}]}]
\index{friends node property@\T{friends} node property|(}
\index{node!friendship|(}
This is a set of atoms that are names of nodes that are friends of node \TZ{N}
(\S\ref{section:friendship}).
It can be accessed on node \TZ{N} by calling the BIF
\T{nodes/0}\index{nodes/0 BIF@\T{nodes/0} BIF}
(\S\ref{section:nodes0}).  The set is
added to when a friendship has been established with another node, typically
because a process on one of the nodes has attempted to communicate with a process
on the other node.  The set is subtracted from when node \TZ{N} loses contact
with a friend node (e.g., because communication between host computers has been
lost, the other node has been terminated, or the host computer on which it
resides has been restarted), or when a process on either node calls the
BIF
\ifOld \T{erlang:disconnect_node/1}\index{erlang:disconnect_node/1
BIF@\T{erlang:disconnect_node/1} BIF} (\S\ref{section:disconnectnode1}) \fi
\ifStd \T{node:disconnect/1}\index{disconnect/1 BIF@\T{disconnect/1} BIF}
(\S\ref{section:node:disconnect1}) \fi
with the name of the other node as argument.
\index{node!friendship|)}
\index{friends node property@\T{friends} node property|)}

\item[\T{garbage_collection[\Z{N}]}]
\index{garbage_collection node property@\T{garbage_collection} node property|(}
\index{current_gc node operation@\T{current_gc} node operation|(}
This is the state of the garbage collection gauge, which is updated
automatically when processes collect garbage to reflect the number of
garbage collection operations that have been carried out by processes
on the node and the number of memory words reclaimed by such
operations.  It can be accessed through the operation
\T{current_gc[\Z{N}]}, which returns a 3-tuple
\T{\{\Z{NumberOfGCs},\Z{WordsReclaimed},0\}} of integers
where \TZ{NumberOfGCs} is the total number of garbage
collection operations that have been carried out by
processes on the node and \TZ{WordsReclaimed} is the total
number of memory words reclaimed by such operations.  (The
third integer is always zero and is there to confuse the
enemy.)
\index{current_gc node operation@\T{current_gc} node operation|)}
\index{garbage_collection node property@\T{garbage_collection} node property|)}

\item[\T{magic_cookie[\Z{N}]}]
\index{magic_cookie node property@\T{magic_cookie} node property|(}
\index{node!magic cookie|(}
This term must be provided by any process on another node that wishes
to communicate with a process on node \TZ{N} (cf.\
\S\ref{section:magic-cookie}).  The value of \T{magic_cookie[\Z{N}]}
can be obtained by a process on node \TZ{N} through the BIF
\ifOld \T{erlang:get_cookie/0}\index{get_cookie/0 BIF@\T{get_cookie/0} BIF}
(\S\ref{section:getcookie0}).\fi
\ifStd \T{node:get_cookie/0}\index{get_cookie/0 BIF@\T{get_cookie/0} BIF}
(\S\ref{section:node:getcookie0}).\fi
\T{magic_cookie[\Z{N}]} can be set by a process on node \TZ{N} by calling
the BIF
\ifOld \T{erlang:set_cookie/2}\index{set_cookie/2 BIF@\T{set_cookie/2} BIF}
(\S\ref{section:setcookie2}) \fi
\ifStd \T{node:set_cookie/2}\index{set_cookie/2 BIF@\T{set_cookie/2} BIF}
(\S\ref{section:node:setcookie2}) \fi
with \T{name[\Z{N}]} and the new magic cookie as arguments.
\index{node!magic cookie|)}
\index{magic_cookie node property@\T{magic_cookie} node property|)}

\item[\T{magic_cookies[\Z{N}]}]
\index{magic_cookies node property@\T{magic_cookies} node property|(}
\index{node!magic cookie|(}
This is a mapping from atoms to terms.  Its role in process communication
across nodes is described in \S\ref{section:magic-cookie}.
\T{magic_cookies[\Z{N}]} cannot be accessed directly but
can be modified by a process on node \TZ{N} by calling
the BIF
\ifOld \T{erlang:set_cookie/2}\index{set_cookie/2 BIF@\T{set_cookie/2} BIF}
(\S\ref{section:setcookie2}) \fi
\ifStd \T{node:set_cookie/2}\index{set_cookie/2 BIF@\T{set_cookie/2} BIF}
(\S\ref{section:node:setcookie2}) \fi
with the name of another node and
the magic cookie to be used as arguments.
\index{node!magic cookie|)}
\index{magic_cookies node property@\T{magic_cookies} node property|)}

\item[\T{module_table[\Z{N}]}]
\index{module_table node property@\T{module_table} node property|(}
\index{node!module table|(}
This is a table where the keys are module names, i.e., atoms.  The table contains
a row with key \TZ{Mod} if a module named \TZ{Mod} has ever been loaded on the node.
The value \TZ{R} of each row contains two fields:
\T{current_version[\Z{R}]} and \T{old_version[\Z{R}]}.
\begin{itemize}
\item \T{current_version[\Z{R}]} is either a binary representing the compiled code
for the current version of the module, or \T{none}.
\item \T{old_version[\Z{R}]} is either a binary representing the compiled code
for the current version of the module, or \T{none}.
\end{itemize}
The table is initially empty.  For a new row both fields are \T{none}.
It cannot be accessed or modified directly but is accessed or updated
by most BIFs for module
dynamics (\S\ref{chapter:module-dynamics},
\ifOld\S\ref{section:module-bifs}\fi
\ifStd\S\ref{section:module-module}\fi).
\index{node!module table|)}
\index{module_table node property@\T{module_table} node property|)}

\item[\T{monitored_nodes[\Z{N}]}]
\index{monitored_nodes node property@\T{monitored_nodes} node property|(}
\index{node!monitored|(}
This is a table where each row has a node name as key and a table as
value.  Each such table has the PID of a process residing on \TZ{N} as
key and a nonnegative integer as value.  If \T{monitored_nodes[\Z{N}]}
contains a row with key $\TZ{N}'$ and value $t$ and $t$ contains a row
with key \TZ{P} and value \TZ{I}, then each time node $\TZ{N}'$ ceases
to be a friend of node \TZ{N}, $\Er[\TZ{I}]$ messages
\T{\{node_down,$\TZ{N}'$\}} will be sent to process \TZ{P}.
The value of \T{monitored_nodes[\Z{N}]} cannot be accessed directly.  It can be
modified using the BIF
\T{monitor_node/2}\index{monitor_node/2@\T{monitor_node/2}} (\S\ref{section:monitornode2}).
It is initially empty.
\index{node!monitored|)}
\index{monitored_nodes node property@\T{monitored_nodes} node property|)}

\item[\T{ports[\Z{N}]}]
\index{ports node property@\T{ports} node property|(}
This is a set of the ports that are open on node \TZ{N}.
It can be accessed by processes on \TZ{N} through the BIF
\ifOld \T{ports/0}\index{ports/0 BIF@\T{ports/0} BIF}
(\S\ref{section:ports0}).\fi
\ifStd \T{node:ports/0}\index{ports/0 BIF@\T{ports/0} BIF}
(\S\ref{section:node:ports0}).\fi
The set is updated implicitly when a
new port is opened on \TZ{N} (\S\ref{section:opening-ports})
and when a port on \TZ{N} is closed (\S\ref{section:closing-ports}).
\index{ports node property@\T{ports} node property|)}

\item[\T{processes[\Z{N}]}]
\index{processes node property@\T{processes} node property|(}
This is a set of the PIDs of the processes that run on node \TZ{N}.
It consists of the PIDs of all processes that have been spawned on
\TZ{N} and that have not yet completed.
It can be accessed by processes on \TZ{N} through the BIF
\ifOld \T{processes/0}\index{processes/0 BIF@\T{processes/0} BIF}
(\S\ref{section:processes0}).\fi
\ifStd \T{node:processes/0}\index{processes/0 BIF@\T{processes/0} BIF}
(\S\ref{section:node:processes0}).\fi
The set is updated implicitly when a
new process is spawned on \TZ{N} (\S\ref{section:spawning-processes})
and when a process on \TZ{N} completes (\S\ref{section:process-completion}).
\index{processes node property@\T{processes} node property|)}

\item[\T{reductions[\Z{N}]}]
\index{reductions@\T{reductions}!node property|(}
This is the state of the reduction gauge.  It
\ifStd should be \fi \ifOld is \fi
updated automatically to reflect the
number of function calls that have been made on node \TZ{N}.
\ifStd
Implementations may
differ in how the number of function calls is measured.  However, the count must
be strictly increasing and it should increase at least
each time a call of remote function application begins.
\fi

\index{current_reductions node operation@\T{current_reductions} node operation|(}
\T{reductions[\Z{N}]} cannot be accessed directly but the atomic operation
\T{current_reductions[\Z{N}]} uses and updates the value of \T{reductions[\Z{N}]}
to return a 2-tuple \T{\{\Z{Total},\Z{SinceLast}\}} of integers where both
measure the number of function calls on node \TZ{N}.
The first is the number of reductions since node \TZ{N} was started while the second is
since the previous invocation of \T{current_reductions[\Z{N}]}.

The first time \T{current_reductions[\Z{N}]} is invoked,
\TZ{Total} and \TZ{SinceLast} will be the same.
This operation should only be invoked as part of an explicit
application \T{statistics(reductions)} (\S\ref{section:statistics1}) in a user program.
\index{current_reductions node operation@\T{current_reductions} node operation|)}
\index{reductions@\T{reductions}!node property|)}

\item[\T{ref_state[\Z{N}]}]
\index{ref_state node property@\T{ref_state} node property|(}
\index{next_ref node operation@\T{next_ref} node operation|(}
This is the state of the ref generator.  It cannot be accessed directly
but the atomic operation \T{next_ref[\Z{N}]} uses the value of \T{ref_state[\Z{N}]}
to provide the ID for a new ref and at the same time increments the value of
\T{ref_state[\Z{N}]}
so that the next invocation of \T{next_ref[\Z{N}]} will produce a unique ref.
\index{ref_state node property@\T{ref_state} node property|)}
\index{next_ref node operation@\T{next_ref} node operation|)}

\item[\T{registry[\Z{N}]}]
\index{registry node property@\T{registry} node property|(}
\index{process!registry|(}
This is a mapping from atoms to PIDs.
\ifStd The implementation must ensure \fi
\ifOld It is ensured \fi
that at any time,
\begin{itemize}
\item the mapping is invertible (i.e., that at most one name is registered
for each PID),
\item that all PIDs refer to processes residing on node \TZ{N}, and
\item that all PIDs refer to live processes.
\end{itemize}
This implies that when a process completes, any name registered for it
must be removed from the registry.

\T{registry[\Z{N}]} can be accessed
by processes running on node \TZ{N} through the BIFs \T{register/2}, \T{unregister/1},
\T{registered/0} and \T{whereis/1}
(\ifOld\S\ref{section:process-registry-bifs}\fi
\ifStd\S\ref{section:process-module}\fi).
It can be accessed indirectly by processes running
on any node when sending messages (\S\ref{section:send-expr}).
\index{process!registry|)}
\index{registry node property@\T{registry} node property|)}

\item[\T{runtime[\Z{N}]}]
\index{runtime node property@\T{runtime} node property|(}
\index{current_runtime node operation@\T{current_runtime} node operation|(}
This is the state of the run time gauge.  It is updated automatically to reflect the
amount of time spent running processes on node \TZ{N}.
It cannot be accessed directly
but the atomic operation \T{current_runtime[\Z{N}]} uses and updates
the value of \T{runtime[\Z{N}]}
to return a 2-tuple \T{\{\Z{Total},\Z{SinceLast}\}} of integers where both
measure the total time in milliseconds spent on running processes on node \TZ{N}.
The first element is the time since node \TZ{N} was started while the second is
since the previous invocation of \T{current_runtime[\Z{N}]}.

The first time \T{current_runtime[\Z{N}]} is invoked,
\TZ{Total} and \TZ{SinceLast} will be the same.
This operation should only be invoked as part of an explicit
BIF application \T{statistics(runtime)}\index{statistics/1 BIF@\T{statistics/1} BIF}
(\S\ref{section:statistics1}) in a user program.
\index{current_runtime node operation@\T{current_runtime} node operation|)}
\index{runtime node property@\T{runtime} node property|)}

\item[\T{wall_clock[\Z{N}]}]
\index{wall_clock node property@\T{wall_clock} node property|(}
\index{current_wall_clock node operation@\T{current_wall_clock} node operation|(}
This is the state of the real time gauge.  It is updated automatically
to reflect the passing of real time in the world where node \TZ{N}
resides.  It cannot be accessed directly but the value of
\T{wall_clock[\Z{N}]} is used and updated by the atomic operation
\T{current_wall_clock[\Z{N}]} to return a 2-tuple
\T{\{\Z{Total},\Z{SinceLast}\}} of integers where both measure the
time in milliseconds that has passed in the world.  The first element
is the time since node \TZ{N} was started while the second is since
the previous invocation of \T{current_wall_clock[\Z{N}]}.

The first time \T{current_wall_clock[\Z{N}]} is invoked, \TZ{Total}
and \TZ{SinceLast} will be the same.  This operation should only be
invoked as part of an explicit BIF application
\T{statistics(wall_clock)}\index{statistics/1 BIF@\T{statistics/1}
BIF} (\S\ref{section:statistics1}) in a user program.
\index{wall_clock node property@\T{wall_clock} node property|)}
\index{current_wall_clock node operation@\T{current_wall_clock} node operation|)}
\end{Lentry}
\index{node!state|)}


%
% %CopyrightBegin%
%
% Copyright Ericsson AB 2017. All Rights Reserved.
%
% Licensed under the Apache License, Version 2.0 (the "License");
% you may not use this file except in compliance with the License.
% You may obtain a copy of the License at
%
%     http://www.apache.org/licenses/LICENSE-2.0
%
% Unless required by applicable law or agreed to in writing, software
% distributed under the License is distributed on an "AS IS" BASIS,
% WITHOUT WARRANTIES OR CONDITIONS OF ANY KIND, either express or implied.
% See the License for the specific language governing permissions and
% limitations under the License.
%
% %CopyrightEnd%
%

\chapter{Ports}

\label{chapter:more-about-ports}

\section{Overview of ports}

\index{port!communication|(}

An \Erlang\ node and thus an \Erlang\ process on it communicates with
resources in the outside world (including the rest of the computer on
which it resides) through \emph{ports}.  Examples of such external
resources are files, drivers (\S\ref{section:drivers}) and non-\Erlang\
processes running on the same
host.  Information sent to a port from outside the \Erlang\ node
can be read by an \Erlang\ process as messages and
messages sent by an \Erlang\ process to a port are made available to
the external program (or written to a file).

\index{port!direction|(}
A port can be unidirectional or bidirectional, as requested when the port
is opened.  If the port is for communication
with an external process, it would typically be bidirectional.
\ifOld
If the port is connected to a file that is being read, it would typically
be input-only.  If the port is connected to a file that is to be written,
it would typically be output-only.
\fi
\index{port!direction|)}

\index{port!ownership|(}
A port is always \emph{owned} by some process that resides on the
same node as the port.
That process will receive input in the form of messages from the
port when information is sent to the port from outside \Erlang.
The port is initially owned by the process that opened the port but ownership
of a port can be transferred to another process on the same node.
\index{port!ownership|)}

An external resource has much in common with a process and thus a
port has much in common with a PID:
\begin{itemize}
\item Communicating with an external resource is similar to communicating
with a process: messages to a port are sent using the \T{!}\ operator and
messages from a port are received using \T{receive} expressions.

\item \index{port!linking|(}
A process can link to a port in order to be notified when an
external resource completes.
\index{port!linking|)}
\end{itemize}
However, ports are restricted as compared with processes.
For example, information written to a port from outside \Erlang\ is always sent
as a message to the process that owns the port.
An external resource thus cannot communicate directly
with an arbitrary \Erlang\ process.\footnote{A typical way to establish
communication between an external resource and arbitrary processes is
to let the process that is connected to the port be a server acting
as a proxy.}  Also, it is not possible to
register a name for a port.

Each port is identified by an \Erlang\ term that is itself referred to
as a port (although calling it a \emph{port identifier} would be more
in harmony with process vs.\ process identifier).

When a port is created, it is connected with an external entity, which
is
\ifOld
either
\begin{itemize}
\item a recently spawned external process or recently opened driver
(\S\ref{section:drivers});
\item an external resource, such as a file.
\end{itemize}
\fi
\ifStd
a recently spawned external process or recently opened driver
(\S\ref{section:drivers}).
\fi

As in the case of PIDs, there can be ports that were created for communication
with an external process or resource that no longer exists.
What happens when a BIF is given such a port varies, cf.\
\ifOld\S\ref{section:port-bifs}\fi
\ifStd\S\ref{section:port-module}\fi.

The interface to a port from the outside of Erlang is dependent on the operating
system where the node resides.  A driver or external process that has been
started by the \Erlang\ system when a port was opened can read from a port and write
to it using a pair of procedures that we refer to below as \I{read} and \I{write}.

\Erlang\ BIFs relating to ports are described in
\ifOld\S\ref{section:port-bifs}\fi
\ifStd\S\ref{section:port-module}\fi.
\index{port!communication|)}

\section{Drivers}

\label{section:drivers}
\index{driver|(}

An \Erlang\ system may provide a way to extend a node with software written in other
programming languages than \Erlang.  Such software is called a \emph{driver} and would
normally communicate with \Erlang\ processes through a port.

The interface to external software is thus the same regardless of whether it runs as
part of the node or as external processes.
\index{driver|)}

\section{I/O terms}

\label{section:io-term}

\index{I/O term!definition of|(}
Define an I/O term to be either
\begin{itemize}
\item a binary,
\item nil, or
\item a cons where the head is a byte\ifStd, a character\fi\ or an I/O term
and the tail is an I/O term.
\end{itemize}
\index{I/O term!definition of|)}

\index{I/O term!contents of|(}
The \emph{contents} of an I/O term is a sequence of bytes defined
recursively as follows:
\begin{itemize}
\item The contents of a binary is the sequence of bytes of the binary,
in the order they appear in the binary.
\item The contents of an empty list is an empty sequence.
\item The contents of a cons of a byte and an I/O term is
the byte followed by the contents of the tail.
\ifStd
\item The contents of a cons of a character and an I/O term is
the code of the character modulo 256 followed by the contents of the tail.
\fi
\item The contents of a cons of two I/O terms is
the contents of the head followed by the contents of the tail.
\end{itemize}
\index{I/O term!contents of|)}

To clarify, the function \T{contents/1} defined below returns the contents
of an I/O term, represented as a list of bytes.\footnote{It is designed for
clarity, not efficiency.}

\ifStd
\begin{verbatim}
contents(B) when is_binary(B) -> binary:to_list(B) ;
contents([]) -> [] ;
contents([I|Y]) when is_integer(I) -> [I | contents(Y)] ;
contents([C|Y]) when is_char(C) ->
    [char:to_integer(C) mod 256 | contents(Y)] ;
contents([X|Y]) -> contents(X) ++ contents(Y).
\end{verbatim}
\fi
\ifOld
\begin{verbatim}
contents(B) when binary(B) -> binary_to_list(B) ;
contents([]) -> [] ;
contents([I|Y]) when integer(I) -> [I | contents(Y)] ;
contents([X|Y]) -> contents(X) ++ contents(Y).
\end{verbatim}
\fi

\section{Opening ports}

\label{section:opening-ports}
\index{port!opening|(}

A port is opened by calling the BIF
\T{open_port/2}\index{open_port/2 BIF@\T{open_port/2} BIF}
(\S\ref{section:openport2}).
In an application \T{open_port(\Z{Resource},\Z{Options})}, the \TZ{Resource}
argument identifies the resource to which a port is being opened and the
\TZ{Options} argument determines the details of the behaviour of the port.

\TZ{Resource} is one of
\begin{itemize}
\item a 2-tuple \T{\{spawn,\Z{Command}\}}, where \TZ{Command} is a string
or an atom (in which case its printname will be used).  A driver is opened
inside the \Erlang\ node or an external
process is spawned and a new port is connected with the driver or the
standard input and output of the external process.
\ifOld
\TZ{Command} must not contain a null character (\T{\char`\\\ifStd u0000\fi\ifOld 000 \fi}).
Suppose that the result of extracting whitespace-separated words from
\TZ{Command} is $\TZ{Wd}_1$, $\TZ{Wd}_2$, \ldots, $\TZ{Wd}_k$.
\begin{itemize}
\item If $\TZ{Wd}_1$ is the name of an installed driver, then such a driver
will be started with the string $\TZ{Wd}_2$, \ldots, $\TZ{Wd}_k$ as
input.
\item Otherwise $\TZ{Wd}_1$ will be interpreted as the name of an
external program which is started with $\TZ{Wd}_2$, \ldots, $\TZ{Wd}_k$ as its arguments.
\end{itemize}
\fi
\ifStd
The exact interpretation of \TZ{Command} is implementation-defined.
\fi
\ifOld
\item a string or atom (in which case its printname will be used) \TZ{Resource}.
If \TZ{Resource} can be interpreted
by \OldErlang\ as identifying a resource (for example, naming a file),
a new port is connected with the resource.  If the resource is a file, then
either the option \T{in} (for reading from the file) or the option \T{out}
(for writing to the file) must be given.
\fi
\item a 3-tuple \T{\{fd, \Z{In}, \Z{Out}\}}, where \TZ{In} and \TZ{Out} are
integers.  On a node running on a POSIX-compliant operating system, \TZ{In}
and \TZ{Out} are interpreted as file descriptors and a new port is connected
writing to \TZ{In} and reading from \TZ{Out}.  On other nodes, the behaviour
is implementation-defined.
\end{itemize}
\ifStd
A \StdErlang\ implementation may accept further alternatives for
\TZ{Resource}.  It should be documented for an implementation which
alternatives for \TZ{Resource} are accepted and how they are
interpreted.
\fi
If \TZ{Resource} is not one of the permitted alternatives, \T{open_port/2}
should exit with \T{badarg}.

\TZ{Options} is a list of items, each of which is a term from one of the
following groups.  At most one term from each group should be present.
If no term from a group is in \TZ{Options}, it is as if the default
term for the group was present.

\begin{Lentry}
\item[Stream/packets]
\index{port!stream mode|(}
\index{port!packet mode|(}
\begin{itemize}
\item \T{stream} (the default). Messages are sent without packet headers.
\item \T{\{packet,\Z{N}\}}, where \TZ{N} is one of the integers \T{1}, \T{2}
and \T{4}.  Messages are sent in packets with the packet headers in \TZ{N} bytes
(\S\ref{section:packets}).
\end{itemize}
\index{port!stream mode|)}
\index{port!packet mode|)}
\item[Process inputs and outputs]
(Only relevant on POSIX-compliant systems and for \T{\{spawn,\TZ{Command}\}}.)
\begin{itemize}
\item \T{use_stdio}.  (The default.)
Use the standard input and standard output (i.e., file descriptors 0 and 1 on UNIX
systems) of the
spawned process for communicating with \Erlang.
\item \T{nouse_stdio}.
Use alternative input and output channels of the spawned process (file descriptors
3 and 4 on UNIX systems) for communicating with \Erlang.
\end{itemize}
\item[Direction]
\index{port!direction|(}
\begin{itemize}
\item \T{in}. The port can only be used for input from the external resource.
\item \T{out}. The port can only be used for output to the external resource.
\item \ifOld \I{None}. \fi \ifStd \T{bi}. (The default.) \fi
The port can be used for input from and output to the external resource.
\end{itemize}
\index{port!direction|)}
\item[Received data]
\index{port!binary mode|(}
\index{port!list mode|(}
\begin{itemize}
\item \T{binary}.  Input from the external resource is received as binaries.
\item \ifOld \I{None}. \fi \ifStd \T{list}. (The default.) \fi
Input from the external resource is received as lists of bytes.
\end{itemize}
\index{port!binary mode|)}
\index{port!list mode|)}
\item[End of stream]
\index{port!closing|(}
\begin{itemize}
\item \T{eof}.  The port is not closed when the external resource is depleted,
instead a message \T{\{\Z{Port},eof\}} is sent to the owning process.
\item \T{noeof}.  (The default.) The port is closed when the external resource is depleted.
All processes linked to the port will then receive exit signals.
\end{itemize}
\index{port!closing|)}
\end{Lentry}
\ifStd
A \StdErlang\ implementation may accept further alternatives for
elements of \TZ{Options} and it should be documented which they are
and how they are interpreted.
\fi
If \TZ{Options} is not a list or if \TZ{Options} contains unrecognized terms,
\T{open_port/2} should exit with \T{badarg}.

\index{port!linking|(}
When a port is opened, it is automatically linked to the owning process
(\S\ref{section:port-linking}).
\index{port!linking|)}
\index{port!opening|)}

\section{Controlling a port from an \Erlang\ process}

\label{section:port-control}
\index{port!control|(}

By controlling a port \TZ{R} to an external resource we mean to
write data to the port or changing properties of the port itself.
Such control is exercised by
sending a message with \TZ{R} as destination
(\S\ref{section:send-expr}).  As before, evaluation
of such a send expression always succeds.  However, the \Erlang\ system
subsequently attempts to interpret the message as port control information and if
that interpretation fails, an exit signal\index{exit!signal} (\S\ref{section:exit-signals})
will be sent to the process owning \TZ{R}, i.e.,
\T{owner[\Z{R}]}\index{owner port property@\T{owner} port property}
(\S\ref{section:port-state-dynamic}).

Any process may control a port.  However, the process that sends
a message to \TZ{R}
must have the PID of the process that owns \TZ{R} (because \T{owner[\Z{R}]} must
be part of the message, as described below).

\index{badsig exit signal@\T{badsig} exit signal|(}
Consider a message \TZ{v} sent to a port \TZ{R}.
\begin{itemize}
\item If the port has been closed, then nothing happens.
\item If \TZ{v} is a tuple \TZ{\{\T{owner[\Z{R}]}, \{command, \Z{Data}\}\}}, then:
\begin{itemize}
\item If \T{in[\Z{R}]} is \T{false}, then
an exit signal \T{badsig} is sent to \T{owner[\Z{R}]}.
\item If \TZ{Data} is an I/O term (\S\ref{section:io-term}), then the contents
of \TZ{Data} will be transmitted through \TZ{R}, as described in
\S\ref{section:send-port} (where it is also described how an exit
signal \T{badsig} is sent to \T{owner[\Z{R}]} if the number of bytes
in the contents of \TZ{Data} is too large to fit in the packet header)
\item Otherwise, an exit signal \T{badsig} is sent to \T{owner[\Z{R}]}.
\end{itemize}
\item If \TZ{v} is a tuple \TZ{\{\T{owner[\Z{R}]}, close\}}, then
the port \TZ{R} is closed and a message \T{\{\Z{R}, closed\}} is sent
to \T{owner[\Z{R}]}.
\item If \TZ{v} is a tuple \TZ{\{\T{owner[\Z{R}]}, \{connect, $\Z{P}_2$\}\}}, then
what happens depends on $\TZ{P}_2$:
\begin{itemize}
\item If $\TZ{P}_2$ is the PID of a process residing on the same node as \TZ{R}
(and thus \T{owner[\Z{R}]}, then a message \T{\{\Z{R}, connected\}} is sent
to \T{owner[\Z{R}]} and the value of \T{owner[\Z{R}]} is changed to $\Z{P}_2$.
\item Otherwise, an exit signal \T{badsig} is sent to \T{owner[\Z{R}]}.
\end{itemize}
\item Otherwise, an exit signal \T{badsig} is sent \T{owner[\Z{R}]}.
\end{itemize}
It should be noted that
all processing when writing to a port happens asynchronously, also error processing.
\index{badsig exit signal@\T{badsig} exit signal|)}
\index{port!control|)}

\section{Transmitting data from an \Erlang\ process to the outside}

\label{section:send-port}
\label{section:packets}
\index{port!sending to a|(}

Suppose that an \Erlang\ process has sent an I/O term \TZ{T} to a port \TZ{R} and that
the contents of the I/O term is a sequence of bytes $\TZ{b}_1$, \ldots, $\TZ{b}_k$.

Depending on the value of
\T{packeting[\Z{R}]}\index{packeting port property@\T{packeting} port property},
the byte sequence transmitted to the
external resource may be prepended with a packet header by the \Erlang\ system:
\begin{itemize}
\item \index{port!stream mode|(}
If \T{packeting[\Z{R}]} is \T{stream}, then the byte sequence will be transmitted
as is.
\index{port!stream mode|)}
\item \index{port!packet mode|(}
If \T{packeting[\Z{R}]} is \T{\{packet,\Z{N}\}}, the transmitted byte sequence
will be $\TZ{h}_1$, \ldots, $\TZ{h}_{\TZm{N}}$, $\TZ{b}_1$, \ldots, $\TZ{b}_k$.
The first \TZ{N} bytes constitute a \emph{packet header}, where for each $i$,
$1\leq i\leq\TZ{N}$, $\TZ{h}_i=\lceil k/256^{\TZm{N}-i}\rceil \bmod 256$.
(That is, the packet header encodes the length of the byte sequence, \TZ{k},
as a big-endian numeral, cf.\ p.~\pageref{page:big-endian}.)
However, if $k\geq256^{\TZm{N}}$, no byte sequence at all will be transmitted to the
external resource and instead an exit signal
\T{badsig} will be sent to \T{owner[\Z{R}]}.
\index{port!packet mode|)}
\end{itemize}
If the external resource is a file opened for writing, then the transmitted byte sequence
written to the file (but the file is not closed until the port is closed).
If the external resource is an opened driver or an external process, it can read the transmitted
byte sequence using the \I{read} function.
However, no
assumption should
be made about how many bytes of data will be read by each invocation of the \I{read}
function.
Note that the transmission of a byte sequence is atomic in that the bytes of a single
transmission (including any
prepended bytes) will never be separated by bytes from other transmissions.
\index{port!sending to a|)}

\section{Transmitting data from the outside to an \Erlang\ process}

\index{port!receiving from a|(}

When the external resource to which a port \TZ{R} is opened is a file and \T{in[\Z{R}]} is
\T{true}, then the contents of the file is written to the port as if by a single invocation
of \I{write}.  We will now describe what happens when an opened driver or an external process writes
data to a port using the \I{write} function.

If \T{out[\Z{R}]} is \T{false}, then nothing is transmitted.
Otherwise messages will be sent to
\T{owner[\Z{R}]}\index{owner port property@\T{owner} port property}\index{port!ownership}.

Let the complete sequence of bytes written (through one or more invocations of \I{write})
by the driver or external process be $\TZ{b}_1$, \ldots, $\TZ{b}_n$.
Depending on the value of \T{packeting[\Z{R}]}, the byte sequence may be
interpreted as containing packet headers, in which case each packet will be
delivered to \T{in[\Z{R}]} as a separate message.
\begin{itemize}
\item \index{port!stream mode|(}
If \T{packeting[\Z{R}]} is \T{stream}, all $n$ bytes will be delivered but
it is not specified how the byte sequence is to be divided into messages.
Each message $i$ (where $i\geq1$) contains the bytes
$\TZ{b}_{k_i}$, \ldots, $\TZ{b}_{k_{i+1}-1}$ where $k_1=1$ and for each $i$, either $k_i\leq n$ and
$k_i<k_{i+1}$, or $k_i=n+1$.  (Thus there should be no empty messages.)  It is encouraged that
each invocation of \I{write} (with a nonempty sequence of bytes) should cause a message to be
sent, in order to facilitate interaction between the driver or external process and the \Erlang\
processes.
\index{port!stream mode|)}
\item \index{port!packet mode|(}
If \T{packeting[\Z{R}]} is \T{\{packet,\Z{N}\}}, the sequence $\TZ{b}_1$, \ldots, $\TZ{b}_n$ is
interpreted as containing packet headers of \TZ{N} bytes each.  Each message $i$ (where $i\geq1$)
will contain the bytes $\TZ{b}_{k_i+\TZm{N}}$, \ldots, $\TZ{b}_{k_{i+1}-1}$ where $k_1=1$ and for each $i$,
either $k_i\leq n$,
\[l_i = \I{BigEndianValue}(\TZ{b}_{k_i}, \ldots, \TZ{b}_{k_i+\TZm{N}})\]
(cf.\ p.~\pageref{page:big-endian})
and $k_i+\TZ{N}+l_i=k_{i+1}$, or $k_i= n+1$.  
(That is, the first \TZ{N} bytes are interpreted as a packet header and the subsequent
bytes [where the number of bytes is given by the packet header] will be contained in a
single message.  Then comes another packet header, etc.  Note that there can be empty messages.)
It is encouraged that a message should be sent as soon as all the bytes of a package have been
written, in order to facilitate interaction between the driver or external process and the \Erlang\
processes.
\index{port!packet mode|)}
\end{itemize}
Each message to \T{owner[\Z{R}]} is on the form \T{\{\Z{R},\Z{T}\}} where \TZ{T} contains $k$ transmitted
bytes, but the term \TZ{T} depends on whether \T{in_format[\Z{R}]} is \T{binary} or \T{list}:
\begin{itemize}
\item \index{port!binary mode|(}
If \T{in_format[\Z{R}]} is \T{binary}, then \TZ{T} is a binary consisting of the $k$ bytes in the
order they were written.
\index{port!binary mode|)}
\item \index{port!list mode|(}
If \T{in_format[\Z{R}]} is \T{list}, then \TZ{T} is a list of the $k$ bytes in the
order they were written.
\index{port!list mode|)}
\end{itemize}
\index{port!receiving from a|)}

\section{Closing ports}

\label{section:closing-ports}
\index{port!closing|(}

A port can be closed (through an implementation-defined method)
by the external resource.  It can also be closed by an \Erlang\ process
using the BIF
\ifOld\T{port_close/1}\index{port_close/1 BIF@\T{port_close/1} BIF} (\S\ref{section:portclose1}).\fi
\ifStd\T{port:close/1}\index{port:close/1 BIF@\T{port:close/1} BIF} (\S\ref{section:port:close1}).\fi

The options \T{eof} and \T{noeof} to the BIF \T{open_port/2}
(\S\ref{section:opening-ports})
determine how the \Erlang\ process that owns
the port is notified if the port is closed by the external resource.
\index{port!closing|)}

\section{Ports, links and exit signals}

\label{section:port-linking}
\index{port!linking|(}

An \Erlang\ process may be linked to a port in the same way that it may be linked to a process
(\S\ref{section:links}).
A process \TZ{P} may set up (or remove) a link with a port \TZ{R} by calling the BIF
\T{link/1}\index{link/1 BIF@\T{link/1} BIF}
(or \T{unlink/1}\index{unlink/1 BIF@\T{unlink/1} BIF}) with \TZ{R} as argument.
When a port is closed, an exit signal will be sent to
every process linked to the port.  The actual exit signal is implementation-defined, except that
implementations are encouraged to use the exit signal \T{normal} when the driver or external process
completes normally.

\index{exit!signal|(}
When a port \TZ{R} receives an exit signal \TZ{T} from a process \TZ{P}, the following happens.
\begin{itemize}
\item If \TZ{T} is \T{normal}\index{normal@\T{normal}!exit signal} and \TZ{P}
is not \T{owner[\Z{R}]}, then nothing happens.
\item Otherwise, if \TZ{T} is \T{kill}\index{kill@\T{kill}!exit signal},
the port is closed and an exit signal \TZ{killed} is sent
to every process that is linked to the port.
\item Otherwise, the port is closed and an exit signal \TZ{T} is sent to every process that is
linked to the port.
\end{itemize}
Note that this behaviour is somewhat different from how a process handles incoming exit signals
(\S\ref{section:receiving-exit-signal}).
\index{exit!signal|)}
\index{port!linking|)}

\ifStd
\section{Ports and Unicode}

\label{section:ports-unicode}
\index{Unicode!and ports|(}

It should be noticed that all communication with ports is through sequences of bytes,
i.e., numbers between 0 and 255.  The codes of Unicode characters require two bytes
and there is no general provision for sending such characters directly to ports.
(However, note that characters \T{\char`\u0000} to \T{\char`\u00FF} can be sent directly
to ports.)

An arbitrary character string that is to be sent through a port must thus be transformed to a
sequence of bytes before being transmitted and there are several such standard transformations,
such as:
\begin{Lentry}
\item[Big-endian.]
Each character is represented by two bytes where the first is the high-order byte and
the second is the low-order byte of the character code.
\item[Little-endian.]
Each character is represented by two bytes where the first is the low-order byte and
the second is the high-order byte of the character code.
\item[UTF-8.]
\index{UTF-8 transformation@\T{UTF-8} transformation|(}
UTF-8 stands for UCS Transformation Format, 8-bit form, and is now part of the ISO/IEC
10646 standard \cite{utf8}.  Each Unicode character is represented by a single byte or
a sequence of two to four bytes.  All ASCII characters are represented by a single byte
and for character sequences that mostly or wholly consist of ASCII characters, the
transformation results in very short byte sequences.
\index{UTF-8 transformation@\T{UTF-8} transformation|)}
\item[UTF-7.]
\index{UTF-7 transformation@\T{UTF-7} transformation|(}
UTF-7 stands for UCS Transformation Format, 7-bit form, and is described in Internet Network
Working Group RFC-1642 \cite{rfc1642}.  Each Unicode character is represented by a single byte
or a sequence of bytes.  The ASCII letters and digits and ten other common ASCII characters
are each representable by a single byte.  The resulting sequence of bytes is safe for
Internet mail transmission (except where line length and line break restrictions are violated).
\index{UTF-7 transformation@\T{UTF-7} transformation|)}
\end{Lentry}

In order for an \Erlang\ process to send an arbitrary string to a port, it should be transformed
to a sequence of
bytes (represented, for example, as a binary) and then transmitted.  Of course, the process
receiving the sequence of bytes must be aware of the transformation if the original Unicode
characters are to be restored.

A similar process would be used to send Unicode character sequences from an external process to
an \Erlang\ process.
\index{Unicode!and ports|)}
\fi

\section{Static and dynamic properties of a port}

\label{section:port-state}
\index{port!state|(}

When a port is created, some properties of the port are determined and will be
in effect until the port is closed.  During that time also a dynamic state
is maintained, consisting of properties of the port that change as time passes.
This state is affected by and affects the behaviour of the port.

We refer to these as the \emph{static} and \emph{dynamic properties} of a port.  We
refer collectively to the values of the latter at a certain time as the \emph{state} of the port
at that time.

\subsection{Static properties}

\label{section:port-state-static}

\begin{Lentry}
\item[\T{command[\Z{R}]}]
\index{command port property@\T{command} port property|(}
For a port that was opened through a call to \T{open_port/2} with
\T{\{spawn,\Z{Cmd}\}}, the
value of \T{command[\Z{R}]} is a string that is either \TZ{Cmd}, if \TZ{Cmd} is a string,
or the printname of \TZ{Cmd}, if \TZ{Cmd} is an atom.
It can be accessed through a BIF call
\ifOld \T{erlang:port_info(\Z{R},name)}\index{port_info/2 BIF@\T{port_info}/2 BIF}
(\S\ref{section:portinfo2}).\fi
\ifStd \T{port:info(\Z{R},name)}\index{port:info/2 BIF@\T{port:info}/2 BIF}
(\S\ref{section:port:info2}).\fi
\index{command port property@\T{command} port property|)}

\item[\T{creation[\Z{R}]}]
\index{creation@\T{creation}!port property|(}
The value of \T{creation[\Z{R}]} is the value of \T{creation[\Z{N}]} for the
node \TZ{N} on which \TZ{R} was created.
\index{creation@\T{creation}!port property|)}

\item[\T{ID[\Z{R}]}]
\index{ID@\T{ID}!port property|(}
The value of \T{ID[\Z{R}]} is a nonnegative integer that is a serial number for \TZ{R}
on the node on which it was
created.  The value is used in the transformation
to the external term format (\S\ref{chapter:external-format}).
It can be accessed through a BIF call
\ifOld \T{erlang:port_info(\Z{R},id)}\index{port_info/2 BIF@\T{port_info}/2 BIF}
(\S\ref{section:portinfo2}).\fi
\ifStd \T{port:info(\Z{R},id)}\index{port:info/2 BIF@\T{port:info}/2 BIF}
(\S\ref{section:port:info2}).\fi
\index{ID@\T{ID}!port property|)}

\item[\T{in[\Z{R}]}]
\index{in port property@\T{in} port property|(}
When a port is opened it is decided whether data can be transmitted from the outside to
the process owning the port.  The value is \T{true} if one of the options \T{in} and \T{bi} was
given when the port was opened and \T{false} otherwise.  The value of \T{in[\Z{R}]} cannot
be accessed directly by \Erlang\ processes.
\index{in port property@\T{in} port property|)}

\item[\T{in_format[\Z{R}]}]
\index{in_format port property@\T{in_format} port property|(}
When a port is opened it is decided whether incoming data is sent as binaries or as lists of
bytes.  The value is \T{binary} or \T{list} depending on which of these options was given when
the port was opened.  The value of \T{in_format[\Z{R}]} cannot be accessed directly by
\Erlang\ processes.
\index{in_format port property@\T{in_format} port property|)}

\item[\T{node[\Z{R}]}]
\index{node@\T{node}!port property|(}
When a port is opened it is created on some node \T{node[\Z{R}]}.
This node never changes. Any process can access \T{node[\Z{R}]}
for a port \TZ{R} by calling the BIF
\T{node/1}\index{node/1 BIF@\T{node/1} BIF} (\S\ref{section:node1})
with \TZ{R} as argument.
\index{node@\T{node}!port property|)}

\item[\T{out[\Z{R}]}]
\index{out port property@\T{out} port property|(}
When a port is opened it is decided whether data can be transmitted from \Erlang\ processes
to the outside.  The value is \T{true} if one of the options \T{out} and \T{bi} was
given when the port was opened and \T{false} otherwise.  The value of \T{out[\Z{R}]} cannot
be accessed directly by \Erlang\ processes.
\index{out port property@\T{out} port property|)}

\item[\T{packeting[\Z{R}]}]
\index{packeting port property@\T{packeting} port property|(}
When a port is opened it is decided whether data is transmitted as a stream or as packets
beginning with a packet header of some length.  The value is \T{stream} or \T{\{packet,\Z{N}\}}
(where \TZ{N} is \T{1}, \T{2} or \T{4}) depending on which of these options was given when
the port was opened.  The value of \T{packeting[\Z{R}]} cannot be accessed directly by
\Erlang\ processes.
\index{packeting port property@\T{packeting} port property|)}
\end{Lentry}

\subsection{Dynamic properties}

\label{section:port-state-dynamic}

\begin{Lentry}
\item[\T{count_in[\Z{R}]}]
\index{count_in port property@\T{count_in} port property|(}
\T{count_in[\Z{R}]} is an integer that is the number of bytes read so far from the
port \TZ{R}.
It can be accessed through a BIF call
\ifOld\begin{alltt}
erlang:port_info(\Z{R},input)
\end{alltt}
\index{port_info/2 BIF@\T{port_info}/2 BIF}(\S\ref{section:portinfo2}).\fi
\ifStd\begin{alltt}
port:info(\Z{R},input)
\end{alltt}
\index{port:info/2 BIF@\T{port:info}/2 BIF}(\S\ref{section:port:info2}).\fi
\index{count_in port property@\T{count_in} port property|)}

\item[\T{count_out[\Z{R}]}]
\index{count_out port property@\T{count_out} port property|(}
\T{count_out[\Z{R}]} is an integer that is the number of bytes written so far to the
port \TZ{R}.
It can be accessed through a BIF call
\ifOld\begin{alltt}
erlang:port_info(\Z{R},output)
\end{alltt}
\index{port_info/2 BIF@\T{port_info}/2 BIF}(\S\ref{section:portinfo2}).\fi
\ifStd\begin{alltt}
port:info(\Z{R},output)
\end{alltt}
\index{port:info/2 BIF@\T{port:info}/2 BIF}(\S\ref{section:port:info2}).\fi
\index{count_out port property@\T{count_out} port property|)}

\item[\T{linked[\Z{R}]}]
\index{linked@\T{linked}!port property|(}
\index{port!linking|(}
The value is a representation of the set of PIDs that identify
the processes to which \TZ{R} is linked (\S\ref{section:links}).
It cannot be set directly but a PID \TZ{P} will be added to \T{linked[\Z{R}]}
if it is not already in it and \TZ{R} receives a linking request from \TZ{P}.

A PID \TZ{P} will be removed from \T{linked[\Z{R}]} if it is in the list and
either
\begin{itemize}
\item \TZ{P} receives an unlinking request from \TZ{Q}, or
\item \TZ{P} receives an exit signal \T{\{'EXIT',\Z{Q},\Z{Reason}\}} for some term
\TZ{Reason}, and the exit signal was sent due to process completion
(\S\ref{section:exit-signals}).
\end{itemize}

The value of \T{linked[\Z{R}]} can be accessed through a BIF call
\ifOld\begin{alltt}
erlang:port_info(\Z{R},links)
\end{alltt}
\index{port_info/2 BIF@\T{port_info}/2 BIF}(\S\ref{section:portinfo2}).\fi
\ifStd\begin{alltt}
port:info(\Z{R},links)
\end{alltt}
\index{port:info/2 BIF@\T{port:info}/2 BIF}(\S\ref{section:port:info2}).\fi
\index{port!linking|)}
\index{linked@\T{linked}!port property|)}

\item[\T{owner[\Z{R}]}]
\index{owner port property@\T{owner} port property|(}
The value is a PID and is the process that will receive messages when
data is written to the port \TZ{R} externally, will receive exit
signals when bad messages are sent to \TZ{R} and becomes linked with
\TZ{R} when \TZ{R} is opened. \T{owner[\Z{R}]} can be changed to a PID
\TZ{P} by sending a message \T{\{owner[\Z{R}],\{connect,\TZ{P}\}\}} to \TZ{R}.
The value of \T{owner[\Z{R}]} can be accessed through a BIF call
\ifOld\begin{alltt}
erlang:port_info(\Z{R},connected)
\end{alltt}
\index{port_info/2 BIF@\T{port_info}/2 BIF}(\S\ref{section:portinfo2}).\fi
\ifStd\begin{alltt}
port:info(\Z{R},connected)
\end{alltt}
\index{port:info/2 BIF@\T{port:info}/2 BIF}(\S\ref{section:port:info2}).\fi
\index{owner port property@\T{owner} port property|)}
\end{Lentry}

There will typically also be buffers for inbound and/or outbound data
but these are not referred to above and we do not describe them in
detail here.
\index{port!state|)}


%
% %CopyrightBegin%
%
% Copyright Ericsson AB 2017. All Rights Reserved.
%
% Licensed under the Apache License, Version 2.0 (the "License");
% you may not use this file except in compliance with the License.
% You may obtain a copy of the License at
%
%     http://www.apache.org/licenses/LICENSE-2.0
%
% Unless required by applicable law or agreed to in writing, software
% distributed under the License is distributed on an "AS IS" BASIS,
% WITHOUT WARRANTIES OR CONDITIONS OF ANY KIND, either express or implied.
% See the License for the specific language governing permissions and
% limitations under the License.
%
% %CopyrightEnd%
%

\chapter{Builtin functions}

\label{chapter:bifs}

As for any function application, when an application of a BIF is
evaluated all arguments are fully evaluated before the BIF itself is
called and begins executing.  When below we discuss the evaluation of
a BIF with $k$ parameters, we assume that the $k$ arguments have
already been evaluated and that their values are $\TZ{v}_1$, \ldots,
$\TZ{v}_k$.

For convenience we abuse our language slightly and may write
\begin{itemize}
\item ``calling a BIF'' when we mean ``evaluating an application of a BIF''
(where the values of the $k$ arguments are the terms
$\TZ{v}_1$, \ldots, $\TZ{v}_k$),
\item ''the BIF returns \ldots''\ when we mean ``the evaluation of
an application of the BIF completes normally with result \ldots'', and
\item ``the BIF exits with reason \ldots'' when we mean
``the evaluation of an application of the BIF exits with reason \ldots''.
\end{itemize}
% There is a proposal to change this!!!
\index{BIF!exit cause of|(}
In this chapter we write that a BIF ``exits with cause
\TZ{R}'' to express that the evaluation of an application of the BIF
completes abruptly with reason
\begin{alltt}
\{'EXIT',\{\Z{R},\{\Z{M},\Z{F},[\(\Z{v}\sb{1}\),\tdots,\(\Z{v}\sb{k}\)]\}\}\R{,}
\end{alltt}
where \TZ{M} is the name of the module and \TZ{F} is the symbol of the
function that called the BIF, and as usual, $\TZ{v}_1$, \ldots,
$\TZ{v}_k$ are the values of the arguments.
\index{BIF!exit cause of|)}

\ifOld
\index{badarg exit cause@\T{badarg} exit cause|(}
If the abnormal completion was because there was something wrong with
the value of an argument, e.g., is was of the wrong type or it was an
index outside the meaningful range, then \TZ{R} is always the atom
\T{badarg}.
\index{badarg exit cause@\T{badarg} exit cause|)}
\fi
\ifStd
\index{badarg exit cause@\T{badarg} exit cause|(}
If the abnormal completion was because the value of an argument was
not valid (e.g., it was not of a permitted type or it was not one of a
group of permitted alternatives such as a set of atoms), then \TZ{R}
is always the atom
\T{badarg}.
\index{badarg exit cause@\T{badarg} exit cause|)}
\index{badindex exit cause@\T{badindex} exit cause|(}
If the cause for the abrupt completion was an invalid index
in some kind of sequence, then \TZ{R} is always the atom
\T{badindex}.
\index{badindex exit cause@\T{badindex} exit cause|)}
\fi

\index{BIF!guard|(}
If it is not explicitly said about a BIF that it is a guard BIF, then
it is not a guard BIF.
\index{BIF!guard|)}

%Some BIFs are mentioned under more than one heading.

\section{Recognizer BIFs}

\label{section:recognizer-bifs}
\index{BIF!recognizer|(}
\ifOld
\index{atom/1 BIF@\T{atom/1} BIF|(}
\index{binary/1 BIF@\T{binary/1} BIF|(}
\index{constant/1 BIF@\T{constant/1} BIF|(}
\index{float/1 BIF@\T{float/1} BIF|(}
\index{function/1 BIF@\T{function/1} BIF|(}
\index{integer/1 BIF@\T{integer/1} BIF|(}
\index{list/1 BIF@\T{list/1} BIF|(}
\index{number/1 BIF@\T{number/1} BIF|(}
\index{pid/1 BIF@\T{pid/1} BIF|(}
\index{port/1 BIF@\T{port/1} BIF|(}
\index{record/2 BIF@\T{record/2} BIF|(}
\index{reference/1 BIF@\T{reference/1} BIF|(}
\index{tuple/1 BIF@\T{tuple/1} BIF|(}
\fi
\ifStd
\index{is_atom/1 BIF@\T{is_atom/1} BIF|(}
\index{is_binary/1 BIF@\T{is_binary/1} BIF|(}
\index{is_char/1 BIF@\T{is_char/1} BIF|(}
\index{is_cons/1 BIF@\T{is_cons/1} BIF|(}
\index{is_compound/1 BIF@\T{is_compound/1} BIF|(}
\index{is_float/1 BIF@\T{is_float/1} BIF|(}
\index{is_function/1 BIF@\T{is_function/1} BIF|(}
\index{is_integer/1 BIF@\T{is_integer/1} BIF|(}
\index{is_list/1 BIF@\T{is_list/1} BIF|(}
\index{is_null/1 BIF@\T{is_null/1} BIF|(}
\index{is_number/1 BIF@\T{is_number/1} BIF|(}
\index{is_pid/1 BIF@\T{is_pid/1} BIF|(}
\index{is_port/1 BIF@\T{is_port/1} BIF|(}
\index{is_ref/1 BIF@\T{is_ref/1} BIF|(}
\index{is_string/1 BIF@\T{is_string/1} BIF|(}
\index{is_tuple/1 BIF@\T{is_tuple/1} BIF|(}
\fi

\ifOld
\addcontentsline{toc}{subsection}{\protect\numberline{}{\T{atom/1}}}
\addcontentsline{toc}{subsection}{\protect\numberline{}{\T{binary/1}}}
\addcontentsline{toc}{subsection}{\protect\numberline{}{\T{constant/1}}}
\addcontentsline{toc}{subsection}{\protect\numberline{}{\T{float/1}}}
\addcontentsline{toc}{subsection}{\protect\numberline{}{\T{function/1}}}
\addcontentsline{toc}{subsection}{\protect\numberline{}{\T{integer/1}}}
\addcontentsline{toc}{subsection}{\protect\numberline{}{\T{list/1}}}
\addcontentsline{toc}{subsection}{\protect\numberline{}{\T{number/1}}}
\addcontentsline{toc}{subsection}{\protect\numberline{}{\T{pid/1}}}
\addcontentsline{toc}{subsection}{\protect\numberline{}{\T{port/1}}}
\addcontentsline{toc}{subsection}{\protect\numberline{}{\T{record/2}}}
\addcontentsline{toc}{subsection}{\protect\numberline{}{\T{reference/1}}}
\addcontentsline{toc}{subsection}{\protect\numberline{}{\T{tuple/1}}}
\fi
\ifStd
\addcontentsline{toc}{subsection}{\protect\numberline{}{\T{is_atom/1}}}
\addcontentsline{toc}{subsection}{\protect\numberline{}{\T{is_binary/1}}}
\addcontentsline{toc}{subsection}{\protect\numberline{}{\T{is_char/1}}}
\addcontentsline{toc}{subsection}{\protect\numberline{}{\T{is_compound/1}}}
\addcontentsline{toc}{subsection}{\protect\numberline{}{\T{is_cons/1}}}
\addcontentsline{toc}{subsection}{\protect\numberline{}{\T{is_float/1}}}
\addcontentsline{toc}{subsection}{\protect\numberline{}{\T{is_function/1}}}
\addcontentsline{toc}{subsection}{\protect\numberline{}{\T{is_integer/1}}}
\addcontentsline{toc}{subsection}{\protect\numberline{}{\T{is_list/1}}}
\addcontentsline{toc}{subsection}{\protect\numberline{}{\T{is_number/1}}}
\addcontentsline{toc}{subsection}{\protect\numberline{}{\T{is_null/1}}}
\addcontentsline{toc}{subsection}{\protect\numberline{}{\T{is_pid/1}}}
\addcontentsline{toc}{subsection}{\protect\numberline{}{\T{is_port/1}}}
\addcontentsline{toc}{subsection}{\protect\numberline{}{\T{is_ref/1}}}
\addcontentsline{toc}{subsection}{\protect\numberline{}{\T{is_string/1}}}
\addcontentsline{toc}{subsection}{\protect\numberline{}{\T{is_tuple/1}}}
\fi

\ifOld
In \OldErlang\ the recognizer BIFs are not true BIFs and can only be
used in guards as \NT{GuardRecognizer}.
There are twelve guard recognizers:
\T{atom/1},
\T{binary/1},
\T{constant/1},
\T{float/1},
\T{function/1},
\T{integer/1},
\T{list/1},
\T{number/1},
\T{pid/1},
\T{port/1},
\T{reference/1}, and
\T{tuple/1}.
As they all behave similarly, we describe them collectively.

\EVALUATION

If $\TZ{v}_1$ is one of the terms indicated in Table~\ref{table:typetests},
then the recognizer in the guard succeeds, otherwise it fails.

\begin{table}[hbp]
\begin{center}
\begin{tabular}{@{}ll@{}}
\hline
Recognizer & Succeeds if and only if the argument is \\
\hline
\T{atom/1} & an atom \\
\T{binary/1} & a binary \\
\T{constant/1} & of an elementary type (cf.\ \S\ref{chapter:types-terms}) \\
\T{float/1} & a float \\
\T{integer/1} & an integer \\
\T{function/1} & a function term \\
\T{list/1} & a cons or nil \\ % BAAAD NAME!!!
\T{number/1} & a number \\
\T{pid/1} & a PID \\
\T{port/1} & a port \\
\T{reference/1} & a reference \\
\T{tuple/1} & a tuple \\
\hline
\end{tabular}
\caption{\Erlang\ recognizer BIFs.}
\label{table:typetests}
\end{center}
\end{table}
\fi

\ifStd
A \emph{recognizer} BIF returns \T{true} for some terms and
\T{false} for all other terms.  Most recognizer BIFs are true
for terms of a certain type.  There are sixteen recognizer BIFs:
\T{is_atom/1},
\T{is_binary/1},
\T{is_char/1},
\T{is_cons/1},
\T{is_compound/1},
\T{is_float/1},
\T{is_function/1},
\T{is_integer/1},
\T{is_list/1},
\T{is_null/1},
\T{is_number/1},
\T{is_pid/1},
\T{is_port/1},
\T{is_ref/1},
\T{is_string/1}, and
\T{is_tuple/1}.
As they all behave similarly, we describe them collectively.

\index{record/2@\T{record/2}!recognizer|(}
Note that although a \NT{GuardRecordTest} \T{record(\Z{E},\Z{R})}
\S\ref{section:record2} is not an application
of a BIF, it is used as a recognizer for records of a certain type.
\index{record/2@\T{record/2}!recognizer|)}

All recognizer BIFs are guard BIFs (to be used in \NT{GuardRecognizer}).

\TYPE

For each recognizer BIF \TZ{F},
\begin{textdisplay}
\T{\Z{F}(term()) -> bool()}.
\end{textdisplay}

\EXITS

All recognizer BIFs always complete normally.

\EVALUATION

If $\TZ{v}_1$ is one of the terms indicated in Table~\ref{table:typetests},
then \T{true} is returned, otherwise \T{false} is returned.

All recognizers except \T{is_list/1} and \T{is_string/1} should take
$O(1)$ time.  \T{is_list/1} and \T{is_string/1} should take $O(n)$
time, where $n$ is the length of the (presumed) list.

\begin{table}[hbp]
\begin{center}
\begin{tabular}{@{}ll@{}}
\hline
BIF & Returns \T{true} if and only if the argument is \\
\hline
\T{is_atom/1} & an atom \\
\T{is_binary/1} & a binary \\
\T{is_char/1} & a character \\
\T{is_compound/1} & of a compound type (cf.\ \S\ref{chapter:types-terms}) \\
\T{is_cons/1} & a cons \\
\T{is_float/1} & a float \\
\T{is_function/1} & a function term \\
\T{is_integer/1} & an integer \\
\T{is_list/1} & a (proper) list \\
\T{is_null/1} & nil \\
\T{is_number/1} & a number \\
\T{is_pid/1} & a PID \\
\T{is_port/1} & a port \\
\T{is_ref/1} & a ref \\
\T{is_string/1} & a (proper) string \\
\T{is_tuple/1} & a tuple \\
\hline
\end{tabular}
\caption{\Erlang\ recognizer BIFs.}
\label{table:typetests}
\end{center}
\end{table}
\fi	%\ifStd

\ifOld
Note that the name of the recognizer \T{list/1} is inaccurate: it does
\emph{not} test whether its argument is a list.  (It succeeds for all
lists but also for some terms that are not lists, such as \T{[a|b]}).

\index{function/1@\T{function/1}!recognizer|(}
\index{record/2@\T{record/2}!recognizer|(}
Note that although a \NT{GuardRecordTest} \T{record(\Z{E},\Z{R})}
\S\ref{section:record2} is not an application of a BIF, it is used as
a recognizer for records of a certain type.  In \OldErlang\ function
terms are implemented with tuples.
\index{function/1@\T{function/1}!recognizer|)}
\index{record/2@\T{record/2}!recognizer|)}
\fi

\ifOld
\index{atom/1 BIF@\T{atom/1} BIF|)}
\index{binary/1 BIF@\T{binary/1} BIF|)}
\index{constant/1 BIF@\T{constant/1} BIF|)}
\index{float/1 BIF@\T{float/1} BIF|)}
\index{function/1 BIF@\T{function/1} BIF|)}
\index{integer/1 BIF@\T{integer/1} BIF|)}
\index{list/1 BIF@\T{list/1} BIF|)}
\index{number/1 BIF@\T{number/1} BIF|)}
\index{pid/1 BIF@\T{pid/1} BIF|)}
\index{port/1 BIF@\T{port/1} BIF|)}
\index{record/2 BIF@\T{record/2} BIF|)}
\index{reference/1 BIF@\T{reference/1} BIF|)}
\index{tuple/1 BIF@\T{tuple/1} BIF|)}
\fi
\ifStd
\index{is_atom/1 BIF@\T{is_atom/1} BIF|)}
\index{is_binary/1 BIF@\T{is_binary/1} BIF|)}
\index{is_char/1 BIF@\T{is_char/1} BIF|)}
\index{is_cons/1 BIF@\T{is_cons/1} BIF|)}
\index{is_compound/1 BIF@\T{is_compound/1} BIF|)}
\index{is_float/1 BIF@\T{is_float/1} BIF|)}
\index{is_function/1 BIF@\T{is_function/1} BIF|)}
\index{is_integer/1 BIF@\T{is_integer/1} BIF|)}
\index{is_list/1 BIF@\T{is_list/1} BIF|)}
\index{is_null/1 BIF@\T{is_null/1} BIF|)}
\index{is_number/1 BIF@\T{is_number/1} BIF|)}
\index{is_pid/1 BIF@\T{is_pid/1} BIF|)}
\index{is_port/1 BIF@\T{is_port/1} BIF|)}
\index{is_ref/1 BIF@\T{is_ref/1} BIF|)}
\index{is_string/1 BIF@\T{is_string/1} BIF|)}
\index{is_tuple/1 BIF@\T{is_tuple/1} BIF|)}
\fi
\index{BIF!recognizer|)}

\section{Builtin functions on atoms}

\label{section:atom-bifs}
\index{atom!BIFs|(}

\subsection{\T{atom\char'137to\char'137list/1}}

\index{atom_to_list/1 BIF@\T{atom_to_list/1} BIF|(}

An atom is converted to a list of \ifOld characters\fi\ifStd character
codes\fi.

\TYPE

\T{atom_to_list(atom()) -> [int()]}.

\EXITS

Exits with cause \T{badarg} if $\TZ{v}_1$ is not an atom.

\EVALUATION

The BIF returns the printname of the atom $\TZ{v}_1$ represented by
a list of
\ifStd integers that are Unicode codes of a sequence of \fi
characters.

\EXAMPLES

\T{atom_to_list(foo)} \RETURNS\ \T{[102,111,111]}\ifOld, i.e., \T{"foo"}\fi; \\
\T{atom_to_list('')} \RETURNS\ \T{[]}\ifOld, i.e., \T{""}\fi; \\
\T{atom_to_list('T \char`\%@\char`\\'\char`\#')} \RETURNS\ \T{[84,32,37,64,39,35]}\ifOld,
i.e., \T{"T \char`\%@'\char`\#"}\fi; \\
\T{atom_to_list('456')} \RETURNS\ \T{[52,53,54]}\ifOld, i.e., \T{"456"}\fi; \\
\T{atom_to_list(456)} \EXITSWITH\ \T{\{badarg,\tdots\}}.
\index{atom_to_list/1 BIF@\T{atom_to_list/1} BIF|)}

\subsection{\T{list\char'137to\char'137atom/1}}

\index{list_to_atom/1 BIF@\T{list_to_atom/1} BIF|(}

A list of \ifOld characters \fi \ifStd character codes \fi is
converted to an atom.

\TYPE

\T{list_to_atom([int()]) -> atom()}.

\EXITS

Exits with cause \T{badarg} if $\TZ{v}_1$ is not a list of
\ifStd integers that are codes of Unicode \fi
characters.

\EVALUATION

The BIF returns the atom that has a printname consisting of the characters
\ifStd obtained by decoding the character codes \fi
in $\TZ{v}_1$.

\EXAMPLES

\T{list_to_atom([102,111,111])}\ifOld, i.e., \T{list_to_atom("foo")}\fi\ \RETURNS\ \T{foo}; \\
\T{list_to_atom([])}\ifOld, i.e., \T{list_to_atom("")}\fi\ \RETURNS\ \T{''}; \\
\T{list_to_atom([84,32,37,64,39,35])}\ifOld, i.e., \T{list_to_atom("T \char`\%@'\char`\#")}\fi\ \RETURNS\ \T{'T \char`\%@\char`\\'\char`\#'}; \\
\T{list_to_atom([52,53,54])}\ifOld, i.e., \T{list_to_atom("456")}\fi\ \RETURNS\ \T{'456'}; \\
\T{list_to_atom([102,-5,111])} \EXITSWITH\ \T{\{badarg,\tdots\}}.
\index{list_to_atom/1 BIF@\T{list_to_atom/1} BIF|)}
\index{atom!BIFs|)}

\section{Builtin arithmetic functions}
\index{number!BIFs|(}

\label{section:number-bifs}

\subsection{\T{abs/1}}

\index{abs/1 BIF@\T{abs/1} BIF|(}

The magnitude of a number is computed.  \T{abs/1} is a guard BIF.

\TYPE

\T{abs(int()) -> int() ; \\
abs(float()) -> float()}.

\EXITS

Exits with cause \T{badarg} if $\TZ{v}_1$ is not a number.
May exit with \T{integer_overflow} when $\TZ{v}_1$ is an integer (see below).

\EVALUATION

If $\TZ{v}_1$ is an integer, compute
$\mathit{abs}_I(\Er[\TZ{v}_1])$; otherwise $\TZ{v}_1$ is a float, compute
$\mathit{abs}_F(\Er[\TZ{v}_1])$. Let $r$ be the result.
If $r$ is a number, the the BIF returns $\Re[r]$;
otherwise the BIF exits with cause $\Re[r]$.

\EXAMPLES

\T{abs(42)} \RETURNS\ \T{42}; \\
\T{abs(-88)} \RETURNS\ \T{88}; \\
\T{abs(5.0)} \RETURNS\ \T{5.0}; \\
\T{abs(-0.1)} \RETURNS\ \T{0.1}; \\
\T{abs(whoopee)} \EXITSWITH\ \T{\{badarg,\tdots\}}.
\index{abs/1 BIF@\T{abs/1} BIF|)}

\ifStd
\subsection{\T{sign/1}}

\label{section:sign1}
\index{sign/1 BIF@\T{sign/1} BIF|(}

The sign of a number is computed.  The result is always
an integer.  \T{sign/1} is a guard BIF.

\TYPE

\T{sign(num()) -> int()}.

\EXITS

Exits with cause \T{badarg} if $\TZ{v}_1$ is not a number.

\EVALUATION

If $\TZ{v}_1$ is an integer, compute
$\mathit{sign}_I(\Er[\TZ{v}_1])$; otherwise $\TZ{v}_1$ is a float, compute
$\mathit{nearest}_{F\to I}(\mathit{sign}(\Er[\TZ{v}_1]))$. Let $r$ be the result.
If $r$ is an integer, the the BIF returns $\Re[r]$;
otherwise the BIF exits with cause $\Re[r]$.

\EXAMPLES

\T{sign(42)} \RETURNS\ \T{1}; \\
\T{sign(-88.56)} \RETURNS\ \T{-1}; \\
\T{sign(0.0)} \RETURNS\ \T{0}; \\
\T{sign(0)} \RETURNS\ \T{0}; \\
\T{sign(whoopee)} \EXITSWITH\ \T{\{badarg,\tdots\}}.
\index{sign/1 BIF@\T{sign/1} BIF|)}
\fi

\subsection{\T{float/1}}

\index{float/1 BIF@\T{float/1} BIF|(}

There are two BIFs named \T{float/1} and which one of them is denoted in an
application \T{float(\Z{E})} depends on the context in which the application appears.

If the application is a \NT{GuardRecognizer} expression
(\S\ref{section:guards}), then it is the guard test \T{float/1}
described in \S\ref{section:recognizer-bifs}; otherwise it is a
function converting numbers to floating-point numbers (which may
appear in guard expressions).  The following description is for the
latter case.

Both BIFs \T{float/1} are guard BIFs.

\TYPE

\T{float(num()) -> float()}.

\EXITS

Exits with cause \T{badarg} if $\TZ{v}_1$ is not a number.
May exit with \T{float_overflow} when $\TZ{v}_1$ is an integer (see below).

\EVALUATION

The evaluation depends on the type of $\TZ{v}_1$:
\begin{itemize}
\item If $\TZ{v}_1$ is a float, it is returned.
\item If $\TZ{v}_1$ is an integer, compute
$\mathit{cvt}_{I\to F}(\Er[\TZ{v}_1])$, let the result be $r$.
If $r$ is a float, the the BIF returns $\Re[r]$;
otherwise the BIF exits with cause $\Re[r]$.
\end{itemize}

\EXAMPLES

\T{float(3)} \RETURNS\ \T{3.0}; \\
\T{float(0)} \RETURNS\ \T{0.0}; \\
\T{float(123456789123456789123456789)} \RETURNS\ \T{1.23456789123457E26}\ifStd
(on an implementation with $r=2$ and $p=53$, cf.\ \S\ref{section:float-type})\fi; \\
\T{float(whoopee)} \EXITSWITH\ \T{\{badarg,\tdots\}}.
\index{float/1 BIF@\T{float/1} BIF|)}

\subsection{\T{float\char'137to\char'137list/1}}

\label{section:floattolist1}
\index{float_to_list/1 BIF@\T{float_to_list/1} BIF|(}

The function produces a list of \ifOld characters \fi
\ifStd integers that are character codes \fi
of a printed representation of a float.

\TYPE

\T{float_to_list(float()) -> [int()]}.

\EXITS

Exits with cause \T{badarg} if $\TZ{v}_1$ is not a float (there is thus no coercion
from integers).

\EVALUATION

The BIF returns the list of
\ifStd integers that are Unicode codes of the sequence of \fi
characters which is like the canonical decimal numeral for $\Er[\TZ{v}_1]$
(\S\ref{section:float-to-numeral}) except that:
\begin{itemize}
\item There are exactly 20 digits between the decimal point and the `e'.
The right end is adjusted by dropping digits or padding with zeroes.
\item If the character after `e' is not a minus sign, then a plus sign is inserted.
\item If the exponent part (i.e., the digits after `e') has only one digit,
a zero is inserted before that digit.
\end{itemize}

\EXAMPLES

\T{float_to_list(-0.00672)} \RETURNS\ \\
\T{[45,54,46,55,50,48,48,48,48,48,48,48,48,48,48,48,48,48,50,55,56,54,55,101,45,48,51]}\ifOld, i.e.,
\T{"-6.72000000000000027867e-03"}\fi; \\
\T{float_to_list(13e4)} \RETURNS\ \\
\T{[49,46,51,48,48,48,48,48,48,48,48,48,48,48,48,48,48,48,48,48,48,48,101,43,48,53]}\ifOld, i.e.,
\T{"1.30000000000000000000e+05"}\fi; \\
\T{float_to_list(0.0)} \RETURNS\ \\
\T{[48,46,48,48,48,48,48,48,48,48,48,48,48,48,48,48,48,48,48,48,48,48,101,43,48,48]}\ifOld, i.e.,
\T{"0.00000000000000000000e+00"}\fi; \\
\T{float_to_list(636.9e121)} \RETURNS\ \\
\T{[54,46,51,54,56,57,57,57,57,57,57,57,57,57,57,57,57,53,52,53,53,51,101,43,49,50,51]}\ifOld, i.e.,
\T{"6.36899999999999954553e+123"}\fi; \\
\T{float_to_list(whoopee)} \EXITSWITH\ \T{\{badarg,\tdots\}}.
\index{float_to_list/1 BIF@\T{float_to_list/1} BIF|)}

\subsection{\T{integer\char'137to\char'137list/1}}

\label{section:integertolist1}
\index{integer_to_list/1 BIF@\T{integer_to_list/1} BIF|(}

The function produces a list of \ifOld characters \fi
\ifStd integers that are character codes \fi
of a printed representation of an integer.

\TYPE

\T{integer_to_list(int()) -> [int()]}.

\EXITS

Exits with cause \T{badarg} if $\TZ{v}_1$ is not an integer.

\EVALUATION

The BIF returns the list of
\ifStd integers that are Unicode codes of the sequence of \fi
characters making up the canonical decimal numeral for $\Er[\TZ{v}_1]$
(\S\ref{section:integer-to-numeral}).

\EXAMPLES

\T{integer_to_list(0)} \RETURNS\ \T{[48]}\ifOld, i.e., \T{"0"}\fi; \\
\T{integer_to_list(42)} \RETURNS\ \T{[52,50]}\ifOld, i.e., \T{"42"}\fi; \\
\T{integer_to_list(-39)} \RETURNS\ \T{[45,51,57]}\ifOld, i.e., \T{"-39"}\fi; \\
\T{integer_to_list(whoopee)} \EXITSWITH\ \T{\{badarg,\tdots\}}.
\index{integer_to_list/1 BIF@\T{integer_to_list/1} BIF|)}

\subsection{\T{list\char'137to\char'137float/1}}

\label{section:listtofloat1}
\index{list_to_float/1 BIF@\T{list_to_float/1} BIF|(}

A float is obtained from a float literal represented by a list of
\ifStd integers that are Unicode codes of \fi the characters of the literal.

\TYPE

\T{list_to_float([int()]) -> float()}.

\EXITS

Exits with cause \T{badarg} if $\TZ{v}_1$ is not a list of
\ifStd integers that are the codes of Unicode \fi
characters making up a float numeral, or if the number
it represents is not representable as a float (see below).
Leading or trailing spaces are not permitted.

\EVALUATION

If the sequence of characters denotes a number $f\in F$
(\S\ref{section:numeral-to-float}), then
$\Re[f]$ is returned, otherwise the BIF exits with cause \T{badarg}.

\EXAMPLES

\T{list_to_float([45,49,55,46,53,48,48])}\ifOld, i.e., \\
\T{list_to_float("-17.500")}\fi\ \RETURNS\ \T{-17.5}; \\
\T{list_to_float([49,50,51,46,52,53,101,56,55])}\ifOld, i.e., \\
\T{list_to_float("123.45e87")}\fi\ \RETURNS\ \T{1.23450e89}; \\
\T{list_to_float([54,50,46,53,69,45,51])}\ifOld, i.e., \\
\T{list_to_float("62.5E-3")}\fi\ \RETURNS\ \T{0.0625}; \\
\T{list_to_float([54,50,53,101,45,52])}\ifOld, i.e., \\
\T{list_to_float("625e-4")}\fi\ \EXITSWITH\ \T{\{badarg,\tdots\}}; \\
\T{list_to_float(whoopee)} \EXITSWITH\ \T{\{badarg,\tdots\}}.
\index{list_to_float/1 BIF@\T{list_to_float/1} BIF|)}

\subsection{\T{list\char'137to\char'137integer/1}}

\label{section:listtointeger1}
\index{list_to_integer/1 BIF@\T{list_to_integer/1} BIF|(}

An integer is obtained from an integer literal represented by a list of
\ifStd integers that are Unicode codes of \fi the characters of the literal.

\TYPE

\T{list_to_integer([int()]) -> integer()}.

\EXITS

Exits with cause \T{badarg} if $\TZ{v}_1$ is not a list of
\ifStd integers that are the codes of Unicode \fi
characters making up a decimal integer numeral, or if the
integer it represents is too large to be represented (see below).
Leading or trailing spaces are not permitted.

\EVALUATION

If the sequence of characters denotes an integer $i\in I$
(\S\ref{section:numeral-to-integer}), then
$\Re[i]$ is returned, otherwise the BIF exits with cause \T{badarg}.

\EXAMPLES

\T{list_to_integer([52,50])}\ifOld, i.e., \T{list_to_integer("42")}\fi\ \RETURNS\ \T{42}; \\
\T{list_to_integer([48,48,48,48,48,48,53,54,55])}\ifOld, i.e., \T{list_to_integer("000000567")}\fi\ \RETURNS\ \T{567}; \\
\T{list_to_integer([45,51,57])}\ifOld, i.e., \T{list_to_integer("-39")}\fi\ \RETURNS\ \T{-39}; \\
\T{list_to_integer([111,105,110,107])}\ifOld, i.e., \T{list_to_integer("oink")}\fi\ \EXITSWITH\ \T{\{badarg,\tdots\}}; \\
\T{list_to_integer(whoopee)} \EXITSWITH\ \T{\{badarg,\tdots\}}.
\index{list_to_integer/1 BIF@\T{list_to_integer/1} BIF|)}

\subsection{\T{round/1}}

\index{round/1 BIF@\T{round/1} BIF|(}

The integer closest to a given number is computed.
\T{round/1} is a guard BIF.

\TYPE

\T{round(num()) -> int()}.

\EXITS

Exits with cause \T{badarg} if $\TZ{v}_1$ is not a number.
May exit with \T{integer_overflow} when $\TZ{v}_1$ is a float (see below).

\EVALUATION

The evaluation depends on the type of $\TZ{v}_1$:
\begin{itemize}
\item If $\TZ{v}_1$ is an integer, it is returned.
\item If $\TZ{v}_1$ is a float, compute
$\mathit{nearest}_{F\to I}(\Er[\TZ{v}_1])$, let the result be $r$.
If $r$ is an integer, the the BIF returns $\Re[r]$;
otherwise the BIF exits with cause $\Re[r]$.
\end{itemize}

\EXAMPLES

\T{round(42)} \RETURNS\ \T{42}; \\
\T{round(88.56)} \RETURNS\ \T{89}; \\
\T{round(-88.56)} \RETURNS\ \T{-89}; \\
\T{round(0.0)} \RETURNS\ \T{0}; \\
\T{round(whoopee)} \EXITSWITH\ \T{\{badarg,\tdots\}}.
\index{round/1 BIF@\T{round/1} BIF|)}

\subsection{\T{trunc/1}}

\index{trunc/1 BIF@\T{trunc/1} BIF|(}

The first float between a given number and zero is computed.
\T{trunc/1} is a guard BIF.

\TYPE

\T{trunc(num()) -> int()}.

\EXITS

Exits with cause \T{badarg} if $\TZ{v}_1$ is not a number.
May exit with \T{integer_overflow} when $\TZ{v}_1$ is a float (see below).

\EVALUATION

The evaluation depends on the type of $\TZ{v}_1$:
\begin{itemize}
\item If $\TZ{v}_1$ is an integer, it is returned.
\item If $\TZ{v}_1$ is a float, compute
$\mathit{truncate}_{F\to I}(\Er[\TZ{v}_1])$, let the result be $r$.
If $r$ is an integer, the the BIF returns $\Re[r]$;
otherwise the BIF exits with cause $\Re[r]$.
\end{itemize}

\EXAMPLES

\T{trunc(42)} \RETURNS\ \T{42}; \\
\T{trunc(88.56)} \RETURNS\ \T{88}; \\
\T{trunc(-88.56)} \RETURNS\ \T{-88}; \\
\T{trunc(0.0)} \RETURNS\ \T{0}; \\
\T{trunc(whoopee)} \EXITSWITH\ \T{\{badarg,\tdots\}}.
\index{trunc/1 BIF@\T{trunc/1} BIF|)}
\index{number!BIFs|)}

\section{Builtin functions on binaries}

\label{section:binary-bifs}
\index{binary!BIFs|(}

Binaries are described in \S\ref{section:binaries}.

\subsection{\T{binary\char'137to\char'137list/1}}
\index{binary_to_list/1 BIF@\T{binary_to_list/1} BIF|(}

A list of bytes that are the elements of a binary, in the same order
as in the binary, is returned.

\TYPE

\T{binary_to_list(bin()) -> [int()]}.

\EXITS

\T{binary_to_list/1} exits with cause \T{badarg} if $\TZ{v}_1$ is not a binary.

\EVALUATION

Suppose that $\TZ{v}_1$ is a binary consisting of the bytes
$\TZ{I}_1$, \ldots, $\TZ{I}_k$.
A list \T{[$\TZ{I}_1$,\tdots,$\TZ{I}_k$]} is returned.
The time for computing the answer should be $O(k)$.
\index{binary_to_list/1 BIF@\T{binary_to_list/1} BIF|)}

\subsection{\T{binary\char'137to\char'137list/3}}
\index{binary_to_list/3 BIF@\T{binary_to_list/3} BIF|(}

A list of integers that are part of the elements of a binary, in the
same order as in the binary, is returned.

\TYPE

\T{binary_to_list(bin(),int(),int()) -> [int()]}.

\EXITS

% Note:
% the proposed binary:to_list takes different arguments.

\T{binary_to_list/3} exits with cause \T{badarg} if $\TZ{v}_1$ is not a binary,
if $\TZ{v}_2$ or $\TZ{v}_3$ is not an integer.
It exits with cause \ifOld \T{badarg} \fi \ifStd \T{\badindex} \fi
if the integers represented
by $\TZ{v}_2$ and $\TZ{v}_3$ are out of range (see below).

\EVALUATION

Suppose that $\TZ{v}_1$ is a binary consisting of the bytes
$\TZ{I}_1$, \ldots, $\TZ{I}_k$ and that $i=\Er[\TZ{v}_2]$ and $j=\Er[\TZ{v}_3]$.
\T{binary_to_list/2} exits with cause \T{badarg} if $i<1$, $j<i$ or $j>k$.
% Should change so j=i-1 or even any j for which j<i is permitted???
A list \T{[$\TZ{I}_i$,\tdots,$\TZ{I}_j$]} is returned.
% Note: This is bad bad bad bad!
The result is thus always a nonempty list.

\iffalse
The list returned depends on $i$ and $j$:
\begin{itemize}
\item If $i\leq j$, then
a list \T{[$\TZ{I}_i$,\tdots,$\TZ{I}_j$]} is returned.
\item If $i>j$, then an empty list is returned.
\end{itemize}
\fi

The time for computing the result should be $O(j)$.
\index{binary_to_list/3 BIF@\T{binary_to_list/3} BIF|)}

\subsection{\T{binary\char'137to\char'137term/1}}

\label{section:binarytoterm1}
\index{binary_to_term/1 BIF@\T{binary_to_term/1} BIF|(}
\index{term!external format|(}

Given a binary that is a representation in the external format
(\S\ref{chapter:external-format}) of a term, that term is returned.

\TYPE

\T{binary_to_term(bin()) -> term()}.

\EXITS

\T{binary_to_term/1} exits with cause \T{badarg} if $\TZ{v}_1$ is not a binary.
It also exits with cause \T{badarg} if the elements of $\TZ{v}_1$ are not
the external format representation of some term (see below).

\EVALUATION

If the bytes that are the elements of $\TZ{v}_1$ constitute a
representation in the external format
(\S\ref{chapter:external-format}) of some term \TZ{t}, then \TZ{t} is
returned; otherwise \T{binary_to_term/1} exits with cause \T{badarg}.

% Is this true???
The time for computing the result should be $O(k)$.
\index{binary_to_term/1 BIF@\T{binary_to_term/1} BIF|)}
\index{term!external format|)}

\subsection{\T{concat\char'137binary/1}}

\index{concat_binary/1 BIF@\T{concat_binary/1} BIF|(}

Given a list of binaries, one binary is returned which has
as elements the elements of the binaries in the list, in the same order as
in which the elements appear in the binaries and the binaries in the list.

\TYPE

\T{concat_binary([bin()]) -> bin()}.

\EXITS

\T{concat_binary/1} exits with cause \T{badarg} if $\TZ{v}_1$ is not a
list of binaries.

\EVALUATION

Suppose that $\TZ{v}_1$ is a list
\T{[$\T{B}_1$,\tdots,$\Z{B}_k$]}, such that each
$\TZ{B}_i$, $1\leq i\leq k$, is a binary consisting of the bytes
$\TZ{b}_{i,1}$, \ldots, $\TZ{b}_{i,n_i}$.
A binary consisting of the bytes
$\TZ{b}_{1,1}$, \ldots, $\TZ{b}_{1,n_1}$, \ldots,
$\TZ{b}_{k,1}$, \ldots, $\TZ{b}_{k,n_k}$
is returned.

The time for computing the result should be $O(\max(\sum_{i=1}^{k}n_i,k))$.
\index{concat_binary/1 BIF@\T{concat_binary/1} BIF|)}

\subsection{\T{list\char'137to\char'137binary/1}}

\index{list_to_binary/1 BIF@\T{list_to_binary/1} BIF|(}

Given a list of bytes, a binary consisting of the same bytes in the
same order is returned.

\TYPE

\T{list_to_binary([int()]) -> bin()}.

\EXITS

\T{list_to_binary/1} exits with cause \T{badarg} if $\TZ{v}_1$ is not a list of bytes.

\EVALUATION

Suppose that $\TZ{v}_1$ is a list of bytes \T{[$\Z{I}_1$,\tdots,$\Z{I}_k$]}.
A binary consisting of the bytes $\TZ{I}_1$, \ldots, $\TZ{I}_k$ is returned.

The time for computing the result should be $O(k)$.
\index{list_to_binary/1 BIF@\T{list_to_binary/1} BIF|)}

\subsection{\T{size/1}}

\index{size/1 BIF@\T{size/1} BIF|(}
See \S\ref{section:size1}.
\index{size/1 BIF@\T{size/1} BIF|)}

\subsection{\T{split\char'137binary/2}}

\label{section:splitbinary2}
\index{split_binary/2 BIF@\T{split_binary/2} BIF|(}

A binary is split into two binaries, where the number of elements in
the first binary is given.

\TYPE

\T{split_binary(bin(),int()) -> \{bin(),bin()\}}.

\EXITS

\T{split_binary/2} exits with cause \T{badarg} if $\TZ{v}_1$ is not a binary or
$\TZ{v}_2$ is not an integer.  It also exits with cause
\ifOld \T{badarg} \fi \ifStd \T{\badindex} \fi
if the number represented by $\TZ{v}_2$
is out of range (see below).

\EVALUATION

Suppose that $\TZ{v}_1$ is a binary consisting of the bytes
$\TZ{I}_1$, \ldots, $\TZ{I}_k$ and that $i=\Er[\TZ{v}_2]$.
The evaluation depends on $i$ and $k$:
\begin{itemize}
\item If $i<0$ or $i>k$, then \T{split_binary/2} exits with cause
\ifOld \T{badarg}\fi \ifStd \T{\badindex}\fi.
\item Otherwise a 2-tuple of binaries \T{\{$\Z{B}_l$,$\Z{B}_r$\}} is returned,
where $\TZ{B}_l$ consists of the bytes $\TZ{I}_1$, \ldots, $\TZ{I}_i$
and $\TZ{B}_r$ consists of the bytes $\TZ{I}_{i+1}$, \ldots, $\TZ{I}_k$.
(If $i=0$ then $\TZ{B}_l$ is an empty binary and $\TZ{B}_r$ equals
$\TZ{v}_1$, while if $i=k$ then $\TZ{B}_l$ equals
$\TZ{v}_1$ and $\TZ{B}_r$ is an empty binary.)
\end{itemize}
The time for computing the answer should be $O(1)$.
\index{split_binary/2 BIF@\T{split_binary/2} BIF|)}

\subsection{\T{term\char'137to\char'137binary/1}}

\label{section:termtobinary1}
\index{term_to_binary/1 BIF@\T{term_to_binary/1} BIF|(}
\index{term!external format|(}

Given a term, a binary having elements that represent the term in the external format
(\S\ref{chapter:external-format}) is returned.

\TYPE

\T{term_to_binary(term()) -> bin()}.

\EXITS

\ifStd
\T{term_to_binary/1} exits with cause \T{badarg} if $\TZ{v}_1$ is not a
term for which the implementation provides an external
representation.\footnote{The only case for which this is allowed to happen is
if $\TZ{v}_1$ contains a function.}
\fi
\ifOld
\T{term_to_binary/1} always completes normally.
\fi

\EVALUATION

Let the representation in the external format
(\S\ref{chapter:external-format}) of $\TZ{v}_1$ be
the sequence of bytes
$\TZ{I}_1$, \ldots, $\TZ{I}_k$.  A binary consisting
of these bytes, in that order, is returned.
% too strong???
The time for computing the answer should be $O(k)$.
\index{term_to_binary/1 BIF@\T{term_to_binary/1} BIF|)}
\index{term!external format|)}
\index{binary!BIFs|)}

\section{Builtin functions on tuples}

\label{section:tuple-bifs}
\index{tuple!BIFs|(}

Tuples are described in \S\ref{section:tuples}.

\subsection{\T{element/2}}

\index{element/2 BIF@\T{element/2} BIF|(}

One element of a tuple is returned.
\T{element/2} is a guard BIF.

\TYPE

\T{element(int(),tuple()) -> term()}.

\EXITS

\T{element/2} exits with cause \T{badarg} if $\TZ{v}_1$ is not an integer or
$\TZ{v}_2$ is not a tuple. It also exits with cause
\ifOld \T{badarg} \fi \ifStd \T{\badindex} \fi
if the number represented by $\TZ{v}_1$ is out of range (see below).

\EVALUATION

Suppose that $i=\Er[\TZ{v}_1]$ and that $\TZ{v}_2$
is a tuple with elements $\TZ{T}_1$, \ldots, $\TZ{T}_k$.
The evaluation depends on $i$ and $k$:
\begin{itemize}
\item If $i<1$ or $i>k$, then \T{element/2} exits with
cause \ifOld \T{badarg}\fi \ifStd \T{\badindex}\fi.
\item Otherwise, $\TZ{T}_i$ is returned.
\end{itemize}
The time for computing the result should be $O(1)$.
\index{element/2 BIF@\T{element/2} BIF|)}

\subsection{\T{list\char'137to\char'137tuple/1}}

\label{section:listtotuple1}
\index{list_to_tuple/1 BIF@\T{list_to_tuple/1} BIF|(}

A tuple with the same elements as a given list is returned.

\TYPE

\T{list_to_tuple([term()]) -> tuple()}.

\EXITS

\T{list_to_tuple/1} exits with cause \T{badarg} if $\TZ{v}_1$ is not a list.

\EVALUATION

Let $\TZ{v}_1$ be a list with elements
$\TZ{T}_1$, \ldots, $\TZ{T}_k$.  A tuple
\T{\{$\Z{T}_1$,\tdots,$\Z{T}_k$\}} is returned.
The time for computing the answer should be $O(k)$.
\index{list_to_tuple/1 BIF@\T{list_to_tuple/1} BIF|)}

\subsection{\T{setelement/3}}

\index{setelement/3 BIF@\T{setelement/3} BIF|(}

A tuple is computed that differs from a given tuple in exactly one element.

\TYPE

\T{setelement(int(),tuple(),term()) -> tuple()}.

\EXITS

\T{setelement/3} exits with cause \T{badarg} if $\TZ{v}_1$ is not an integer or
$\TZ{v}_2$ is not a tuple. It also exits with cause
\ifOld \T{badarg} \fi \ifStd \T{\badindex} \fi
if the number represented by $\TZ{v}_1$ is out of range (see below).

\EVALUATION

Suppose that $i=\Er[\TZ{v}_1]$ and that $\TZ{v}_2$
with elements $\TZ{T}_1$, \ldots, $\TZ{T}_k$.
The evaluation depends on $i$ and $k$:
\begin{itemize}
\item If $i<1$ or $i>k$, then \T{setelement/3} exits with
cause \ifOld \T{badarg}\fi \ifStd \T{\badindex}\fi.
\item Otherwise, a tuple
\T{\{$\Z{T}_1$,\tdots,$\Z{T}_{i-1}$,$\TZ{v}_3$,$\Z{T}_{i+1}$,$\Z{T}_k$\}}
is returned.  (That is, a tuple that is exactly like $\TZ{v}_2$ except
that the element at position $i$ is $\TZ{v}_3$.)
\end{itemize}
The time for computing the answer should be $O(k)$.  (This is not a
destructive operation so the tuple given as argument must not be observably affected.)
\index{setelement/3 BIF@\T{setelement/3} BIF|)}

\subsection{\T{size/1}}

\label{section:size1}
\index{size/1 BIF@\T{size/1} BIF|(}

The number of elements of a binary or a tuple is returned.  \T{size/1}
is a guard BIF.

\TYPE

\T{size(bin()) -> int()} ; \\
\T{size(tuple()) -> int()}.

\EXITS

\T{size/1} exits with cause \T{badarg} if $\TZ{v}_1$ is neither a binary, nor a tuple.

\EVALUATION

The integer $\Re[k]$ is returned, where $k$ is the number of elements in the binary
or tuple $\TZ{v}_1$.
The time for computing the answer should be $O(1)$.
\index{size/1 BIF@\T{size/1} BIF|)}

\subsection{\T{tuple\char'137to\char'137list/1}}

\index{tuple_to_list/1 BIF@\T{tuple_to_list/1} BIF|(}

A list with the same elements as a given tuple is returned.

\TYPE

\T{tuple_to_list(tuple()) -> [term()]}.

\EXITS

\T{tuple_to_list/1} exits with cause \T{badarg} if $\TZ{v}_1$ is not a tuple.

\EVALUATION

Let $\TZ{v}_1$ be a tuple with elements
$\TZ{T}_1$, \ldots, $\TZ{T}_k$.  A list
\T{[$\Z{T}_1$,\tdots,$\Z{T}_k$]} is returned.
The time for computing the result should be $O(k)$.
\index{tuple_to_list/1 BIF@\T{tuple_to_list/1} BIF|)}
\index{tuple!BIFs|)}

\section{Builtin functions on lists and conses}

\label{section:list-bifs}
\index{list!BIFs|(}
\index{cons!BIFs|(}

Lists and conses are described in \S\ref{section:lists}.

\subsection{\T{hd/1}}

\index{hd/1 BIF@\T{hd/1} BIF|(}

The head of a cons, e.g., the first element of a list, is returned.
\T{hd/1} is a guard BIF.

\TYPE

If cons is used as intended, the type is
\begin{textdisplay}
\T{hd([T]) -> T}.
\end{textdisplay}
If cons is used as a general pairing operator, the type is instead
\begin{textdisplay}
\T{hd([T|_]) -> T}.
\end{textdisplay}

\EXITS

\T{hd/1} exits with cause \T{badarg} if $\TZ{v}_1$ is not a cons.

\EVALUATION

The head of the cons $\TZ{v}_1$ is returned.
The time for computing the result should be $O(1)$.
\index{hd/1 BIF@\T{hd/1} BIF|)}

\subsection{\T{length/1}}

\index{length/1 BIF@\T{length/1} BIF|(}

The number of elements of a list is returned.
\T{length/1} is a guard BIF.

\TYPE

\T{length([term()]) -> int()}.

\EXITS

\T{length/1} exits with cause \T{badarg} if $\TZ{v}_1$ is not a list.

\EVALUATION

The integer $\Re[k]$ is returned, where $k$ is the number of elements
in the list.
The time for computing the answer should be $O(1)$.
\index{length/1 BIF@\T{length/1} BIF|)}

\subsection{\T{tl/1}}

\index{tl/1 BIF@\T{tl/1} BIF|(}

The tail of a cons, e.g., all but the first element of a list, is
returned.  \T{tl/1} is a guard BIF.

\TYPE

If cons is used as intended, the type is
\begin{textdisplay}
\T{tl([T]) -> [T]}.
\end{textdisplay}
If cons is used as a general pairing operator, the type is instead
\begin{textdisplay}
\T{tl([_|T]) -> T}.
\end{textdisplay}

\EXITS

\T{tl/1} exits with cause \T{badarg} if $\TZ{v}_1$ is not a cons.

\EVALUATION

The tail of the cons $\TZ{v}_1$ is returned.
The time for computing the result should be $O(1)$.
\index{tl/1 BIF@\T{tl/1} BIF|)}
\index{list!BIFs|)}
\index{cons!BIFs|)}

\iffalse
% There are such functions in \StdErlang\ but in a separate module.
\section{Builtin functions on strings}

\label{section:string-bifs}

TO BE DECIDED.
\fi

\iffalse
% It could be that ord and chr will be provided in \OldErlang.
\section{Builtin functions on characters}

\label{section:char-bifs}

ord/1
chr/1

TO BE DECIDED.
\fi

\section{Builtin functions for modules}

\label{section:module-bifs}
\index{module!BIFs|(}

The syntax of modules is described in \S\ref{section:module-declarations}
and their dynamics in \S\ref{chapter:module-dynamics}.

The BIFs in this section are not designed to be used directly in applications.
Rather,
\ifStd
a \StdErlang\ implementation is expected to use them for providing a more
\fi
\ifOld
they are provided for implementing more
\fi
high-level interface to dynamic loading and replacement
of modules.\footnote{Cf.\ the \T{code} module of
OTP \cite[pp.~158--167]{otp-dev-ref}.}

\subsection{\T{erlang:check_process_code/2}}

\label{section:checkprocesscode2}
\index{erlang:check_process_code/2 BIF@\T{erlang:check_process_code/2} BIF|(}
\index{check_process_code/2 BIF@\T{check_process_code/2} BIF|(}

A check is made whether a particular process is using a given module
(\S\ref{section:process-using-module}).

\TYPE

\T{erlang:check_process_code(pid(),atom()) -> bool()}.

\EXITS

\T{erlang:check_process_code/2} exits with cause \T{badarg} if $\TZ{v}_1$ is not a pid
or $\TZ{v}_2$ is not an atom.\ifOld\footnote{\OldErlang\ actually allows $\TZ{v}_2$
to be anything and always returns \T{false} if $\TZ{v}_2$ is not an atom.}\fi

\EVALUATION

Let \TZ{N} be \T{node[$\TZ{v}_1$]}.
\begin{itemize}
\item If module_table[\Z{N}] contains a row with $\TZ{v}_2$ as key and \TZ{R} as value,
\T{old_version[\TZ{R}]} is not \T{none} and 
a reference to \T{old_version[\TZ{R}]} is found in \T{stack_trace[\Z{P}]},
then the BIF returns \T{true} (cf.\ \S\ref{section:checking-process-module}).
\item Otherwise, it returns \T{false}.
\end{itemize}
\index{erlang:check_process_code/2 BIF@\T{erlang:check_process_code/2} BIF|)}
\index{check_process_code/2 BIF@\T{check_process_code/2} BIF|)}

\subsection{\T{erlang:delete_module/1}}

\label{section:deletemodule1}
\index{erlang:delete_module/1 BIF@\T{erlang:delete_module/1} BIF|(}
\index{delete_module/1 BIF@\T{delete_module/1} BIF|(}

The current version of a module is changed to be the old version
(\S\ref{section:replacing-module}).

\TYPE

\T{erlang:delete_module(atom()) -> atom()}.

\EXITS

\T{erlang:delete_module/1} exits with cause \T{badarg} if $\TZ{v}_1$ is not
an atom.  Moreover, it will exit with cause \T{badarg} % bad choice!!!
if there is already an old version of the module (see below)

\EVALUATION

Let \TZ{P} be the process calling \T{delete_module/1} and
let \TZ{N} be \T{node[\Z{P}]}.
\begin{itemize}
% WEIRD!!! it should look for a current version!
\item If there is no row with key $\TZ{v}_1$ in \T{module_table[\Z{N}]},
then \T{undefined} is returned.\footnote{This is contrary to the reasonable intuition
that the BIF should return \T{undefined} if, and only if, \T{module_loaded/1}
returns \T{false} for the module.}
Otherwise, let \TZ{R} be the value for $\TZ{v}_1$ in \T{module_table[\Z{N}]}.
\item If \T{old_version[\Z{R}]} is not \T{none},
the BIF exits with cause \T{badarg}. % yeah, bad choice!!!
% I think this one should be checked first
\item Otherwise, if \T{current_version[\Z{R}]} is not \T{none},
it is made the old version of module $\TZ{v}_1$ as described in
\S\ref{section:making-old-version} and \T{true} is returned.
\item Otherwise, no action is taken and \T{true} is returned.
\end{itemize}
\index{erlang:delete_module/1 BIF@\T{erlang:delete_module/1} BIF|)}
\index{delete_module/1 BIF@\T{delete_module/1} BIF|)}

\subsection{\T{erlang:load_module/2}}

\label{section:loadmodule2}
\index{erlang:load_module/2 BIF@\T{erlang:load_module/2} BIF|(}
\index{load_module/2 BIF@\T{load_module/2} BIF|(}

A compiled module is loaded as the current version of the module
(\S\ref{section:loading}).
If there is already a current version of the module, it is made the
old version of the module.

\TYPE

\T{erlang:load_module(atom(),bin()) -> atom()}.

\EXITS

\T{erlang:load_module/2} exits with cause \T{badarg} if $\TZ{v}_1$ is not
an atom or $\TZ{v}_2$ is not a binary.

\EVALUATION

Let \TZ{P} be the process calling \T{load_module/2} and
let \TZ{N} be \T{node[\Z{P}]}.
\begin{itemize}
\item If the binary $\TZ{v}_2$ does not contain compiled code for a module
named $\TZ{v}_1$, the BIF returns a tuple \T{\{error, badfile\}}.
\item Otherwise, the following is done:

First, if \T{module_table[\Z{N}]} contains no row with $\TZ{v}_1$ as
key, a row is added with $\TZ{v}_1$ as key and \TZ{R} as value, such that
\T{old_version[\Z{R}]} and \T{current_version[\Z{R}]]} are both \T{none}.

Otherwise, let \TZ{R} be the value of the row with $\TZ{v}_1$ as key;
if neither of \T{old_version[\Z{R}]} and \T{current_version[\Z{R}]]}
is \T{none}, the BIF returns a tuple \T{\{error, not_purged\}}.

Next, if \T{current_version[\Z{R}]} is not \T{none},
it is made the old version of module $\TZ{v}_1$ on node \TZ{N} as described in
\S\ref{section:making-old-version}.

Finally, the binary $\TZ{v}_2$ is made the current version of module
$\TZ{v}_1$ on node \TZ{N} as described in
\S\ref{section:making-current-version}.
\end{itemize}
\index{erlang:load_module/2 BIF@\T{erlang:load_module/2} BIF|)}
\index{load_module/2 BIF@\T{load_module/2} BIF|)}

\subsection{\T{erlang:preloaded/0}}

\label{section:preloaded0}
\index{erlang:preloaded/0 BIF@\T{erlang:preloaded/0} BIF|(}
\index{preloaded/0 BIF@\T{preloaded/0} BIF|(}

A list is returned of the names of all modules that were loaded
as part of starting the current node.

\TYPE

\T{erlang:preloaded() -> [atom()]}.

\EXITS

\T{erlang:preloaded/0} always completes normally.

\EVALUATION

The BIF returns a list representing the value of
\T{preloaded[node[\Z{P}]]}, where \TZ{P} is the
process calling the BIF.
\index{erlang:preloaded/0 BIF@\T{erlang:preloaded/0} BIF|)}
\index{preloaded/0 BIF@\T{preloaded/0} BIF|)}

\subsection{\T{erlang:purge_module/1}}

\label{section:purgemodule1}
\index{erlang:purge_module/1 BIF@\T{erlang:purge_module/1} BIF|(}
\index{purge_module/1 BIF@\T{purge_module/1} BIF|(}

The old version of a module is purged (\S\ref{section:replacing-module}).

\TYPE

\T{erlang:purge_module(atom()) -> true}.

\EXITS

\T{erlang:purge_module/1} exits with cause \T{badarg} if $\TZ{v}_1$ is not
an atom.  Moreover, it will exit with cause \T{badarg} % bad choice!!!
if there is no old version of the module (see below)

\EVALUATION

Let \TZ{P} be the process calling \T{purge_module/1} and
let \TZ{N} be \T{node[\Z{P}]}.
\begin{itemize}
\item If there is a row in \T{module_table[\Z{N}]} with $\TZ{v}_1$ as key and 
some \TZ{R} as value, and \T{old_version[\TZ{R}]} is not \T{none}, then
\T{old_version[\TZ{R}]} is purged as described in
\S\ref{section:purging-old-version} and \T{true} is returned.
\item Otherwise, the BIF exits with cause \T{badarg}. % yuck, bad choice!!!
\end{itemize}
\index{erlang:purge_module/1 BIF@\T{erlang:purge_module/1} BIF|)}
\index{purge_module/1 BIF@\T{purge_module/1} BIF|)}

\subsection{\T{erlang:module_loaded/1}}

\label{section:moduleloaded1}
\index{erlang:module_loaded/1 BIF@\T{erlang:module_loaded/1} BIF|(}
\index{module_loaded/1 BIF@\T{module_loaded/1} BIF|(}

It is found out whether there is a current version (\S\ref{section:current-version}) of some module.

\TYPE

\T{erlang:module_loaded(atom()) -> bool()}.

\EXITS

\T{erlang:module_loaded/1} exits with cause \T{badarg} if $\TZ{v}_1$ is not
an atom.

\EVALUATION

Let \TZ{P} be the process calling \T{module_loaded/1} and
let \TZ{N} be \T{node[\Z{P}]}.
\begin{itemize}
\item If there is a row in \T{module_table[\Z{N}]} with $\TZ{v}_1$ as key and 
some \TZ{R} as value, and \T{current_version[\TZ{R}]} is not \T{none}, then
the BIF returns \T{true}.
\item Otherwise, the BIF returns \T{false}.
\end{itemize}
\index{erlang:module_loaded/1 BIF@\T{erlang:module_loaded/1} BIF|)}
\index{module_loaded/1 BIF@\T{module_loaded/1} BIF|)}
\index{module!BIFs|)}

\section{Builtin functions for functions and processes}

\label{section:process-bifs}
\index{function!BIFs|(}
\index{process!BIFs|(}

\subsection{\T{apply/2}}

\label{section:apply2}
\index{apply/2 BIF@\T{apply/2} BIF|(}
\index{function!application|(}

A given function is applied to a sequence of arguments.

\TYPE

\T{apply(\{atom(),atom()\},[term()]) -> term()} ; \\
\T{apply(function(),[term()]) -> term()}

\EXITS

\T{apply/2} exits with cause \T{badarg} if $\TZ{v}_1$ is neither a 2-tuple of atoms,
nor a function,
or $\TZ{v}_2$ is not a list. \T{apply/2} may also exit with the reasons described in
\S\ref{section:function-application}.
(In addition, the function being applied may complete abnormally
with any reason.)

\EVALUATION

The evaluation depends on the type of the first argument:
\begin{itemize}
\item If $\TZ{v}_1$ is a 2-tuple of atoms \TZ{Mod} and \TZ{Fun}, then
evaluation proceeds as described by case~\ref{item:explicit-mod-fun}
in \S\ref{section:function-application} with the atoms \TZ{Mod} and
\TZ{Fun} specifying the module name and function symbol, respectively, and
the list $\TZ{v}_2$ specifying the values of the arguments (and the arity).
\item If $\TZ{v}_1$ is a function, then evaluation proceeds as
described by case~\ref{item:function-application-impl-fun}
or~\ref{item:function-application-expl-fun}
in \S\ref{section:function-application} with $\TZ{v}_1$ being
the function and
the list $\TZ{v}_2$ specifying the values of the arguments.
\end{itemize}
\index{apply/2 BIF@\T{apply/2} BIF|)}
\index{function!application|)}

\subsection{\T{apply/3}}

\label{section:apply3}
\index{apply/3 BIF@\T{apply/3} BIF|(}
\index{function!application|(}

Given a module name, a function symbol and a list of arguments, a function
is looked up and applied to the arguments.

\TYPE

\T{apply(atom(),atom(),[term()]) -> term()}.

\EXITS

\T{apply/3} exits with cause \T{badarg} if $\TZ{v}_1$ is not an atom,
$\TZ{v}_2$ is not an atom, or $\TZ{v}_3$ is not a list.
\T{apply/3} may also exit with the reasons described in
\S\ref{section:function-application}.
(In addition, the function being applied may complete abnormally
with any reason.)

\EVALUATION

The evaluation of \T{apply/3} is described by case~\ref{item:explicit-mod-fun}
in \S\ref{section:function-application} with the atoms $\TZ{v}_1$ and
$\TZ{v}_2$ specifying the module name and function symbol, respectively, and
the list $\TZ{v}_3$ specifying the values of the arguments.
\index{apply/3 BIF@\T{apply/3} BIF|)}
\index{function!application|)}

\subsection{\T{exit/1}}

\label{section:exit1}
\index{exit/1 BIF@\T{exit/1} BIF|(}
\index{exit|(}

The BIF always exits with the argument as reason.  Unless the application is
governed by a
\ifOld \T{catch} expression (\S\ref{section:catch})\fi
\ifStd \T{try} expression (\S\ref{section:try-expr})\fi,
the process calling the BIF will exit.

\TYPE

\T{exit(term()) -> _}.

\EXITS

\T{exit/1} always exits, see below.

\EVALUATION

The evaluation of \T{exit/1} exits with reason $\TZ{v}_1$.
\index{exit/1 BIF@\T{exit/1} BIF|)}
\index{exit|)}

\subsection{\T{exit/2}}

\label{section:exit2}
\index{exit/2 BIF@\T{exit/2} BIF|(}
\index{exit!signal|(}

An exit signal with the given reason is sent to a process or port.
Reception of the exit signal will cause
the receiving process to complete abruptly\index{completion!abrupt}
unless it traps exit signals\index{exit!signal!trapping} or the
reason is the atom \T{normal}\index{normal@\T{normal}!exit signal},
cf.~\S\ref{section:receiving-exit-signal}.
If the reason is the atom \T{kill}\index{kill@\T{kill}!exit signal},
the receiving process will always complete abruptly.

\TYPE

\T{exit(pid(),term()) -> true} ; \\
\T{exit(port(),term()) -> true}.

\EXITS

\T{exit/2} exits with cause \T{badarg} if $\TZ{v}_1$ is neither a PID,
nor a port.

\EVALUATION

An exit signal with $\TZ{v}_2$ as reason is sent to the process or port
identified by $\TZ{v}_1$, as described in \S\ref{section:sending-exit-signal}.
The result is always the atom \T{true}.
\index{exit/2 BIF@\T{exit/2} BIF|)}
\index{exit!signal|)}

\subsection{\T{group_leader/0}}

\label{section:groupleader0}
\index{group_leader/0 BIF@\T{group_leader/0} BIF|(}
\index{process!group|(}

The BIF returns the group leader (\S\ref{section:group-leader}) of the calling process.

\TYPE

\T{group_leader() -> pid()}.

\EXITS

\T{group_leader/0} always completes normally.

\EVALUATION

The result is \T{group_leader[\Z{P}]}, where \TZ{P} is the process calling the
BIF.
\index{group_leader/0 BIF@\T{group_leader/0} BIF|)}
\index{process!group|)}

\subsection{\T{group_leader/2}}

\label{section:groupleader2}
\index{group_leader/2 BIF@\T{group_leader/2} BIF|(}
\index{process!group|(}

The BIF changes the group leader of a process.

\TYPE

\T{group_leader(pid(),pid()) -> true}.

\EXITS

\T{group_leader/2} exits with cause \T{badarg} if $\TZ{v}_1$ or $\TZ{v}_2$
is not a PID.

\EVALUATION

A group leader signal (\S\ref{section:signals}) is sent to $\TZ{v}_2$ with
$\TZ{v}_1$ as additional data.  When the signal is received by $\TZ{v}_2$,
\T{group_leader[$\Z{v}_2$]} will be set to $\TZ{v}_1$
(\S\ref{section:signal-arrival}).

The result is always the atom \T{true}.
\index{group_leader/2 BIF@\T{group_leader/2} BIF|)}
\index{process!group|)}

\subsection{\T{link/1}}

\label{section:link1}
\index{link/1 BIF@\T{link/1} BIF|(}
\index{process!linking|(}

A request to add a link between the process calling the BIF and some process
or port is submitted.

\TYPE

\T{link(pid()) -> true} ; \\
\T{link(port()) -> true}.

\EXITS

\T{link/1} exits with cause \T{badarg} if $\TZ{v}_1$ is not a PID or a port.

\EVALUATION

Let \TZ{P} be the process evaluating the application of \T{link/1}.
\begin{itemize}
\item If $\TZ{P}\neq\TZ{v}_1$ and $\TZ{v}_1$ is not
in \T{linked[\Z{P}]}, then $\TZ{v}_1$ is added to
\T{linked[\Z{P}]} and a \I{link} signal with \TZ{P} as sender
is dispatched to process $\TZ{v}_1$ (\S\ref{section:links}, \S\ref{section:signals}).
\item Otherwise, nothing is done.
\end{itemize}
The result is always the atom \T{true}.
\index{link/1 BIF@\T{link/1} BIF|)}
\index{process!linking|)}

\ifOld

\subsection{\T{list_to_pid/1}}

\label{section:listtopid1}
\index{list_to_pid/1 BIF@\T{list_to_pid/1} BIF|(}
\index{process!PID|(}

A list of characters is converted to a PID.

\TYPE

\T{list_to_pid([int()]) -> pid()}.

\EXITS

\T{list_to_pid/1} exits with cause \T{badarg} if $\TZ{v}_1$ is not a
list of characters that represents a PID.

\EVALUATION

The result of a BIF call of \T{list_to_pid/1} is a PID \TZ{P}
such that the value of \T{pid_to_list(\Z{P})}
(cf.\ \S\ref{section:pidtolist1}) on the same node
equals $\TZ{v}_1$.
\index{list_to_pid/1 BIF@\T{list_to_pid/1} BIF|)}
\index{process!PID|)}

\subsection{\T{pid_to_list/1}}

\label{section:pidtolist1}
\index{pid_to_list/1 BIF@\T{pid_to_list/1} BIF|(}
\index{process!PID|(}

A PID is converted to a list of characters.

\TYPE

\T{list_to_pid(pid()) -> [int()]}.

\EXITS

\T{list_to_pid/1} exits with cause \T{badarg} if $\TZ{v}_1$ is not a PID.

\EVALUATION

The result of a BIF call of \T{list_to_pid/1} is
\ifStd an implementation-defined \fi
\ifOld some \fi
list of characters.  It
\ifStd must hold \fi
\ifOld is guaranteed \fi
that for any PID \TZ{P}, the value
of an expression \T{list_to_pid(pid_to_list(\Z{P}))} equals \TZ{P}.
\index{pid_to_list/1 BIF@\T{pid_to_list/1} BIF|)}
\index{process!PID|)}

\fi % ifOld

\subsection{\T{process_flag/2}}

\label{section:processflag2}
\index{process_flag/2 BIF@\T{process_flag/2} BIF|(}
\index{process!flag|(}

The value of a process flag is read and updated.

\TYPE

\T{process_flag(trap_exit,bool()) -> bool()} ; \\
\T{process_flag(error_handler,atom()) -> atom()} ; \\
\T{process_flag(priority,atom()) -> atom()}.

\EXITS

\T{process_flag/2} exits with cause \T{badarg} if $\TZ{v}_1$ and $\TZ{v}_2$
is not one of the following combinations:
\begin{itemize}
\item The atom \T{trap_exit} and a Boolean atom.
\item The atom \T{error_handler} and a module name (i.e., an atom).
\item The atom \T{priority} and a priority atom, i.e., one of
\T{normal}, \T{high} and \T{low} \S\ref{section:scheduling}).
\end{itemize}

\EVALUATION

The action depends on $\TZ{v}_1$:
\begin{itemize}
\item If $\TZ{v}_1$ is \T{trap_exit}, then
\T{trap_exit[\Z{P}]} is set to $\TZ{v}_2$ and
the previous value of \T{trap_exit[\Z{P}]} is returned.
\item If $\TZ{v}_1$ is \T{error_handler}, then
\T{error_handler[\Z{P}]} is set to $\TZ{v}_2$ and
the previous value of \T{error_handler[\Z{P}]} is returned.
\item If $\TZ{v}_1$ is \T{priority}, then
\T{priority[\Z{P}]} is set to $\TZ{v}_2$ and
the previous value of \T{priority[\Z{P}]} is returned.
\end{itemize}
\index{process_flag/2 BIF@\T{process_flag/2} BIF|)}
\index{process!flag|)}

\subsection{\T{process_info/1}}

\label{section:processinfo1}
\index{process_info/1 BIF@\T{process_info/1} BIF|(}
\index{process!information|(}

Information about various properties of a process is returned.

\TYPE

\iftrue
\T{process_info(pid()) -> term()}.
\else
% This just isn't correct: the atom undefined may be returned.
\T{process_info(pid()) -> [\{atom(),term()\}]}.
\fi

\EXITS

\T{process_info/1} exits with cause \T{badarg} if $\TZ{v}_1$ is not a PID.

\EVALUATION

The BIF returns information about the process $\TZ{v}_1$.
\begin{itemize}
\item If that process is not alive, the result is the atom \T{undefined}.
\item Otherwise, the result is
a list of 2-tuples, each of which is the same as the result of an
application of the BIF \T{process_info/2} (\S\ref{section:processinfo2})
to $\TZ{v}_1$ and a distinct atom in the left column of Table~\ref{table:processinfo}.
The list should include all properties for which \T{process_info/2} gives meaningful
information.
\end{itemize}
\index{process_info/1 BIF@\T{process_info/1} BIF|)}
\index{process!information|)}

\subsection{\T{process_info/2}}

\label{section:processinfo2}
\index{process_info/2 BIF@\T{process_info/2} BIF|(}
\index{process!information|(}

Information about some property of a process is returned,
as described in Table~\ref{table:processinfo}.

\begin{table}[htb]
\begin{center}
\begin{tabular}{@{}ll@{}}
\hline
Second argument & Information returned about the process \\
\hline
\T{current_function} & The most recently entered function. \\ 
\T{dictionary} & The process dictionary. \\
\T{error_handler} & The error handler module. \\
\T{group_leader} & The group leader of the process. \\
\T{heap_size} & The current heap size. \\
\T{initial_call} & The initial application of the process. \\
\T{links} & The processes and ports to which the process is \\
& linked. \\
\T{memory} & The total amount of memory occupied by the \\
& process. \\
\T{message_queue_len} & The number of unprocessed messages in the queue. \\
\T{messages} & The unprocessed messages in the queue. \\
\T{priority} & The scheduling priority of the process. \\
\T{reductions} & The current number of reductions. \\
\T{registered_name} & The registered name of the process, if any. \\
\T{stack_size} & The current stack size. \\
\T{status} & The scheduling status: waiting, runnable or \\
& running. \\
\T{trap_exit} & Whether exit signals are trapped. \\
\hline
\end{tabular}
\caption{Alternatives for the BIF \T{process_info/2}.}
\label{table:processinfo}
\end{center}
\end{table}

\TYPE

\iftrue
\T{process_info(pid(),atom()) -> term()}.
\else
% This just isn't correct: the atom undefined may be returned, and for
% current_function we may also get the atom undefined instead of a 3-tuple.
\T{process_info(pid(),current_function) -> \{current_function,\{atom(),atom(),[term()]\}\}} ; \\
\T{process_info(pid(),dictionary) -> \{dictionary,[\{term(),term()\}]\}} ; \\
\T{process_info(pid(),error_handler) -> \{error_handler,atom()\}} ; \\
\T{process_info(pid(),group_leader) -> \{group_leader,pid()\}} ; \\
\T{process_info(pid(),heap_size) -> \{heap_size,int()\}} ; \\
\T{process_info(pid(),initial_call) -> \{initial_call,\{atom(),atom(),[term()]\}\}} ; \\
\T{process_info(pid(),links) -> \{links,[pid()]\}} ; \\
\T{process_info(pid(),memory) -> \{memory,int()\}} ; \\
\T{process_info(pid(),message_queue_len) -> \{message_queue_len,int()\}} ; \\
\T{process_info(pid(),messages) -> \{messages,[term()]\}} ; \\
\T{process_info(pid(),priority) -> \{priority,atom()\}} ; \\
\T{process_info(pid(),reductions) -> \{reductions,int()\}} ; \\
\T{process_info(pid(),registered_name) -> \{registered_name,atom()\}} ; \\
\T{process_info(pid(),stack_size) -> \{stack_size,int()\}} ; \\
\T{process_info(pid(),status) -> \{status,atom()\}} ; \\
\T{process_info(pid(),trap_exit) -> \{trap_exit,bool()\}}.
\fi

\EXITS

\T{process_info/2} exits with cause \T{badarg} if $\TZ{v}_1$ is not a PID
or $\TZ{v}_2$ is not one of the atoms in the left column of
Table~\ref{table:processinfo}.

\EVALUATION

\begin{itemize}
\item If process $\TZ{v}_1$ is not alive, the result is the atom \T{undefined}.
\item Otherwise, the BIF returns information about the process $\TZ{v}_1$ and
the result is always a 2-tuple where the first element is $\TZ{v}_2$:
\begin{itemize}
\item If $\TZ{v}_2$ is \T{current_function}, then
return \T{\{current_function,\linebreak[0]current_function[$\TZ{v}_1$]\}}.
\item If $\TZ{v}_2$ is \T{dictionary}, then
return \T{\{dictionary,\Z{Dict}\}}, where \TZ{Dict} is an
association list representing the contents of \T{dictionary[$\TZ{v}_1$]}
(cf.\ the BIF \T{get/0} [\S\ref{section:get0}]).
\item If $\TZ{v}_2$ is \T{error_handler}, then
return \T{\{error_handler,\linebreak[0]error_handler[$\TZ{v}_1$]\}}.
\item If $\TZ{v}_2$ is \T{group_leader}, then
return \T{\{group_leader,\linebreak[0]group_leader[$\TZ{v}_1$]\}}.
\item If $\TZ{v}_2$ is \T{heap_size}, then
return \T{\{heap_size,heap_size[$\TZ{v}_1$]\}}.
\item If $\TZ{v}_2$ is \T{initial_call}, then
return \T{\{initial_call,\linebreak[0]initial_call[$\TZ{v}_1$]\}}.
\item If $\TZ{v}_2$ is \T{links}, then
return \T{\{links,\Z{Lst}\}}, where \TZ{Lst} is a
list representing the value of \T{linked[$\TZ{v}_1$]}.
\item If $\TZ{v}_2$ is \T{memory}, then
return \T{\{memory,memory_in_use[$\TZ{v}_1$]\}}.
\item If $\TZ{v}_2$ is \T{message_queue_len}, then
return \T{\{message_queue_len,\linebreak[0]$\Re[l]$\}}, where $l$ is the
length of \T{message_queue[$\TZ{v}_1$]}.
\item If $\TZ{v}_2$ is \T{messages}, then
return \T{\{messages,\Z{Lst}\}}, where \TZ{Lst} is a
list representing the value of \T{message_queue[$\TZ{v}_1$]}
(i.e., a list of the messages in the queue, in the same order).
\item If $\TZ{v}_2$ is \T{priority}, then
return \T{\{priority,priority[$\TZ{v}_1$]\}}.
\item If $\TZ{v}_2$ is \T{reductions}, then
return \T{\{reductions,reductions[$\TZ{v}_1$]\}}.
\item If $\TZ{v}_2$ is \T{registered_name}, then
return \T{\{registered_name,\linebreak[0]registered_name[$\TZ{v}_1$]\}}.
\item If $\TZ{v}_2$ is \T{stack_size}, then
return \T{\{stack_size,$\Re[s]$\}}, where $s$ is a measure of
the size of \T{stack_trace[$\TZ{v}_1$]}.
\item If $\TZ{v}_2$ is \T{status}, then
return \T{\{status,status[$\TZ{v}_1$]\}}.
\item If $\TZ{v}_2$ is \T{trap_exit}, then
return \T{\{trap_exit,trap_exit[$\TZ{v}_1$]\}}.
\end{itemize}
\end{itemize}
Let \TZ{P} be the process calling the BIF.  The behaviour with respect
to signals (\S\ref{section:signals}) should be as if the result was
obtained by \TZ{P} sending an \I{info request} signal to process $\TZ{v}_1$
with $\TZ{v}_2$ as additional information, and process $\TZ{v}_1$ responding
with a message to \TZ{P} containing the result.
\index{process_info/2 BIF@\T{process_info/2} BIF|)}
\index{process!information|)}

\subsection{\T{processes/0}}

\index{processes/0 BIF@\T{processes/0} BIF|(}
See \S\ref{section:processes0}.
\index{processes/0 BIF@\T{processes/0} BIF|)}

\subsection{\T{self/0}}

\label{section:self0}
\index{self/0 BIF@\T{self/0} BIF|(}

The PID of the process calling the BIF is returned.
\T{self/0} is a guard BIF.

\TYPE

\T{self() -> pid()}.

\EXITS

\T{self/0} always completes normally.

\EVALUATION

The PID of the process calling the BIF is returned.
\index{self/0 BIF@\T{self/0} BIF|)}

\subsection{\T{spawn/3}}

\label{section:spawn3}
\index{spawn/3 BIF@\T{spawn/3} BIF|(}
\index{process!spawning a|(}

A new process is spawned on the same node.

\TYPE

\T{spawn(atom(),atom(),[term()]) -> pid()}.

\EXITS

\T{spawn/3} exits with cause \T{badarg} if $\TZ{v}_1$ or $\TZ{v}_2$ is not an atom,
or if $\TZ{v}_3$ is not a list.

\EVALUATION

Let the elements of $\TZ{v}_3$ be $\TZ{T}_1$, \ldots, $\TZ{T}_k$.
A new process is spawned on the same node as the process calling the BIF
(\S\ref{section:spawning-processes}).
The initial call of the process is \T{$\Z{v}_1$:$\Z{v}_2$($\Z{T}_1$,\tdots,$\Z{T}_k$)}.
The PID of the newly spawned process is returned.
\index{spawn/3 BIF@\T{spawn/3} BIF|)}
\index{process!spawning a|)}

\subsection{\T{spawn/4}}

\label{section:spawn4}
\index{spawn/4 BIF@\T{spawn/4} BIF|(}
\index{process!spawning a|(}

A new process is spawned on a particular node.

\TYPE

\T{spawn(atom(),atom(),atom(),[term()]) -> pid()}.

\EXITS

\T{spawn/4} exits with cause \T{badarg} if $\TZ{v}_1$, $\TZ{v}_2$ or $\TZ{v}_3$ is not an atom,
or if $\TZ{v}_4$ is not a list.

\EVALUATION

Let the elements of $\TZ{v}_4$ be $\TZ{T}_1$, \ldots, $\TZ{T}_k$.
A new process is spawned on node $\TZ{v}_1$ (\S\ref{section:spawning-processes}).
The initial call of the process is \T{$\Z{v}_2$:$\Z{v}_3$($\Z{T}_1$,\tdots,$\Z{T}_k$)}.
The PID of the newly spawned process is returned.
\index{spawn/4 BIF@\T{spawn/4} BIF|)}
\index{process!spawning a|)}

\subsection{\T{spawn_link/3}}

\label{section:spawnlink3}
\index{spawn_link/3 BIF@\T{spawn_link/3} BIF|(}
\index{process!spawning a|(}

A new process, initially linked to its creator, is spawned on the same node.

\TYPE

\T{spawn_link(atom(),atom(),[term()]) -> pid()}.

\EXITS

\T{spawn_link/3} exits with cause \T{badarg} if $\TZ{v}_1$ or $\TZ{v}_2$ is not an atom,
or if $\TZ{v}_3$ is not a list.

\EVALUATION

The BIF does exactly the same thing as \T{spawn/3} (\S\ref{section:spawn3}),
except that when the BIF returns, the newly spawned process is linked with
the process calling \T{spawn_link/3}.
\index{spawn_link/3 BIF@\T{spawn_link/3} BIF|)}
\index{process!spawning a|)}

\subsection{\T{spawn_link/4}}

\label{section:spawnlink4}
\index{spawn_link/4 BIF@\T{spawn_link/4} BIF|(}
\index{process!spawning a|(}

A new process, initially linked to its creator, is spawned on a particular node.

\TYPE

\T{spawn_link(atom(),atom(),atom(),[term()]) -> pid()}.

\EXITS

\T{spawn/4} exits with cause \T{badarg} if $\TZ{v}_1$, $\TZ{v}_2$ or $\TZ{v}_3$ is not an atom,
or if $\TZ{v}_4$ is not a list.

\EVALUATION

The BIF does exactly the same thing as \T{spawn/4} (\S\ref{section:spawn4}),
except that when the BIF returns, the newly spawned process is linked with
the process calling \T{spawn_link/4}.
\index{spawn_link/4 BIF@\T{spawn_link/4} BIF|)}
\index{process!spawning a|)}

\subsection{\T{unlink/1}}

\label{section:unlink1}
\index{unlink/1 BIF@\T{unlink/1} BIF|(}
\index{process!linking|(}

A request to remove any link between the process calling the BIF and
some process or port is submitted.

\TYPE

\T{unlink(pid()) -> true} ; \\
\T{unlink(port()) -> true}.

\EXITS

\T{unlink/1} exits with cause \T{badarg} if $\TZ{v}_1$ is not a PID or a port.

\EVALUATION

Let \TZ{P} be the process evaluating the application of \T{unlink/1}.
\begin{itemize}
\item If $\TZ{P}\neq\TZ{v}_1$ and $\TZ{v}_1$ is
in \T{linked[\Z{P}]}, then $\TZ{v}_1$ is removed from
\T{linked[\Z{P}]} and an \I{unlink} signal with \TZ{P} as sender
is dispatched to process $\TZ{v}_1$.
\item Otherwise, nothing is done.
\end{itemize}
The result is always the atom \T{true}.
\index{unlink/1 BIF@\T{unlink/1} BIF|)}
\index{process!linking|)}
\index{function!BIFs|)}
\index{process!BIFs|)}

\section{Builtin functions for process dictionaries}

\label{section:dictionary-bifs}
\index{process!dictionary!BIFs|(}

As described in \S\ref{section:process-state-dynamic},
each process \TZ{P} has associated with it a table
\T{dictionary[\Z{P}]} (\S\ref{section:tables}).

\subsection{\T{erase/0}}
\index{erase/0 BIF@\T{erase/0} BIF|(}

The process calling the BIF has every row of its dictionary removed
but a representation of the previous contents is returned.

\TYPE

\T{erase() -> [\{term(),term()\}]}.

\EXITS

\T{erase/0} always completes normally.

\EVALUATION

Let $d$ be the value of \T{dictionary[\Z{P}]}, where \TZ{P} is the
process calling \T{erase/0} and let \TZ{lst} be an association list
representing the contents of $d$ (\S\ref{section:tables}).
The effect of the call is to remove every
row of $d$ and the result is \TZ{lst}.

The operation must be atomic.

\iffalse
The time for producing the effect and computing the answer should
be $O(\mathit{size}(d))$.
\fi
\index{erase/0 BIF@\T{erase/0} BIF|)}

\subsection{\T{erase/1}}
\index{erase/1 BIF@\T{erase/1} BIF|(}

In the dictionary of the process calling the BIF, the value recorded
for a certain key (if any) is erased, but the previously recorded
value (or \T{undefined}) is returned.

\TYPE

\T{erase(term()) -> term()}.

\EXITS

\T{erase/1} always completes normally.

\EVALUATION

Let $d$ be the value of \T{dictionary[\Z{P}]}, where \TZ{P} is the
process calling \T{erase/1}.
There are two cases depending on whether $\TZ{v}_1$ is a key in $d$ or not:
\begin{itemize}
\item If $d$ contains a row with $\TZ{v}_1$ as key and some term $\TZ{T}_1$ as
value, then that row is removed from $d$ and \TZ{t} is $\TZ{t}_1$.
\item Otherwise, $d$ is unchanged and \TZ{t} is the atom \T{undefined}.
\end{itemize}
The result is \TZ{t}.

The operation must be atomic.

\iffalse
The time for producing the effect and computing the answer should
be $O(\mathit{size}(\mathit{keys}(d))+size(\TZ{t}))$.
\fi
\index{erase/1 BIF@\T{erase/1} BIF|)}

\subsection{\T{get/0}}

\label{section:get0}
\index{get/0 BIF@\T{get/0} BIF|(}

A representation of the dictionary of
the process calling the BIF is returned.

\TYPE

\T{get() -> [\{term(),term()\}]}.

\EXITS

\T{get/0} always completes normally.

\EVALUATION

Let $d$ be the value of \T{dictionary[\Z{P}]}, where \TZ{P} is the
process calling \T{get/0}.
The result of calling \T{get/0} is a list representing $d$
(\S\ref{section:tables}).

\iffalse
The time for computing the answer should
be $O(\mathit{size}(d))$.
\fi
\index{get/0 BIF@\T{get/0} BIF|)}

\subsection{\T{get/1}}

\index{get/1 BIF@\T{get/1} BIF|(}

The value recorded for a certain
key in the dictionary of the process calling the BIF is returned
(or \T{undefined} is returned if there is no value recorded).

\TYPE

\T{get(term()) -> term()}.

\EXITS

\T{get/1} always completes normally.

\EVALUATION

Let $d$ be the value of \T{dictionary[\Z{P}]}, where \TZ{P} is the
process calling \T{get/1}.
There are two cases depending on whether $\TZ{v}_1$ is a key in $d$ or not:
\begin{itemize}
\item If $d$ contains a row with $\TZ{v}_1$ as key and some term $\TZ{T}_1$ as
value, then \TZ{t} is $\TZ{t}_1$.
\item Otherwise, \TZ{t} is  the atom \T{undefined}.
\end{itemize}
The result is \TZ{t}.

\iffalse
The time for computing the answer cannot be expected to be less than
$O(\mathit{size}(\mathit{keys}(d))+size(\TZ{t}))$.
\fi
\index{get/1 BIF@\T{get/1} BIF|)}

\subsection{\T{get\char'137keys/1}}

\index{get_keys/1 BIF@\T{get_keys/1} BIF|(}

A list of all keys in the dictionary of the process calling the BIF
that have a certain value is returned.

\TYPE

\T{get_keys(term()) -> [term()]}.

\EXITS

\T{get_keys/1} always completes normally.

\EVALUATION

Let $d$ be the value of \T{dictionary[\Z{P}]}, where \TZ{P} is the
process calling \T{get/0}.
The result of calling \T{get_keys/1} is a list without duplicates
that contains each key
of $d$ for which the value is $\TZ{v}_1$.  The elements
of the resulting list may be in any order.

\iffalse
The time for computing the answer should be
$O(\mathit{size}(\mathit{keys}(d)))$.
\fi
\index{get_keys/1 BIF@\T{get_keys/1} BIF|)}

\subsection{\T{put/2}}

\index{put/2 BIF@\T{put/2} BIF|(}

In the dictionary of the process calling the BIF, a value
is set for some key, replacing any previous value, which is returned
(or \T{undefined} is returned if there was no value recorded previously).

\TYPE

\T{put(term(),term()) -> term()}.

\EXITS

\T{put/2} always completes normally.

\EVALUATION

Let $d$ be the value of \T{dictionary[\Z{P}]}, where \TZ{P} is the
process calling \T{put/2}.
There are two cases depending on whether $\TZ{v}_1$ is a key in $d$ or not:
\begin{itemize}
\item If $d$ contains a row with $\TZ{v}_1$ as key and some term $\TZ{t}_1$ as value,
then that row is replaced with one having $\TZ{v}_1$ as key and
$\TZ{v}_2$ as value; \TZ{t} is $\TZ{t}_1$.
\item Otherwise, a row with key $\TZ{v}_1$ and value $\TZ{v}_2$ is added
to $d$ and \TZ{t} is \T{undefined}.
\end{itemize}
The result is \TZ{t}.

The operation must be atomic.

\iffalse
The time for producing the effect and computing the answer should
be $O(\mathit{size}(\mathit{keys}(d))+size(\TZ{t})+size(\TZ{v}_2))$.
\fi
\index{put/2 BIF@\T{put/2} BIF|)}
\index{process!dictionary!BIFs|)}

\section{Builtin functions for nodes}

% Have a look at all these when we are an isolated node!!!

\label{section:node-bifs}
\index{node!BIFs|(}

\iffalse
% Not part of the specification (or even \OldErlang).
\subsection{\T{alive/2}}

\label{section:alive2}
\index{erlang:alive/2 BIF@\T{erlang:alive/2} BIF|(}
\index{alive/2 BIF@\T{alive/2} BIF|(}
\index{node!communicating|(}

% ASKED VARIOUS QUESTIONS BY EMAIL 980524:
\iffalse
Hej!

BIFen alive/2 �r lite b�kig f�r mig att testa s� jag undrar:

1. Vad returnerar den?

2. Vad h�nder om den anropas p� en redan kommunicerande nod?

3. Vad h�nder om den f�rs�ker ange ett namn som redan anv�nds?

Mvh,

-- Jonas
\fi

Calling this BIF has as effect to make the node on which the calling process
resides a communicating node, provided that certain preconditions are met.

\TYPE

\T{alive(atom(),port()) -> ???}.

\EXITS

\T{alive/2} exits with cause \T{badarg} if $\TZ{v}_1$ is
not an atom or if $\TZ{v}_2$ is not a port.  \T{register/2} may also
exit with cause \T{badarg} % really lousy choice...
if certain preconditions are not met (see below).

\EVALUATION

Let \TZ{P} be the process calling \T{alive/2}.
\begin{itemize}
\item If there is no process registered under the name \T{net_kernel}
on \T{node[\Z{P}]}, then the BIF exits with cause \T{badarg}.  % yuck
\item Something about EPMD???
\item Otherwise, the node becomes a communicating node, as described in
\S\ref{section:single-multi-node} and \S\ref{section:registering-nodes}.
\end{itemize}
The result is ???.
\index{erlang:alive/2 BIF@\T{erlang:alive/2} BIF|)}
\index{alive/2 BIF@\T{alive/2} BIF|)}
\index{node!communicating|)}
\fi % iffalse

\subsection{\T{erlang:disconnect_node/1}}

\label{section:disconnectnode1}
\index{erlang:disconnect_node/1 BIF@\T{erlang:disconnect_node/1} BIF|(}
\index{disconnect_node/1 BIF@\T{disconnect_node/1} BIF|(}

Friendship is terminated with a given node.

\TYPE

\T{disconnect_node(term()) -> atom()}. % yuck, should be atom() -> bool().

\EXITS

\T{disconnect_node/1} always completes normally.

\EVALUATION

Let \TZ{P} be the process calling \T{disconnect_node/1}.
\begin{itemize}
\item If \T{node[\Z{P}]} is not communicating, then \T{ignored} is returned.
\item Otherwise, if $\TZ{v}_1$ is not an atom or $\TZ{v}_1$ is not in
\T{friends[node[\Z{P}]]}, \T{false} is returned.
\item Otherwise the same things happen as when node \T{node[\Z{P}]} has
lost contact with node $\TZ{v}_1$, as described in \S\ref{section:friendship}
and \S\ref{section:monitor-node}, and then \T{true} is returned.
\end{itemize}
\index{erlang:disconnect_node/1 BIF@\T{erlang:disconnect_node/1} BIF|)}
\index{disconnect_node/1 BIF@\T{disconnect_node/1} BIF|)}

\subsection{\T{erlang:get_cookie/0}}

\label{section:getcookie0}
\index{erlang:get_cookie/0 BIF@\T{erlang:get_cookie/0} BIF|(}
\index{get_cookie/0 BIF@\T{get_cookie/0} BIF|(}
\index{node!magic cookie|(}

The magic cookie of the current node is returned.

\TYPE

\T{get_cookie() -> atom()}.

\EXITS

\T{get_cookie/0} always completes normally.

\EVALUATION

The BIF returns \T{magic_cookie[node[\Z{P}]]}, where \TZ{P} is the process
calling \T{get_cookie/0}.
\index{erlang:get_cookie/0 BIF@\T{erlang:get_cookie/0} BIF|)}
\index{get_cookie/0 BIF@\T{get_cookie/0} BIF|)}
\index{node!magic cookie|)}

\subsection{\T{erlang:halt/0}}

\label{section:halt0}
\index{halt/0 BIF@\T{halt/0} BIF|(}
\index{node!termination|(}

The current node is terminated.

\TYPE

\T{halt() -> _}.

\EXITS

\T{halt/0} never completes abnormally.

\EVALUATION

The node \T{node[\Z{P}]}, where \TZ{P} is the process
calling \T{halt/0}, is terminated immediately.  The BIF thus never returns.
\index{node!termination|)}
\index{halt/0 BIF@\T{halt/0} BIF|)}

\subsection{\T{is_alive/0}}

\label{section:isalive0}
\index{is_alive/0 BIF@\T{is_alive/0} BIF|(}
\index{node!communicating|(}

The BIF tests whether the current node is communicating or not.

\TYPE

\T{is_alive() -> bool()}.

\EXITS

\T{is_alive/0} always completes normally.

\EVALUATION

The BIF returns the value of \T{communicating[node[\Z{P}]]},
where \TZ{P} is the process calling the BIF.
\index{is_alive/0 BIF@\T{is_alive/0} BIF|)}
\index{node!communicating|)}

\subsection{\T{monitor_node/2}}

\label{section:monitornode2}
\index{monitor_node/2 BIF@\T{monitor_node/2} BIF|(}
\index{node!monitoring|(}

Calling this BIF has as effect to increase or decrease the number of messages
that the current process will receive as notification that friendship between
the current node and a certain (other) node has ceased.

\TYPE

\T{monitor_node(atom(),bool()) -> true}.

\EXITS

\T{monitor_node/2} exits with cause \T{badarg} if $\TZ{v}_1$ is
not an atom or if $\TZ{v}_2$ is not a Boolean atom.  On an isolated node,
it will also exit with cause \T{badarg} % baaaad
if the node to be monitored is not the current node.

\EVALUATION

Let \TZ{P} be the process calling \T{monitor_node/2}.
\begin{itemize}
\item If \T{alive[node[\Z{P}]]} is \T{false} and $\TZ{v}_1$ is not
\T{node[\Z{P}]}, then the BIF exits with cause \T{badarg}. % questionable
\item If $\TZ{v}_1$ is \T{node[\Z{P}]}, do nothing.
\item If $\TZ{v}_1$ is \T{true}, do the following:
\begin{enumerate}
\item If there is not already a row in \T{monitored_nodes[node[\Z{P}]]}
with $\TZ{v}_1$ as key, then add one with an empty table as value.
\item Let $t$ be the value for $\TZ{v}_1$ in \T{monitored_nodes[node[\Z{P}]]}.
If there is no row with \TZ{P} as key in that table, add such a row with value 1.
Otherwise, add 1 to the value for \TZ{P} in $t$.
\end{enumerate}
\item Otherwise ($\TZ{v}_1$ is \T{false}):
\begin{itemize}
\item If there is no row with $\TZ{v}_1$ as key in
\T{monitored_nodes[node[\Z{P}]]}, then do nothing.
\item Otherwise, let $t$ be the value for $\TZ{v}_1$.
If there is no entry for \TZ{P} in $t$, do nothing.
\item Otherwise, if the row for \TZ{P} in $t$ has value 1, then remove the
row for \TZ{P}.
\item Otherwise, subtract 1 from the value for \TZ{P} in $t$.
\end{itemize}
\end{itemize}
The result is always \T{true}.
\index{monitor_node/2 BIF@\T{monitor_node/2} BIF|)}
\index{node!monitoring|)}

\subsection{\T{node/0}}

\label{section:node0}
\index{node/0 BIF@\T{node/0} BIF|(}

This BIF returns (the name of) the node on which the calling process
resides.  \T{node/0} is a guard BIF.

\TYPE

\T{node() -> atom()}.

\EXITS

\T{node/0} always completes normally.

\EVALUATION

The result is \T{node[\Z{P}]}, where \TZ{P} is the
process calling \T{node/0} (\S\ref{section:process-state-static}).
\index{node/0 BIF@\T{node/0} BIF|)}

\subsection{\T{node/1}}

\label{section:node1}
\index{node/1 BIF@\T{node/1} BIF|(}

This BIF returns (the name of) the node on which a given ref, PID or
port was created.  \T{node/1} is a guard BIF.

\TYPE

\T{node(ref()) -> atom()} ; \\
\T{node(pid()) -> atom()} ; \\
\T{node(port()) -> atom()}.

\EXITS

\T{node/1} exits with cause \T{badarg} if $\TZ{v}_1$ is
not a ref, nor a PID or a port.

\EVALUATION

The result is \T{node[$\Z{v}_1$]} (\S\ref{section:process-state-static},
\S\ref{section:port-state-static}).
\index{node/1 BIF@\T{node/1} BIF|)}

\subsection{\T{nodes/0}}

\label{section:nodes0}
\index{nodes/0 BIF@\T{nodes/0} BIF|(}

A list is returned of all friends
of the node on which the BIF is called.

\TYPE

\T{nodes() -> [atom()]}.

\EXITS

\T{nodes/0} always completes normally.

\EVALUATION

The BIF returns a list representing the value of
\T{friends[node[\Z{P}]]}, where \TZ{P} is the
process calling the BIF.
\index{nodes/0 BIF@\T{nodes/0} BIF|)}

\subsection{\T{processes/0}}

\ifStd \label{section:node:processes0} \fi
\ifOld \label{section:processes0} \fi
\index{processes/0 BIF@\T{processes/0} BIF|(}

A list is returned of the PIDs of all live processes
on the node on which the BIF is called.

\TYPE

\T{processes() -> [pid()]}.

\EXITS

\T{processes/0} always completes normally.

\EVALUATION

The BIF returns a list representing the value of \T{processes[node[\Z{P}]]},
where \TZ{P} is the process calling the BIF.
\index{processes/0 BIF@\T{processes/0} BIF|)}

\subsection{\T{erlang:set_cookie/2}}

\label{section:setcookie2}
\index{erlang:set_cookie/2 BIF@\T{erlang:set_cookie/2} BIF|(}
\index{set_cookie/2 BIF@\T{set_cookie/2} BIF|(}
\index{node!magic cookie|(}

The magic cookie of the current node and/or the presumed magic cookie of
another node is set.

\TYPE

\T{set_cookie(atom(),atom()) -> bool()}.

\EXITS

\T{set_cookie/2} exits with cause \T{badarg} if $\TZ{v}_1$ or $\TZ{v}_2$ is
not an atom, or if the atom $\TZ{v}_1$ could not be the name of a node (i.e.,
it does not contain exactly one `\T{@}' character,
cf.\ \S\ref{section:single-multi-node}).

\EVALUATION

There are two cases, depending on the value of $\TZ{v}_1$.  Let \TZ{P} be
the process calling \T{set_cookie/2} and let \TZ{N} be \T{node[\Z{P}]}.
\begin{itemize}
\item If $\TZ{v}_1$ equals \TZ{N}, then
\begin{enumerate}
\item Set \T{magic_cookie[\Z{N}]} to $\TZ{v}_2$.
\item For each pair in \T{magic_cookies[\Z{N}]} where the presumed magic cookie
is \T{nocookie}, change the presumed magic cookie to $\TZ{v}_2$.
\end{enumerate}
\item If $\TZ{v}_1$ does not equal \TZ{N}, then
delete any pair for $\TZ{v}_1$ in \T{magic_cookies[\Z{N}]} and add a pair
$(\TZ{v}_1,\TZ{v}_2)$ to \T{magic_cookies[\Z{N}]}.
\end{itemize}

\index{erlang:set_cookie/2 BIF@\T{erlang:set_cookie/2} BIF|)}
\index{set_cookie/2 BIF@\T{set_cookie/2} BIF|)}
\index{node!magic cookie|)}

\subsection{\T{set_node/2}}

\label{section:setnode2}
\index{erlang:set_node/2 BIF@\T{erlang:set_node/2} BIF|(}
\index{set_node/2 BIF@\T{set_node/2} BIF|(}
\index{node!communicating|(}

Calling this BIF has as effect to make the node on which the calling process
resides a communicating node, provided that certain preconditions are met.

\TYPE

\T{set_node(???,???) -> ???}.

\EXITS

???
\iffalse
\T{alive/2} exits with cause \T{badarg} if $\TZ{v}_1$ is
not an atom or if $\TZ{v}_2$ is not a port.  \T{register/2} may also
exit with cause \T{badarg} % really lousy choice...
if certain preconditions are not met (see below).
\fi

\EVALUATION

???
\iffalse
Let \TZ{P} be the process calling \T{alive/2}.
\begin{itemize}
\item If there is no process registered under the name \T{net_kernel}
on \T{node[\Z{P}]}, then the BIF exits with cause \T{badarg}.  % yuck
\item Something about EPMD???
\item Otherwise, the node becomes a communicating node, as described in
\S\ref{section:single-multi-node} and \S\ref{section:registering-nodes}.
\end{itemize}
The result is ???.
\fi
\index{erlang:set_node/2 BIF@\T{erlang:set_node/2} BIF|)}
\index{set_node/2 BIF@\T{set_node/2} BIF|)}

\subsection{\T{erlang:set_node/3}}

\label{section:setnode3}
\index{erlang:set_node/3 BIF@\T{erlang:set_node/3} BIF|(}
\index{set_node/3 BIF@\T{set_node/3} BIF|(}

Ugga mugga.

\TYPE

\T{set_node(???,???,???) -> ???}.

\EXITS

\T{set_node/3} \ldots

\EVALUATION

Yumm yumm!
\index{erlang:set_node/3 BIF@\T{erlang:set_node/3} BIF|)}
\index{set_node/3 BIF@\T{set_node/3} BIF|)}

\subsection{\T{statistics/1}}

\label{section:statistics1}
\index{statistics/1 BIF@\T{statistics/1} BIF|(}

Information about the current state of the node is returned.

\TYPE

\ifOld
\T{statistics(garbage_collection) -> \{int(),int(),int()\}} \\
\fi
\T{statistics(reductions) -> \{int(),int()\}} \\
\T{statistics(runtime) -> \{int(),int()\}} \\
\T{statistics(run_queue) -> int()} \\
\T{statistics(wall_clock) -> \{int(),int()\}}.

\EXITS

\T{statistics/1} exits with cause \T{badarg} if $\TZ{v}_1$ is not an atom
or if it is not one of the atoms \T{runtime}, \T{wall_clock} or
\T{reductions}.
\ifStd
An implementation may extend \T{statistics/1} to
accept additional atoms as argument.
\fi

\EVALUATION

Let \TZ{N} be the node on which the BIF is called.
The evaluation depends on $\TZ{v}_1$:
\begin{itemize}
\item \T{garbage_collection}: return the result of \T{current_gc[\Z{N}]}
(\S\ref{section:node-state-dynamic}), i.e., a 3-tuple
\T{\{\Z{NumberOfGCs},\Z{WordsReclaimed},0\}} of integers where \TZ{NumberOfGCs} is the
total number of garbage collection operations
that have been carried out by processes on the node and \TZ{WordsReclaimed} is
the total number of memory words reclaimed by such operations, and the third integer
is always 0.

\item \T{reductions}: return the result of \T{current_reductions[\Z{N}]}
(\S\ref{section:node-state-dynamic}), i.e.,
a 2-tuple \T{\{\Z{TotalReductions},\Z{RecentReductions}\}} where \TZ{TotalReductions}
is an \Erlang\ integer representing the number of function calls made
on the node and \Z{RecentReductions} is the same but
only since the last call \T{statistics(reductions)}.

\item \T{run_queue}: return $\Re[i]$, where $i$ is the number
of processes on the node that have \T{runnable} or \T{running} status.

\item \T{runtime}: return the result of \T{current_runtime[\Z{N}]}
(\S\ref{section:node-state-dynamic}), i.e.,
a 2-tuple \T{\{\Z{TotalRuntime},\Z{RecentRuntime}\}} where \TZ{TotalRuntime}
is an \Erlang\ integer representing the total time spent
running processes on the node and \Z{RecentRunTime} is the same but
only since the last call \T{statistics(runtime)}.

\item \T{wall_clock}: return the result of \T{current_wall_clock[\Z{N}]}
(\S\ref{section:node-state-dynamic}), i.e.,
a 2-tuple \T{\{\Z{TotalWallClock},\Z{RecentWallClock}\}} where \TZ{TotalWallClock}
is an \Erlang\ integer representing the time which has passed since the
node was started and \Z{RecentWallClock} is the same but
only since the last call \T{statistics(wall_clock)}.
\end{itemize}
% garbage_collection omitted.
\index{statistics/1 BIF@\T{statistics/1} BIF|)}
\index{node!communicating|)}
\index{node!BIFs|)}

\section{Builtin functions for process registries}

\label{section:process-registry-bifs}
\index{process!registry!BIFs|(}

The process registry of a node is described in \S\ref{section:process-registry}.

\subsection{\T{register/2}}

\label{section:register2}
\index{register/2 BIF@\T{register/2} BIF|(}

A name is registered for a process on a node.

\TYPE

\T{register(atom(),pid()) -> true}.

\EXITS

\T{register/2} exits with cause \T{badarg} if $\TZ{v}_1$ is
not an atom or if $\TZ{v}_2$ is not a PID.  \T{register/2} may also
exit with cause
\ifOld \T{badarg} \fi
\ifStd \T{registry} \fi
if some process is already registered
under the name, there is already a name registered for the process,
the process resides on a different node than the one on which the BIF
is called, or the process has completed (see below).

\EVALUATION

Let \TZ{N} be the node on which the BIF is called.

\begin{itemize}
\item
If \T{node[$\Z{v}_2$]} is not \TZ{N}
or \T{registry[\Z{N}]} already contains a process with name $\TZ{v}_1$
or \T{registry[\Z{N}]} already contains a name for process $\TZ{v}_2$,
or process $\Z{v}_2$ has completed,
then the BIF exits with cause \ifStd\T{registry}\fi\ifOld\T{badarg}\fi.
\item
Otherwise, the name $\TZ{v}_1$ is added for $\TZ{v}_2$ in \T{registry[\Z{N}]}.
The BIF always returns the atom \T{true}.
\end{itemize}
The operation should be atomic to ensure the integrity of the registry.
\index{register/2 BIF@\T{register/2} BIF|)}

\subsection{\T{registered/0}}

\label{section:registered0}
\index{registered/0 BIF@\T{registered/0} BIF|(}

A list of all names in the registry on the current node is returned.

\TYPE

\T{registered() -> [atom()]}.

\EXITS

\T{registered/0} always completes normally.

\EVALUATION

Let \TZ{N} be the node on which the BIF is called.
A list without duplicates of all names in \T{registry[\Z{N}]} is returned.
The order of the atoms in the list is not defined.
The operation should be atomic to ensure the integrity of the registry.
\index{registered/0 BIF@\T{registered/0} BIF|)}

\subsection{\T{unregister/1}}

\label{section:unregister1}
\index{unregister/1 BIF@\T{unregister/1} BIF|(}

Any registration for a name is removed.

\TYPE

\T{unregister(atom()) -> true}.

\EXITS

\T{unregister/1} exits with cause \T{badarg} if $\TZ{v}_1$ is
not an atom.

\EVALUATION

Let \TZ{N} be the node on which the BIF is called.
\begin{itemize}
\item If \T{registry[\Z{N}]} does not contains the name $\TZ{v}_1$,
then the BIF has no effect.
\item Otherwise, the association for the name $\TZ{v}_1$ is removed
from \T{registry[\Z{N}]}.
\end{itemize}
In either case, the atom \T{true} is returned.
The operation should be atomic to ensure the integrity of the registry.
\index{unregister/1 BIF@\T{unregister/1} BIF|)}

\subsection{\T{whereis/1}}

\label{section:whereis1}
\index{whereis/1 BIF@\T{whereis/1} BIF|(}

The PID of a process registered under the given name is returned, if any.

\TYPE

\T{whereis(atom()) -> term()}.

\EXITS

\T{whereis/1} exits with cause \T{badarg} if $\TZ{v}_1$ is
not an atom.

\EVALUATION

Let \TZ{N} be the node on which the BIF is called.
\begin{itemize}
\item If \T{registry[\Z{N}]} does not contains the name $\TZ{v}_1$,
then the BIF returns the atom \T{undefined}.
\item Otherwise, the PID that is the value for $\TZ{v}_1$ in
\T{registry[\Z{N}]} is returned.
\end{itemize}
The operation should be atomic to ensure the integrity of the registry.
\index{whereis/1 BIF@\T{whereis/1} BIF|)}
\index{process!registry!BIFs|)}

\section{Builtin functions for I/O and ports}

\label{section:port-bifs}
\index{port!BIFs|(}

\subsection{\T{open_port/2}}

\label{section:openport2}
\index{open_port/2 BIF@\T{open_port/2} BIF|(}
\index{port!opening|(}

A new port is opened to a recently opened driver or
recently spawned external process.

\TYPE

\T{open_port(term(),[term()]) -> port()}.

\EXITS

\T{open_port/2} exits with cause \T{badarg} if $\TZ{v}_1$ is not one
of the permitted alternatives or $\TZ{v}_2$ contains an invalid
option (\S\ref{section:opening-ports}).

\EVALUATION

The BIF \T{open_port/2} is described fully in \S\ref{section:opening-ports}.
\index{open_port/2 BIF@\T{open_port/2} BIF|)}
\index{port!opening|)}

\subsection{\T{port_close/1}}

\label{section:portclose1}
\index{port_close/1 BIF@\T{port_close/1} BIF|(}
\index{port!closing|(}

A port is closed.

\TYPE

\T{port_close(port()]) -> true}.

\EXITS

\T{port_close/1} exits with cause \T{badarg} if $\TZ{v}_1$ is not a port.
It may also exit with cause \T{badarg} if $\TZ{v}_1$ is  already closed.

\EVALUATION

If $\TZ{v}_1$ is already closed, exit with cause \T{badarg}.  % bad choice
Otherwise, close port $\TZ{v}_1$.
\index{port_close/1 BIF@\T{port_close/1} BIF|)}
\index{port!closing|)}

\subsection{\T{port_info/1}}

\label{section:portinfo1}
\index{port_info/1 BIF@\T{port_info/1} BIF|(}
\index{port!information|(}

Information about various properties of a port is returned.

\TYPE

\iftrue
\T{port_info(port()) -> term()}.
\else
% This just isn't correct: the atom undefined may be returned.
\T{port_info(port()) -> [\{atom(),term()\}]}.
\fi

\EXITS

\T{port_info/1} exits with cause \T{badarg} if $\TZ{v}_1$ is not a port.

\EVALUATION

The BIF returns information about the port $\TZ{v}_1$.
\begin{itemize}
\item If that port is not open, the result is the atom \T{undefined}.
\item Otherwise, the result is
a list of 2-tuples, each of which is the same as the result of an
application of the BIF \T{port_info/2} (\S\ref{section:portinfo2})
to $\TZ{v}_1$ and a distinct atom in the left column of Table~\ref{table:portinfo}.
The list should include all properties for which \T{port_info/2} gives meaningful
information.
\end{itemize}
\index{port_info/1 BIF@\T{port_info/1} BIF|)}
\index{port!information|)}

\subsection{\T{port_info/2}}

\label{section:portinfo2}
\index{port_info/2 BIF@\T{port_info/2} BIF|(}
\index{port!information|(}

Information about some property of a port is returned,
as described in Table~\ref{table:portinfo}.

\begin{table}[hbp]
\begin{center}
\begin{tabular}{@{}ll@{}}
\hline
Second argument & Information returned about the port \\
\hline
\T{id} & The ID of the port. \\ 
\T{connected} & The process owning the port. \\
\T{input} & The number of bytes read from the port. \\
\T{links} & The processes to which the port is linked. \\
\T{name} & The driver or external process to which the port \\
& is opened. \\
\T{output} & The number of bytes written to the port. \\
\hline
\end{tabular}
\caption{Alternatives for the BIF \T{port_info/2}.}
\label{table:portinfo}
\end{center}
\end{table}

\TYPE

\T{port_info(pid(),atom()) -> term()}.

\EXITS

\T{port_info/2} exits with cause \T{badarg} if $\TZ{v}_1$ is not a port
or $\TZ{v}_2$ is not one of the atoms in the left column of
Table~\ref{table:portinfo}.

\EVALUATION

\begin{itemize}
\item If port $\TZ{v}_1$ is closed, the result is the atom \T{undefined}.
\item Otherwise, the BIF returns information about the port $\TZ{v}_1$ and
the result is always a 2-tuple where the first element is $\TZ{v}_2$:
\begin{itemize}
\item If $\TZ{v}_2$ is \T{id}, then
return \T{\{id,ID[$\TZ{v}_1$]\}}.
\item If $\TZ{v}_2$ is \T{connected}, then
return \T{\{connected,owner[$\TZ{v}_1$]\}}.
\item If $\TZ{v}_2$ is \T{input}, then
return \T{\{input,count_in[$\TZ{v}_1$]\}}.
\item If $\TZ{v}_2$ is \T{links}, then
return \T{\{links,\Z{Lst}\}}, where \TZ{Lst} is a
list representing the value of \T{linked[$\TZ{v}_1$]}.
\item If $\TZ{v}_2$ is \T{name}, then
return \T{\{name,command[$\TZ{v}_1$]\}}.
\item If $\TZ{v}_2$ is \T{output}, then
return \T{\{output,count_out[$\TZ{v}_1$]\}}.
\end{itemize}
\end{itemize}
Let \TZ{P} be the process calling the BIF.  The behaviour with respect
to signals (\S\ref{section:signals}) should be as if the result was
obtained by \TZ{P} sending an \I{info request} signal to port $\TZ{v}_1$
with $\TZ{v}_2$ as additional information, and port $\TZ{v}_1$ responding
with a message to \TZ{P} containing the result.
\index{port_info/2 BIF@\T{port_info/2} BIF|)}
\index{port!information|)}

\subsection{\T{ports/0}}

\label{section:ports0}
\index{ports/0 BIF@\T{ports/0} BIF|(}

A list is returned of all open ports
on the node on which the BIF is called.

\TYPE

\T{ports() -> [port()]}.

\EXITS

\T{ports/0} always completes normally.

\EVALUATION

The BIF returns a list representing the value of \T{ports[node[\Z{P}]]},
where \TZ{P} is the process calling the BIF.
\index{ports/0 BIF@\T{ports/0} BIF|)}

\iffalse
WHAT ABOUT THESE BIFS???
\begin{verbatim}
port_command
port_control
port_connect
\end{verbatim}
\fi
\index{port!BIFs|)}

\section{Miscellaneous builtin functions}

\label{section:misc-bifs}

\subsection{\T{date/0}}

\label{section:date0}
\index{date/0 BIF@\T{date/0} BIF|(}

The local date when the BIF is called is returned as a triple.

\TYPE

\T{date() -> \{int(),int(),int()\}}.

\EXITS

\T{date/0} always completes normally.

\EVALUATION

Let the current local year, month and day at the time of evaluation be
$y$, $m$ and $d$.  Let \I{Month} be a function mapping January to 1,
February to 2, \ldots, December to 12.

A triple \T{\{\Z{Year},\Z{Month},\Z{Day}\}} is returned, where
\TZ{Year} is the \Erlang\ integer $\Er[y]$, \TZ{Month} is
the \Erlang\ integer $\Er[\I{Month}(m)]$, and
\TZ{Day} is the \Erlang\ integer $\Er[d]$.

\EXAMPLES

\begin{itemize}
\item \T{date()} evaluated on the 29th of June, 1996, would return \T{\{1996,6,29\}}.
\item \T{date()} evaluated on the 1st of January, 2000, would return \T{\{2000,1,1\}}.
\end{itemize}
\index{date/0 BIF@\T{date/0} BIF|)}

\subsection{\T{erlang:hash/2}}

\label{section:hash2}
\index{erlang:hash/2 BIF@\T{erlang:hash/2} BIF|(}
\index{hash/2 BIF@\T{hash/2} BIF|(}
\index{term!hashing|(}

A hash value in a specified range for an arbitrary term is returned.

\TYPE

\T{hash(term(),fixnum()) -> int()}.

\EXITS

\T{hash/2} exits with cause \T{badarg} if $\TZ{v}_2$ is not a nonnegative
integer.

\EVALUATION

The BIF returns $\Re[\I{Hash}(\TZ{v}_1,\Er[\TZ{v}_2])+1]$, where the function
\I{Hash} is as defined in \S\ref{chapter:hashing}.  That is, the BIF maps each
\Erlang\ term to an integer in the range $[1,\TZ{v}_2]$. The function \I{Hash}
is defined in such a way as to be portable across nodes and
independent of time.  (It is obviously not invertible.)
\index{erlang:hash/2 BIF@\T{erlang:hash/2} BIF|)}
\index{hash/2 BIF@\T{hash/2} BIF|)}
\index{term!hashing|)}

\subsection{\T{make_ref/0}}

\label{section:makeref0}
\index{make_ref/0 BIF@\T{make_ref/0} BIF|(}
\index{ref!creating new|(}

A ref is returned that is different from all refs created previously on the node
and that is different from all refs created on other nodes.

\TYPE

\T{make_ref() -> ref()}.

\EXITS

\T{make_ref/0} always completes normally.

\EVALUATION

Let \TZ{N} be the node on which \T{make_ref/0} is called.  The BIF
invokes the operation \T{next_ref[\Z{N}]}
(\S\ref{section:node-state-dynamic}) and the result is a new ref
that is returned.  This also modifies the state
\T{ref_state[\Z{N}]} so future invocations of
\T{next_ref[\Z{N}]} will produce different refs.
\index{make_ref/0 BIF@\T{make_ref/0} BIF|)}
\index{ref!creating new|)}

\subsection{\T{now/0}}

\label{section:now0}
\index{now/0 BIF@\T{now/0} BIF|(}

A 3-tuple of integers is returned that is guaranteed to be different
for each invocation on a node.

\TYPE

\T{now() -> \{int(),int(),int()\}}.

\EXITS

\T{now/0} always completes normally.

\EVALUATION

Two calls of the BIF \T{now()} by processes residing on the same node can never
return the same term.\footnote{The name of the BIF comes from the fact that the
three integers normally represent the universal time (with microsecond resolution).
However, the time might be inaccurate if several calls are made within a microsecond.}
\index{now/0 BIF@\T{now/0} BIF|)}

\subsection{\T{throw/1}}

\label{section:throw1}
\index{throw/1 BIF@\T{throw/1} BIF|(}

A value is thrown.

\TYPE

\T{throw(term()) -> _}.

\EXITS

\T{throw/1} always completes abruptly with reason \T{\{'THROW',$\Z{v}_1$\}}.
\index{throw/1 BIF@\T{throw/1} BIF|)}

\subsection{\T{time/0}}

\label{section:time0}
\index{time/0 BIF@\T{time/0} BIF|(}

The local time of day when the BIF is called is returned as a triple.

\TYPE

\T{time() -> \{int(),int(),int()\}}.

\EXITS

\T{time/0} always completes normally.

\EVALUATION

Let the current local hour, minute and second at the time of evaluation be
$h$, $m$ and $s$.  The hour is on 24-hour format.

A triple \T{\{\Z{Hour},\Z{Minute},\Z{Second}\}} is returned, where
\TZ{Hour} is the \Erlang\ integer $\Er[h]$, \TZ{Minute} is
the \Erlang\ integer $\Er[m]$, and
\TZ{Second} is the \Erlang\ integer $\Er[s]$.

\EXAMPLES

\begin{itemize}
\item \T{time()} evaluated at five minutes and forty-two seconds
past midnight would return \T{\{0,5,42\}}.
\item \T{time()} evaluated at five minutes and forty-two seconds
past noon would return \T{\{12,5,42\}}.
\item \T{time()} evaluated at five minutes and forty-two seconds
before midnight would return \T{\{23,44,18\}}.
\end{itemize}
\index{time/0 BIF@\T{time/0} BIF|)}

\section{Reserved function names}

\label{section:reserved-function-names}
\index{function!reserved names|(}
\index{apply_lambda/2@\T{apply_lambda/2} reserved|(}
\index{module_info/0@\T{module_info/0}!reserved|(}
\index{module_info/1@\T{module_info/1}!reserved|(}
\index{module_lambdas/4 reserved@\T{module_lambdas/4} reserved|(}
\index{record/2@\T{record/2}!reserved|(}
\index{record_index/2@\T{record_index/2} reserved|(}
\index{record_info/2@\T{record_info/2} reserved|(}

The function names in Table~\ref{table:reserved-functions}
do not name BIFs but
are recognized by the compiler and
a module must not define any function with one of these names.

The reason may be that the compiler automatically generates
a definition of a function with such a name (e.g., \T{module_info/0}
and \T{module_info/1}), or that applications
of function named as such are treated specially (e.g., \T{record_info/2}).

Function names without a reference in Table~\ref{table:reserved-functions}
are reserved because
\ifStd it is known that existing implementations use \fi
\ifOld \OldErlang\ uses \fi
them for internal purposes.  If a module defines a function with one
of the names below, the compiler gives a compile-time error.

\begin{table}[hbp]
\begin{center}
\begin{tabular}{@{}ll@{}}
\hline
Function name & Described \\ \hline
\T{apply_lambda/2} \\
\T{module_info/0} & \S\ref{section:moduleinfo0} \\
\T{module_info/1} & \S\ref{section:moduleinfo1} \\
\T{module_lambdas/4} \\
\T{record/2} & \S\ref{section:record2} \\
\T{record_index/2} \\
\T{record_info/2} \\ \hline
\end{tabular}
\caption{Reserved function names}
\label{table:reserved-functions}
\end{center}
\end{table}
\index{function!reserved names|)}
\index{apply_lambda/2@\T{apply_lambda/2} reserved|)}
\index{module_info/0@\T{module_info/0}!reserved|)}
\index{module_info/1@\T{module_info/1}!reserved|)}
\index{module_lambdas/4 reserved@\T{module_lambdas/4} reserved|)}
\index{record/2@\T{record/2}!reserved|)}
\index{record_index/2@\T{record_index/2} reserved|)}
\index{record_info/2@\T{record_info/2} reserved|)}

\iffalse
\chapter{BIFs in the standard libraries}

% this is sort of \StdErlang\ stuff, except limits.
\section{The \T{atom} library}

\label{section:atom-library}

\subsection{\T{to_string/1}}

The printname of an atom is obtained as a string.

\TYPE

\T{atom:to_string(atom()) -> string()}.

\EXITS

\T{atom:to_string/1} exits with cause \T{badarg} if $\TZ{v}_1$ is not an atom.

\EVALUATION

A string representing the printname of the atom $\TZ{v}_1$ is returned.

\subsection{\T{from_string/1}}

Given a string, an atom with that printname is obtained.

\TYPE

\T{atom:from_string(string()) -> atom()}.

\EXITS

\T{atom:from_string/1} exits with cause \T{badarg} if $\TZ{v}_1$ is not a string.

\EVALUATION

An atom having the printname represented by the string $\TZ{v}_1$ is returned.

\subsection{\T{size/1}}

The length of the printname of an atom is returned.
\T{size/1} is a guard BIF.

\TYPE

\T{atom:size(atom()) -> int()}.

\EXITS

\T{atom:size/1} exits with cause \T{badarg} if $\TZ{v}_1$ is not an atom.

\EVALUATION

The size of the printname of the atom $\TZ{v}_1$ is returned.

\section{The \T{char} library}

\label{section:char-module}

\section{The \T{float} library}

\label{section:float-module}

\section{The \T{integer} library}

\label{section:integer-module}

\subsection{\T{to_string/1}}

The shortest decimal numeral denoting an integer
is obtained as a string.

\TYPE

\T{integer:to_string(int()) -> string()}.

\EXITS

\T{integer:to_string/1} exits with cause \T{badarg} if $\TZ{v}_1$ is not an integer.

\EVALUATION

Let $n$ be a numeral satisfying satisfies
the following axioms:
\begin{itemize}
\item $n$ interpreted as a decimal numeral denotes $\Re[\TZ{v}_1]$;
\item $n$ consists of decimal digits, except that the leftmost character
is \T{-} when $\Re[\TZ{v}_1]<0$;
\item the leftmost digit of $n$ is not \T{0} when $\Re[\TZ{v}_1]<>0$;
\item $n$ is \T{0} when $\Re[\TZ{v}_1]=0$.
\end{itemize}
$n$ is unique and is the shortest numeral
denoting the integer $\Re[\TZ{v}_1]$.

\T{integer:to_string/1} returns a string representing $n$.

\subsection{\T{from_string/1}}

Given a string that is a decimal numeral, the corresponding integer is obtained.

\TYPE

\T{integer:from_string(string()) -> int()}.

\EXITS

\T{integer:from_string/1} exits with cause \T{badarg} if $\TZ{v}_1$ is
not a string or if that string does not represent an
\NT{IntegerLiteral} (\S\ref{section:integer-literals}).

\EVALUATION

An \Erlang\ integer representing the number denoted by the \NT{IntegerLiteral}
is returned.

\section{The \T{list} library}

\label{section:list:nth2}

\section{The \T{math} library}

\label{section:math-library}

This library contains a collection of simple arithmetic functions
on floats.

\subsection{Void functions}

\addcontentsline{toc}{subsection}{\protect\numberline{}{\T{pi/0}}}

\TYPE

\T{math:pi() -> float()}.

\EXITS

\T{pi/0} always completes normally.

\EVALUATION

\begin{tabular}{@{}ll@{}}
\T{pi/0} & $\Re[\pi]$
\end{tabular}

\subsection{Unary functions}

\addcontentsline{toc}{subsection}{\protect\numberline{}{\T{cos/1}}}
\addcontentsline{toc}{subsection}{\protect\numberline{}{\T{cosh/1}}}
\addcontentsline{toc}{subsection}{\protect\numberline{}{\T{sin/1}}}
\addcontentsline{toc}{subsection}{\protect\numberline{}{\T{sinh/1}}}
\addcontentsline{toc}{subsection}{\protect\numberline{}{\T{tan/1}}}
\addcontentsline{toc}{subsection}{\protect\numberline{}{\T{tanh/1}}}
\addcontentsline{toc}{subsection}{\protect\numberline{}{\T{acos/1}}}
\addcontentsline{toc}{subsection}{\protect\numberline{}{\T{acosh/1}}}
\addcontentsline{toc}{subsection}{\protect\numberline{}{\T{asin/1}}}
\addcontentsline{toc}{subsection}{\protect\numberline{}{\T{asinh/1}}}
\addcontentsline{toc}{subsection}{\protect\numberline{}{\T{atan/1}}}
\addcontentsline{toc}{subsection}{\protect\numberline{}{\T{atanh/1}}}
\addcontentsline{toc}{subsection}{\protect\numberline{}{\T{erf/1}}}
\addcontentsline{toc}{subsection}{\protect\numberline{}{\T{erfc/1}}}
\addcontentsline{toc}{subsection}{\protect\numberline{}{\T{exp/1}}}
\addcontentsline{toc}{subsection}{\protect\numberline{}{\T{log/1}}}
\addcontentsline{toc}{subsection}{\protect\numberline{}{\T{log10/1}}}
\addcontentsline{toc}{subsection}{\protect\numberline{}{\T{sqrt/1}}}

\TYPE

\T{math:cos(num()) -> float()}, \\
\T{math:cosh(num()) -> float()}, \\
\T{math:sin(num()) -> float()}, \\
\T{math:sinh(num()) -> float()}, \\
\T{math:tan(num()) -> float()}, \\
\T{math:tanh(num()) -> float()}, \\
\T{math:acos(num()) -> float()}, \\
\T{math:acosh(num()) -> float()}, \\
\T{math:asin(num()) -> float()}, \\
\T{math:asinh(num()) -> float()}, \\
\T{math:atan(num()) -> float()}, \\
\T{math:atanh(num()) -> float()}, \\
\T{math:erf(num()) -> float()}, \\
\T{math:erfc(num()) -> float()}, \\
\T{math:exp(num()) -> float()}, \\
\T{math:log(num()) -> float()}, \\
\T{math:log10(num()) -> float()}, \\
\T{math:sqrt(num()) -> float()}.

\EXITS

TBW. % !!!

\EVALUATION

\begin{tabular}{@{}ll@{}}
\T{cos/1} & $\cos \TZ{v}_1$ \\
\T{cosh/1} & $\cosh \TZ{v}_1$ \\
\T{sin/1} & $\sin \TZ{v}_1$ \\
\T{sinh/1} & $\sinh \TZ{v}_1$ \\
\T{tan/1} & $\tan \TZ{v}_1$ \\
\T{tanh/1} & $\tanh \TZ{v}_1$ \\
\T{acos/1} & $\arccos \TZ{v}_1$ \\
\T{acosh/1} & hyperbolic $\arccos \TZ{v}_1$ \\
\T{asin/1} & $\arcsin \TZ{v}_1$ \\
\T{asinh/1} & hyperbolic $\arcsin \TZ{v}_1$ \\
\T{atan/1} & $\arctan \TZ{v}_1$ \\
\T{atanh/1} & hyperbolic $\arctan \TZ{v}_1$ \\
\T{erf/1} & error function of $\TZ{v}_1$ \\
\T{erfc/1} &  error function of $\TZ{v}_1$ \\
\T{exp/1} & $e^{\TZ{v}_1}$ \\
\T{log/1} & $\ln\TZ{v}_1$ \\
\T{log10/1} & $\log_{10}\TZ{v}_1$ \\
\T{sqrt/1} & $\sqrt{\TZ{v}_1}$
\end{tabular}

\subsection{Binary functions}

\addcontentsline{toc}{subsection}{\protect\numberline{}{\T{pow/2}}}
\addcontentsline{toc}{subsection}{\protect\numberline{}{\T{atan2/2}}}

\TYPE

\T{math:pow(num(),num()) -> float()}, \\
\T{math:atan2(num(),num()) -> float()}.

\EXITS

TBW. % !!!

\EVALUATION

\begin{tabular}{@{}ll@{}}
\T{pow/2} & $\TZ{v}_1^{\TZ{v}_1}$ \\
\T{atan2/2} & $\arctan \frac{\TZ{v}_1}{\TZ{v}_2}$
\end{tabular}

\section{The \T{string} library}
\fi

\iffalse
% Don't include this stuff either until we know.
\section{The \T{limits} library}

\label{section:limits-library}

\subsection{\T{atom_parameter/1}}

A parameter of the atom type of the implementation is returned
(cf.\ \S\ref{section:atoms}).

\TYPE

\T{limits:atom_parameter(maxatomlength) -> int()}.

\EXITS

\T{limits:atom_parameter/1} exits with cause \T{badarg} if $\TZ{v}_1$ is not an atom
or if it is not the atom \T{maxatomlength}.

\EVALUATION

$\TZ{v}_1$ must be \T{maxatomlength} and the BIF returns
$\Re[\mathit{maxatomlength}]$.

\subsection{\T{integer_parameter/1}}

A parameter of the integer type of the implementation is returned
(cf.\ \S\ref{section:integer-type}).

\TYPE

\T{limits:integer_parameter(bounded) -> bool()} ; \\
\T{limits:integer_parameter(minint) -> int()} ; \\
\T{limits:integer_parameter(maxint) -> int()} ; \\
\T{limits:integer_parameter(minfixnum) -> int()} ; \\
\T{limits:integer_parameter(maxfixnum) -> int()}.

\EXITS

\T{limits:integer_parameter/1} exits with cause \T{badarg} if $\TZ{v}_1$ is not an atom
or if it is not one of the atoms \T{bounded}, \T{minint}, \T{maxint},
\T{minfixnum} or \T{maxfixnum}.

\EVALUATION

The evaluation depends on $\TZ{v}_1$:
\begin{itemize}
\item \T{bounded}: return $\Re[\mathit{bounded}]$.
\item \T{minint}: return $\Re[\mathit{minint}]$.
\item \T{maxint}: return $\Re[\mathit{maxint}]$.
\item \T{minfixnum}: return $\Re[\mathit{minfixnum}]$.
\item \T{maxfixnum}: return $\Re[\mathit{maxfixnum}]$.
\end{itemize}

\subsection{\T{float_parameter/1}}

A parameter of the float type of the implementation is returned
(cf.\ \S\ref{section:float-type}).

\TYPE

\T{limits:float_parameter(radix) -> int()} ; \\
\T{limits:float_parameter(precision) -> int()} ; \\
\T{limits:float_parameter(emin) -> int()} ; \\
\T{limits:float_parameter(emax) -> int()} ; \\
\T{limits:float_parameter(denorm) -> bool()} ; \\
\T{limits:float_parameter(fmax) -> float()} ; \\
\T{limits:float_parameter(fmin) -> float()} ; \\
\T{limits:float_parameter(fmin_norm) -> float()} ; \\
\T{limits:float_parameter(epsilon) -> float()} ; \\
\T{limits:float_parameter(rnd_error) -> float()} ; \\
\T{limits:float_parameter(rnd_style) -> atom()} ; \\
\T{limits:float_parameter(iec_559) -> bool()}.

\EXITS

\T{limits:float_parameter/1} exits with cause \T{badarg} if $\TZ{v}_1$ is not an atom
or if it is not one of the atoms \T{radix}, \T{precision}, \T{emin},
\T{emax}, \T{denorm}, \T{fmax}, \T{fmin}, \T{fmin_norm}, \T{epsilon}, \T{rnd_error},
\T{rnd_style} or \T{iec_559}.

\EVALUATION

The evaluation depends on $\TZ{v}_1$:
\begin{itemize}
\item \T{radix}: return $\Re[r]$.
\item \T{precision}: return $\Re[p]$.
\item \T{emin}: return $\Re[\mathit{emin}]$.
\item \T{emax}: return $\Re[\mathit{emax}]$.
\item \T{denorm}: return $\Re[\mathit{denorm}]$.
\item \T{fmax}: return $\Re[\mathit{fmax}]$.
\item \T{fmin}: return $\Re[\mathit{fmin}]$.
\item \T{fmin_norm}: return $\Re[\mathit{fmin}_N]$.
\item \T{epsilon}: return $\Re[\mathit{epsilon}]$.
\item \T{rnd_error}: return $\Re[\mathit{rnd\_error}]$ (\S\ref{section:float-operations}).
\item \T{rnd_style}: return $\begin{cases}
\T{nearest} & \text{if $\I{rnd\_style}=\B{nearest}$,} \\
\T{truncate} & \text{if $\I{rnd\_style}=\B{truncate}$, and} \\
\T{other} & \text{if $\I{rnd\_style}=\B{other}$.}
\end{cases}$ (\S\ref{section:float-operations}).
\item \T{iec_559}: return $\Re[\mathit{iec\_559}]$ (\S\ref{section:arith-iec559}).
\end{itemize}

\subsection{\T{tuple_parameter/1}}

A parameter of the tuple type of the implementation is returned
(cf.\ \S\ref{section:atoms}).

\TYPE

\T{limits:tuple_parameter(maxtuplesize) -> int()}.

\EXITS

\T{limits:tuple_parameter/1} exits with cause \T{badarg} if $\TZ{v}_1$ is not an atom
or if it is not the atom \T{maxtuplesize}.

\EVALUATION

$\TZ{v}_1$ must be \T{maxtuplesize} and the BIF returns
$\Re[\mathit{maxtuplesize}]$.
\fi


\ifOld
%
% %CopyrightBegin%
%
% Copyright Ericsson AB 2017. All Rights Reserved.
%
% Licensed under the Apache License, Version 2.0 (the "License");
% you may not use this file except in compliance with the License.
% You may obtain a copy of the License at
%
%     http://www.apache.org/licenses/LICENSE-2.0
%
% Unless required by applicable law or agreed to in writing, software
% distributed under the License is distributed on an "AS IS" BASIS,
% WITHOUT WARRANTIES OR CONDITIONS OF ANY KIND, either express or implied.
% See the License for the specific language governing permissions and
% limitations under the License.
%
% %CopyrightEnd%
%

\chapter{Libraries}

\label{chapter:libraries}

There are some standard libraries that belong to \OldErlang.  Each library is one \Erlang\ module:

\begin{table}
\centering
\begin{tabular}{@{}ll@{}}
\hline
Module name & Contents \\ \hline
\T{file} & File handling \\
\T{io} & Simple input and output \\
\T{lists} & List functions \\
\T{math} & Logarithmic and arithmetic functions \\
\T{string} & String functions \\ \hline
\end{tabular}
\caption{Standard libraries of \OldErlang}
\label{table:standard-libraries}
\end{table}

Further standard libraries are provided with \OTP\ \cite{otp-dev-ref}.

\section{The \T{file} library}

TO BE WRITTEN!

\section{The \T{io} library}

TO BE WRITTEN!

\section{The \T{lists} library}

TO BE WRITTEN!

\section{The \T{math} library}

The library contains trigonometric and logarithmic functions.
All trigonometric functions work with angles in radians.
The functions correspond to the functions in the \T{math} facilities of ISO C
\cite{harbison+steele:c,iso-c}.

In the description of a function, $\TZ{v}_i$ refers to the value of the $i$th argument.


\subsection{\T{acos/1}}

\label{section:math:acos1}
\index{math:acos/1 function@\T{math:acos/1} function|(}
\index{acos/1 function of math library@\T{acos/1} function of \T{math} library|(}

Computes the \I{arccosine} function.

\TYPE

\T{acos(num()) -> float()}.

\EXITS

\T{acos/1} exits with cause \T{badarg} if $\TZ{v}_1$ is not a number.
\T{acos/1} exits with cause \T{badarith} if $\TZ{v}_1$ is not in the range $[-1,1]$.

\EVALUATION

The function returns $\mathit{result}_F(\arccos(\TZ{v}_1),\mathit{rnd}_F)$.
The result is always in the range $[0,\pi]$.
\index{math:acos/1 function@\T{math:acos/1} function|)}
\index{acos/1 function of math library@\T{acos/1} function of \T{math} library|)}


\iffalse
\subsection{\T{acosh/1}}

\label{section:math:acosh1}
\index{math:acosh/1 function@\T{math:acosh/1} function|(}
\index{acosh/1 function of math library@\T{acosh/1} function of \T{math} library|(}

Computes the \I{hyperbolic arccosine} function.

\TYPE

\T{acosh(num()) -> float()}.

\EXITS

\T{acosh/1} exits with cause \T{badarg} if $\TZ{v}_1$ is not a number.
% Other restrictions???

\EVALUATION

The function returns $\mathit{result}_F(\arccosh(\TZ{v}_1),\mathit{rnd}_F)$.
\index{math:acosh/1 function@\T{math:acosh/1} function|)}
\index{acosh/1 function of math library@\T{acosh/1} function of \T{math} library|)}
\fi


\subsection{\T{asin/1}}

\label{section:math:asin1}
\index{math:asin/1 function@\T{math:asin/1} function|(}
\index{asin/1 function of math library@\T{asin/1} function of \T{math} library|(}

Computes the \I{arcsine} function.

\TYPE

\T{asin(num()) -> float()}.

\EXITS

\T{asin/1} exits with cause \T{badarg} if $\TZ{v}_1$ is not a number.
\T{asin/1} exits with cause \T{badarith} if $\TZ{v}_1$ is not in the range $[-1,1]$.

\EVALUATION

The function returns $\mathit{result}_F(\arcsin(\TZ{v}_1),\mathit{rnd}_F)$.
The result is always in the range $[-\frac{\pi}{2},\frac{\pi}{2}]$.
\index{math:asin/1 function@\T{math:asin/1} function|)}
\index{asin/1 function of math library@\T{asin/1} function of \T{math} library|)}


\iffalse
\subsection{\T{asinh/1}}

\label{section:math:asinh1}
\index{math:asinh/1 function@\T{math:asinh/1} function|(}
\index{asinh/1 function of math library@\T{asinh/1} function of \T{math} library|(}

Computes the \I{hyperbolic arcsine} function.

\TYPE

\T{asinh(num()) -> float()}.

\EXITS

\T{asinh/1} exits with cause \T{badarg} if $\TZ{v}_1$ is not a number.
% Other restrictions???

\EVALUATION

The function returns $\mathit{result}_F(\arcsinh(\TZ{v}_1),\mathit{rnd}_F)$.
\index{math:asinh/1 function@\T{math:asinh/1} function|)}
\index{asinh/1 function of math library@\T{asinh/1} function of \T{math} library|)}
\fi


\subsection{\T{atan/1}}

\label{section:math:atan1}
\index{math:atan/1 function@\T{math:atan/1} function|(}
\index{atan/1 function of math library@\T{atan/1} function of \T{math} library|(}

Computes the \I{arctangent} function.

\TYPE

\T{atan(num()) -> float()}.

\EXITS

\T{atan/1} exits with cause \T{badarg} if $\TZ{v}_1$ is not a number.

\EVALUATION

The function returns $\mathit{result}_F(\arctan(\TZ{v}_1),\mathit{rnd}_F)$.
The result is always in the range $[-\frac{\pi}{2},\frac{\pi}{2}]$.
\index{math:atan/1 function@\T{math:atan/1} function|)}
\index{atan/1 function of math library@\T{atan/1} function of \T{math} library|)}


\subsection{\T{atan2/2}}

\label{section:math:atan22}
\index{math:atan2/2 function@\T{math:atan2/2} function|(}
\index{atan2/2 function of math library@\T{atan2/2} function of \T{math} library|(}

Computes the \I{arctangent} function of a quotient, taking the signs of the arguments
into account for determining the quadrant.

\TYPE

\T{atan2(num(), num()) -> float()}.

\EXITS

\T{atan2/2} exits with cause \T{badarg} if $\TZ{v}_1$ or $\TZ{v}_2$ is not a number.
\T{atan2/2} may exit with cause \T{badarith} if $\TZ{v}_1$ and $\TZ{v}_2$ are both zero.

\EVALUATION

The function returns $\mathit{result}_F(\arctan(\frac{\TZ{v}_1}{\TZ{v}_2}),\mathit{rnd}_F)$.
The result is always in the range $[-\frac{\pi}{2},\frac{\pi}{2}]$.
\begin{itemize}
\item If $\TZ{v}_2$ is zero, then the result is $\pi/2$ if $\TZ{v}_1$ is positive and $-\pi/2$ if
$\TZ{v}_1$ is negative.
\item If $\TZ{v}_1$ and $\TZ{v}_2$ are both positive,
\T{atan2($\Z{v}_1$,$\Z{v}_2$)} equals \T{atan($\Z{v}_1$/$\Z{v}_2$)}.
\item If $\TZ{v}_1$ is negative and $\TZ{v}_2$ is positive,
\T{atan2($\Z{v}_1$,$\Z{v}_2$)} equals \T{-atan((-$\Z{v}_1$)/$\Z{v}_2$)}.
\item If $\TZ{v}_1$ is positive and $\TZ{v}_2$ is negative,
\T{atan2($\Z{v}_1$,$\Z{v}_2$)} equals \T{math:pi()-atan($\Z{v}_1$/(-$\Z{v}_2$))}.
\item If $\TZ{v}_1$ and $\TZ{v}_2$ are both negative,
\T{atan2($\Z{v}_1$,$\Z{v}_2$)} equals \T{atan($\Z{v}_1$/$\Z{v}_2$)-math:pi()}.
\end{itemize}
\index{math:atan2/2 function@\T{math:atan2/2} function|)}
\index{atan2/2 function of math library@\T{atan2/2} function of \T{math} library|)}


\iffalse
\subsection{\T{atanh/1}}

\label{section:math:atanh1}
\index{math:atanh/1 function@\T{math:atanh/1} function|(}
\index{atanh/1 function of math library@\T{atanh/1} function of \T{math} library|(}

Computes the \I{hyperbolic arctangent} function.

\TYPE

\T{atanh(num()) -> float()}.

\EXITS

\T{atanh/1} exits with cause \T{badarg} if $\TZ{v}_1$ is not a number.
% Other restrictions???

\EVALUATION

The function returns $\mathit{result}_F(\arctanh(\TZ{v}_1),\mathit{rnd}_F)$.
\index{math:atanh/1 function@\T{math:atanh/1} function|)}
\index{atanh/1 function of math library@\T{atanh/1} function of \T{math} library|)}
\fi


\subsection{\T{cos/1}}

\label{section:math:cos1}
\index{math:cos/1 function@\T{math:cos/1} function|(}
\index{cos/1 function of math library@\T{cos/1} function of \T{math} library|(}

Computes the \I{cosine} function.

\TYPE

\T{cos(num()) -> float()}.

\EXITS

\T{cos/1} exits with cause \T{badarg} if $\TZ{v}_1$ is not a number.

\EVALUATION

The function returns $\mathit{result}_F(\cos(\TZ{v}_1),\mathit{rnd}_F)$.
\index{math:cos/1 function@\T{math:cos/1} function|)}
\index{cos/1 function of math library@\T{cos/1} function of \T{math} library|)}


\subsection{\T{cosh/1}}

\label{section:math:cosh1}
\index{math:cosh/1 function@\T{math:cosh/1} function|(}
\index{cosh/1 function of math library@\T{cosh/1} function of \T{math} library|(}

Computes the \I{hyperbolic cosine} function.

\TYPE

\T{cosh(num()) -> float()}.

\EXITS

\T{cosh/1} exits with cause \T{badarg} if $\TZ{v}_1$ is not a number.

\EVALUATION

The function returns $\mathit{result}_F(\cosh(\TZ{v}_1),\mathit{rnd}_F)$.
\index{math:cosh/1 function@\T{math:cosh/1} function|)}
\index{cosh/1 function of math library@\T{cosh/1} function of \T{math} library|)}

\iffalse
\subsection{\T{erf/1}}

\label{section:math:erf1}
\index{math:erf/1 function@\T{math:erf/1} function|(}
\index{erf/1 function of math library@\T{erf/1} function of \T{math} library|(}

Computes the ``error function''.

\TYPE

\T{erf(num()) -> float()}.

\EXITS

\T{erf/1} exits with cause \T{badarg} if $\TZ{v}_1$ is not a number.

\EVALUATION

The function returns $\mathit{result}_F(\mathop{\rm erf}\nolimits(\TZ{v}_1),\mathit{rnd}_F)$,
where \R{erf} is the ``error function''.
\index{math:erf/1 function@\T{math:erf/1} function|)}
\index{erf/1 function of math library@\T{erf/1} function of \T{math} library|)}


\subsection{\T{erfc/1}}

\label{section:math:erfc1}
\index{math:erfc/1 function@\T{math:erfc/1} function|(}
\index{erfc/1 function of math library@\T{erfc/1} function of \T{math} library|(}

Computes the ``error function correction''. % ???

\TYPE

\T{erfc(num()) -> float()}.

\EXITS

\T{erfc/1} exits with cause \T{badarg} if $\TZ{v}_1$ is not a number.

\EVALUATION

The function returns $\mathit{result}_F(\mathop{\rm erfc}\nolimits(\TZ{v}_1),\mathit{rnd}_F)$,
where \R{erfc} is the ``error function correction''.
\index{math:erfc/1 function@\T{math:erfc/1} function|)}
\index{erfc/1 function of math library@\T{erfc/1} function of \T{math} library|)}
\fi


\subsection{\T{exp/1}}

\label{section:math:exp1}
\index{math:exp/1 function@\T{math:exp/1} function|(}
\index{exp/1 function of math library@\T{exp/1} function of \T{math} library|(}

Computes exponentiation with base $e$.

\TYPE

\T{exp(num()) -> float()}.

\EXITS

\T{exp/1} exits with cause \T{badarg} if $\TZ{v}_1$ is not a number.

\EVALUATION

The function returns $\mathit{result}_F(e^{\TZ{v}_1},\mathit{rnd}_F)$.
\index{math:exp/1 function@\T{math:exp/1} function|)}
\index{exp/1 function of math library@\T{exp/1} function of \T{math} library|)}


\subsection{\T{log/1}}

\label{section:math:log1}
\index{math:log/1 function@\T{math:log/1} function|(}
\index{log/1 function of math library@\T{log/1} function of \T{math} library|(}

Computes logarithm with base $e$.

\TYPE

\T{log(num()) -> float()}.

\EXITS

\T{log/1} exits with cause \T{badarg} if $\TZ{v}_1$ is not a number.
\T{log/1} exits with cause \T{badarith} if $\TZ{v}_1\leq0$.

\EVALUATION

The function returns $\mathit{result}_F(\log_e(\TZ{v}_1),\mathit{rnd}_F)$.
\index{math:log/1 function@\T{math:log/1} function|)}
\index{log/1 function of math library@\T{log/1} function of \T{math} library|)}


\subsection{\T{log10/1}}

\label{section:math:log101}
\index{math:log10/1 function@\T{math:log10/1} function|(}
\index{log10/1 function of math library@\T{log10/1} function of \T{math} library|(}

Computes logarithm with base $10$.

\TYPE

\T{log10(num()) -> float()}.

\EXITS

\T{log10/1} exits with cause \T{badarg} if $\TZ{v}_1$ is not a number.
\T{log10/1} exits with cause \T{badarith} if $\TZ{v}_1\leq0$.

\EVALUATION

The function returns $\mathit{result}_F(\log_{10}(\TZ{v}_1),\mathit{rnd}_F)$.
\index{math:log10/1 function@\T{math:log10/1} function|)}
\index{log10/1 function of math library@\T{log10/1} function of \T{math} library|)}


\subsection{\T{pi/0}}

\label{section:math:pi0}
\index{math:pi/0 function@\T{math:pi/0} function|(}
\index{pi/0 function of math library@\T{pi/0} function of \T{math} library|(}

Returns an approximation of $\pi$.

\TYPE

\T{pi() -> float()}.

\EXITS

\T{pi/0} always completes normally.

\EVALUATION

$\mathit{result}_F(\pi,\mathit{rnd}_F)$ is returned.
\index{math:pi/0 function@\T{math:pi/0} function|)}
\index{pi/0 function of math library@\T{pi/0} function of \T{math} library|)}


\subsection{\T{pow/2}}

\label{section:math:pow2}
\index{math:pow/2 function@\T{math:pow/2} function|(}
\index{pow/2 function of math library@\T{pow/2} function of \T{math} library|(}

Computes exponentiation.

\TYPE

\T{pow(num(),num()) -> float()}.

\EXITS

\T{pow/2} exits with cause \T{badarg} if $\TZ{v}_1$ or $\TZ{v}_2$ is not a number.
\T{pow/2} exits with cause \T{badarith} if $\TZ{v}_1<0$ and $\TZ{v}_2$ is a float for which
the fraction part is not zero.

\EVALUATION

The function returns $\TZ{v}_1^{\TZ{v}_2},\mathit{rnd}_F)$.
\index{math:pow/1 function@\T{math:pow/1} function|)}
\index{pow/1 function of math library@\T{pow/1} function of \T{math} library|)}


\subsection{\T{sin/1}}

\label{section:math:sin1}
\index{math:sin/1 function@\T{math:sin/1} function|(}
\index{sin/1 function of math library@\T{sin/1} function of \T{math} library|(}

Computes the \I{sine} function.

\TYPE

\T{sin(num()) -> float()}.

\EXITS

\T{sin/1} exits with cause \T{badarg} if $\TZ{v}_1$ is not a number.

\EVALUATION

The function returns $\mathit{result}_F(\sin(\TZ{v}_1),\mathit{rnd}_F)$.
\index{math:sin/1 function@\T{math:sin/1} function|)}
\index{sin/1 function of math library@\T{sin/1} function of \T{math} library|)}


\subsection{\T{sinh/1}}

\label{section:math:sinh1}
\index{math:sinh/1 function@\T{math:sinh/1} function|(}
\index{sinh/1 function of math library@\T{sinh/1} function of \T{math} library|(}

Computes the \I{hyperbolic sine} function.

\TYPE

\T{sinh(num()) -> float()}.

\EXITS

\T{sinh/1} exits with cause \T{badarg} if $\TZ{v}_1$ is not a number.

\EVALUATION

The function returns $\mathit{result}_F(\sinh(\TZ{v}_1),\mathit{rnd}_F)$.
\index{math:sinh/1 function@\T{math:sinh/1} function|)}
\index{sinh/1 function of math library@\T{sinh/1} function of \T{math} library|)}


\subsection{\T{sqrt/1}}

\label{section:math:sqrt1}
\index{math:sqrt/1 function@\T{math:sqrt/1} function|(}
\index{sqrt/1 function of math library@\T{sqrt/1} function of \T{math} library|(}

Computes square roots.

\TYPE

\T{sqrt(num()) -> float()}.

\EXITS

\T{sqrt/1} exits with cause \T{badarg} if $\TZ{v}_1$ is not a number.
\T{sqrt/1} exits with cause \T{badarith} if $\TZ{v}_1<0$.

\EVALUATION

The function returns $\sqrt{\TZ{v}_1},\mathit{rnd}_F)$.
\index{math:sqrt/1 function@\T{math:sqrt/1} function|)}
\index{sqrt/1 function of math library@\T{sqrt/1} function of \T{math} library|)}


\subsection{\T{tan/1}}

\label{section:math:tan1}
\index{math:tan/1 function@\T{math:tan/1} function|(}
\index{tan/1 function of math library@\T{tan/1} function of \T{math} library|(}

Computes the \I{tangent} function.

\TYPE

\T{tan(num()) -> float()}.

\EXITS

\T{tan/1} exits with cause \T{badarg} if $\TZ{v}_1$ is not a number and may exit
with \T{badarith} if $cos(\TZ{v}_1)$ is zero.

\EVALUATION

The function returns $\mathit{result}_F(\tan(\TZ{v}_1),\mathit{rnd}_F)$.
\index{math:tan/1 function@\T{math:tan/1} function|)}
\index{tan/1 function of math library@\T{tan/1} function of \T{math} library|)}


\subsection{\T{tanh/1}}

\label{section:math:tanh1}
\index{math:tanh/1 function@\T{math:tanh/1} function|(}
\index{tanh/1 function of math library@\T{tanh/1} function of \T{math} library|(}

Computes the \I{hyperbolic tangent} function.

\TYPE

\T{tanh(num()) -> float()}.

\EXITS

\T{tanh/1} exits with cause \T{badarg} if $\TZ{v}_1$ is not a number.

\EVALUATION

The function returns $\mathit{result}_F(\tanh(\TZ{v}_1),\mathit{rnd}_F)$.
\index{math:tanh/1 function@\T{math:tanh/1} function|)}
\index{tanh/1 function of math library@\T{tanh/1} function of \T{math} library|)}

\section{The \T{string} library}

TO BE WRITTEN!

\fi

\bibliographystyle{plain}

\bibliography{es}

\appendix

\appendixtrue

%\part{Appendices}

%
% %CopyrightBegin%
%
% Copyright Ericsson AB 2017. All Rights Reserved.
%
% Licensed under the Apache License, Version 2.0 (the "License");
% you may not use this file except in compliance with the License.
% You may obtain a copy of the License at
%
%     http://www.apache.org/licenses/LICENSE-2.0
%
% Unless required by applicable law or agreed to in writing, software
% distributed under the License is distributed on an "AS IS" BASIS,
% WITHOUT WARRANTIES OR CONDITIONS OF ANY KIND, either express or implied.
% See the License for the specific language governing permissions and
% limitations under the License.
%
% %CopyrightEnd%
%

\chapter{Summary of \Erlang\ expressions}

\label{chapter:summary-exprs}

This is a summary of the expressions of \Erlang.  Each group of
expressions is annotated with a precedence.
\begin{itemize}
\item $\alpha^{+}$ means a nonempty comma-separated sequence of $\alpha$.
\item $\alpha^{@}$ means a nonempty semicolon-separated sequence of $\alpha$.
\item $\alpha\mid\beta$ means either $\alpha$ or $\beta$.
\item $[\alpha]$ means $\alpha$ or nothing.
\item $(\alpha)$ means $\alpha$.
\item $\TZ{E}_p$ means an expression with precedence $p$ or less.
\item \TZ{A} means an atom.
\item \TZ{A} means an integer literal.
\item \TZ{L} means an atomic literal.
\item \TZ{V} means a variable.
\item \TZ{P} means a pattern.
\item \TZ{G} means a guard.
\end{itemize}

\begin{center}
\begin{tabular}{@{}rlllll@{}}
\hline
10 & \multicolumn{2}{l}{\T{catch $\Z{E}_{12}$}} & & & \S\ref{section:catch} \\ \hline
 9 & \T{\Z{P} = $\Z{E}_{\ifStd10\fi\ifOld11\fi}$} & & & & \S\ref{section:match-expr} \\ \hline
 8 & \T{$\Z{E}_9$ = $\Z{E}_{10}$} & & & & \S\ref{section:send-expr} \\ \hline
 7 & \T{$\Z{E}_6$ < $\Z{E}_6$} & \T{$\Z{E}_6$ =< $\Z{E}_6$} & \T{$\Z{E}_6$ > $\Z{E}_6$} & \T{$\Z{E}_6$ >= $\Z{E}_6$} & \S\ref{section:relational} \\
   & \T{$\Z{E}_6$ =:= $\Z{E}_6$} & \T{$\Z{E}_6$ =/= $\Z{E}_6$} & \T{$\Z{E}_6$ == $\Z{E}_6$} & \T{$\Z{E}_6$ /= $\Z{E}_6$} & \S\ref{section:relational} \\ \hline
 6 & \T{$\Z{E}_5$ ++ $\Z{E}_6$} & \T{$\Z{E}_5$ -- $\Z{E}_6$} & & & \S\ref{section:listconc-exprs} \\ \hline
 5 & \T{$\Z{E}_5$ + $\Z{E}_4$} & \T{$\Z{E}_5$ - $\Z{E}_4$} & \T{$\Z{E}_5$ bor $\Z{E}_4$} & \T{$\Z{E}_5$ bxor $\Z{E}_4$} & \S\ref{section:additive} \\
   & \T{$\Z{E}_5$ bsl $\Z{E}_4$} & \T{$\Z{E}_5$ bsr $\Z{E}_4$} & & & \S\ref{section:shift} \\
   & \T{$\Z{E}_9$ or $\Z{E}_8$} & \T{$\Z{E}_9$ xor $\Z{E}_8$} & & &\S\ref{section:additive} \\ \hline
 4 & \T{$\Z{E}_4$ * $\Z{E}_3$} & \T{$\Z{E}_4$ / $\Z{E}_3$} & & & \S\ref{section:multiplicative} \\
   & \ifStd \T{$\Z{E}_4$ // $\Z{E}_3$} & \fi \T{$\Z{E}_4$ div $\Z{E}_3$} & \ifStd \T{$\Z{E}_4$ mod $\Z{E}_3$} & \fi \T{$\Z{E}_4$ rem $\Z{E}_3$} & \ifOld & & \fi \S\ref{section:multiplicative} \\
   & \T{$\Z{E}_4$ band $\Z{E}_3$} & & & & \S\ref{section:multiplicative} \\
   & \T{$\Z{E}_8$ and $\Z{E}_7$} & & & & \S\ref{section:multiplicative} \\ \hline
 3 & \T{+ $\Z{E}_2$} & \T{- $\Z{E}_2$} & \T{bnot $\Z{E}_2$} & \T{not $\Z{E}_2$} & \S\ref{section:unary} \\ \hline
 2 & \T{\char`\#\Z{A}.\Z{A}} & \multicolumn{3}{l}{\T{\char`\#\Z{A}\{$(\text{\T{\Z{A}=$\Z{E}_{12}$}})^{+}$\}}} & \S\ref{section:record-exprs} \\
   & \T{$\Z{E}_2$\char`\#\Z{A}.\Z{A}} & \multicolumn{3}{l}{\T{$\Z{E}_2$\char`\#\Z{A}\{$(\text{\T{\Z{A}=$\Z{E}_{12}$}})^{+}$\}}} & \S\ref{section:record-exprs} \\ \hline
 1 & \T{\Z{A}($\Z{E}_{12}^{+}$)} & \T{$\Z{E}_0$($\Z{E}_{12}^{+}$)} & \T{$\Z{E}_0$:$\Z{E}_0$($\Z{E}_{12}^{+}$)} & & \S\ref{section:application-exprs} \\ \hline
 0 & \multicolumn{4}{l}{\T{\Z{V}}} & \S\ref{section:variables-eval} \\
   & \multicolumn{4}{l}{\T{\Z{L}}} & \S\ref{section:atomic-literals} \\
   & \multicolumn{4}{l}{\T{\{$\Z{E}_{12}^{+}$\}}} & \S\ref{section:tuple-skeletons} \\
   & \T{[]} & \multicolumn{3}{l}{\T{[$\Z{E}_{12}^{*}$|$\Z{E}_{12}$]}} & \S\ref{section:list-skeletons} \\
   & \multicolumn{4}{l}{\T{[$\Z{E}_{12}$||$(\text{\T{\Z{P}<-$\Z{E}_{12}$}}\mid\Z{E}_{12})^{+}$]}} & \S\ref{section:list-comprehensions} \\
   & \multicolumn{4}{l}{\T{begin $\Z{E}_{12}^{+}$ end}} & \S\ref{section:block-exprs} \\
\ifStd
   & \multicolumn{4}{l}{\T{all_true $\Z{E}_{12}^{+}$ end}} & \S\ref{section:alltrue-exprs} \\
   & \multicolumn{4}{l}{\T{some_true $\Z{E}_{12}^{+}$ end}} & \S\ref{section:sometrue-exprs} \\
   & \multicolumn{4}{l}{\T{cond $(\text{\T{$\Z{E}_{12}$ -> $\Z{E}_{12}^{+}$}})^{@}$ end}} & \S\ref{section:cond-expr} \\
\fi
   & \multicolumn{4}{l}{\T{if $(\text{\T{\Z{G} -> $\Z{E}_{12}^{+}$}})^{@}$ end}} & \S\ref{section:if-expr} \\
   & \multicolumn{4}{l}{\T{case $\Z{E}_{12}$ of $(\text{\T{\Z{P} $[\text{\T{when \Z{G}}}]$ -> $\Z{E}_{12}^{+}$}})^{@}$ end}} & \S\ref{section:case-expr} \\
   & \multicolumn{4}{l}{\T{receive $(\text{\T{\Z{P} $[\text{\T{when \Z{G}}}]$ -> $\Z{E}_{12}^{+}$}})^{@}$ $[\text{\T{after $\Z{E}_{12}$ -> $\Z{E}_{12}^{+}$}}]$ end}} & \S\ref{section:receive-expr} \\
\ifStd
   & \multicolumn{4}{l}{\T{try $\Z{E}_{12}^{+}$ catch $(\text{\T{\Z{P} $[\text{\T{when \Z{G}}}]$ -> $\Z{E}_{12}^{+}$}})^{@}$ end}} & \S\ref{section:try-expr} \\
\fi
   & \multicolumn{4}{l}{\T{fun \Z{A}/\Z{I}}} & \S\ref{section:fun-exprs} \\
   & \multicolumn{4}{l}{\T{fun $(\text{\T{($\Z{P}^{+}$) $[\text{\T{when \Z{G}}}]$ -> $\Z{E}_{12}^{+}$}})^{@}$ end}} & \S\ref{section:fun-exprs} \\
\ifOld
   & \multicolumn{4}{l}{\T{query [$\Z{E}_{12}$||$(\text{\T{\Z{P}<-$\Z{E}_{12}$}}\mid\Z{E}_{12})^{+}$] end}} & \S\ref{section:query-exprs} \\
\fi
   & \multicolumn{4}{l}{\T{($\Z{E}_{12}$)}} & \S\ref{section:paren-expr} \\ \hline
\end{tabular}
\end{center}


\ifStd
%
% %CopyrightBegin%
%
% Copyright Ericsson AB 2017. All Rights Reserved.
%
% Licensed under the Apache License, Version 2.0 (the "License");
% you may not use this file except in compliance with the License.
% You may obtain a copy of the License at
%
%     http://www.apache.org/licenses/LICENSE-2.0
%
% Unless required by applicable law or agreed to in writing, software
% distributed under the License is distributed on an "AS IS" BASIS,
% WITHOUT WARRANTIES OR CONDITIONS OF ANY KIND, either express or implied.
% See the License for the specific language governing permissions and
% limitations under the License.
%
% %CopyrightEnd%
%

\chapter{Summary of differences between \OldErlang\ and \StdErlang}

\begin{Lentry}
\item[\ifStd\S\ref{section:unicode}\fi\ifOld\S\ref{chapter:lexical}\fi]
\OldErlang\ uses the ASCII character set, in which character codes are 7 bits wide.
In \StdErlang\ the character set is Unicode, in which character codes are 16 bits wide.
\NewErlang\
would be expected to support at least the 256 first Unicode characters, which coincide with
Latin-1.

This leads to minor differences between \OldErlang\ and \StdErlang\
in many parts of the specification where characters are mentioned.

Also there are Unicode escape sequences in \StdErlang\ but not in \OldErlang.

\item[\S\ref{section:keywords}]
\StdErlang\ has three keywords that \OldErlang\ does not: \T{all_true}, \T{some_true}
and \T{try}.

\item[\S\ref{section:operators}]
\StdErlang\ has two operators that \OldErlang\ does not: \T{//} and \T{mod}.

\item[\S\ref{section:escapes}]
\StdErlang\ has an escape sequence that \OldErlang\ does not: \T{\char`\\s} for space.

In \OldErlang, an octal escape may be from \T{\char`\\000} to \T{\char`\\777}
while in \StdErlang, only bytes can be expressed, i.e.,
\T{\char`\\000} to \T{\char`\\377}.

\item[\S\ref{section:integer-literals}]
In \StdErlang\ integer literals may contain underscore characters to separate groups of digits.

\item[\S\ref{section:float-literals}]
In \StdErlang\ float literals may contain underscore characters to separate groups of digits.

\item[\S\ref{section:char-literals}]
In \OldErlang\ a \T{\char`\$} character may be followed by any character and its value would be
the code of that character.  In \StdErlang\ a \T{\char`\$} character must not be followed by
whitespace.

\item[\S\ref{section:string-literals}]
In \OldErlang\ a control character may be included in a string literal.
In \StdErlang\ it must not.

\item[\S\ref{section:atoms}]
In \StdErlang\ the maximum length of atoms' printnames is an implementation-dependent
constant while in \OldErlang\ it is always 255.

\item[\S\ref{section:chars}]
In \StdErlang\ there is a separate character type while in \OldErlang, integers in the
range $[0,127]$ serve this purpose.

\item[\S\ref{section:functions}]
In \OldErlang, functions are transparently represented by tuples, while in
\StdErlang\ they must be a proper type.

\item[\S\ref{chapter:arithmetics}]
In \StdErlang, the arithmetics have been generalized, as compared with
\OldErlang, in the sense that there are
various requirements on the arithmetics but few exact prescriptions.
For example, an implementation may
or may not have bignums and the floating-point representation can vary
between implementations (and on platforms). The international standard LIA-1 has been followed in
the description.

In \OldErlang, the exit cause for all arithmetic errors is \T{badarith} (except that
a few BIFs will give \T{badarg} instead) while in
\StdErlang\ there are four exit causes, depending on the problem.

\item[\S\ref{section:scope}]
The scope of variables is different in \OldErlang\ and \StdErlang. In \StdErlang\
the binding occurrence can be determined at compile-time while in \OldErlang\ it
cannot, in general.  This forces \OldErlang\ to verify that for \emph{any} order of
evaluation that can be chosen, no applied variable occurrence will be unbound.

\item[\S\ref{section:evorder}]
In \OldErlang\ it is not defined in which order the subexpressions of an expression
will be evaluated, except that a body is evaluated from left to right.  In \StdErlang\
there is a left-to-right order also for arguments of function applications, list
literals, etc.

\item[\S\ref{section:patterns}]
In \StdErlang\ there is an additional pattern \T{$\Z{P}_1$ = $\Z{P}_2$} that will
match a term if, and only if, both $\TZ{P}_1$ and $\TZ{P}_2$ match it.

\item[\S\ref{section:function-application}]
The order in which shadowing will be resolved is different in \OldErlang\ and
\StdErlang.

\item[\S\ref{section:appl-unnamed-function}]
If an application of an explicit \T{fun} expression completes
abruptly because there is no clause with a matching pattern and a successful
guard, it exits with reason
\T{\{lambda_clause,\{\Z{Mod},\Z{T},[$\Z{v}_1$,\tdots,$\Z{v}_{\TZm{n}}$]\}\}}
in \StdErlang\ but
\T{\{lambda_clause,\{\Z{Mod},[$\Z{v}_1$,\tdots,$\Z{v}_{\TZm{n}}$]\}\}}
in \OldErlang.  \TZ{T} is an implementation-defined term.

\item[\S\ref{section:catch}]
In \OldErlang\ \T{catch} expressions are the only expressions for restoring
normal mode of evaluation.

\item[\ifStd\S\ref{section:intdiv-f}\fi\ifOld\S\ref{section:multiplicative}\fi]
In \StdErlang\ there is an operator \T{//} for integer division with rounding
towards negative infinity.

\item[\ifStd\S\ref{section:intmod}\fi\ifOld\S\ref{section:multiplicative}\fi]
In \StdErlang\ there is an operator \T{mod} for integer remainder with rounding
towards negative infinity.

\item[\S\ref{section:unaryplus}]
In \OldErlang\ the unary \T{+} operator can be applied to a term of any type.
In \StdErlang\ it can only be applied to numbers.

\item[\S\ref{section:tuple-skeletons}]
In \StdErlang\ there is no restriction on the size of tuple skeletons while in
\OldErlang\ they can have at most 255 elements.

\item[\ifStd\S\ref{section:alltrue-exprs}\fi\ifOld\S\ref{section:primary-exprs}\fi]
In \StdErlang\ there are expressions \T{all_true $\Z{E}_1$ ; \tdots ; $\Z{E}_k$ end}
for expressing nonstrict conjunction.

\item[\ifStd\S\ref{section:sometrue-exprs}\fi\ifOld\S\ref{section:primary-exprs}\fi]
In \StdErlang\ there are expressions \T{some_true $\Z{E}_1$ ; \tdots ; $\Z{E}_k$ end}
for expressing nonstrict disjunction.

\item[\ifStd\S\ref{section:cond-expr}\fi\ifOld\S\ref{section:primary-exprs}\fi]
In \StdErlang\ there are \T{cond} expressions
for expressing \T{if}-like choices but where the conditions are arbitrary Boolean
expressions rather than guards.

\item[\ifStd\S\ref{section:try-expr}\fi\ifOld\S\ref{section:primary-exprs}\fi]
In \StdErlang\ there are \T{try} expressions that are more powerful than
\T{catch} expressions for capturing abrupt completion.

\T{try} expressions have the following advantages over \T{catch} expressions:
\begin{itemize}
\item It is possible to distinguish safely between the three possible
outcomes of evaluating an expression:
\begin{itemize}
\item Normal completion.
\item Abrupt completion due to a \T{throw}.
\item Abrupt completion due to an \T{exit} (possibly caused by an error).
\end{itemize}
As an example, here is how to define a function \T{foo_with_log/1} which
is exactly like a function \T{foo/1} except that on abrupt completion,
the reason is logged:
\begin{verbatim}
foo_with_log(X) ->
    try
        foo(X)
    catch
        {'THROW',T} ->
            io:format(logger,
                      "foo(~w) threw ~w.\n",
                      [X,T]),
            throw(T) ;
        {'EXIT',E} ->
            io:format(logger,
                      "foo(~w) exited due to ~w.\n",
                      [X,E]),
            exit(E)
    end.
\end{verbatim}
\item It is possible to catch only certain thrown terms or exits.  For example,
in the following function, throws of a particular reference \T{BadSyntax} are
caught but any other thrown terms or exits still cause abrupt completion of the
function and can be caught outside it:
\begin{verbatim}
read_entry_safely(Device) ->
    BadSyntax = make_ref(),
    try
        read_entry(Device,BadSyntax)
    catch
        {'THROW',BadSyntax} ->
            recover(Device),
            read_entry_safely(Device)
    end.
\end{verbatim}
The function above shows a useful programming idiom.
\item \T{try} expressions do not have the syntactic inconvenience of \T{catch}
expressions (that many expressions have to be enclosed in parentheses when
being the argument of \T{catch} and that one has to use \T{begin} \ldots\ \T{end}
in order to give a body to \T{catch}).
\end{itemize}

\item[\S\ref{section:signals}]
In the signal model of \OldErlang, there are three kinds of signals that are not included
in \StdErlang: \I{Suspend/resume request}, \I{Garbage collection request} and
\I{Trace/notrace request}.

\item[\S\ref{section:process-state-dynamic}]
In \StdErlang\ it is possible for a process to \emph{monitor} another. It is like
monitoring of nodes but for processes.  It is like linking but one-way.

\item[\S\ref{section:magic-cookie}]
In \OldErlang\ the magic cookie mechanism is used only for messages, not for other communication.
In \StdErlang\ all communication between nodes is subject to magic cookies.

\item[\S\ref{section:opening-ports}]
In \OldErlang\ the first argument to the BIF \T{open_port/2} can be an atom or
a string.  This is not included in \StdErlang\ (but \NewErlang\ may well inlude it).

\item[\S\ref{chapter:bifs}]
There are (proposed) changes to the exit causes.  For example:

In \StdErlang, only if abnormal completion of a BIF was because there was something
wrong with the
value of an argument, e.g., is was of the wrong type or it was an index
outside the meaningful range, the exit cause will be \T{badarg}.  In \OldErlang,
\T{badarg} is given for various conditions.

In \StdErlang\ some BIFs will give a new exit \T{badindex} if they are given an
index argument that is not in the permitted range.

\item[\S\ref{section:recognizer-bifs}]
In \StdErlang\ the recognizer BIFs all have names that begin with \T{is_} and there
have been some additions and changes in the repertoire.

\item[\ifStd\S\ref{section:sign1}\fi\ifOld\S\ref{section:number-bifs}\fi]
In \StdErlang\ there is a new BIF \T{sign/1}.

\item[\S\ref{section:termtobinary1}]
The presence of functions in \StdErlang\ and the fact that they cannot be portably
communicated means that \T{term_to_binary/1} may exit if given a term containing a
function.

\item[\ifOld\S\ref{section:listtopid1}\fi\ifStd\S\ref{section:process-bifs}\fi]
The BIFs \T{list_to_pid/1} and \T{pid_to_list/1} are not part of \StdErlang\
(but may well be included in \NewErlang).

\item[\S\ref{section:atom-tables}]
In \StdErlang\ the size of atom tables is an implementation-dependent constant
whereas in \OldErlang\ it is always 256.
\end{Lentry}


%
% %CopyrightBegin%
%
% Copyright Ericsson AB 2017. All Rights Reserved.
%
% Licensed under the Apache License, Version 2.0 (the "License");
% you may not use this file except in compliance with the License.
% You may obtain a copy of the License at
%
%     http://www.apache.org/licenses/LICENSE-2.0
%
% Unless required by applicable law or agreed to in writing, software
% distributed under the License is distributed on an "AS IS" BASIS,
% WITHOUT WARRANTIES OR CONDITIONS OF ANY KIND, either express or implied.
% See the License for the specific language governing permissions and
% limitations under the License.
%
% %CopyrightEnd%
%

\chapter{Implementation constants}

This appendix summarizes the constants that characterize a \StdErlang\ implementation
and requirements on the values for these constants.  An implementation should
document the values, most of which are also available to programs.

\section*{Atoms}

The constant
\I{maxatomlength} (\S\ref{section:atoms}) must be at least $2^8-1$ ($255$)
and at most $2^{16}-1$ ($65\,535$).

The constant $\mathit{atom\_table\_size}$ gives
the size of the atom tables used when transforming terms to and from
the external representation (\S\ref{section:atom-tables}).
It must be at least $2^8$ ($256$).

\section*{Integers}

The integers in an implementation of \StdErlang\ are characterized by
five constants $\mathit{bounded}$, $\mathit{maxint}$, $\mathit{minint}$,
$\mathit{minfixnum}$ and $\mathit{maxfixnum}$
(\S\ref{section:integer-type}):
\begin{itemize}
\item There is no requirement on the constant $\mathit{bounded}$.
\item The constant $\mathit{maxint}$ is only relevant if $\it{bounded}=\B{true}$,
in which case $\mathit{maxint}$ must be at least $2^{59}-1$ (576\,460\,752\,303\,423\,487).
\item The only requirement on the constant $\mathit{maxfixnum}$ is the obvious
condition that $0 < \I{maxfixnum} \leq \I{maxint}$.
\item Either
$\I{minint} = -(\I{maxint}+1)$ or $\I{minint} = -\I{maxint}$ must hold.
\end{itemize}

\section*{Floats}

The floating-point numbers in an implementation of \StdErlang\ are characterized by
five constants $r$, $p$, $\mathit{emin}$, $\mathit{emax}$ and $\mathit{denorm}$
(\S\ref{section:float-type}):
\begin{itemize}
\item The radix $r$ should be even.
\item The precision $p$ should be such that $r^{p-1}\geq 10^6$.
\item For the constants $\mathit{emin}$ and $\mathit{emax}$ it should hold that
$\mathit{emin}-1 \leq k*(p-1)$ and  $\mathit{emax} > k*(p-1)$,
with $k\geq 2$ and $k$ as large an integer as practical, and that
$-2 \leq (emin-1) + emax \leq 2$.
\item There is no requirement on the constant $\mathit{denorm}$.
\end{itemize}

\section*{Refs}

The refs in an implementation are characterized by two constants:
\I{refs\_bounded} and \I{maxrefs} (\S\ref{section:refs}).
If \I{refs\_bounded} is \B{true},
then the value of \I{maxrefs} must be at least XXX.

\section*{PIDs}

The PIDs in an implementation are characterized by two constants:
\I{pids\_bounded} and \I{maxpids} (\S\ref{section:pids}).
If \I{pids\_bounded} is \B{true},
then the value of \I{maxpids} must be at least XXX.

\section*{Ports}

The ports in an implementation are characterized by two constants:
\I{ports\_bounded} and \I{maxports} (\S\ref{section:ports}).
If \I{ports\_bounded} is \B{true},
then the value of \I{maxports} must be at least XXX.

\section*{Tuples}

The constant \I{maxtuplesize} (\S\ref{section:tuples})
must be at least $2^{16}-1$ ($65\,535$) and at most
$2^{32}-1$ ($4\,294\,967\,296$).

\section*{Scheduling}

The constant \I{normal\_advantage} (\S\ref{section:scheduling})
should be between $4$ and $32$.

\fi

%
% %CopyrightBegin%
%
% Copyright Ericsson AB 2017. All Rights Reserved.
%
% Licensed under the Apache License, Version 2.0 (the "License");
% you may not use this file except in compliance with the License.
% You may obtain a copy of the License at
%
%     http://www.apache.org/licenses/LICENSE-2.0
%
% Unless required by applicable law or agreed to in writing, software
% distributed under the License is distributed on an "AS IS" BASIS,
% WITHOUT WARRANTIES OR CONDITIONS OF ANY KIND, either express or implied.
% See the License for the specific language governing permissions and
% limitations under the License.
%
% %CopyrightEnd%
%

\chapter{Parse trees}

\label{chapter:parse-trees}
\index{parse tree!standard representation|(}

In this appendix we define the standard representation of parse trees
for \Erlang\ programs as \Erlang\ terms.  A
\T{parse_transform/1}\index{parse_transform/1
function@\T{parse_transform/1} function}
(\S\ref{section:parse-transform}) takes a list of such terms as input
and is expected to return a list of such terms.  We define the
representation in a top-down way by first examining the forms that
make up a module declaration and then going through their lexical
constituents, etc.

The representation admits representation of many parse trees that
would be rejected by the grammar of \Erlang.  For example, it allows
representation of a module declaration in which forms are not in a
permitted order.

\index{Rep@\Rep|(}
We define the representation semi-formally through a function \Rep\
that maps \Erlang\ formulas or sequences of formulas to \Erlang\
terms.

\index{L@\LINE|(}
When we write \LINE, we mean an \Erlang\ integer that may be interpreted by
software tools as a line number when reporting errors, etc.
\index{L@\LINE|)}

\section{Module declarations and forms}

A module declaration consists of a sequence of forms that are either
function declarations or attributes
(\S\ref{section:module-declarations}).
\begin{itemize}
\item If \TZ{D} is a module declaration consisting of the forms
$\TZ{F}_1$, \ldots, $\TZ{F}_k$, then
\[\Rep(\TZ{D}) = \text{\T{[$\Rep(\Z{F}_1)$,\tdots,$\Rep(\Z{F}_k)$]}}.\]
\item If \TZ{F} is an attribute \T{-module(\Z{Mod})}, then
\[\Rep(\TZ{F}) = \text{\T{\{attribute,\LINE,module,\Z{Mod}\}}}.\]
\item If \TZ{F} is an attribute \T{-export([$\Z{Fun}_1$/$\Z{A}_1$,\tdots,$\Z{Fun}_k$/$\Z{A}_k$])}, then
\[\Rep(\TZ{F}) = \text{\T{\{attribute,\LINE,export,[\{$\Z{Fun}_1$,$\Z{A}_1$\},\tdots,\{$\Z{Fun}_k$,$\Z{A}_k$\}]\}}}.\]
\item If \TZ{F} is an attribute \T{-import(\Z{Mod},[$\Z{Fun}_1$/$\Z{A}_1$,\tdots,$\Z{Fun}_k$/$\Z{A}_k$])},
then
\begin{align*}
\Rep(\TZ{F}) =
\text{\T{\{attribute,\LINE,import,\{\Z{Mod},[}}&\text{\T{\{$\Z{Fun}_1$,$\Z{A}_1$\},\tdots,}} \\
                                               &\text{\T{\{$\Z{Fun}_k$,$\Z{A}_k$\}]\}\}}.}
\end{align*}
\item If \TZ{F} is an attribute \T{-compile([$\Z{T}_1$,\tdots,$\Z{T}_k$])}, then
\[\Rep(\TZ{F}) = \text{\T{\{attribute,\LINE,compile,[$\Z{T}_1$,\tdots,$\Z{T}_k$]\}}}.\]
\item If \TZ{F} is an attribute \T{-file(\Z{File},\Z{Line})}, then
\[\Rep(\TZ{F}) = \text{\T{\{attribute,\LINE,file,\{$\Rep(\Z{File})$,$\Rep(\Z{Line})$\}\}}}.\]
\item If \TZ{F} is a record declaration \T{-record(\Z{Name},\{$\Z{V}_1$,\tdots,$\Z{V}_k$\})}, then
\begin{align*}
\Rep(\TZ{F}) =
\text{\T{\{attribute,\LINE,record,\{\Z{Name},[}}&\text{\T{$\Rep(\Z{V}_1)$,\tdots,}} \\
                                                &\text{\T{$\Rep(\Z{V}_k)$]\}\}}.}
\end{align*}
\item If \TZ{F} is a wild attribute \T{-\Z{A}(\Z{T})}, then
\[\Rep(\TZ{F}) = \text{\T{\{attribute,\LINE,\Z{A},\Z{T}\}}}.\]
\item If \TZ{F} is a function declaration \T{\Z{Name}($\Z{Ps}_1$) $[\text{\T{when $\Z{Gs}_1$}}]$ -> $\Z{B}_1$ ; \tdots ; \Z{Name}($\Z{Ps}_k$) $[\text{\T{when $\Z{Gs}_k$}}]$ -> $\Z{B}_k$ end},
where $k\geq1$ and each $\Z{Ps}_i$, $\Z{Gs}_i$ and $\Z{B}_i$, $1\leq i\leq k$,
is a pattern sequence, a guard and a body, respectively (and $\Z{Gs}_i$ is \T{true} if
omitted), and each $\Z{Ps}_i$, $1\leq i\leq k$, has the same length \Z{Arity}, then
\begin{align*}
\Rep(\TZ{F}) =
\text{\T{\{}}&\text{\T{function,\LINE,\Z{Name},\Z{Arity},}} \\
             &\text{\T{[\{clause,\LINE,$\Rep(\Z{Ps}_1)$,$\Rep(\Z{Gs}_1)$,$\Rep(\Z{B}_1)$\},\tdots,}} \\
             &\text{\T{~\{clause,\LINE,$\Rep(\Z{Ps}_k)$,$\Rep(\Z{Gs}_k)$,$\Rep(\Z{B}_k)$\}]\}}.}
\end{align*}
\end{itemize}
In addition to the representations of forms, the list that represents
a module declaration may contain tuples \T{\{error,\Z{E}\}}, denoting
syntactically incorrect forms, and \T{\{eof,\LINE\}}, denoting an end
of stream encountered before a complete form had been parsed.

Each field declaration in a record declaration is with or without an
explicit default initializer expression
(\S\ref{section:record-declarations}).
\begin{itemize}
\item If \TZ{V} is \TZ{A}, then
\[\Rep(\TZ{V}) = \text{\T{\{record_field,\LINE,$\Rep(\Z{A})$\}}}.\]
\item If \TZ{V} is \T{\Z{A} = \Z{E}}, then
\[\Rep(\TZ{V}) = \text{\T{\{record_field,\LINE,$\Rep(\Z{A})$,$\Rep(\Z{E})$\}}}.\]
\end{itemize}

\section{Atomic literals}

\label{section:atomic-literal-rep}

There are five kinds of atomic literals, which are represented in the
same way in patterns, expressions and guard expressions:
\begin{itemize}
\item If \TZ{L} is an integer literal, then
\[\Rep(\TZ{L}) = \text{\T{\{integer,\LINE,\Z{L}\}}}.\]
\item If \TZ{L} is a float literal, then
\[\Rep(\TZ{L}) = \text{\T{\{float,\LINE,\Z{L}\}}}.\]
\item If \TZ{L} is a character literal, then
\ifStd
\[\Rep(\TZ{L}) = \text{\T{\{char,\LINE,\Z{L}\}}}.\]
\else
\[\Rep(\TZ{L}) = \text{\T{\{integer,\LINE,\Z{L}\}}}.\]
\fi
\item If \TZ{L} is a string literal consisting of the characters
$\TZ{C}_1$, \ldots, $\TZ{C}_k$, then
\[\Rep(\TZ{L}) = \text{\T{\{string,\LINE,[$\Z{C}_1$,\tdots,$\Z{C}_k$]\}}}.\]
\item If \TZ{L} is an atom literal, then
\[\Rep(\TZ{L}) = \text{\T{\{atom,\LINE,\Z{L}\}}}.\]
\end{itemize}

\section{Patterns}

For a sequence \TZ{Ps} of patterns $\TZ{P}_1$, \ldots, $\TZ{P}_k$,
\[\Rep(\TZ{Ps}) = \text{\T{[$\Rep(\Z{P}_1)$,\tdots,$\Rep(\Z{P}_k)$]}}.\]

\noindent Individual patterns are represented as follows:
\begin{itemize}
\item If \TZ{P} is a compound pattern \T{$\Z{P}_1$ = $\Z{P}_2$}, then
\[\Rep(\TZ{P}) = \text{\T{\{match,\LINE,$\Rep(\Z{P}_1)$,$\Rep(\Z{P}_2)$\}}}.\]
\item If \TZ{P} is an atomic literal \TZ{L}, then
\[\Rep(\TZ{P}) = \Rep(\TZ{L}),\]
cf.\ \S\ref{section:atomic-literal-rep}.
\item If \TZ{P} is a variable pattern \TZ{V}, then
\[\Rep(\TZ{P}) = \text{\T{\{var,\LINE,\TZ{A}\}}},\]
where \TZ{A} is an atom with a printname that constitutes the same characters as \TZ{V}.
\item If \TZ{P} is a universal pattern \T{_}, then
\[\Rep(\TZ{P}) = \text{\T{\{var,\LINE,'_'\}}}.\]
\item If \TZ{P} is a tuple pattern \T{\{$\Z{P}_1$,\tdots,$\Z{P}_k$\}}, then
\[\Rep(\TZ{P}) = \text{\T{\{tuple,\LINE,[$\Rep(\Z{P}_1)$,\tdots,$\Rep(\Z{P}_k)$]\}}}.\]
\item If \TZ{P} is a nil pattern \T{[]}, then
\[\Rep(\TZ{P}) = \text{\T{\{nil,\LINE\}}}.\]
\item If \TZ{P} is a cons pattern \T{[$\Z{P}_h$|$\Z{P}_t$]}, then
\[\Rep(\TZ{P}) = \text{\T{\{cons,\LINE,$\Rep(\Z{P}_h)$,$\Rep(\Z{P}_t)$\}}}.\]
\item If \TZ{P} is a record pattern \T{\char`\#\TZ{Name}\{$\Z{Field}_1$=$\Z{P}_1$,\tdots,$\Z{Field}_k$=$\Z{P}_k$\}}, then
\begin{align*}
\Rep(\TZ{P}) =
\text{\T{\{}}&\text{\T{record,\LINE,\Z{Name},}} \\
             &\text{\T{[\{record_field,\LINE,$\Rep(\Z{Field}_1)$,$\Rep(\Z{P}_1)$\},\tdots,}} \\
             &\text{\T{~\{record_field,\LINE,$\Rep(\Z{Field}_k)$,$\Rep(\Z{P}_k)$\}]\}}.}
\end{align*}
\end{itemize}

\section{Expressions}

\label{section:expressions-rep}

A body \TZ{B} is a sequence of expressions \T{$\Z{E}_1$, \ldots, $\Z{E}_k$},
where $k\geq 1$, and
\[\Rep(\TZ{B}) = \text{\T{[$\Rep(\Z{E}_1)$,\tdots,$\Rep(\Z{E}_k)$]}}.\]

%For a sequence \TZ{Es} of expressions $\TZ{E}_1$, \ldots, $\TZ{E}_k$,
%\[\Rep(\TZ{Es}) = \text{\T{[$\Rep(\Z{E}_1)$,\tdots,$\Rep(\Z{E}_k)$]}}.\]

\noindent An expression \TZ{E} is one of the following alternatives:
\begin{itemize}
\item If \TZ{E} is \T{catch $\Z{E}_0$}, then
\[\Rep(\TZ{E}) = \text{\T{\{'catch',\LINE,$\Rep(\Z{E}_0)$\}}}.\]
\item If \TZ{E} is \T{\Z{P} = $\Z{E}_0$}, then
\[\Rep(\TZ{E}) = \text{\T{\{match,\LINE,$\Rep(\Z{P})$,$\Rep(\Z{E}_0)$\}}}.\]
%\item If \TZ{E} is \T{$\Z{E}_1$ !\ $\Z{E}_2$}, then
%\[\Rep(\TZ{E}) = \text{\T{\{op,\LINE,'!',$\Rep(\Z{E}_1)$,$\Rep(\Z{E}_2)$\}}}.\]
\item If \TZ{E} is \T{$\Z{E}_1$ \Z{Op} $\Z{E}_2$}, where \TZ{Op} is \T{!}, \T{or}, \T{and},
a \NT{RelationalOp}, an \NT{EqualityOp}, a \NT{ListConcOp}, an \NT{AdditionOp},
a \NT{ShiftOp} or a \NT{MultiplicationOp}, then
\[\Rep(\TZ{E}) = \text{\T{\{op,\LINE,\Z{Op},$\Rep(\Z{E}_1)$,$\Rep(\Z{E}_2)$\}}}.\]
\item If \TZ{E} is \T{\Z{Op} $\Z{E}_0$}, where \TZ{Op} is a \NT{PrefixOp}, then
\[\Rep(\TZ{E}) = \text{\T{\{op,\LINE,\Z{Op},$\Rep(\Z{E}_0)$\}}}.\]
\item If \TZ{E} is \T{\char`\#\Z{Name}.\Z{Field}}, then
\[\Rep(\TZ{E}) = \text{\T{\{record_index,\LINE,\Z{Name},$\Rep(\Z{Field})$\}}}.\]
\item If \TZ{E} is \T{$\Z{E}_0$\char`\#\Z{Name}.\Z{Field}}, then
\[\Rep(\TZ{E}) = \text{\T{\{record_field,\LINE,$\Rep(\Z{E}_0)$,\Z{Name},$\Rep(\Z{Field})$\}}}.\]
\item If \TZ{E} is \T{\char`\#\Z{Name}\{$\Z{Field}_1$=$\Z{E}_1$,\tdots,$\Z{Field}_k$=$\Z{E}_k$\}}, then
\begin{align*}
\Rep(\TZ{E}) =
\text{\T{\{}}&\text{\T{record,\LINE,\Z{Name},}} \\
             &\text{\T{[\{record_field,\LINE,$\Rep(\Z{Field}_1)$,$\Rep(\Z{E}_1)$\},\tdots,}} \\
             &\text{\T{~\{record_field,\LINE,$\Rep(\Z{Field}_k)$,$\Rep(\Z{E}_k)$\}]\}}.}
\end{align*}
\item If \TZ{E} is \T{$\Z{E}_0$\char`\#\Z{Name}\{$\Z{Field}_1$=$\Z{E}_1$,\tdots,$\Z{Field}_k$=$\Z{E}_k$\}}, then
\begin{align*}
\Rep(\TZ{E}) =
\text{\T{\{}}&\text{\T{record,\LINE,$\Rep(\Z{E}_0)$,\Z{Name},}} \\
             &\text{\T{[\{record_field,\LINE,$\Rep(\Z{Field}_1)$,$\Rep(\Z{E}_1)$\},\tdots,}} \\
             &\text{\T{~\{record_field,\LINE,$\Rep(\Z{Field}_k)$,$\Rep(\Z{E}_k)$\}]\}}.}
\end{align*}
\item If \TZ{E} is \T{$\Z{E}_0$($\Z{E}_1$,\tdots,$\Z{E}_k$)} (the case where
$\Z{E}_0$ is an atom is not distinguished), then
\[\Rep(\TZ{E}) = \text{\T{\{call,\LINE,$\Rep(\Z{E}_0)$,[$\Rep(\Z{E}_1)$,\tdots,$\Rep(\Z{E}_k)$]\}}}.\]
\item If \TZ{E} is \T{$\TZ{E}_m$:$\Z{E}_0$($\Z{E}_1$,\tdots,$\Z{E}_k$)}, then
\begin{align*}
\Rep(\TZ{E}) =
\text{\T{\{call,\LINE,\{remote,\LINE,$\Rep(\Z{E}_m)$,$\Rep(\Z{E}_0)$\},[}}&\text{\T{$\Rep(\Z{E}_1)$,\tdots,}} \\
                                                                          &\text{\T{$\Rep(\Z{E}_k)$]\}}.}
\end{align*}
\item If \TZ{E} is a variable \TZ{V}, then
\[\Rep(\TZ{E}) = \text{\T{\{var,\LINE,\TZ{A}\}}},\]
where \TZ{A} is an atom with a printname that constitutes the same characters as \TZ{V}.
\item If \TZ{P} is an atomic literal \TZ{L}, then
\[\Rep(\TZ{P}) = \Rep(\TZ{L}),\]
cf.\ \S\ref{section:atomic-literal-rep}.
\item If \TZ{E} is a tuple skeleton \T{\{$\Z{E}_1$,\tdots,$\Z{E}_k$\}}, then
\[\Rep(\TZ{E}) = \text{\T{\{tuple,\LINE,[$\Rep(\Z{E}_1)$,\tdots,$\Rep(\Z{E}_k)$]\}}}.\]
\item If \TZ{E} is \T{[]}, then
\[\Rep(\TZ{E}) = \text{\T{\{nil,\LINE\}}}.\]
\item If \TZ{E} is a cons skeleton \T{[$\Z{E}_h$|$\Z{E}_t$]}, then
\[\Rep(\TZ{E}) = \text{\T{\{cons,\LINE,$\Rep(\Z{E}_h)$,$\Rep(\Z{E}_t)$\}}}.\]
\item If \TZ{E} is a list comprehension \T{[$\Z{E}_0$ || $\Z{W}_1$, \tdots, $\Z{W}_k$]},
where each $\TZ{W}_i$, $1\leq i\leq k$, is a generator or a filter, then
\[\Rep(\TZ{E}) = \text{\T{\{lc,\LINE,$\Rep(\Z{E}_0)$,[$\Rep(\Z{W}_1)$,\tdots,$\Rep(\Z{W}_k)$]\}}}.\]
\item If \TZ{E} is \T{begin \TZ{B} end}, where \TZ{B} is a body, then
\[\Rep(\TZ{E}) = \text{\T{\{block,\LINE,$\Rep(\Z{B})$\}}}.\]
\ifStd
\item If \TZ{E} is \T{cond $\Z{E}_1$ -> $\Z{B}_1$ ; \tdots ; $\Z{E}_k$ -> $\Z{B}_k$ end},
where $k\geq1$ and each $\Z{E}_i$ and $\Z{B}_i$, $1\leq i\leq k$, is an expression
and a body, respectively, then
\begin{align*}
\Rep(\TZ{E}) =
\text{\T{\{'cond',\LINE,[}}&\text{\T{\{c_clause,\LINE,$\Rep(\Z{E}_1)$,$\Rep(\Z{B}_1)$\},\tdots,}} \\
                           &\text{\T{\{c_clause,\LINE,$\Rep(\Z{E}_k)$,$\Rep(\Z{B}_k)$\}]\}}.}
\end{align*}
\fi
\item If \TZ{E} is \T{if $\Z{Gs}_1$ -> $\Z{B}_1$ ; \tdots ; $\Z{Gs}_k$ -> $\Z{B}_k$ end},
where $k\geq1$ and each $\Z{Gs}_i$ and $\Z{B}_i$, $1\leq i\leq k$, is a guard and a body,
respectively, then
\begin{align*}
\Rep(\TZ{E}) =
\text{\T{\{'if',\LINE,[}}&\text{\T{\{clause,\LINE,[],$\Rep(\Z{Gs}_1)$,$\Rep(\Z{B}_1)$\},\tdots,}} \\
                         &\text{\T{\{clause,\LINE,[],$\Rep(\Z{Gs}_k)$,$\Rep(\Z{B}_k)$\}]\}}.}
\end{align*}
\item If \TZ{E} is \T{case $\Z{E}_0$ of $\Z{P}_1$ $[\text{\T{when $\Z{Gs}_1$}}]$ -> $\Z{B}_1$ ; \tdots ; $\Z{P}_k$ $[\text{\T{when $\Z{Gs}_k$}}]$ -> \\ $\Z{B}_k$ end},
where $\TZ{E}'$ is an expression, $k\geq1$ and each $\Z{P}_i$, $\Z{Gs}_i$ and $\Z{B}_i$, $1\leq i\leq k$,
is a pattern, a guard and a body, respectively (and $\Z{Gs}_i$ is \T{true} if
omitted), then
\begin{align*}
\Rep(\TZ{E}) =
\text{\T{\{}}&\text{\T{'case',\LINE,$\Rep(\Z{E}_0)$,}} \\
             &\text{\T{[\{clause,\LINE,[$\Rep(\Z{P}_1)$],$\Rep(\Z{Gs}_1)$,$\Rep(\Z{B}_1)$\},\tdots,}} \\
             &\text{\T{~\{clause,\LINE,[$\Rep(\Z{P}_k)$],$\Rep(\Z{Gs}_k)$,$\Rep(\Z{B}_k)$\}]\}}.}
\end{align*}
\item If \TZ{E} is \T{receive $\Z{P}_1$ $[\text{\T{when $\Z{Gs}_1$}}]$ -> $\Z{B}_1$ ; \tdots ; $\Z{P}_k$ $[\text{\T{when $\Z{Gs}_k$}}]$ -> $\Z{B}_k$ end},
where $k\geq1$ and each $\Z{P}_i$, $\Z{Gs}_i$ and $\Z{B}_i$, $1\leq i\leq k$,
is a pattern, a guard and a body, respectively (and $\Z{Gs}_i$ is \T{true} if
omitted), then
\begin{align*}
\Rep(\TZ{E}) =
\text{\T{\{}}&\text{\T{'receive',\LINE,}} \\
             &\text{\T{[\{clause,\LINE,[$\Rep(\Z{P}_1)$],$\Rep(\Z{Gs}_1)$,$\Rep(\Z{B}_1)$\},\tdots,}} \\
             &\text{\T{~\{clause,\LINE,[$\Rep(\Z{P}_k)$],$\Rep(\Z{Gs}_k)$,$\Rep(\Z{B}_k)$\}]\}}.}
\end{align*}
\item If \TZ{E} is \T{receive $\Z{P}_1$ $[\text{\T{when $\Z{Gs}_1$}}]$ -> $\Z{B}_1$ ; \tdots ; $\Z{P}_k$ $[\text{\T{when $\Z{Gs}_k$}}]$ -> $\Z{B}_k$ after $\TZ{E}'$ -> $\Z{B}_{k+1}$ end},
where $k\geq1$, each $\Z{P}_i$, $\Z{Gs}_i$ and $\Z{B}_i$, $1\leq i\leq k$,
is a pattern, a guard and a body, respectively (and $\Z{Gs}_i$ is \T{true} if
omitted), $\TZ{E}'$ is an expression and $\TZ{B}_{k+1}$ is a body, then
\begin{align*}
\Rep(\TZ{E}) =
\text{\T{\{}}&\text{\T{'receive',\LINE,}} \\
             &\text{\T{[\{clause,\LINE,[$\Rep(\Z{P}_1)$],$\Rep(\Z{Gs}_1)$,$\Rep(\Z{B}_1)$\},\tdots,}} \\
             &\text{\T{~\{clause,\LINE,[$\Rep(\Z{P}_k)$],$\Rep(\Z{Gs}_k)$,$\Rep(\Z{B}_k)$\}],}} \\
             &\text{\T{$\Rep(\Z{E}_0)$,$\Rep(\Z{B}_{k+1})$\}}.}
\end{align*}
\ifStd
\item If \TZ{E} is \T{try $\Z{B}$ catch $\Z{P}_1$ $[\text{\T{when $\Z{Gs}_1$}}]$ -> $\Z{B}_1$ ; \tdots ; $\Z{P}_k$ $[\text{\T{when $\Z{Gs}_k$}}]$ -> $\Z{B}_k$ end},
where \TZ{B} is a body, $k\geq1$ and each $\Z{P}_i$, $\Z{Gs}_i$ and $\Z{B}_i$, $1\leq i\leq k$,
is a pattern, a guard and a body, respectively (and $\Z{Gs}_i$ is \T{true} if
omitted), then
\begin{align*}
\Rep(\TZ{E}) =
\text{\T{\{}}&\text{\T{'try',\LINE,$\Rep(\Z{B})$,}} \\
             &\text{\T{[\{clause,\LINE,[$\Rep(\Z{P}_1)$],$\Rep(\Z{Gs}_1)$,$\Rep(\Z{B}_1)$\},\tdots,}} \\
             &\text{\T{~\{clause,\LINE,[$\Rep(\Z{P}_k)$],$\Rep(\Z{Gs}_k)$,$\Rep(\Z{B}_k)$\}]\}}.}
\end{align*}
\item If \TZ{E} is \T{try $\Z{B}$ end},
where \TZ{B} is a body, then
\[\Rep(\TZ{E}) = \text{\T{\{'try',\LINE,$\Rep(\Z{B})$,[]\}}}.\]
\fi
\item If \TZ{E} is \T{fun \Z{Name}/\Z{Arity}}, then
\[\Rep(\TZ{E}) = \text{\T{\{'fun',\LINE,\{function,\Z{Name},\Z{Arity}\}\}}}.\]
\item If \TZ{E} is \T{fun $\Z{P}_1$ $[\text{\T{when $\Z{Gs}_1$}}]$ -> $\Z{B}_1$ ; \tdots ; $\Z{P}_k$ $[\text{\T{when $\Z{Gs}_k$}}]$ -> $\Z{B}_k$ end},
where $k\geq1$ and each $\Z{P}_i$, $\Z{Gs}_i$ and $\Z{B}_i$, $1\leq i\leq k$,
is a pattern, a guard and a body, respectively (and $\Z{Gs}_i$ is \T{true} if
omitted), then
\begin{align*}
\Rep(\TZ{E}) =
\text{\T{\{'fun',\LINE,\{}}&\text{\T{clauses,}} \\
                           &\text{\T{[\{clause,\LINE,[$\Rep(\Z{P}_1)$],$\Rep(\Z{Gs}_1)$,$\Rep(\Z{B}_1)$\}]\},}} \\
                           &\text{\T{\tdots,}} \\
                           &\text{\T{~\{clause,\LINE,[$\Rep(\Z{P}_k)$],$\Rep(\Z{Gs}_k)$,$\Rep(\Z{B}_k)$\}]\}\}}.}
\end{align*}
\ifOld
\item If \TZ{E} is \T{query [$\Z{E}_0$ || $\Z{W}_1$,\tdots,$\Z{W}_k$] end},
where each $\TZ{W}_i$, $1\leq i\leq k$, is a generator or a filter, then
\[\Rep(\TZ{E}) = \text{\T{\{'query',\LINE,\{lc,\LINE,$\Rep(\Z{E}_0)$,[$\Rep(\Z{W}_1)$,\tdots,$\Rep(\Z{W}_k)$]\}\}}}.\]
\item If \TZ{E} is \T{$\Z{E}_0$.\Z{Field}}, a Mnesia record access
inside a \T{query}, then
\[\Rep(\TZ{E}) = \text{\T{\{record_field,\LINE,$\Rep(\Z{E}_0)$,$\Rep(\Z{Field})$\}}}.\]
\fi
\item If \TZ{E} is \T{( $\TZ{E}'$ )}, then
\[\Rep(\TZ{E}) = \Rep(\TZ{E}'),\]
i.e., parenthesized expressions cannot be distinguished from their bodies.
\end{itemize}
When \TZ{W} is a generator or a filter (in the body of a list comprehension), then:
\begin{itemize}
\item If \TZ{W} is a generator \T{\Z{P} <- \Z{E}}, where \TZ{P} is a pattern and \TZ{E}
is an expression, then
\[\Rep(\TZ{W}) = \text{\T{\{generate,\LINE,$\Rep(\Z{P})$,$\Rep(\Z{E})$\}}}.\]
\item If \TZ{W} is a filter \TZ{E}, which is an expression, then
\[\Rep(\TZ{W}) = \Rep(\TZ{E}).\]
\end{itemize}

\section{Guards}

A guard \TZ{Gs} is a nonempty sequence of guard tests $\TZ{G}_1$, \ldots, $\TZ{G}_k$, and
\[\Rep(\TZ{Gs}) = \text{\T{[$\Rep(\Z{G}_1)$,\tdots,$\Rep(\Z{G}_k)$]}}.\]

A guard test \TZ{G} is either \T{true}, an application of a BIF to a sequence of guard
expressions (syntactically this includes guard record tests), or a binary operator
applied to two guard expressions.
\begin{itemize}
\item If \TZ{G} is \T{true}, then
\[\Rep(\TZ{G}) = \text{\T{\{atom,\LINE,true\}}}.\]
\item If \TZ{G} is an application \T{\Z{A}($\Z{E}_1$,\tdots,$\Z{E}_k$)}, where \TZ{A}
is an atom and $\TZ{E}_1$, \ldots, $\TZ{E}_k$ are guard expressions, then
\[\Rep(\TZ{G}) = \text{\T{\{call,\LINE,\{atom,\LINE,\Z{A}\},[$\Rep(\Z{E}_1)$,\tdots,$\Rep(\Z{E}_k)$]\}}}.\]
\item If \TZ{G} is an operator expression \T{$\Z{E}_1$ \Z{Op} $\Z{E}_2$}, where \TZ{Op}
is a \NT{RelationalOp} or an \NT{EqualityOp}, and $\TZ{E}_1$, $\TZ{E}_2$ are guard
expressions, then
\[\Rep(\TZ{G}) = \text{\T{\{op,\LINE,\Z{Op},$\Rep(\Z{E}_1)$,$\Rep(\Z{E}_2)$\}}}.\]
\end{itemize}
All guard expressions are expressions and are represented in the same way as
the corresponding expressions, cf.\ \S\ref{section:expressions-rep}.
\index{Rep@\Rep|)}
\index{parse tree!standard representation|)}


%
% %CopyrightBegin%
%
% Copyright Ericsson AB 2017. All Rights Reserved.
%
% Licensed under the Apache License, Version 2.0 (the "License");
% you may not use this file except in compliance with the License.
% You may obtain a copy of the License at
%
%     http://www.apache.org/licenses/LICENSE-2.0
%
% Unless required by applicable law or agreed to in writing, software
% distributed under the License is distributed on an "AS IS" BASIS,
% WITHOUT WARRANTIES OR CONDITIONS OF ANY KIND, either express or implied.
% See the License for the specific language governing permissions and
% limitations under the License.
%
% %CopyrightEnd%
%

\chapter{Portable hashing}

\label{chapter:hashing}
\index{term!hashing|(}
\index{erlang:hash/2 BIF@\T{erlang:hash/2} BIF|(}
\index{hash/2 BIF@\T{hash/2} BIF|(}

The function \I{Hash} defined in this appendix is used as part of the
definition of the BIF \T{erlang:hash/2} (\S\ref{section:hash2}).
Given an arbitrary \Erlang\ term and a positive integer $r$, it returns
an integer in the range $[0,r-1]$.  The function has been designed with
the aim to make it a good hash function, i.e., that it spreads function
values evenly across the range.
\index{erlang:hash/2 BIF@\T{erlang:hash/2} BIF|)}
\index{hash/2 BIF@\T{hash/2} BIF|)}

\section{Definitions}

%\index{C1, ..., C9@$C_1$, \ldots, $C_9$|(}
We make use of nine constants $C_1$, \ldots, $C_9$:
\begin{align*}
C_1 &= 268440163 \\
C_2 &= 268439161 \\
C_3 &= 268435459 \\
C_4 &= 268436141 \\
C_5 &= 268438633 \\
C_6 &= 268437017 \\
C_7 &= 268438039 \\
C_8 &= 268437511 \\
C_9 &= 268439627
\end{align*}
%\index{C1, ..., C9@$C_1$, \ldots, $C_9$|)}
%\index{Foldl@\I{Foldl}|(}
We will use a helper function \I{Foldl}, such that
\[\I{Foldl}(F,E,\langle v_1,\ldots,v_k\rangle) =
F(v_k,F(v_{k-1},\ldots F(v_2,F(v_1,E)) \ldots))\]
(Note that $\I{Foldl}(F,E,\langle\rangle) = E$, regardless of $F$.)
%\index{Foldl@\I{Foldl}|)}

\noindent\index{  bitwise exclusive OR@$\otimes$}
$w_1 \otimes w_2$ denotes the bitwise exclusive OR of $w_1$ and $w_2$.

\noindent All arithmetic operations in this appendix are modulo $2^{32}$.

\section{The hash function}

%\index{Hash@\I{Hash}|(}
The main function \I{Hash} is defined as follows:
\[\I{Hash}(\TZ{t},r) = H(\TZ{t},0) \bmod r\]
%\index{Hash@\I{Hash}|)}
The auxiliary function $H$ is defined by cases.
\begin{itemize}
\item If \TZ{t} is an atom having a printname with character codes $i_1$, ..., $i_k$,
where for all $j$, $1\leq j\leq k$, $i_j\in[0,255]$, then
\[H(\TZ{t},h) = C_1*h+\I{Foldl}(F,0,\langle i_1,\ldots,i_k\rangle),\]
where
\begin{align*}
F(i,h) &= G(16h + i) \\
G(j) &= (j \bmod 2^{28}) \otimes 16(\lfloor j / 2^{28}\rfloor).
\end{align*}
(Note that for any application of $F$, $i\in[0,255]$ and $h\in[0,2^{28}-1]$,
and for any application of $G$, $j\in[0,2^{28}-1]$.)
\item If \TZ{t} is a fixnum, then
\[H(\TZ{t},h) = C_2*h+(\Er[t] \bmod 2^{32}).\]
I'M NOT SURE I GOT THIS RIGHT AND I'D RATHER NOT MENTION FIXNUMS AND BIGNUMS!!!
\item If \TZ{t} is a bignum where the 32-bit words of its absolute value
in little-endian order is $w_1$, ..., $w_k$, then
\[H(\TZ{t},h) = C*\I{Foldl}(F,h,\langle w_1,\ldots,w_k\rangle)+k,\]
where
\begin{alignat*}{2}
     C &= C_2 && \qquad\text{if $\Er[t]\geq0$;} \\
       &= C_3 && \qquad\text{if $\Er[t]<0$.} \displaybreak[0]\\[\smallskipamount]
F(w,h) &= C_2*h+w
\end{alignat*}
\item If \TZ{t} is \T{[]}, then
\[H(\TZ{t},h) = C_3*h+1.\]
\item If \TZ{t} is a binary consisting of the bytes $i_1$, \ldots, $i_k$, then
\[H(\TZ{t},h) = C_4*\I{Foldl}(F,h,\langle i_1,\ldots,i_l\rangle)+k,\]
where
\begin{align*}
l &= \min(k,15) \\
F(i,h) = C_1*h+i.
\end{align*}
\item If \TZ{t} is a PID, then
\[H(\TZ{t},h) = C_5*h+\I{MagicPid}(\TZ{t}).\]
\item If \TZ{t} is a port or a ref, then
\[H(\TZ{t},h) = C_9*h+\I{MagicPortRef}(\TZ{t}).\]
\item If \TZ{t} is a float represented by the two unsigned 32-bit quantities
$w_1$ and $w_2$, then (THIS IS NOT VERY PORTABLE!!!)
\[H(\TZ{t},h) = C_6*h+(w_1 \otimes w_2).\]
\item If \TZ{t} is a term \T{[$\Z{t}_1$,\tdots,$\Z{t}_k$|$\Z{t}_{k+1}$]}, then
\[H(\TZ{t},h) = C_8*H(\TZ{t}_{k+1},\I{Foldl}(H,h,\langle t_1,\ldots,t_k\rangle)).\]
\item If \TZ{t} is a tuple \T{\{$\Z{t}_1$,\tdots,$\Z{t}_k$\}}, then
\[H(\TZ{t},h) = C_9*\I{Foldl}(H,h,\langle t_1,\ldots,t_k\rangle)+k.\]
\end{itemize}
\index{term!hashing|)}


%
% %CopyrightBegin%
%
% Copyright Ericsson AB 2017. All Rights Reserved.
%
% Licensed under the Apache License, Version 2.0 (the "License");
% you may not use this file except in compliance with the License.
% You may obtain a copy of the License at
%
%     http://www.apache.org/licenses/LICENSE-2.0
%
% Unless required by applicable law or agreed to in writing, software
% distributed under the License is distributed on an "AS IS" BASIS,
% WITHOUT WARRANTIES OR CONDITIONS OF ANY KIND, either express or implied.
% See the License for the specific language governing permissions and
% limitations under the License.
%
% %CopyrightEnd%
%

\chapter{The external term format}

\label{chapter:external-format}
\index{term!external format|(}

The external term format is a representation of any \Erlang\ term
\ifStd (although possibly subject to tighter implementation limits than elsewhere) \fi
as a sequence of bytes.  It is used as part of the \Erlang\
distribution protocol but can also be accessed explicitly through the
BIFs
\ifOld \T{binary_to_term/1}\index{binary_to_term/1 BIF@\T{binary_to_term/1} BIF}
(\S\ref{section:binarytoterm1}) \fi
\ifStd \T{binary:to_term/1}\index{binary:to_term/1 BIF@\T{binary:to_term/1} BIF}
(\S\ref{section:binary:toterm1}) \fi
and
\ifOld \T{term_to_binary/1}\index{term_to_binary/1 BIF@\T{term_to_binary/1} BIF}
(\S\ref{section:termtobinary1}),\fi
\ifStd \T{term:to_binary/1}\index{term:to_binary/1 BIF@\T{term:to_binary/1} BIF}
(\S\ref{section:term:tobinary1}),\fi
in which a sequence of bytes is represented as
a binary.

The version of the external term format described here is 4.7.  It can
be recognized by the sequence of bytes beginning with a byte \T{131}.
Future versions of the external term format that are not compatible
with version 4.7 must begin the sequence of bytes differently.

We will describe the external term format as a function
$\I{TermRep}_{4.7}$ that given an \Erlang\ term returns a sequence of
bytes.  We use a mathematical notation rather than \Erlang\ function
declarations, although it would not be difficult to transform our
function definitions to an \Erlang\ program that returns a list of
bytes.

\section{Context}

\label{section:atom-tables}

The transformation from a term to a sequence of bytes is context-dependent.
We assume that we can access the name of the node on which a term \T{t}
resides.

\index{node!atom table|(}
\ifOld
There is also an \emph{atom table} that has 256 rows with keys
$0$ to $255$, each of which has an \Erlang\ atom as value.
\fi
\ifStd
There is also an \emph{atom table} with keys $0$ to
$\mathit{atom_table_size}-1$ (where $\mathit{atom_table_size}$ is an
implementation-defined parameter), each of which has an \Erlang\ atom
as value.
\fi
Its purpose is to reduce the communication when terms are being
transmitted between two nodes (although it would be possible to use an
atom table also, for example, when writing terms to a file).  The idea
is that every node has one atom table for each of its
friends\index{node!friendship} (\S\ref{chapter:nodes}).  Two nodes
that are friends, say $\TZ{N}_1$ and $\TZ{N}_2$, agree to ensure that
$\TZ{N}_1$'s atom table for $\TZ{N}_2$ (i.e.,
\T{atom_tables[$\Z{N}_1$]($\Z{N}_2$)}\index{atom_tables node property@\T{atom_tables} node property},
cf.\ \S\ref{section:node-state-dynamic})
will have the same contents as
$\TZ{N}_2$'s atom table for
$\TZ{N}_1$ (i.e., \T{atom_tables[$\Z{N}_2$]($\Z{N}_1$)}).

When an atom \TZ{A} is to be transmitted from node $\TZ{N}_1$ to
node $\TZ{N}_2$, the following
happens at node $\TZ{N}_1$.  First a hash value \TZ{I} between 0 and
\ifOld 255 \fi
\ifStd $\mathit{atom_table_size}-1$ \fi
is computed for \TZ{A} (cf.\ \S\ref{chapter:hashing}).  Then row
\TZ{I} of $\TZ{N}_1$'s atom table for $\TZ{N}_2$ is inspected.  If
that row has \TZ{A} as value, then only \TZ{I} is transmitted to
$\TZ{N}_2$ because it is assumed that also $\TZ{N}_2$'s atom table for
$\TZ{N}_1$ has \TZ{A} as value for \TZ{I}.  Otherwise row \TZ{I} of
$\TZ{N}_1$'s atom table for $\TZ{N}_2$ is updated to contain \TZ{A}
and the whole printname of \TZ{A} is transmitted together with \TZ{I}
to $\TZ{N}_2$.  Note that sequences of bytes must be transformed back
into terms at node $\TZ{N}_2$ in the same order as they were produced
at node $\TZ{N}_1$, for the atom tables to have the correct contents
at all times.
\index{node!atom table|)}

\section{Definitions}

We will use the following auxiliary functions.  When we write that
something is an error, it means that the transformation should fail.

\begin{itemize}
\item $\I{BE}(k,i)$ (for Big Endian) is the sequence of bytes
$\TZ{b}_1$, \ldots, $\TZ{b}_k$ such that
$\I{BigEndianValue}(\langle\TZ{b}_1, \ldots, \TZ{b}_k\rangle)$ is $i$.
It is an error if $i\notin[0,256^k-1]$.
\item $\I{BES}(k,i)$ (for Big Endian Signed) is the sequence of bytes
$\TZ{b}_1$, \ldots, $\TZ{b}_k$ such that $\I{BigEndianSignedValue}(\langle\TZ{b}_1, \ldots, \TZ{b}_k\rangle)$ is $i$.
It is an error if $i\notin[0,256^k-1]$.
\item $\I{ID}(\TZ{t})$, where \TZ{t} is a ref or a port, is the value of
\T{ID[\Z{t}]}, which uniquely identifies the ref or port on the node on
which it resides.
\ifOld
If \TZ{t} is a PID, $\I{ID}(\TZ{t})$ is the XXX least significant bits
of \T{ID[\Z{t}]}, cf.\ $\I{PS}(\TZ{t})$.
\fi
\ifStd
If \TZ{t} is a PID, $\I{ID}(\TZ{t})$ is the $32$ least significant
bits of \T{ID[\Z{t}]}, cf.\ $\I{PS}(\TZ{t})$.
\fi
\item $\I{FloatString}(f)$ is a sequence of 31 bytes where the first 26 (if $f$ is
nonnegative) or 27 (if $f$ is negative) are the character codes of the
string produced by the ISO C \T{printf} facility given the format
string \T{"\char`\%.20e"} and the remaining 4 or 5 bytes are \T{0}.
\item $\I{CR}(\TZ{t})$, where \TZ{t} is a ref, PID or port, is the value of
\T{creation[\Z{t}]} (a nonnegative integer), which distinguishes
between different invocations of nodes with the same name.
$\I{CR}(\TZ{t})$ must be in the range $[0,255]$.
\item $\I{LE}(k,i)$ (for Little Endian) is the unique sequence of bytes
$\TZ{b}_1$, \ldots, $\TZ{b}_k$ such that $\I{LittleEndianValue}(\langle\TZ{b}_1, \ldots, \TZ{b}_k\rangle)$
is $i$.  It is an error if $i\notin[0,256^k-1]$.
\item $\I{Log}(k,i)$ is the base $k$ logarithm of $i$ rounded towards zero.
It is an error if $i\leq 0$.
\item $\I{Node}(t)$ is an atom that names the node on which the term $t$ resides.
\ifOld
\item $\I{PS}(\TZ{t})$, where \TZ{t} is a PID is \T{ID[\Z{t}]} with the XXX least
significant bits removed, cf.\ $\I{ID}(\TZ{t})$.
\fi
\ifStd
\item $\I{PS}(\TZ{t})$, where \TZ{t} is a PID is \T{ID[\Z{t}]} with the $32$ least
significant bits removed, cf.\ $\I{ID}(\TZ{t})$.
\fi
%\item $\I{PrintLength}(a)$ is the length of the printname of the atom $a$.
%\item $\I{PrintName}(a)$ is a sequence of bytes that constitute the character codes
%of the printname of the atom $a$.  It is an error if the printname of $a$ contains
%characters with codes greater than 255.
\item $SignBit(i)$ is \T{0} if $i\geq 0$ and \T{1} if $i<0$.
%\item $Size(t)$ where $t$ is a tuple or binary is the number of elements it has.
\end{itemize}

We write a sequence of bytes $b_1$, \ldots, $b_k$ as $\langle
b_1,\ldots,b_k\rangle$.  Juxtaposition denotes concatenation, so
\[\langle b_1,\ldots,b_k\rangle
\langle b'_1,\ldots,b'_l\rangle = \langle b_1,\ldots,b_k,b'_1,\ldots,b'_l\rangle.\]

\section{The transformation}

\[\I{TermRep}_{4.7}(\TZ{T}) = \langle\T{131}\rangle\,\I{TR}_{4.7}(\TZ{T})\]

The value of $\I{TR}_{4.7}(\TZ{T})$ is defined by cases.  When cases
overlap, either case could be used but the more specific case is
preferred.
\begin{itemize}
\item If \TZ{T} is an integer in the range $[0,255]$, then
\[\I{TR}_{4.7}(\TZ{T}) = \langle\T{97}\rangle\,\langle\TZ{T}\rangle.\]
\item If \TZ{T} is an integer in the range $[-2^{31},2^{31}-1]$
(but typically not in the range $[0,255]$), then
\[\I{TR}_{4.7}(\TZ{T}) = \langle\T{98}\rangle\,\I{BES}(4,\TZ{T}).\]
\item If \TZ{T} is an integer in the range $[-(256^{255}-1),256^{255}-1]$
(but typically not in the range $[-2^{31},2^{31}-1]$), then
\[\I{TR}_{4.7}(\TZ{T}) =
\langle\T{110},l,\I{SignBit}(t)\rangle\,\I{LittleEndian}(l,\TZ{T}),\]
where $l=\I{Log}(256,|\TZ{T}|)+1$.
\item If \TZ{T} is an integer in the range $[-(256^{2^{32}-1}-1),256^{2^{32}-1}-1]$
(but typically not in the range $[-(256^{255}-1),256^{255}-1]$), then
\[\I{TR}_{4.7}(\TZ{T}) =
\langle\T{111}\rangle\,\I{BE}(4,l)\,
\langle\I{SignBit}(\TZ{T})\rangle\,
\I{LittleEndian}(l,|\TZ{T}|),\]
where $l=\I{Log}(256,|\TZ{T}|)+1$.
\item If \TZ{T} is a float, then
\[\I{TR}_{4.7}(\TZ{T}) = \langle\T{99}\rangle,\I{FloatString}(\TZ{T}).\]
\item If \TZ{T} is an atom where the printname consists of $k$ characters with codes
$i_1$, \ldots, $i_k$, such that $k\leq255$ and for all $j$, $1\leq j\leq k$,
$i_1\in[0,255]$, then
\[\I{TR}_{4.7}(\TZ{T}) = \langle\T{100},\T{0},k,i_1,\ldots,i_k\rangle,\]
or if the atom is also to be stored in row $p$ of the atom table (where $p\in[0,255]$),
then
\[\I{TR}_{4.7}(\TZ{T}) = \langle\T{78},p,\T{0},k,i_1,\ldots,i_k\rangle,\]
or if the atom is already stored in row $p$ of the atom table (where $p\in[0,255]$),
then
\[\I{TR}_{4.7}(\TZ{T}) = \langle\T{67},p\rangle.\]
\item If \TZ{T} is a ref, then
\[\I{TR}_{4.7}(\TZ{T}) = \langle\T{101}\rangle\,\I{TR}_{4.7}(\I{Node}(\TZ{T}))\,
\I{BE}(4,\I{ID}(\TZ{T}))\,\langle\I{CR}(\TZ{T})\rangle.\]
\item If \TZ{T} is a port, then
\[\I{TR}_{4.7}(\TZ{T}) = \langle\T{102}\rangle\,\I{TR}_{4.7}(\I{Node}(\TZ{T}))\,
\I{BE}(4,\I{ID}(\TZ{T})),\langle\I{CR}(\TZ{T})\rangle.\]
\item If \TZ{T} is a PID, then
\[\I{TR}_{4.7}(\TZ{T}) = \langle\T{103}\rangle\,\I{TR}_{4.7}(\I{Node}(\TZ{T}))\,
\I{BE}(4,\I{ID}(\TZ{T}))\,\I{BE}(4,\I{PS}(\TZ{T}))
\langle\I{CR}(\TZ{T})\rangle.\]
\item If \TZ{T}is a tuple with elements $\TZ{T}_1$, \ldots, $\TZ{T}_k$, where $k\leq255$, then
\[\I{TR}_{4.7}(\TZ{T}) = \langle\T{104},k\rangle\,
\I{TR}_{4.7}(\TZ{T}_1)\,\cdots\,\I{TR}_{4.7}(\TZ{T}_k).\]
\item If \TZ{T} is a tuple with elements $\TZ{T}_1$, \ldots, $\TZ{T}_k$, where $k\leq2^{32}-1$, then
\[\I{TR}_{4.7}(\TZ{T}) = \langle\T{104}\rangle\,\I{BE}(4,k)\,
\I{TR}_{4.7}(\TZ{T}_1)\,\cdots\,\I{TR}_{4.7}(\TZ{T}_k).\]
\item If \TZ{T} is \T{[]}, then
\[\I{TR}_{4.7}(\TZ{T}) = \langle\T{106}\rangle.\]
\item If \TZ{T} is a (typically nonempty) string of Latin-1 characters having codes
$i_1$, \ldots, $i_k$, where $k\leq2^{16}-1$, then
\[\I{TR}_{4.7}(\TZ{T}) = \langle\T{107}\rangle\,\I{BE}(2,k)\,
\langle i_1,\ldots,i_k\rangle.\]
\item If \TZ{T} is a term \T{[$\TZ{T}_1$,\ldots,$\TZ{T}_k$|$\TZ{T}_{k+1}$]}, where
$k\leq2^{32}-1$, (but typically not a string covered by the previous case) then
\[\I{TR}_{4.7}(\TZ{T}) = \langle\T{108}\rangle\,\I{BE}(4,k)\,
\I{TR}_{4.7}(\TZ{T}_1)\,\cdots\,\I{TR}_{4.7}(\TZ{T}_{k+1}).\]
\item If \TZ{T} is a binary with elements $i_1$, \ldots, $i_k$, where $k\leq2^{32}-1$, then
\[\I{TR}_{4.7}(\TZ{T}) = \langle\T{109}\rangle\,\I{BE}(4,k)\,
\langle i_1,\ldots,i_k\rangle.\]
\item Otherwise, it is an error.
\end{itemize}

The inverse transformation is obvious.
\index{term!external format|)}


\ifStd
%
% %CopyrightBegin%
%
% Copyright Ericsson AB 2017. All Rights Reserved.
%
% Licensed under the Apache License, Version 2.0 (the "License");
% you may not use this file except in compliance with the License.
% You may obtain a copy of the License at
%
%     http://www.apache.org/licenses/LICENSE-2.0
%
% Unless required by applicable law or agreed to in writing, software
% distributed under the License is distributed on an "AS IS" BASIS,
% WITHOUT WARRANTIES OR CONDITIONS OF ANY KIND, either express or implied.
% See the License for the specific language governing permissions and
% limitations under the License.
%
% %CopyrightEnd%
%

\chapter{Inter-node communication}

\section{The Erlang Port Mapper Daemon}

\label{section:epmd}

\subsection{Registering a new node}

\label{section:epmd-register}

Don't forget to tell how to find the EPMD.

\subsection{Finding a node}

\label{section:epmd-find}

\section{The Port External Interface}

\label{chapter:port-interface}

Two \T{net_kernel} processes negotiating.

\fi

\grammarindextrue

%
% %CopyrightBegin%
%
% Copyright Ericsson AB 2017. All Rights Reserved.
%
% Licensed under the Apache License, Version 2.0 (the "License");
% you may not use this file except in compliance with the License.
% You may obtain a copy of the License at
%
%     http://www.apache.org/licenses/LICENSE-2.0
%
% Unless required by applicable law or agreed to in writing, software
% distributed under the License is distributed on an "AS IS" BASIS,
% WITHOUT WARRANTIES OR CONDITIONS OF ANY KIND, either express or implied.
% See the License for the specific language governing permissions and
% limitations under the License.
%
% %CopyrightEnd%
%

\chapter{Grammar}

\section{The lexical grammar}

\label{section:lex-gram-summary}
\index{grammar!lexical|(}

\begin{small}
\begin{rules}
\input{es-lex-gram}
\end{rules}
\end{small}
\index{grammar!lexical|)}

\section{The main grammar}

\label{section:main-gram-summary}
\index{grammar!main|(}

\begin{small}
\begin{rules}
\input{es-main-gram}
\end{rules}
\end{small}
\index{grammar!main|)}

\section{The preprocessor grammar}

\label{section:preproc-gram-summary}
\index{grammar!preprocessor|(}

\begin{small}
\begin{rules}
\input{es-preproc-gram}
\end{rules}
\end{small}
\index{grammar!preprocessor|)}


\grammarindexfalse

%
% %CopyrightBegin%
%
% Copyright Ericsson AB 2017. All Rights Reserved.
%
% Licensed under the Apache License, Version 2.0 (the "License");
% you may not use this file except in compliance with the License.
% You may obtain a copy of the License at
%
%     http://www.apache.org/licenses/LICENSE-2.0
%
% Unless required by applicable law or agreed to in writing, software
% distributed under the License is distributed on an "AS IS" BASIS,
% WITHOUT WARRANTIES OR CONDITIONS OF ANY KIND, either express or implied.
% See the License for the specific language governing permissions and
% limitations under the License.
%
% %CopyrightEnd%
%

% cross references for the index

\index{\ character@\T{\char`\\} character|see{escapes, character}}
\index{ANSI X3.4|see{ASCII}}
\index{application!of a function|see{function, application of}}
\index{applied occurrence|see{variable, applied occurrence}}
\index{argument!evaluation of|see{evaluation, of arguments}}
\index{arithmetic!float operations|see{float, arithmetic operations}}
\index{arithmetic!integer operations|see{integer, arithmetic operations}}
\index{arity of a function|see{function, arity}}
\index{association list|see{list, association}}
\index{atom!table|see{node, atom table}}
\index{bignum|see{integer, bignum}}
\index{binding occurrence|see{variable, binding occurrence}}
\index{built-in function|see{BIF}}
\index{call (a function)|see{function, call}}
\index{cause!for BIF exit|see{BIF, exit cause}}
\index{clause!function|see{function, clause}}
\index{code!generation|see{module, code generation}}
\index{code!part|see{module, code part of declaration}}
\index{communicating node|see{node, communicating}}
\index{comparison of terms|see{term, comparison}}
\index{compilation|see{module, compilation}}
\index{compound!term|see{term, compound}}
\index{compound!type|see{type, compound}}
\index{elementary!term|see{term, elementary}}
\index{elementary!type|see{type, elementary}}
\index{error!compile-time|see{compile-time, error}}
\index{error!run-time|see{run-time, error}}
\index{exceptional values|see{arithmetic, exceptional values}}
\index{extent!of function call|see{function, extent of call}}
\index{extent|see{term, life time of}}
\index{external format|see{term, external format}}
\index{field (of record)|see{record, declaration}}
\index{fixnum|see{integer, fixnum}}
\index{float!coercion to|see{coercion, to float}}
\index{floating-point number|see{float}}
\index{free variable|see{variable, free}}
\index{function!optimization of last call|see{last call optimization}}
\index{garbage collection|see{memory management}}
\index{group (of processes)|see{process, group}}
\index{guard!BIF|see{BIF, guard}}
\index{header part|see{module, header part of declaration}}
\index{immediate subterm|see{subterm, immediate}}
\index{ISO 8859-1|see{Latin-1}}
\index{ISO/IEC 10967-1|see{LIA-1}}
\index{ISO/IEC 10967-2|see{LIA-2}}
\index{isolated node|see{node, isolated}}
\index{left-to-right evaluation|see{evaluation, left-to-right}}
\index{linking (processes and ports)|see{process, linking}}
\index{list!of 2-tuples|see{association list}}
\index{literal!atomic|see{atomic literal}}
\index{local function application|see{function, local application}}
\index{magic cookie (of node)|see{node, magic cookie}}
\index{metavariable|see{variable, meta-}}
\index{module!applying parse transform to|see{parse transform}}
\index{module!loading|see{module, making current version}}
\index{module!reloading|see{module, replacing a version}}
\index{monitoring!a node|see{node, monitoring}}
\index{mutual recursion|see{recursion, mutual}}
\index{node!alive|see{node, communicating}}
\index{node!process registry|see{process, registry}}
\index{Open Telecom Platform|see{OTP}}
\index{operator!unary|see{operator, prefix}}
\index{parsing|see{module, parsing}}
\index{pattern matching!in match expression|see{match expression}}
\index{pattern!universal|see{universal pattern}}
\index{port!identifier|see{port, identifier}}
\index{port!linking|see{also process, linking}}
\index{preprocessing|see{module, preprocessing of}}
\index{preprocessing|see{module, preprocessing of}}
\index{printname|see{atom, printname}}
\index{priority (of a process)|see{process, priority}}
\index{process!identifier|see{PID}}
\index{process!completion|see{also completion}}
\index{process!name|see{process, registry}}
\index{process!status|see{process, scheduling}}
\index{production|see{grammar, production}}
\index{recognizer|see{BIF, recognizer}}
\index{record@\T{record}!expression|see{\T{record/2}}}
\index{recursion!mutual|see{mutual recursion}}
\index{reference|see{ref}}
\index{remote function application|see{function, remote application}}
\index{strict function|see{function, strict}}
%\index{sugar!syntactic|see{syntactic sugar}}
%\index{syntactic!category|see{grammar, syntactic category}}
\index{syntactic category|see{grammar, syntactic category}}
\index{term!order|see{term, comparison}}
\index{terminal|see{grammar, terminal}}
\index{test!comparison|see{term, comparison}}
\index{test!recognizer|see{BIF, recognizer}}
\index{test!record|see{\T{record/2}}}
\index{test!trivially true|see{\T{true} expression}}
\index{token|see{grammar, token}}
\index{trapping exit signal|see{exit, signal, trapping}}
\ifNew
\index{implementation|see{\Erlang, implementation of \Std}}
\index{Unicode!escape|see{escape, Unicode}}
\fi
\ifOld
\index{implementation|see{\OldErlang, implementation of}}
\fi

 % index cross-references

\backmatter

\printindex

\end{document}
