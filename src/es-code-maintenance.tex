%
% %CopyrightBegin%
%
% Copyright Ericsson AB 2017. All Rights Reserved.
%
% Licensed under the Apache License, Version 2.0 (the "License");
% you may not use this file except in compliance with the License.
% You may obtain a copy of the License at
%
%     http://www.apache.org/licenses/LICENSE-2.0
%
% Unless required by applicable law or agreed to in writing, software
% distributed under the License is distributed on an "AS IS" BASIS,
% WITHOUT WARRANTIES OR CONDITIONS OF ANY KIND, either express or implied.
% See the License for the specific language governing permissions and
% limitations under the License.
%
% %CopyrightEnd%
%

\chapter{Dynamics of modules}

\label{chapter:module-dynamics}

\Erlang\ has been designed to make it possible to incorporate functionality for
replacing a version of a module with a new version of that module, even though
at the same time there are processes executing the old version of the module.

\ifOld
An example of such functionality is provided by a collection of BIFs:
\T{load_module/2}, \T{delete_module/1}, \T{purge_module/1}, etc.,
which are described in this chapter.\footnote{These BIFs are, in turn, used for
implementing the \T{code} module of OTP \cite[p.~158--167]{otp-dev-ref}.}
\fi
\ifStd
An example of such functionality is provided by the standard module
\T{codeload} which is described in this chapter.\footnote{The module
\T{codeload} is, in turn, used for
implementing the \T{code} module of OTP \cite[p.~158--167]{otp-dev-ref}.}
Other (nonstandard) modules for code maintenance could behave differently.
\fi

A process that is evaluating
an application of a function in the old version of module can finish the evaluation
of that application even though a new version of the function has been loaded.
However, there can be only one ``old version'' of a module so
it is presumed that the \Erlang\ code of the modules has been written
so that processes will begin using the new version of the module as soon as possible.

As soon as a new version of a module named \TZ{Mod} has been loaded, evaluation of
remote applications of functions in module \TZ{Mod} will use the new version of module
(\S\ref{section:exported-functions}).

\index{function!tail recursive|(}
This implies, for example, that a tail recursive function which is intended to run
for a long time should typically use a remote function application
for the tail recursive call.  The tail recursive call will then use
the most recently loaded version of the module.  (Note also that
\ifStd
the requirement that an \Erlang\ implementation must provide
\fi
\ifOld the \fi
\emph{last call optimization}\index{last call optimization}
[\S\ref{section:lco}] entails that evaluation of such a tail recursive function
can run using stack space bounded only by the need for each iteration.)
\index{function!tail recursive|)}

Obviously, the programmer must be aware of the possibility that modules may be
replaced by new versions and design the code so it will cope gracefully with such
replacements.

\section{Loading or replacing a module}

\label{section:current-version}
\label{section:replacing-module}
\label{section:loading}

\index{module!version of|(}
A \emph{version} of a module named \TZ{Mod} is the result of compiling a
\NT{ModuleDeclaration} with a module attribute \T{-module(\Z{Mod})}, represented
as a binary (\S\ref{section:code-generation}).
(Module declarations are described in \S\ref{chapter:programs-modules}
and their compilation in \S\ref{chapter:compilation}.)

\index{module!version of!current|(}
The \emph{current version} of a module named \TZ{Mod} on a node \TZ{N} is the
one for which there are rows in \T{entry_points[\Z{N}]} (\S\ref{section:exported-functions}).
the current version of \TZ{Mod} on \TZ{N} is accessible as
\T{current_version[module_table[\Z{N}](\Z{Mod})]} (\S\ref{section:loaded-modules}).

\index{module!version of!old|(}
A version of a module named \TZ{Mod}, represented as a binary \TZ{B},
is the \emph{current version} on node \TZ{N} from the time it has been loaded on node \TZ{N}
(\S\ref{section:making-current-version}) until it is made old
on node \TZ{N} (\S\ref{section:making-old-version}).
\index{module!version of!current|)}

The \emph{old version} of a module named \TZ{Mod} on a node \TZ{N} has all its code
intact on \TZ{N} but there are no longer any rows for it in \T{entry_points[\Z{N}]} so
no new remote function calls can reach code in it.  The version is the old version
until it is purged
\index{module!version of!old|)}
(\S\ref{section:purging-old-version}).

When we write that a process \TZ{P} uses some version of a module \TZ{Mod},
represented as a binary \TZ{B}, we mean that \TZ{P} is
using\label{section:process-using-module}
some exported function \T{\Z{F}/\Z{A}} in \TZ{B} (\S\ref{section:function-use},
\S\ref{section:checking-process-module}).
\index{module!version of|)}

\index{module!replacing a version|(}
When a version of a module named \TZ{Mod} is to be replaced with a
new version on an \Erlang\ node
\TZ{N}, the following sequence of steps is intended to be followed:
\begin{itemize}
\item The current version of module \TZ{Mod} on node \TZ{N}, if any,
is made the old version (\S\ref{section:making-old-version}).
\item A version of module \TZ{Mod} is made the current version of \TZ{Mod}
on node \TZ{N} (\S\ref{section:making-current-version}).
\item When it has been ensured that no process on node \TZ{N} is using the old version
of module \TZ{Mod} anymore,
the old version of \TZ{Mod} is purged (\S\ref{section:purging-old-version}).
This reclaims the space used on node \TZ{N} by the old version of the code.
\index{module!replacing a version|)}

Ensuring that no process is using the old version
of a module can be accomplished in several ways, e.g.:
\begin{itemize}
\item Waiting until all such processes complete or no longer use the old version
of the code (cf.\ \S\ref{section:checking-process-module}).
\item Killing such processes.
\item Designing the code of the module so that a process using it can be sent a message that
makes it prepare for a version change by, e.g., completing or evaluating a remote
tail recursive call.
\end{itemize}
\end{itemize}

\section{Loaded modules}

\label{section:loaded-modules}
\index{module!loaded on a node|(}

At any time, a module is \emph{loaded} on node \TZ{N} if there is a current version of
\TZ{Mod} on \TZ{N}, i.e., if \T{current_version[module_table[\Z{N}](\Z{Mod})]} is a binary.

\S\ref{section:making-current-version} and \S\ref{section:making-old-version}
describe how \T{current_version[module_table[\Z{N}](\Z{Mod})]} is set to a binary or
to \T{none}, respectively.

A process residing on node \TZ{N} can find out whether module \TZ{Mod} is loaded on
node \TZ{N} through a BIF call
\ifOld \T{erlang:module_loaded(\Z{Mod})}\fi
\ifStd \T{codeload:module_loaded(\Z{Mod})}\fi,
which inspects \T{current_version[module_table[\Z{N}](\Z{Mod})]} and
returns \T{true} if the value is a binary
and \T{false} if it is \T{none} (\S\ref{section:moduleloaded1}).

% 0.7 removed, there is enugh elsewhere.
\iffalse
\index{module!preloaded on a node|(}
A process residing on node \TZ{N} can obtain a list of the module names in
\T{preloaded[\Z{N}]}, i.e., the names of the modules
that were loaded as part
of starting the node \TZ{N}, through a BIF call
\ifOld \T{erlang:pre_loaded()}\fi
\ifStd \T{codeload:pre_loaded()}\fi.
The order of the elements in the list is not specified.
\index{module!preloaded on a node|)}
\fi
\index{module!loaded on a node|)}

\section{Exported functions of a module}

\label{section:exported-functions}
\index{module!exported functions of|(}
\index{function!exported|(}

Each node maintains a table \T{entry_points[\Z{N}]} (\S\ref{section:node-state-dynamic})
in which the keys are triples consisting of a module name (an atom), a function
symbol (an atom) and an arity (a nonnegative integer) and the values are
\ifStd implementation-defined \fi
entry points to executable code.  The table contains one row for
each exported function of each loaded module.

\S\ref{section:making-current-version} and \S\ref{section:making-old-version}
describe how rows are are added to and removed from
\T{entry_points[\Z{N}]}, respectively.

The table \T{entry_points[\Z{N}]} is used implicitly
when evaluating remote function applications (\S\ref{section:appl-named-function}).
Only while a version \TZ{B} of a module \TZ{Mod} is current on a node \TZ{N} can
a remote call of an exported function \T{\Z{F}/\Z{n}} in \TZ{B} begin.
\index{module!exported functions of|)}
\index{function!exported|)}

\section{Loading a new current version}

\label{section:making-current-version}
\index{module!making version current|(}

When a binary \TZ{B} is to be made the current version of a module \TZ{Mod}
on a node \TZ{N}, there are two preconditions:
\begin{itemize}
\item \TZ{B} must represent the result of compiling a module declaration for
a module \TZ{Mod}, and
\item there must be no current version of module \TZ{Mod} on node \TZ{N}, i.e.,
\T{current_version[module_table[\Z{N}](\Z{Mod})]} should be \T{none}.
\end{itemize}
The actions required to make \TZ{B} the current version are
\begin{enumerate}
\item For each exported function \T{\Z{Name}/\Z{Arity}} of the version
of \TZ{Mod} represented by \TZ{B}, add to \T{entry_points[\Z{N}]} a
row with $(\TZ{Mod},\TZ{Name},\linebreak[0]\TZ{Arity})$ as key and an
entry point of the executable code for that function as value.
\item Set \T{current_version[module_table[\Z{N}](\Z{Mod})]} to \TZ{B}.
\end{enumerate}
\index{module!making version current|)}

\section{Making a current version old}

\label{section:making-old-version}
\index{module!making version old|(}

When the current version of a module \TZ{Mod} on a node \TZ{N} is to become the
old version of that module on the node, the precondition is that there is
not already an old version, i.e., that 
\T{old_version[module_table[\Z{N}](\Z{Mod})]} is \T{none}.

Let the value of
\T{current_version[module_table[\Z{N}](\Z{Mod})]}, i.e.,
the current version of \TZ{Mod} on \TZ{N} be \TZ{B}.
The actions required to make \TZ{B} the old version of \TZ{Mod} on \TZ{N} are
\begin{enumerate}
\item Remove every row from
\T{entry_points[\Z{N}]} that has \TZ{Mod} as key.
\item Set \T{old_version[module_table[\Z{N}](\Z{Mod})]} to \TZ{B}.
\item Set \T{current_version[module_table[\Z{N}](\Z{Mod})]} to \T{none}.
\end{enumerate}
\index{module!making version old|)}

\section{Purging an old version}

\label{section:purging-old-version}
\index{module!purging old version|(}

When the old version of a module \TZ{Mod} on a node \TZ{N} is to be
purged, there had better be no process using it.  If some process
residing on \TZ{N} is using the old version of \TZ{Mod}, the behaviour
of that process is thereafter undefined.

The action required to purge the old version of \TZ{Mod} on \TZ{N} is
setting\linebreak[2] \T{old_version[module_table[\Z{N}](\Z{Mod})]} to
\T{none}.  If that was the last reference to the binary that was the
old version of \TZ{Mod} on \TZ{N}, then the memory management
subsystem (\S\ref{section:memory-management}) will eventually reclaim
the memory occupied by it.  (The reason that processes still using the
old version may behave erratically is that it cannot be expected that
their references through return addresses will prevent the binary from
being ``garbage collected''.)
\index{module!purging old version|)}

\section{Checking a process for module usage}

\label{section:checking-process-module}
\index{module!used by process|(}
\index{process!using a module|(}

A BIF call
\ifOld \T{erlang:check_process_code(\Z{P},\Z{Mod})}
(\S\ref{section:checkprocesscode2}) \fi
\ifStd \T{codeload:check_process_code(\Z{P},\Z{Mod})}
(\S\ref{section:codeload:checkprocesscode2}) \fi
inspects the value of \T{stack_trace[\Z{P}]} and returns \T{true} if there is
a reference to the code of some function in
\T{old_version[module_table[node[\Z{P}]](\Z{Mod})]} and \T{false} otherwise.

A list of the PIDs of all processes residing on node \TZ{N} can be obtained
through a BIF call
\ifOld \T{processes()} (\S\ref{section:processes0}) \fi
\ifStd \T{node:processes()} (\S\ref{section:node:processes0}) \fi
on node \TZ{N}.
\index{module!used by process|)}
\index{process!using a module|)}


%%% OLD STUFF FROM HERE ON!


\iffalse
\section{Terminology}

% This is MID stuff.
%The most recently loaded module named \TZ{Mod} is called the \emph{current version of
%\TZ{Mod}}.  A module named \TZ{Mod} that is not the current version of \TZ{Mod}
%is called an \emph{old version of \TZ{Mod}}.

% This is MID stuff.
%\section{Modules}

% This is MID stuff with structs too.
Each node \TZ{N} has as part of its state a dictionary
\T{modules[\Z{N}]} mapping atoms (module names) to modules.
This dictionary is used at run time for obtaining the
current version of a named module when evaluating
a remote function application or resolving a remote struct name.
It can also be accessed through the BIF \T{get_module/1}.

A module \TZ{M} consists of
\begin{itemize}
\item A module name (an atom).
\item A mapping \T{exported_functions[\Z{M}]} from exported function
names to functions.
\item A mapping \T{internal_functions[\Z{M}]} from internal function
names to functions.
\item A mapping \T{exported_structs[\Z{M}]} from exported struct names
(atoms) to struct descriptors.
\item A mapping \T{internal_structs[\Z{M}]} from internal struct names
(atoms) to struct descriptors.
\end{itemize}

Given a module \TZ{M}, the mappings \T{exported_functions[\Z{M}]} and
\T{exported_structs[\Z{M}]} can be accessed from within any module
either
\begin{itemize}
\item by evaluating
a remote function application or resolving a remote struct name, or
\item through the BIFs \T{get_function/3} and \T{get_struct/2}.
\end{itemize}
The mappings \T{internal_functions[\Z{M}]} and
\T{internal_structs[\Z{M}]} can only be accessed from code occurring
lexically inside the module declaration for \TZ{M} by evaluating a
local function application or resolving a local struct name.

\section{The module dictionary}

Modules having the same name are presumed to provide similar
functionality and a module is thus presumed to supercede any
previously loaded module with the same name.  A remote function
application or remote struct name uses the module with that name most
recently loaded onto the node.  This is accomplished through the table
\T{modules[\Z{N}]} for each node \TZ{N}, which has module names as
keys BLURP...  to the most recently loaded module with that name.

Loading a module \TZ{M} with name \TZ{Mod} onto a node \TZ{N} means
that \T{modules[\Z{N}]} should be updated so that it maps the module
name \TZ{Mod} to \TZ{M}.  This does not have any effect on any
previously loaded module $\TZ{M}'$ named \TZ{Mod}.  A process that is
evaluating an application of a function in such a module $\TZ{M}'$
will continue to use functions in $\TZ{M}'$ for evaluating local
function applications and resolving local struct names.  However,
should it evaluate a remote function application or resolve a remote
struct name, that module name will be looked up in
\T{modules[\Z{N}]} and may yield \TZ{M} but never $\TZ{M}'$.

When there is no longer any reference to a module, the memory it
occupies can be reclaimed, as for any other data structure.

This means that the \emph{language} permits the simultaneous existence
of an unlimited number of versions of a module. For some
\emph{applications} this may not be desirable.  There are thus means
for obtaining at run time information about which processes have
ongoing evaluations of functions in a certain module. Such processes
can then be killed in order to purge all references to a module so
that its memory will eventually be reclaimed.

\section{Loading modules}

\T{codeload:load_module/1} or \T{2}?
\fi
